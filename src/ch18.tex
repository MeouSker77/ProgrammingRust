\chapter{输入输出}\label{ch18}

\emph{Doolittle: What concrete evidence do you have that you exist?\\
Bomb \#20: Hmmmm... well... I think, therefore I am.\\
Doolittle: That’s good. That’s very good. But how do you know that anything else exists?\\
Bomb \#20: My sensory apparatus reveals it to me.}

\begin{flushright}
    ——Dark Star
\end{flushright}

Rust中有关输入输出的特性围绕着三个trait:\texttt{Read}、\texttt{BufRead}、\texttt{Write}来组织:
\begin{enumerate}
    \item 实现了\texttt{Read}的值有读取字节输入的方法。它们被称为\emph{读者(reader)}。
    \item 实现了\texttt{BufRead}的值是\emph{buffered reader(有缓存的读者)}。它们支持\texttt{Read}的所有方法,加上读取文本的一行的方法,等等。
    \item 实现了\texttt{Write}的值支持字节和UTF-8文本输出。它们被称为\emph{写者(writer)}。
\end{enumerate}

\hyperref[f18-1]{图18-1}展示了这三个trait以及一些reader和writer类型的示例。

在本章中,我们将解释如何使用这些trait和它们的方法,包括图中出现的reader和writer类型,还有一些其他的和文件、终端、网络交互的方法。

\begin{figure}[htbp]
    \centering
    \includegraphics[width=0.8\textwidth]{../img/f18-1.png}
    \caption{Rust的三个主要的I/O trait以及一些实现了它们的类型}
    \label{f18-1}
\end{figure}

\section{Reader和Writer}

\emph{Reader}是你的程序可以从中读取字节的值。例如:
\begin{enumerate}
    \item 使用\texttt{std::fs::File::open(filename)}打开的文件
    \item 用于从网络中接收数据的\texttt{std::net::TcpStream}
    \item 进程用来读取标准输入的\texttt{std::io::stdin()}
    \item \texttt{std::io::Cursor<\&[u8]>}和\texttt{std::io::Cursor<Vec<u8>>}值,它们是从内存中的字节数组或vector中“读取”数据的reader
\end{enumerate}

\emph{Writer}是你的程序可以向其中写入字节的值。例如:
\begin{enumerate}
    \item 使用\texttt{std::fs::File::create(filename)}打开的文件
    \item 用于向网络中发送数据的\texttt{std::net::TcpStream}
    \item 用于写入到终端的\texttt{std::io::stdout()}和\texttt{std::io::stderr()}
    \item \texttt{Vec<u8>},它也是一个writer,它的\texttt{write}方法把数据附加到尾部
    \item \texttt{std::io::Cursor<Vec<u8>>},类似于上面,但允许你同时读取和写入数据,并可以在vector中定位到不同位置
    \item \texttt{std::io::Cursor<\&mut [u8]>},和\texttt{std::io::Cursor<Vec<u8>>}很像,除了它不能让缓冲区增长,因为它只是已经存在的字节数组的切片
\end{enumerate}

因为有为reader和writer设计的标准trait(\texttt{std::io::Read}和\texttt{std::io::Write}),所以编写可以处理多种输入输出通道的泛型代码是非常普遍的。例如,这里有一个函数拷贝任意reader中的所有字节到任意writer:
\begin{minted}{Rust}
    use std::io::{self, Read, Write, ErrorKind};

    const DEFAULT_BUF_SIZE: usize = 8 * 1024;

    pub fn copy<R: ?Sized, W: ?Sized>(reader: &mut R, writer: &mut W)
        -> io::Result<u64>
        where R: Read, W: Write
    {
        let mut buf = [0; DEFAULT_BUF_SIZE];
        let mut written = 0;
        loop {
            let len = match reader.read(&mut buf) {
                Ok(0) => return Ok(written),
                Ok(len) => len,
                Err(ref e) if e.kind() == ErrorKind::Interrupted => continue,
                Err(e) => return Err(e),
            };
            writer.write_all(&buf[..len])?;
            written += len as u64;
        }
    }
\end{minted}

这是Rust的标准库中的\texttt{std::io::copy()}的实现。因为它是泛型的,你可以使用它从\texttt{File}中读取数据然后写入到\texttt{TcpStream},或者从\texttt{Stdin}读取,然后写入到内存中的\texttt{Vec<u8>},等等。

如果你看不明白这里的错误处理代码,请复习\hyperref[ch07]{第7章}。我们将在接下来的内容中一直使用\texttt{Result}类型,掌握它的工作原理很重要。

这三个\texttt{std::io}的trait:\texttt{Read}、\texttt{BufRead}、\texttt{Write},以及\texttt{Seek}如此常用,以至于有一个只包含这些trait的\texttt{prelude}模块:
\begin{minted}{Rust}
    use std::io::prelude::*;
\end{minted}

本章中你还会见到它一到两次。我们通常也习惯导入\texttt{std::io}模块自身:
\begin{minted}{Rust}
    use std::io::{self, Read, Write, ErrorKind};
\end{minted}

这里的\texttt{self}关键字声明了\texttt{io}作为\texttt{std::io}模块的一个别名。这样,\texttt{std::io::Result}和\\
\texttt{std::io::Error}可以用\texttt{io::Result}和\texttt{io::Error}更简洁地表示出来,等等。

\subsection{Reader}
\texttt{std::io::Read}有几个方法用于读取数据。所有这些方法都通过\texttt{mut}引用获取self参数。

\codeentry{reader.read(\&mut buffer)}
\hangparagraph{从数据源读取一些字节,并存储到给定的\texttt{buffer}中。\texttt{buffer}参数的类型是\texttt{\&mut [u8]}。它最多读取\texttt{buffer.len()}个字节。}
\hangparagraph{返回类型是\texttt{io::Result<u64>},它是\texttt{Result<u64, io::Error>}的类型别名。当成功时,\texttt{u64}值是读取到的字节数,它可能等于或者小于\texttt{buffer.len()},\emph{即使还有更多的数据可以读取}。\texttt{Ok(0)}意味着没有更多的输入可以读取。}

\hangparagraph{当出错时,\texttt{.read()}返回\texttt{Err(err)},其中\texttt{err}是一个\texttt{io::Error}值。为了便于人类阅读,\texttt{io::Error}是可打印的;而对于程序,它有一个\texttt{.kind()}方法返回一个\texttt{io::ErrorKind}类型的错误码。这个枚举的成员有例如\texttt{PermissionDenied}和\texttt{ConnectionReset}。大多数的错误都不能被忽略,但有一种错误应该进行特殊处理。\texttt{io::ErrorKind::Interrupted}对应Unix的错误码\texttt{EINTR},它意味着读取过程恰好被一个信号打断。除非你的程序想设计为根据信号做一些聪明的操作,否则它应该简单地重试读取操作。上一节中的\texttt{copy()}的代码,就是一个例子。}

\hangparagraph{如你所见,\texttt{.read()}方法非常底层,甚至直接继承了底层操作系统的怪癖。如果你要为一个新的数据源类型实现\texttt{Read} trait,这会赋予你极大的灵活性。但如果你尝试读取一些数据,就会非常难受。因此,Rust提供了几个更高级的便捷方法。它们都有基于\texttt{.read()}的默认实现。它们都处理了\texttt{ErrorKind::Interrupted},因此你不需要再处理。}

\codeentry{reader.read\_to\_end(\&mut byte\_vec)}
\hangparagraph{读取reader中剩余的所有输入,将读到的数据附加到\texttt{byte\_vec}尾部,\texttt{byte\_vec}是一个\texttt{Vec<u8>}。返回一个\texttt{io::Result<uszie>},表示读取到的字节数。}

\hangparagraph{这个方法读取的数据的大小没有限制,因此不要将它用于不受信任的源。(你可以使用\texttt{.take()}方法施加限制,如后文所述。)}

\codeentry{reader.read\_to\_string(\&mut string)}
\hangparagraph{和上面相同,不过把数据附加到给定的\texttt{String}。如果流不是有效的UTF-8,它会返回一个\texttt{ErrorKind::InvalidData}错误。}

\hangparagraph{在一些编程语言中,字节输入和字符输入由不同的类型来处理。如今,UTF-8占据主导地位,Rust承认这一事实标准,并且完全支持UTF-8。其他字符集由开源的\texttt{encoding} crate提供支持。}

\codeentry{reader.read\_exact(\&mut buf)}
\hangparagraph{读取恰好足以填满给定缓冲区的数据。参数的类型是\texttt{\&mut [u8]},如果在读取够\texttt{buf.len()}个字节之前reader的数据就已经耗光,那么会返回一个\texttt{ErrorKind:: UnexpectedEof}错误。}

上面这些是\texttt{Read} trait的主要方法。除此之外,还有三个以值获取\texttt{reader}的适配器方法,将它转换为一个迭代器或者一个不同的reader:

\codeentry{reader.bytes()}
\hangparagraph{返回一个输入流的字节的迭代器。item的类型是\texttt{io::Result<u8>},因此每一个字节都需要进行错误检查。另外,它会逐字节调用\texttt{reader.read()},因此如果reader没有缓存的话会非常低效。}

\codeentry{reader.chain(reader2)}
\hangparagraph{返回一个新的reader,首先产生\texttt{reader}的所有输入,然后产生\texttt{reader2}的所有输入。}

\codeentry{reader.take(n)}
\hangparagraph{返回一个新的reader,从和\texttt{reader}相同的数据源读取数据,但最多只读取\texttt{n}个字节。}

没有关闭reader的方法。reader和writer通常实现了\texttt{Drop},因此它们会自动关闭。

\subsection{有缓冲的Reader}
出于性能考虑,reader和writer可以进行\emph{缓存(buffer)},意思是它们有一块内存(缓冲区)用来存储一些输入或输出数据。这样可以减少系统调用的次数,如\hyperref[f18-2]{图18-2}所示。在这个例子中,应用调用\texttt{.read\_line()}方法从\texttt{BufReader}中读取数据,\texttt{BufReader}从操作系统获取更大块的输入。

\begin{figure}[htbp]
    \centering
    \includegraphics[width=0.9\textwidth]{../img/f18-2.png}
    \caption{一个有缓冲的文件reader}
    \label{f18-2}
\end{figure}

这张图并不是按比例的,一个\texttt{BufReader}的实际大小是几千字节,因此一次系统的\texttt{read}调用可以提供上百次\texttt{.read\_line()}调用。这么做之所以能提高性能是因为系统调用很慢。(如图所示,操作系统也有一个缓冲区,原因与此相同:系统调用很慢,但从磁盘读取数据更慢。)

有缓冲的reader实现了\texttt{Read}和另一个trait \texttt{BufRead},它添加了下面的方法:

\codeentry{reader.read\_line(\&mut line)}
\hangparagraph{读取一行文本并将它附加到\texttt{line},\texttt{line}是一个\texttt{String}。行尾的换行符\texttt{'\textbackslash{}n'}}也会包含在\texttt{line}中。如果输入中有Windows风格的换行符\texttt{"\textbackslash{}r\textbackslash{}n"},这两个字符都会包含进\texttt{line}。
\hangparagraph{返回值是一个\texttt{io::Result<usize>},代表读取到的字节数,包括行尾的换行符。}
\hangparagraph{如果reader到达输入结尾,\texttt{line}会保持不变,并返回\texttt{Ok(0)}。}

\codeentry{reader.lines()}
\hangparagraph{返回一个迭代输入中每一行的迭代器。item的类型是\texttt{io::Result<String>}。换行符\emph{不}包含在字符串中。如果输入中有Windows风格的换行符\texttt{"\textbackslash{}r\textbackslash{}n"},这两个字符都会被丢弃。}

\hangparagraph{这个方法几乎总是你需要的文本输入方法。下面的两节会通过例子展示如何使用它。}

\codeentry{reader.read\_until(stop\_byte, \&mut byte\_vec), reader.split(stop\_byte)}
\hangparagraph{这两个方法类似于\texttt{.read\_line()}和\texttt{.lines()},但是是面向字节的,产生\texttt{Vec<u8>}而不是\texttt{String}。你可以选择终止符\texttt{stop\_byte}。}

\texttt{BufRead}还提供两个底层的方法\texttt{.fill\_buf()}和\texttt{.consume(n)},用来直接访问reader的内部缓冲区。更多有关这些方法的信息,可以查阅在线文档。

接下来的两节详细介绍了有缓冲的reader。

\subsection{读取行}\label{ReadLines}
这里有一个实现了Unix \texttt{grep}工具的函数。它搜索文本的每一行,文本通常通过管道从另一个命令输入。对于一个给定的字符串:
\begin{minted}{Rust}
    use std::io;
    use std::io::prelude::*;

    fn grep(target: &str) -> io::Result<()> {
        let stdin = io::stdin();
        for line_result in stdin.lock().lines() {
            let line = line_result?;
            if line.contains(target) {
                println!("{}", line);
            }
        }
        Ok(())
    }
\end{minted}

因为我们想调用\texttt{.lines()},所以我们需要一个实现了\texttt{BufRead}的输入源。在这个例子中,我们调用了\texttt{io::stdin()}来获取通过管道传入的数据。然而,Rust标准库使用了一个mutex来保护\texttt{stdin},我们调用\texttt{.lock()}来锁住\texttt{stdin}以让当前的线程独占使用,它返回一个实现了\texttt{BufRead}的\texttt{StdinLock}值。在循环的结尾,\texttt{StdinLock}被丢弃,释放mutex。(如果没有mutex,那么如果两个线程同时从\texttt{stdin}中读取数据,会导致未定义行为。C里也有这个问题,它通过这种方式解决它:C中所有的输入和输出函数会在幕后获取一个锁。Rust中唯一的不同就是锁是API的一部分。)

函数的剩余部分非常直观:它调用\texttt{.lines()}并迭代返回的迭代器。因为这个迭代器产生\texttt{Result}值,所以我们使用\texttt{?}操作符来检查错误。

假设我们想进一步扩展我们的\texttt{grep}程序,让它支持搜索磁盘中的文件。我们可以把函数修改为泛型的:
\begin{minted}{Rust}
    fn grep<R>(target: &str, reader: R) -> io::Result<()>
        where R: BufRead
    {
        for line_result in reader.lines() {
            let line = line_result?;
            if line.contains(target) {
                println!("{}", line);
            }
        }
        Ok(())
    }
\end{minted}

现在我们可以向它传递一个\texttt{StdinLock}或者一个有缓存的\texttt{File}:
\begin{minted}{Rust}
    let stdin = io::stdin();
    grep(&target, stdin.lock())?;   // ok

    let f = File::open(file)?;
    grep(&target, BufReader::new(f))?;  // ok
\end{minted}

注意\texttt{File}并不是自动缓存的。\texttt{File}实现了\texttt{Read}但没有实现\texttt{BufRead}。然而,很容易为\texttt{File}或者其他任何无缓存的reader创建一个有缓存的reader。\texttt{BufReader::new(reader)}可以实现这个功能。(可以使用\texttt{BufReader::with\_capacity(size, reader)}设置缓冲区的大小。)

在大多数语言中,文件都是默认有缓存的。如果你想要无缓存的输入或输出,你必须知道如何关闭缓存。在Rust中,\texttt{File}和\texttt{BufReader}是两个单独的库特性,因为有时你可能需要没有缓冲的文件,或者需要缓存文件之外的内容(例如,你可能会想要缓存来自网络的输入)。

包含错误处理和一些参数解析的完整的程序,如下所示:
\begin{minted}{Rust}
    // grep - 搜索stdin或文件中匹配给定string的行
    use std::error::Error;
    use std::io::{self, BufReader};
    use std::io::prelude::*;
    use std::fs::File;
    use std::path::PathBuf;

    fn grep<R>(target: &str, reader: R) -> io::Result<()>
        where R: BufRead
    {
        for line_result in reader.lines() {
            let line = line_result?;
            if line.contains(target) {
                println!("{}", line);
            }
        }
        Ok(())
    }

    fn grep_main() -> Result<(), Box<dyn Error>> {
        // 获取命令行参数。第一个参数是要搜索的字符串;
        // 剩余的是文件名。
        let mut args = std::env::args().skip(1);
        let target = match args.next() {
            Some(s) => s,
            None => Err("usage: grep PATTERN FILE...")?
        };
        let files: Vec<PathBuf> = args.map(PathBuf::from).collect();

        if files.is_empty() {
            let stdin = io::stdin();
            grep(&target, stdin.lock())?;
        } else {
            for file in files {
                let f = File::open(file)?;
                grep(&target, BufReader::new(f))?;
            }
        }

        Ok(())
    }

    fn main() {
        let result = grep_main();
        if let Err(err) = result {
            eprintln!("{}", err);
            std::process::exit(1);
        }
    }
\end{minted}

\subsection{收集行}
包括\texttt{.lines()}在内的几个reader方法返回产生\texttt{Result}的迭代器。当你第一次尝试将一个文件的每一行收集到一个很大的vector中时,你可能会遇到需要摆脱\texttt{Result}的问题:
\begin{minted}{Rust}
    // ok,但不是你想要的
    let results: Vec<io::Result<String>> = reader.lines().collect();

    // error: 不能将Result的集合转换成Vec<String>
    let lines: Vec<String> = reader.lines().collect();
\end{minted}

第二次尝试不能编译:哪里出错了?直观的解决方法是编写一个\texttt{for}循环并为每一个item检查错误:
\begin{minted}{Rust}
    let mut lines = vec![];
    for line_result in reader.lines() {
        lines.push(line_result?);
    }
\end{minted}

不错;但这里如果使用\texttt{.collect()}会更好,并且我们确实可以这么做。我们只需要知道需要什么样的类型:
\begin{minted}{Rust}
    let lines = reader.lines().collect::<io::Result<Vec<String>>>()?;
\end{minted}

为什么这能工作?标准库里为\texttt{Result}包含了一个\texttt{FromIterator}的实现——在在线文档中容易忽略——让这变为了可能:
\begin{minted}{Rust}
    impl<T, E, C> FromIterator<Result<T, E>> for Result<C, E>
        where C: FromIterator<T>
    {
        ...
    }
\end{minted}
这个签名需要仔细阅读,但它是一个漂亮的技巧。假设\texttt{C}是任意集合类型,例如\texttt{Vec}或者\texttt{HashSet}。只要我们已经知道了如何从一个产生\texttt{T}值的迭代器构建一个\texttt{C},我们就可以从一个产生\texttt{Result<T, E>}值的迭代器构建一个\texttt{Result<C, E>}。我们只需要遍历迭代器产生的值,用其中的\texttt{Ok}值构建集合,但如何遇到了一个\texttt{Err},就停止并传递它。

换句话说,\texttt{io::Result<Vec<String>>}是一个集合类型,所以\texttt{.collect()}方法可以创建并填充这种类型的值。

\subsection{Writer}
正如我们所见,使用方法就基本可以完成输入。输出有一些不同。

在整本书中,我们都在使用\texttt{println!()}来产生普通文本输出:
\begin{minted}{Rust}
    println!("Hello, world!");

    println!("The greatest common divisor of {:?} is {}", numbers, d);

    println!();     // 打印空白行
\end{minted}

还有一个\texttt{print!()}宏,它不会在最后加上一个换行符,\texttt{eprintln!}和\texttt{eprint!}宏写入到标准错误流。这些函数的格式化代码都和\texttt{format!}宏一样,见\nameref{format}。

使用\texttt{write!()}和\texttt{writeln!()}宏可以把输出写入一个writer。它们与\texttt{print!()}和\\
\texttt{println!()}类似,除了两个不同点:
\begin{minted}{Rust}
    writeln!(io::stderr(), "error: world not helloable")?;

    writeln!(&mut byte_vec, "The greatest common divisor of {:?} is {}", numbers, d)?;
\end{minted}

一是\texttt{write}宏有一个额外的第一个参数:writer。另一个不同是它们返回一个\texttt{Result},因此必须进行错误处理。这就是为什么我们在每一行的结尾都使用了\texttt{?}运算符。

\texttt{print}宏不返回一个\texttt{Result},如果写入失败它们会直接panic。因为它们会写入到终端,写入终端很少会失败。

\texttt{Write} trait有这些方法:
\codeentry{writer.write(\&buf)}
\hangparagraph{将切片\texttt{buf}中的字节写入到底层的流中。它返回一个\texttt{io::Result<usize>}。成功时,它返回写入的字节数量,可能会小于\texttt{buf.len()},取决于流。}
\hangparagraph{类似于\texttt{Reader::read()},这是一个你应该避免直接使用的底层方法。}

\codeentry{writer.write\_all(\&buf)}
\hangparagraph{写入切片\texttt{buf}中的所有字节。返回\texttt{Result<()>}。}

\codeentry{writer.flush()}
\hangparagraph{冲洗底层流中所有缓存的数据。返回\texttt{Result<()>}。}
\hangparagraph{注意尽管\texttt{println!}和\texttt{eprintln!}宏会自动冲洗标准输出和标准错误流,但\texttt{print!}和\texttt{eprint!}不会。使用它们之后你可能需要手动调用\texttt{flush()}。}

类似于reader,writer也是在丢弃时自动关闭。

类似于\texttt{BufReader::new(reader)}为任意reader添加缓存,\texttt{BufWriter::new(writer)}为任意writer添加缓存:
\begin{minted}{Rust}
    let file = File::create("tmp.txt")?;
    let writer = BufWriter::new(file);
\end{minted}

为了设置缓冲区的大小,使用\texttt{BufWriter::with\_capacity(size, writer)。}

当\texttt{BufWriter}被丢弃时,它剩余的所有被缓存的数据都会被写入到底层的writer。然而,如果这次写入时出现了错误,这个错误会被\emph{忽略}。(因为这个错误是在\texttt{BufWriter}的\texttt{.drop()}方法中发生,没有汇报错误的地方。)为了保证你的应用能够注意到所有的输出错误,可以在drop有缓存的writer之前手动调用\texttt{.flush()}。

\subsection{File}\label{file}
我们已经看到过两种打开文件的方式:
\codeentry{File::open(filename)}
\hangparagraph{打开一个已存在的文件。它返回一个\texttt{io::Result<File>},如果文件不存在将返回一个错误。}

\codeentry{File::create(filename)}
\hangparagraph{创建一个新的文件用于写入。如果已经有同名文件,它会被截断。}

注意\texttt{File}类型在文件系统模块\texttt{std::fs}中,而不是在\texttt{std::io}中。

当这两个文件都不符合要求时,你可以使用\texttt{OpenOptions}来指定额外的期望行为:
\begin{minted}{Rust}
    use std::fs::OpenOptions;

    let log = OpenOptions::new()
        .append(true)   // 如果文件存在,就追加到末尾
        .open("server.log")?;

    let file = OpenOptions::new()
        .write(true)
        .create_new(true)   // 如果文件存在就失败
        .open("new_file.txt")?;
\end{minted}

方法\texttt{.append(), .write(), .create\_new()}等,被设计用来进行类似这样的链式调用:每一个都返回\texttt{self}。这种链式方法的设计模式在Rust中太过普遍以至于有一个专门的名字:它被称为\emph{builder(构建器)}。\texttt{std::process::Command}是另一个例子。更多关于\texttt{OpenOptions}的细节可以查阅在线文档。

\texttt{File}被打开后,它的行为就类似于其他的reader和writer。如果需要的话你可以添加一个缓冲区。当你drop一个\texttt{File}时它会自动关闭。

\subsection{Seek}
\texttt{File}还实现了\texttt{Seek} trait,它意味着你可以在一个\texttt{File}中跳来跳去,而不是只能从开始单调地读到尾。\texttt{Seek}的定义类似如下:
\begin{minted}{Rust}
    pub trait Seek {
        fn seek(&mut self, pos: SeekFrom) -> io::Result<u64>;
    }

    pub enum SeekFrom {
        Start(u64),
        End(i64),
        Current(i64)
    }
\end{minted}

得益于这个枚举,\texttt{seek}方法变得很有表达力:使用\texttt{file.seek(SeekFrom::Start(0))}来定位到开始,使用\texttt{file.seek(SeekFrom::Current(-8))}来回退一些字节,等等。

在一个文件中定位很慢。不管你是在硬盘还是固态盘(SSD)上,定位都要消耗和读取几M数据一样长的时间。

\subsection{其他Reader和Writer类型}
目前为止,本章主要使用了\texttt{File}作为示例,但还有很多其他有用的reader和writer类型:

\codeentry{io::stdin()}
\hangparagraph{返回一个标准输入流的reader。它的类型是\texttt{io::Stdin}。因为它被多个线程共享,所以每一次读取都需要请求并释放mutex。}

\hangparagraph{\texttt{Stdin}有一个\texttt{.lock()}方法获取mutex并返回一个\texttt{io::StdinLock},这是一个有缓存的reader,它会持有mutex,直到它被丢弃。因此对\texttt{StdinLock}的单独操作可以避免mutex的开销。我们在\nameref{ReadLines}中展示过使用这个方法的示例代码。}

\hangparagraph{出于技术原因,\texttt{io::stdin().lock()}不能工作。这个锁持有一个\texttt{Stdin}值的引用,这意味着\texttt{Stdin}值必须被存储起来,这样它才能生存的足够久:}
\begin{minted}{Rust}
    let stdin = io::stdin();
    let lines = stdin.lock().lines();   // ok
\end{minted}

\codeentry{io::stdout(), io::stderr()}
\hangparagraph{返回标准输出和标准错误流的\texttt{Stdout}和\texttt{Stderr} writer类型。这两个类型也持有互斥锁和\texttt{.lock()}方法。}

\codeentry{Vec<u8>}
\hangparagraph{实现了\texttt{Write}。写入到一个\texttt{Vec<u8>}会把新的数据附加到vector尾部。}

\hangparagraph{然而,\texttt{String}\emph{并没有}实现\texttt{Write}。为了使用\texttt{Write}构建一个字符串,首先要写入到一个\texttt{Vec<u8>},然后使用\texttt{String::from\_utf8(vec)}来把vector转换为字符串。}

\codeentry{Cursor::new(buf)}
\hangparagraph{创建一个\texttt{Cursor},它是一个从\texttt{buf}中读取的有缓存的reader。这也是一个创建从\texttt{String}读取的reader的方法。参数\texttt{buf}可以是任何实现了\texttt{AsRef<[u8]>}的类型,因此你也可以传递一个\texttt{\&[u8], \&str, Vec<u8>}。}
\hangparagraph{\texttt{Cursor}内部的结构非常简单。它只有两个字段:\texttt{buf}和一个整数,用来表示下一次读取开始的偏移量。初始时为0。}

\hangparagraph{\texttt{Cursor}实现了\texttt{Read, BufRead, Seek}。如果\texttt{buf}的类型是\texttt{\&mut [u8]}或者\texttt{Vec<u8>},那么\texttt{Cursor}还会实现\texttt{Write}。写入一个\texttt{Cursor}会覆盖\texttt{buf}中从当前位置开始的字节。如果你试图越界写入一个\texttt{\&mut [u8]},结果会是部分写入或者一个\texttt{io::Error}。使用Cursor越界写入一个\texttt{Vec<u8>}没有问题,因为它会让vector变长。因此\texttt{Cursor<\&mut [u8]>}和\texttt{Cursor<Vec<u8>>}实现了\texttt{std::io::prelude}中全部的4个trait。}

\codeentry{std::net::TcpStream}
\hangparagraph{代表一个TCP网络连接。因为TCP允许双向连接,所以它既是reader又是writer。}
          
\hangparagraph{类型关联函数\texttt{TcpStream::connect(("hostname", PORT))}尝试连接到服务器,并返回一个\texttt{io::Result<TcpStream>}。}

\codeentry{std::process::Command}
\hangparagraph{支持创建一个子进程并把数据管道连接到它的标准输入,例如:}
\begin{minted}{Rust}
    use std::process::{Command, Stdio};

    let mut child =
        Command::new("grep")
         .arg("-e")
         .arg("a.*e.*i.*o.*u")
         .stdin(Stdio::piped())
         .spawn()?;
    
    let mut to_child = child.stdin.take().unwrap();
    for word in my_words {
        writeln!(to_child, "{}", word)?;
    }

    drop(to_child); // 关闭grep的stdin,所以它会退出
    child.wait()?;
\end{minted}

\hangparagraph{\texttt{child.stdin}的类型是\texttt{Option<std::process:ChildStdin>};这里我们在创建子进程的时候使用了\texttt{.stdin(Stdio::piped())},因此\texttt{.spawn()}成功后\texttt{child.stdin}肯定是\texttt{Some}。否则\texttt{child.stdin}将是\texttt{None}。}

\hangparagraph{\texttt{Command}还有类似的\texttt{.stdout()}和\texttt{.stderr()}方法,它们可以用来请求\texttt{child.stdout}和\\\texttt{child.stderr}中的reader。}

\texttt{std::io}模块还提供了很多返回简单reader和writer的函数:
\codeentry{io::sink()}
\hangparagraph{这是一个无操作的writer。所有的写入方法都会返回\texttt{Ok},但数据都会被丢弃。}

\codeentry{io::empty()}
\hangparagraph{这是一个无操作的reader。所有的读取都会成功,但总是返回输入结束。}

\codeentry{io::repeat(byte)}
\hangparagraph{返回一个无限重复给定字节的reader。}

\subsection{二进制数据,压缩和序列化}
有很多基于\texttt{std::io}框架的开源crate提供额外的特性。

\texttt{byteorder} crate提供\texttt{ReadBytesExt}和\texttt{WriteBytesExt} trait,它们为所有reader和writer添加二进制输入和输出的方法:
\begin{minted}{Rust}
    use byteorder::{ReadBytesExt, WriteBytesExt, LittleEndian};

    let n = reader.read_u32::<LittleEndian>()?;
    writer.write_i64::<LittleEndian>(n as i64)?;
\end{minted}

\texttt{flate2} crate提供读取和写入\texttt{gzip}数据的适配器方法:
\begin{minted}{Rust}
    use flate2::read::GzDecoder;
    let file = File::open("access.log.gz")?;
    let mut gzip_reader = GzDecoder::new(file);
\end{minted}

\texttt{serde} crate以及它关联的格式化crate例如\texttt{serde\_json},实现了序列化和反序列化:它们在Rust结构体和字节流之间来回转换。我们之前在\nameref{OrphanRule}中提到过它们一次。现在让我们仔细看看。

假设我们有一些数据,即一个文字冒险游戏的地图,存储在一个\texttt{HashMap}中:
\begin{minted}{Rust}
    type RoomId = String;                       // 每一个房间有一个独一无二的名字
    type RoomExits = Vec<(char, RoomId)>;       // ...和一个通向的房间的名字的列表
    type RoomMap = HashMap<RoomId, RoomExits>;

    // 创建一个简单的地图。
    let mut map = RoomMap::new();
    map.insert("Cobble Crawl".to_string(),
               vec![('W', "Debris Room".to_string())]);
    map.insert("Debris Room".to_string(),
               vec![('E', "Cobble Crawl".to_string()),
                    ('W', "Sloping Canyon".to_string())]);
    ...
\end{minted}

将这个数据转换为JSON并输出只需要一行代码:
\begin{minted}{Rust}
    serde_json::to_writer(&mut std::io::stdout(), &map)?;
\end{minted}

在内部,\texttt{serde\_json::to\_writer}使用了\texttt{serde::Serialize} trait的\texttt{serialize}方法。这个库给所有它知道如何序列化的类型附加了这个trait,其中包括我们的数据中出现的类型:字符串、字符、元组、vector、\texttt{HashMap}。

\texttt{serde}非常灵活。在我们的程序中,输出是JSON数据,因为我们选择了\texttt{serde\_json}序列化器。其他格式例如\texttt{MessagePack}也是可用的。同样地,你可以把输出送到文件、\texttt{Vec<u8>}或其他任何writer中。上面的代码通过\texttt{stdout}打印了数据,内容如下:
\begin{minted}{json}
    {"Debris Room":[["E","Cobble Crawl"],["W","Sloping Canyon"]],"Cobble Crawl": [["W","Debris Room"]]}
\end{minted}

\texttt{serde}还包括派生两个关键trait的支持:
\begin{minted}{Rust}
    #[derive(Serialize, Deserialize)]
    struct Player {
        location: String,
        items: Vec<String>,
        health: u32
    }
\end{minted}

这个\texttt{\#[derive]}属性会让编译过程稍微变长,因此当你在\emph{Cargo.toml}文件中将\texttt{serde}列为依赖时需要要求它支持这个特性。这是我们上面的代码用到的依赖:
\begin{minted}{toml}
    [dependencies]
    serde = { version = "1.0", features = ["derive"] }
    serde_json = "1.0"
\end{minted}

更多的细节可以查阅\texttt{serde}的文档。简单来说,构建系统可以自动为\texttt{Player}生成\\
\texttt{serde::Serialize}和\texttt{serde::Deserialize},因此序列化一个\texttt{Player}值非常简单:
\begin{minted}{Rust}
    serde_json::to_writer(&mut std::io::stdout(), &player)?;
\end{minted}

输出看起来是这样的:
\begin{minted}{json}
    {"location":"Cobble Crawl","items":["a wand"],"health":3}
\end{minted}

\section{文件和目录}
现在我们已经展示了如何使用reader和writer,下面的几节将介绍Rust中处理文件和目录的特性,它们在\texttt{std::path}和\texttt{std::fs}模块中。这些特性都涉及到文件名,所以我们将以文件名类型开始。

\subsection{\texttt{OsStr}和\texttt{Path}}
很不方便的一点是,你的操作系统并不一定强制文件名是有效的Unicode。这里有两个创建文本文件的Linux shell命令。只有第一个是有效的UTF-8文件名:
\begin{minted}{text}
    $ echo "hello world" > ô.txt
    $ echo "O brave new world, that has such filenames in't" > $'\xf4'.txt
\end{minted}

两条命令都可以运行,因为Linux内核不知道来自Ogg Vorbis的UTF-8。对于内核来说,任何字节(除了null字节和斜杠)组成的字符串都是可接受的文件名。Windows上也类似:几乎任何16位“宽字符”组成的字符串都是可接受的文件名,即使字符串并不是有效的UTF-16。操作系统处理的其他字符串也是这样,例如命令行参数和环境变量。

Rust的字符串总是有效的Unicode。在实践中文件名\emph{几乎}总是Unicode,但Rust必须提供方式以应对少数不是Unicode的情况。这就是为什么Rust有\texttt{std::ffi::OsStr}和\texttt{OsString}。

\texttt{OsStr}是一个作为UTF-8超集的字符串类型。它的任务是能表示当前系统中的所有文件名、命令行参数、环境变量,\emph{不管它们是不是Unicode}。在Unix上,\texttt{OsStr}可以存储任意字节序列。在Windows上,\texttt{OsStr}以UTF-8的扩展格式存储,它可以编码任何16位值的序列。

所以我们有了两种字符串类型:\texttt{str}用于实际的Unicode字符串;\texttt{OsStr}用于操作系统可能用到的字符串。我们将再介绍一个用于文件名的\texttt{std::path::Path},它纯粹是为了方便。\texttt{Path}实际上很像\texttt{OsStr},但它添加了很多和文件名相关的方法,我们将在下一节中介绍。可以使用\texttt{Path}表示绝对路径和相对路径。对于路径中每个单独的部分,使用\texttt{OsStr}。

最后,每个字符串类型都有一个相应的\emph{有所有权的(owning)}类型:\texttt{String}拥有一个堆上分配的\texttt{str},一个\texttt{std::ffi::OsString}拥有一个堆上分配的\texttt{OsStr},一个\texttt{std::path::PathBuf}拥有一个堆上分配的\texttt{Path}。\hyperref[t18-1]{表18-1}列出了每个类型的一些特性。

\begin{table}[htbp]
    \centering
    \caption{文件名类型}
    \label{t18-1}
    \begin{tabular}{llll}
        \hline
        & \textbf{str} & \textbf{OsStr} & \textbf{Path} \\
        \hline
        非固定大小类型,总是以引用传递 & 是 & 是 & 是 \\
        \rowcolor{tablecolor}
        包含任意Unicode文本 & 是 & 是 & 是 \\
        通常看起来就像UTF-8 & 是 & 是 & 是 \\
        \rowcolor{tablecolor}
        可以包含非Unicode数据 & 否 & 是 & 是 \\
        文本处理方法    & 是 & 否 & 否 \\
        \rowcolor{tablecolor}
        文件名相关方法  & 否 & 是 & 是 \\
        对应的有所有权、可增长的、堆上分配的类型 & \texttt{String} & \texttt{OsString} & \texttt{PathBuf} \\
        \rowcolor{tablecolor}
        转换为有所有权的类型  & \texttt{.to\_string()} & \texttt{.to\_os\_string()} & \texttt{.to\_path\_buf()} \\
    \end{tabular}
\end{table}

所有这些类型都实现了一个公共的trait:\texttt{AsRef<Path>},所以我们可以轻易地声明一个泛型函数接受“任何文件名类型”作为参数。这使用到了我们之前展示过的\nameref{asref}:
\begin{minted}{Rust}
    use std::path::Path;
    use std::io;

    fn swizzle_file<P>(path_arg: P) -> io::Result<()>
        where P: AsRef<Path>
    {
        let path = path_arg.as_ref();
        ...
    }
\end{minted}

所有接受\texttt{path}参数的标准函数和方法都使用了这项技术,因此你可以自由地向它们传递字符串字面量。

\subsection{\texttt{Path}和\texttt{PathBuf}方法}
\texttt{Path}提供了下面这些方法:

\codeentry{Path::new(str)}
\hangparagraph{将一个\texttt{\&str}或者\texttt{\&OsStr}转换为\texttt{\&Path}。它不会拷贝字符串,新的\texttt{\&Path}和原本的\texttt{\&str}\\或\texttt{\&OsStr}指向相同的字节流:}
\begin{minted}{Rust}
    use std::path::Path;
    let home_dir = Path::new("/home/fwolfe");
\end{minted}

\hangparagraph{(类似的方法\texttt{OsStr::new(str)}将\texttt{\&str}转换为\texttt{\&OsStr}。)}

\codeentry{path.parent()}
\hangparagraph{返回\texttt{path}的父目录,如果有的话。返回类型是\texttt{Option<\&Path>}。}

\hangparagraph{它也不会拷贝路径,\texttt{path}的父目录总是\texttt{path}的一个子串:}
\begin{minted}{Rust}
    assert_eq!(Path::new("/home/fwolfe/program.txt").parent(),
               Some(Path::new("/home/fwolfe")));
\end{minted}

\codeentry{path.file\_name()}
\hangparagraph{返回\texttt{path}的最后一个部分,如果有的话。返回类型是\texttt{Option<\&OsStr>}。}

\hangparagraph{在通常的情况下,\texttt{path}由一个目录、一个斜杠、然后是一个文件名组成,这会返回文件名:}
\begin{minted}{Rust}
    use std::ffi::OsStr;
    assert_eq!(Path::new("/home/fwolfe/program.txt").file_name(),
               Some(OsStr::new("program.txt")));
\end{minted}

\codeentry{path.is\_absolute(), path.is\_relative()}
\hangparagraph{这些方法判断路径是绝对的(例如Unix路径\emph{/usr/bin/advent}或者Windows路径\\\emph{C:\textbackslash{}Program Files})还是相对的(例如\emph{src/main.rs})。}

\codeentry{path1.join(path2)}
\hangparagraph{连接两个路径,返回一个新的\texttt{PathBuf}:}
\begin{minted}{Rust}
    let path1 = Path::new("/usr/share/dict");
    assert_eq!(path1.join("words"),
               Path::new("/usr/share/dict/words"));
\end{minted}

\hangparagraph{如果\texttt{path2}是一个绝对路径,这会简单地返回\texttt{path2}的拷贝,因此这个方法可以用于将任何路径转换为一个绝对路径:}
\begin{minted}{Rust}
    let abs_path = std::env::current_dir()?.join(any_path);
\end{minted}

\codeentry{path.components()}
\hangparagraph{返回一个从左到右迭代给定路径的所有部分的迭代器。这个迭代器的item类型是\texttt{std::path::Component},它可以代表任何可能出现在文件名中的部分:}
\begin{minted}{Rust}
    pub enum Component<'a> {
        Prefix(PrefixComponent<'a>),  // 一个驱动器字母或者共享设备(在Windows上)
        RootDir,            // 根目录,`/`或`\`
        CurDir,             // `.`特殊目录
        ParentDir,          // `..`特殊目录
        Normal(&'a OsStr)   // 普通的文件和目录名
    }
\end{minted}

\hangparagraph{例如,Windows路径\emph{\textbackslash\textbackslash{}venice\textbackslash{}Music\textbackslash{}A Love Supreme\textbackslash{}04-Psalm.mp3}由一个\texttt{Prefix}(表示\emph{\textbackslash\textbackslash{}venice\textbackslash{}Music})、后跟一个\texttt{RootDir},然后是两个\texttt{Normal}组件(分别是\emph{A Love Supreme}和\emph{04-Psalm.mp3})组成。}

\hangparagraph{细节见\href{https://doc.rust-lang.org/std/path/struct.Path.html\#method.components}{在线文档}。}

\codeentry{path.ancestors()}
\hangparagraph{返回一个从\texttt{path}一直回溯到根目录的迭代器。每一个产生的item都是一个\texttt{Path}:第一个是\texttt{path}本身,然后是它的父目录、它的父目录的父目录,等等:}
\begin{minted}{Rust}
    let file = Path::new("/home/jimb/calendars/calendar-18x18.pdf");
    assert_eq!(file.ancestors().collect::<Vec<_>>(),
               vec![Path::new("/home/jimb/calendars/calendar-18x18.pdf"),
                    Path::new("/home/jimb/calendars"),
                    Path::new("/home/jimb"),
                    Path::new("/home"),
                    Path::new("/")]);
\end{minted}

\hangparagraph{这很像一直调用\texttt{parent}直到它返回\texttt{None}。最终的item总是一个根目录或者前缀路径。}

这些方法只考虑内存中的字符串。\texttt{Path}还有一些会查询文件系统的方法:\texttt{.exists(), .is\_file(), .is\_dir(), .read\_dir(), .canonicalize()}等等。更多内容请查阅在线文档。

有三个将\texttt{Path}转换为字符串的方法。每一个都允许\texttt{Path}中可能含有无效的UTF-8:
\codeentry{path.to\_str()}
\hangparagraph{将一个\texttt{Path}转换成字符串,返回一个\texttt{Option<\&str>}。如果\texttt{path}不是有效的UTF-8,它返回\texttt{None}:}
\begin{minted}{Rust}
    if let Some(file_str) = path.to_str() {
        println!("{}", file_str);
    }   // ...否则跳过这个文件名
\end{minted}

\codeentry{path.to\_string\_lossy()}
\hangparagraph{这个方法功能基本和上面一样,但它在所有情况下都会返回字符串。如果\texttt{path}不是有效的UTF-8,这个方法会创建拷贝,然后将每一个无效的字节序列替换为Unicode占位字符:U+FFFD('�')。}

\hangparagraph{返回值类型是\texttt{std::borrow::Cow<str>}:可能是字符串的借用也可能是有所有权的字符串。为了从这个值得到一个\texttt{String},使用它的\texttt{.to\_owned()}方法。(更多有关\texttt{Cow}的内容,见\nameref{Cow}。)}

\codeentry{path.display()}
\hangparagraph{这用于打印路径:}
\begin{minted}{Rust}
    println!("Download found. You put it in: {}", dir_path.display());
\end{minted}

\hangparagraph{它返回的值并不是字符串,但它实现了\texttt{std::fmt::Display},所以它可以和\texttt{format!(), println!()}等一起使用。如果路径不是有效的UTF-8,输出可能会含有�字符。}

\subsection{文件系统访问函数}
\hyperref[t18-2]{表18-2}展示了\texttt{std::fs}中的一些函数以及它们在Unix和Windows中的类似等价物。所有这些函数都返回\texttt{io::Result}值。除非特意提及,不然就是\texttt{io::Result<()>}。

\begin{table}[htbp]
    \centering
    \caption{文件系统访问函数总结}
    \label{t18-2}
    \begin{tabular}{lp{0.3\textwidth}ll}
        \hline
        & \textbf{Rust函数} & \textbf{Unix} & \textbf{Windows} \\
        \hline
        \multirow{5}{*}{创建和删除} & \texttt{create\_dir(path)} & \texttt{mkdir()} & \texttt{CreateDirectory()} \\
        & \texttt{create\_dir\_all(path)} \cellcolor{tablecolor} & 类似\texttt{mkdir -p} \cellcolor{tablecolor} & 类似\texttt{mkdir} \cellcolor{tablecolor} \\
        & \texttt{remove\_dir(path)} & \texttt{rmdir()} & \texttt{RemoveDirectory()} \\
        & \texttt{remove\_dir\_all(path)} \cellcolor{tablecolor} & 类似\texttt{rm -r} \cellcolor{tablecolor} & 类似\texttt{rmdir /s} \cellcolor{tablecolor} \\
        & \texttt{remove\_file(path)} & \texttt{unlink()} & \texttt{DeleteFile()} \\
        \hline
        \multirow{3}{*}{拷贝,移动和链接} & \texttt{copy(src\_path, dest\_path) -> Result<u64>} \cellcolor{tablecolor} & 类似\texttt{cp -p} \cellcolor{tablecolor} & \texttt{CopyFileEx()} \cellcolor{tablecolor} \\
        & \texttt{rename(src\_path, dest\_path)} & \texttt{rename()} & \texttt{MoveFileex()} \\
        & \texttt{hard\_link(src\_path, dest\_path)} \cellcolor{tablecolor} & \texttt{link()} \cellcolor{tablecolor} & \texttt{CreateHardLink()} \cellcolor{tablecolor} \\
        \hline
        \multirow{5}{*}{检查} & \texttt{canonicalize(path) -> Result<PathBuf>} & \texttt{realpath} & \texttt{GetFinalPathNameByHandle()} \\
        & \texttt{metadata(path) -> Result<Metadata>} \cellcolor{tablecolor} & \texttt{stat()} \cellcolor{tablecolor} & \texttt{GetFileInformationByHandle()} \cellcolor{tablecolor} \\
        & \texttt{symlink\_metadata(path) -> Result<Metadata>} & \texttt{lstat()} & \texttt{GetFileInformationByHandle()} \\
        & \texttt{read\_dir(path) -> Result<ReadDir>} \cellcolor{tablecolor} & \texttt{opendir()} \cellcolor{tablecolor} & \texttt{FindFirstFile()} \cellcolor{tablecolor} \\
        & \texttt{read\_link(path) -> Result<PathBuf>} & \texttt{readlink()} & \texttt{FSCTL\_GET\_REPARSE\_POINT} \\
        \hline 
        权限 & \texttt{set\_permission(path, perm)} \cellcolor{tablecolor} & \texttt{chmod()} \cellcolor{tablecolor} & \texttt{SetFileAttributes()} \cellcolor{tablecolor} \\
    \end{tabular}
\end{table}

(\texttt{copy()}返回的数字是被拷贝的文件的大小,以字节为单位。有关创建符号链接,见\nameref{PlatSpec}。)

如你所见,Rust努力提供可以在Windows、macOS、Linux以及其他Unix系统上工作的可移植函数。

文件系统的完整说明超出了本书的范围,但如果你对这些函数中的某些更感兴趣,你可以在网上轻松地找到有关他们的更多信息。我们将在下一节中展示更多示例。

所有这些函数都是通过调用操作系统的功能来实现。例如\texttt{std::fs::canonicalize(path)}不只是使用字符串处理来消除给定的\texttt{path}中的\texttt{.}和\texttt{..} 。它使用当前的工作目录来解析相对路径,并且它会解析符号链接。如果路径不存在它会报错。

\texttt{std::fs::metadata(path)}和\texttt{std::fs::symlink\_metadata(path)}产生的\texttt{Metadata}类型包含类似于文件类型和大小、权限、时间戳等信息。同样,详细的内容请查阅文档。

为了方便,\texttt{Path}类型将一些这样的函数内建为方法:例如\texttt{path.metadata()}和\\
\texttt{std::fs::metadata(path)是一样的。}

\subsection{读取目录}
可以使用\texttt{std::fs::read\_dir}列出目录中的内容。或者等价的\texttt{Path}的\texttt{.read\_dir()}方法:
\begin{minted}{Rust}
    for entry_result in path.read_dir()? {
        let entry = entry_result?;
        println!("{}", entry.file_name().to_string_lossy());
    }
\end{minted}

注意这段代码中\texttt{?}的两次使用。第一行的检查打开目录的错误。第二行的检查读取下一个条目的错误。

\texttt{entry}的类型是\texttt{std::fs::DirEntry},它有如下方法:
\codeentry{entry.file\_name()}
\hangparagraph{文件或目录的名字,是一个\texttt{OsString}。}

\codeentry{entry.path()}
\hangparagraph{和上面相同,但和原本的路径连接在一起,产生一个新的\texttt{PathBuf}。如果我们正在列出的目录是\texttt{"/home/jimb"},并且\texttt{entry.file\_name()}是\texttt{".emacs"},那么\texttt{entry.path()}将会返回\texttt{PathBuf::from("/home/jimb/.emacs")}。}

\codeentry{entry.file\_type()}
\hangparagraph{返回一个\texttt{io::Result<FileType>}。\texttt{FileType}类型有\texttt{.is\_file()}、\texttt{.is\_dir()}、\\
\texttt{.is\_symlink()}方法。}

\codeentry{entry.metadata()}
\hangparagraph{获取这个条目的其他元数据。}

在读取目录时特殊目录\texttt{.}和\texttt{..}\emph{不会被}列出。

这里还有另一个示例。下面的代码递归拷贝磁盘上的一个目录树:
\begin{minted}{Rust}
    use std::fs;
    use std::io;
    use std::path::Path;

    /// 拷贝现有的目录`src`到目标路径`dst`
    fn copy_dir_to(src: &Path, dst: &Path) -> io::Result<()> {
        if !dst.is_dir() { 
            fs::create_dir(dst)?;
        }

        for entry_result in src.read_dir()? {
            let entry = entry_result?;
            let file_type = entry.file_type()?;
            copy_to(&entry.path(), &file_type, &dst.join(entry.file_name()))?;
        }

        Ok(())
    }
\end{minted}

用一个单独的函数\texttt{copy\_to}来拷贝单独的目录项:
\begin{minted}{Rust}
    /// 拷贝`src`中的所有东西到目标路径`dst`。
    fn copy_to(src: &Path, src_type: &fs::FileType, dst: &Path)
        -> io::Result<()>
    {
        if src_type.is_file() {
            fs::copy(src, dst)?;
        } else if src_type.is_dir() {
            copy_dir_to(src, dst)?;
        } else {
            return Err(io::Error::new(io::ErrorKind::Other,
                                      format!("don't know how to copy: {}", src.display())));
        }                                                            
        Ok(())
    }
\end{minted}

\subsection{平台特定特性}\label{PlatSpec}
到目前为止,我们的\texttt{copy\_to}函数可以拷贝文件和目录。假设我们还想在Unix上支持符号链接。

目前并没有可移值的方式能创建同时在Unix和Windows上工作的符号链接,但标准库提供了一个Unix特定的\texttt{symlink}函数:
\begin{minted}{Rust}
    use std::os::unix::fs::symlink;
\end{minted}

有了这个,我们的工作就变得很简单。我们只需要给\texttt{copy\_to}里的\texttt{if}表达式添加一个分支:
\begin{minted}{Rust}
    ...
    } else if src_type.is_symlink() {
        let target = src.read_link()?;
        symlink(target, dst)?;
    ...
\end{minted}

只要我们在Unix系统例如Linux和macOS上编译程序,它就可以工作。

\texttt{std::os}模块包含很多平台特定的特性,例如\texttt{symlink}。\texttt{std::os}在标准库中的实际内容看起来像这样(取得了许可):
\begin{minted}{Rust}
    //! OS特定的功能

    #[cfg(unix)]                pub mod unix;
    #[cfg(windows)]             pub mod windows;
    #[cfg(target_os = "ios")]   pub mod ios;
    #[cfg(target_os = "linux")] pub mod linux;
    #[cfg(target_os = "macos")] pub mod macos;
\end{minted}

\texttt{\#[cfg]}属性表示条件编译:这些模块中的每一个都只在特定平台上可用。这也是为什么我们的修改后使用了\texttt{std::os::unix}的程序在Unix上将会成功编译:在其他平台上,\texttt{std::os::unix}不存在。

如果我们想让我们的代码在所有平台上编译,并且支持Unix上的符号链接,我们必须在我们的程序中也使用\texttt{\#[cfg]}。在这种情况下,最简单的方法是在Unix上时导入\texttt{symlink},而在其它系统上定义我们自己的\texttt{symlink}:
\begin{minted}{Rust}
    #[cfg(unix)]
    use std::os::unix::fs::symlink;

    /// 为不支持`symlink`的平台的实现
    #[cfg(not(unix))]
    fn symlink<P: AsRef<Path>, Q: AsRef<Path>>(src: P, _dst: Q)
        -> std::io::Result<()>
    {
        Err(io::Error::new(io::ErrorKind::Other,
                           format!("can't copy symbolic link: {}", 
                                   src.as_ref().display())))
    }
\end{minted}

事实证明\texttt{symlink}是一种特殊情况。大多数Unix特定的特性并不是单独的函数而是一些扩展的trait,它们为标准库类型添加了一些的方法(我们在\nameref{OrphanRule}中介绍过扩展trait)。这里有一个可以一次性启用所有这些扩展的\texttt{prelude}模块:
\begin{minted}{Rust}
    use std::os::unix::prelude::*;
\end{minted}

例如,在Unix上这会给\texttt{std::fs::Permissions}添加一个\texttt{.mode()}方法,它提供Unix上表示权限的底层\texttt{u32}值的访问。类似的,它还扩展了\texttt{std::fs::Metadata},添加了一些访问底层的\texttt{struct stat}的字段的方法——例如\texttt{.uid()}返回文件所有者的ID。

总而言之,\texttt{std::os}中的内容非常基础。更多的平台特定功能通过第三方crate提供,例如\href{https://crates.io/crates/winreg}{\texttt{winreg}}提供了访问Windows注册表的支持。

\section{网络}

有关网络编程的教程超出了本书的范围。然而,如果你已经知道一些有关网络编程的知识,那么这一节可以帮助你在Rust中开始网络编程。

底层的网络编程需要使用\texttt{std::net}模块,它提供了TCP和UDP网络的跨平台支持。使用\texttt{native\_tls} crate来提供SSL/TLS支持。

这些模块提供了通过网络的直观的、阻塞式的输入和输出。你可以通过\texttt{std::net}用很少的代码编写一个简单的服务器,为每一个连接创建一个线程。例如,这里有一个“echo”服务器:
\begin{minted}{Rust}
    use std::net::TcpListener;
    use std::io;
    use std::thread::spawn;

    /// 一直等待并接受连接,为每个连接新建一个线程处理。
    fn echo_main(addr: &str) -> io::Result<()> {
        let listener = TcpListener::bind(addr)?;
        println!("listening on {}", addr);
        loop {
            // 等待客户端连接。
            let (mut stream, addr) = listener.accept()?;
            println!("connection received from {}", addr);

            // 创建一个线程来服务这个客户端。
            let mut write_stream = stream.try_clone()?;
            spawn(move || {
                // 把我们从`stream`接收到的所有内容写回。
                io::copy(&mut stream, &mut write_stream)
                    .expect("error in client thread: ");
                println!("connection closed");
            });
        }
    }

    fn main() {
        echo_main("127.0.0.1:17007").expect("error: ");
    }
\end{minted}

一个回声服务器简单地把你发送给它的数据返回。这些代码和你在Java或Python中编写的代码并没有多少不同。(我们将在\hyperref[ch19]{下一章}中介绍\texttt{std::thread::spawn()})

然而,对于高性能的服务器,你将需要使用异步的输入和输出。\hyperref[ch20]{第20章}会介绍Rust对异步编程的支持,并展示编写网络客户端和服务器的完整代码。

更高层的协议由第三方crate支持。例如,\texttt{reqwest} crate为HTTP客户端提供了一个漂亮的API。这里有一个完整的命令行程序获取\texttt{http:}或者\texttt{https:} URL的文档并输出到终端。这段代码使用\texttt{reqwest = "0.11"}编写,并启用了它的\texttt{"blocking"}特性。\texttt{reqwest}还提供了一套异步的接口。
\begin{minted}{Rust}
    use std::error::Error;
    use std::io;

    fn http_get_main(url: &str) -> Result<(), Box<dyn Error>> {
        // 发送HTTP请求并获取一个响应。
        let mut response = reqwest::blocking::get(url)?;
        if !response.status().is_success() {
            Err(format!("{}", response.status()))?;
        }

        // 读取响应的body并写入到标准输出。
        let stdout = io::stdout();
        io::copy(&mut response, &mut stdout.lock())?;

        Ok(())
    }

    fn main() {
        let args: Vec<String> = std::env::args().collect();
        if args.len() != 2 {
            eprintln!("usage: http-get URL");
            return;
        }

        if let Err(err) = http_get_main(&args[1]) {
            eprintln!("error: {}", err);
        }
    }
\end{minted}

用于HTTP服务器的\texttt{actix-web}框架提供了更高层的特性,例如\texttt{Service}和\texttt{Transform} trait,它们可以帮助你通过可组合的部分构建一个app。\texttt{websocket} crate实现了WebSocket协议,等等。Rust是一门年轻的语言,有一个繁荣的开源生态系统。对网络的支持正在快速扩张。
