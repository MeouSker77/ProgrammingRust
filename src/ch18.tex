\chapter{输入输出}\label{ch18}

\emph{Doolittle: What concrete evidence do you have that you exist?
Bomb \#20: Hmmmm... well... I think, therefore I am.
Doolittle: That’s good. That’s very good. But how do you know that anything else exists?
Bomb \#20: My sensory apparatus reveals it to me.}

\begin{flushright}
    ——Dark Star
\end{flushright}

Rust中有关输入输出的特性围绕着三个trait:\texttt{Read}、\texttt{BufRead}、\texttt{Write}来组织:
\begin{enumerate}
    \item 实现了\texttt{Read}的值有读取字节输入的方法。它们被称为\emph{读者(reader)}。
    \item 实现了\texttt{BufRead}的值是\emph{buffered reader(有缓存的读者)}。它们支持\texttt{Read}的所有方法,加上读取文本的一行的方法,等等。
    \item 实现了\texttt{Write}的值支持字节和UTF-8文本输出。它们被称为\emph{写者(writer)}。
\end{enumerate}

\autoref{f18-1}展示了这三个trait以及一些reader和writer类型的示例。

在本章中,我们将解释如何使用这些trait和它们的方法,包括图中出现的reader和writer类型,还有一些其他的和文件、终端、网络交互的方法。

\begin{figure}[htbp]
    \centering
    \includegraphics[width=0.8\textwidth]{../img/f18-1.png}
    \caption{Rust的三个主要的I/O trait以及一些实现了它们的类型}
    \label{f18-1}
\end{figure}

\section{Reader和Writer}

\emph{Reader}是你的程序可以从中读取字节的值。例如:
\begin{enumerate}
    \item 使用\texttt{std::fs::File::open(filename)}打开的文件
    \item 用于从网络中接收数据的\texttt{std::net::TcpStream}
    \item 进程用来读取标准输入的\texttt{std::io::stdin()}
    \item \texttt{std::io::Cursor<\&[u8]>}和\texttt{std::io::Cursor<Vec<u8>>}值,它们是从内存中的字节数组或vector中“读取”数据的reader
\end{enumerate}

\emph{Writer}是你的程序可以向其中写入字节的值。例如:
\begin{enumerate}
    \item 使用\texttt{std::fs::File::create(filename)}打开的文件
    \item 用于向网络中发送数据的\texttt{std::net::TcpStream}
    \item 用于写入到终端的\texttt{std::io::stdout()}和\texttt{std::io::stderr()}
    \item \texttt{Vec<u8>},它也是一个writer,它的\texttt{write}方法把数据附加到尾部
    \item std::io::Cursor<Vec<u8>>,类似于上面,但允许你同时读取和写入数据,并可以在vector中定位到不同位置
    \item std::io::Cursor<\&mut [u8]>,和\texttt{std::io::Cursor<Vec<u8>>}很像,除了它不能让缓冲区增长,因为它只是已经存在的字节数组的切片
\end{enumerate}

因为有为reader和writer设计的标准trait(\texttt{std::io::Read}和\texttt{std::io::Write}),所以编写可以处理多种输入输出通道的泛型代码是非常普遍的。例如,这里有一个函数拷贝任意reader中的所有字节到任意writer:
\begin{minted}{Rust}
    use std::io::{self, Read, Write, ErrorKind};

    const DEFAULT_BUF_SIZE: usize = 8 * 1024;

    pub fn copy<R: ?Sized, W: ?Sized>(reader: &mut R, writer: &mut W)
        -> io::Result<u64>
        where R: Read, W: Write
    {
        let mut buf = [0; DEFAULT_BUF_SIZE];
        let mut written = 0;
        loop {
            let len = match reader.read(&mut buf) {
                Ok(0) => return Ok(written),
                Ok(len) => len,
                Err(ref e) if e.kind() == ErrorKind::Interrupted => continue,
                Err(e) => return Err(e),
            };
            writer.write_all(&buf[..len])?;
            written += len as u64;
        }
    }
\end{minted}

这是Rust的标准库中的\texttt{std::io::copy()}的实现。因为它是泛型的,你可以使用它从\texttt{File}中读取数据然后写入到\texttt{TcpStream},或者从\texttt{Stdin}读取,然后写入到内存中的\texttt{Vec<u8>},等等。

如果你看不明白这里的错误处理代码,请复习\hyperref[ch07]{第7章}。我们将在接下来的页面中一直使用\texttt{Result}类型;掌握它的工作原理很重要。

这三个\texttt{std::io}的trait:\texttt{Read}、\texttt{BufRead}、\texttt{Write},以及\texttt{Seek}如此常用,以至于有一个只包含这些trait的\texttt{prelude}模块:
\begin{minted}{Rust}
    use std::io::prelude::*;
\end{minted}

本章中你还会见到它一到两次。我们通常也习惯于导入\texttt{std::io}模块自身:
\begin{minted}{Rust}
    use std::io::{self, Read, Write, ErrorKind};
\end{minted}

这里的\texttt{self}关键字声明了\texttt{io}作为\texttt{std::io}模块的一个别名。这样,\texttt{std::io::Result}和\texttt{std::io::Error}可以用\texttt{io::Result}和\texttt{io::Error}更精确地表示出来,等等。

\subsection{Reader}
\texttt{std::io::Read}有几个方法用于读取数据。所有这些方法都通过\texttt{mut}引用获取self参数。

\codeentry{reader.read(\&mut buffer)}
\hangparagraph{从数据源读取一些字节,并存储到给定的\texttt{buffer}中。\texttt{buffer}参数的类型是\texttt{\&mut [u8]}。它最多读取\texttt{buffer.len()}个字节。}
\hangparagraph{返回类型是\texttt{io::Result<u64>},它是\texttt{Result<u64, io::Error>}的类型别名。当成功时,\texttt{u64}值是读取到的字节数——可能等于或着小于\texttt{buffer.len()},\emph{即使还有更多的数据可以读取}。\texttt{Ok(0)}意味着没有更多的输入可以读取。}
\hangparagraph{当出错时,\texttt{.read()}返回\texttt{Err(err)},其中\texttt{err}是一个\texttt{io::Error}值。\texttt{io::Error}是可打印的,为了便于阅读;对于程序来讲,它有一个\texttt{.kind()}方法返回一个\texttt{io::ErrorKind}类型的错误码。这个枚举的成员有例如\texttt{PermissionDenied}和\texttt{ConnectionReset}的名称。大多数的错误类型都不能被忽略,但有一种错误应该进行特殊处理。\texttt{io::ErrorKind::Interrupted}对应Unix的错误码\texttt{EINTR},它意味着读取过程恰好被一个信号打断。除非你的程序想设计为根据信号做一些聪明的操作,否则它应该简单地重试读取操作。上一节中的\texttt{copy()}的代码,就是一个例子。}
\hangparagraph{如你所见,\texttt{.read()}方法非常底层,甚至直接继承了底层操作系统的怪癖。如果你要为一个新的数据源类型实现\texttt{Read} trait,这会赋予你极大的灵活性。但如果你尝试读取一些数据,就会非常难受。因此,Rust提供了几个更高级的便捷方法。它们都有基于\texttt{.read()}的默认实现。它们都处理了\texttt{ErrorKind::Interrupted},因此你不需要再处理。}

\codeentry{reader.read\_to\_end(\&mut byte\_vec)}
\hangparagraph{读取reader种剩余的所有输入,将读到的数据附加到\texttt{byte\_vec}尾部,\texttt{byte\_vec}是一个\texttt{Vec<u8>}。返回一个\texttt{io::Result<uszie>},表示读取到的字节数。}
\hangparagraph{这个方法读取的数据的大小没有限制,因此不要将它用于不受信任的源。(你可以使用\texttt{.take()}方法施加限制,如后文所述。}

\codeentry{reader.read\_to\_string(\&mut string)}
\hangparagraph{和上面相同,不过把数据附加到给定的\texttt{String}。如果流不是有效的UTF-8,它会返回一个\texttt{ErrorKind::InvalidData}错误。}
\hangparagraph{在一些编程语言中,字节输入和字符输入由不同的类型来处理。如今,UTF-8占据主导地位,Rust承认这一事实标准,并且完全支持UTF-8。其他字符集由开源的\texttt{encoding} crate提供支持。}

\codeentry{reader.read\_exact(\&mut buf)}
\hangparagraph{读取恰好足够的数据来填充给定的缓冲区。参数的类型是\texttt{\&mut [u8]},如果在读取够\texttt{buf.len()}个字节之前reader的数据就已经耗光,那么它会返回一个\texttt{ErrorKind::UnexpectedEof}错误。}

上面这些是\texttt{Read} trait的主要方法。除此之外,还有三个以值获取\texttt{reader}的适配器方法,将它转换为一个迭代器或者一个不同的reader:

\codeentry{reader.bytes()}
\hangparagraph{返回一个输入流的字节的迭代器。item的类型是\texttt{io::Result<u8>},因此每一个字节都需要进行错误检查。另外,它会逐字节调用\texttt{reader.read()},因此如果reader没有缓存的话会非常低效。}

\codeentry{reader.chain(reader2)}
\hangparagraph{返回一个新的reader,首先产生\texttt{reader}的所有输入,然后产生\texttt{reader2}的所有输入。}

\codeentry{reader.take(n)}
\hangparagraph{返回一个新的reader,从和\texttt{reader}相同的数据源读取数据,但最多只读取\texttt{n}个字节。}

没有关闭reader的方法。reader和writer通常实现了\texttt{Drop},因此它们会自动关闭。

\subsection{Buffered Reader}

\subsection{读取行}

\subsection{收集行}

\subsection{Writer}

\subsection{File}\label{file}