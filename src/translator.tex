\chapter{译者序}
这是我翻译的第二本英文资料。

\section*{相识}
初次听说Rust的传闻还是大二时候听说有位清华的大佬用Rust实现了自己的编译器和操作系统作为毕设,当时虽然不懂但大为震撼(笑)。
初次接触Rust应该还是大三,当时是看官方的书入门的。
但我相信一定有很多人和我一样,看完官方的书之后合上书发现自己其实还是什么也写不出来(笑)。

初次接触本书是在21年7月初从Rust中文社区的推送消息中得知,当发现本书有一千多页时推测本书应该是类似C/C++ Primer [Plus]那样的比较全面的书籍,事实果然如此。
不过因为当时忙于科研,没有空余时间阅读本书,就一直搁置了下去(实名吐槽某实验室让大四刚毕业的学生搞科研搞到8月份才能放暑假回家)。

初次开始阅读和翻译本书是在2021.9.19日,撰写本序是在2022.9.18日晚,历时恰好一年完成了本书的翻译,不得不感叹命运总是如此巧妙。(整整一年,鬼知道我这一年都经历了什么。)

\section*{相知}
研一刚开学之后忙于科研,到9月下旬才有时间开始翻译,之后连续翻译了几个月,直到快到21年底的时候科研任务又紧张起来才中断了翻译。
再之后就是寒假前后才又有时间继续翻译,在寒假结束之后开学不久我就做出了人生中也许是第二大也是第二正确的选择,随之而来的又是很长一段时间的翻译中断。
终于在22年9月重新拾起了翻译,坚持到了最后。

在翻译本书的前期,我最大的感受只有两个字——绝望。
一千多页的压力让我几乎窒息,即使我每天把所有的时间用来翻译,也只能翻译20-30页,而当时研一的我科研任务繁重,经常每天只能抽出一两个小时来进行翻译,结果就是每天只能翻译几页。
也只有在周末,才能拿出全天时间来翻译,但也只能翻译20-30页。
就这样持续了两周之后,突然发现自己坚持了两周把每天所有空闲时间都拿来翻译,竟然只翻译了$\frac{1}{5}$还不到?
再加上科研压力越来越大,翻译的时间越来越短,绝望之感尤甚。

还记得心态的转变是在翻译到了$\frac{1}{3}$左右的时候,就像是看到了曙光一样,第一次有了自己可以翻译完这本书的信心。
也是从那时起我不再经常犹豫是否还要继续翻译下去。
在那之前我也深刻地感受到了坚持做一件看不到希望的事情是多么痛苦。

22年5月于Rust中文社区推送的消息中得知已有人完成了本书的翻译,当时我离弃坑就差那么亿点点的距离:
直到那时本译文仍然几乎无人问津,且我已经明白了我以后的工作是不可能用到Rust的,既然我已用不到Rust而且本书也有人翻译过了,那我为什么还要继续下去?继续学习Rust和翻译本书对我而言又有什么意义?
幸运的是我很快就得到了这个问题的答案,再加上强迫症的影响,我还是决定继续下去。

\section*{尾声}
本书于2022.9.18日完成翻译(除了第17章不打算进行翻译),在完成时我感受到的并非是完成了一项艰难的任务之后的成就感和幸福感,而是一种平静和释然的感觉。成就感和幸福感不能说没有,只能说是被一年的时长稀释了,留下的只有平淡如水的心境,似乎翻译逐渐成为了习惯、成为了一有空闲时间就会选择的消磨时间的方式。

我在翻译的过程中可以说是收获满满,(虽然长达一年的时长让我已经几乎完全忘记了自己的收获),希望阅读本书的你也能有所收获。

\begin{flushright}
    ——汪屹硕(MeouSker77)

    (真正的实名,笑)

    2022.9.18
\end{flushright}

\section*{致谢}
感谢所有在github上star、fork、issue、pr的读者,你们对本译文的关注和支持也是我坚持下去的主要动力之一。