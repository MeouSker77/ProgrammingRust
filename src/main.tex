\documentclass[twoside,10pt,AutoFakeBold,AutoFakeSlant]{book}

\usepackage{xeCJK}
\usepackage{indentfirst}
\usepackage{minted}
\usepackage{xcolor}
\usepackage{graphicx}
\usepackage{longtable}
\usepackage{colortbl}
\usepackage{enumitem}
\usepackage{framed}
\usepackage[colorlinks,breaklinks,linktoc=page,allcolors=linkcolor]{hyperref}
\usepackage[numbered]{bookmark}     % 设置书签
\usepackage[pagestyles,clearempty]{titlesec}   % 设置页眉格式
\usepackage[a4paper,left=2cm,right=2cm,bottom=3.67cm]{geometry} % 设置页边距

\linespread{1.25}\linespread{1.4}   % 设置行距
\CJKsetecglue{\,}                   % 设置中英文间的空格

\setCJKmainfont[ItalicFont=楷体,BoldItalicFont=楷体]{宋体}  % 设置中文正文字体
\setCJKmonofont{宋体}                                       % 设置中文等宽字体
\setmainfont{Times New Roman}                               % 设置英文正文字体
\setmonofont[Path=../fonts/,BoldFont=Inconsolata-Bold.ttf]{Inconsolata-Regular.ttf}       % 设置英文等宽字体
\setlist{nosep}                 % 设置列表垂直间距为0

\usemintedstyle{solarized-light}

\newpagestyle{front}{               % 自定义页眉格式
\sethead[\thepage][][\chaptertitle] % 偶数页页眉格式
{\chaptertitle}{}{\thepage}         % 奇数页页眉格式
\headrule                           % 页眉下水平分隔线
}

\newpagestyle{main}{        % 自定义页眉格式
\sethead[\thepage][][Chapter \thechapter: \quad \chaptertitle] % 偶数页页眉格式
{\thesection \quad \sectiontitle}{}{\thepage}   % 奇数页页眉格式
\headrule                   % 页眉下水平分隔线
}

\definecolor{linkcolor}{RGB}{175, 74, 87}       % 定义链接颜色
\definecolor{tablecolor}{RGB}{241, 246, 252}    % 定义表格背景颜色

\newenvironment{note}{
    \begin{framed}
        \begin{center}
            {\color{gray} \huge{\textbf{\texttt{N\,O\,T\,E}}}}
        \end{center}
}{
    \end{framed}
}

\newcommand{\codeentry}[1]{
    \begin{flushleft}
        \emph{\texttt{#1}}
    \end{flushleft}
}

\newcommand{\hangparagraph}[1]{
    \hangafter 0
    \hangindent 2em
    \noindent
    #1
}

\includeonly{ch07}

\begin{document}
    \frontmatter
    \pagestyle{front}
    \pdfbookmark[0]{目录}{\indexname}
    \tableofcontents
    \chapter{前言}
Rust是一门系统编程语言。

因为很多业务程序员对系统编程并不是很熟悉,所以这里首先简单解释一下什么是系统编程,为之后的内容奠定基础。

当你合上笔记本电脑时,操作系统检测到了这一行为,然后把所有正在运行的程序挂起、关掉屏幕、并把电脑设置为睡眠;之后,当你打开笔记本电脑时,屏幕和其他组件被再次唤醒,并且每个程序可以在它中断的地方继续运行。我们对此习以为常,但这都多亏了系统程序员为此编写的很多代码。

系统编程被用于以下领域:
\begin{itemize}
    \item 操作系统
    \item 各种设备的驱动
    \item 文件系统
    \item 数据库
    \item 在非常廉价或需要极高的可靠性的设备上运行的代码
    \item 密码学
    \item 多媒体编解码器(用于读写音频、视频、图片文件的软件)
    \item 多媒体处理(例如,语音识别或图像处理软件)
    \item 内存管理(例如,实现一个垃圾回收器)
    \item 文本渲染(把文本和字体转换为像素点的过程)
    \item 实现更高级的编程语言(例如JavaScript和Python)
    \item 网络
    \item 虚拟化和容器
    \item 科学仿真
    \item 游戏
\end{itemize}

简而言之,系统编程是一种\emph{资源受限(resource-constrained)}的编程方式、是一种每个字节和每个CPU时钟都需要考虑的编程方式。仅仅是为了支持一个基本的应用所需要的系统代码的数量也是非常惊人的。

本书并不会教你系统编程。事实上,这本书包含了很多有关内存管理的细节,如果你没有自己进行过系统编程,你会感觉这些内容乍一看似乎没有必要。但如果你是一个熟练的系统程序员,你将会发现Rust是一门非常优秀的语言:它可以解决困扰了整个工业界几十年的主要问题。

\section*{谁应该阅读这本书}
如果你已经是一名系统程序员并且已经准备好替换C++,那么这本书就是为你而生。如果你是一名有其他任何语言的经验的开发者,不管是C\#、Java、Python、JavaScript还是其他语言,这本书同样适用于你。

然而,这阅读这本书的过程中,你需要学习的不止是Rust。为了充分利用这门语言,你还需要一些系统编程的经验。我们推荐在阅读这本书的同时用Rust实现一些系统编程的项目,构建一些你以前从来没有构建过的东西、一些充分利用Rust的速度、并发和安全性的东西。本前言开头的列表也许能给你一些启发。

\section*{我们为什么要撰写这本书}
早在我们开始学习Rust时我们就开始着手编写这本书。我们的目标是首先讲解Rust中主要的、新的概念,清晰而深入地呈现它们,以最大限度地避免通过试错来学习。

\section*{本书概览}
本书的前两章简单地介绍了Rust,并提供了一些简单的示例。之后我们转到\hyperref[ch03]{第3章}的基本数据类型。\hyperref[ch04]{第4章}、\hyperref[ch05]{第5章}专注于介绍所有权和引用的核心概念。我们推荐按照顺序阅读前5章。

第\hyperref[ch06]{6}到第\hyperref[ch10]{10}章覆盖了语言的基础部分:表达式(\hyperref[ch06]{第6章}),错误处理(\hyperref[ch07]{第7章}),crate和模块(\hyperref[ch08]{第8章}),结构体(\hyperref[ch09]{第9章}),枚举和模式(\hyperref[ch10]{第10章})。在这一部分可以跳过一些内容,但请不要跳过错误处理的章节。\hyperref[ch11]{第11章}包括trait和泛型,这是最后两个必需的重要概念。trait类似于Java或C\#中的接口。它们也是Rust支持把你自己的类型集成到语言中的主要手段。\hyperref[ch12]{第12章}展示了trait怎么支持运算符重载,\hyperref[ch13]{第13章}包括了很多有用的工具trait。

理解了trait和泛型就可以解锁本书的剩余部分了。闭包和迭代器,这两个你绝对不想错过的强大工具,分别在\hyperref[ch14]{第14章}和\hyperref[ch15]{第15章}中介绍。你可以以任意顺序阅读剩下的章节,或者按需阅读。它们包括了剩余的语言部分:集合(\hyperref[ch16]{第16章}),字符串和文本(\hyperref[ch17]{第17章}),输入和输出(\hyperref[ch18]{第18章}),并发(\hyperref[ch19]{第19章}),异步代码(\hyperref[ch20]{第20章}),宏(\hyperref[ch21]{第21章}),unsafe代码(\hyperref[ch22]{第22章})和调用其它语言中的函数(\hyperref[ch23]{第23章})。

\section*{本书中的约定}
本书中用到了以下约定:\footnote{译者注:因为我比较懒,所以基本上不会用等宽加粗和等宽斜体这两种字体。}

\hangafter 1
\hangindent 2em
\noindent
\emph{斜体}\\
表示新术语、URL、电子邮件地址、文件名和文件拓展名。

\hangafter 1
\hangindent 2em
\noindent
\texttt{等宽}\\
用于代码环境和在段落中引用代码中的元素例如变量或函数名、数据库、数据类型、环境变量、语句和关键字等。

\hangafter 1
\hangindent 2em
\noindent
\textbf{\texttt{等宽加粗}}\\
表示需要用户逐字输入的命令或其他文本。

\hangafter 1
\hangindent 2em
\noindent
\emph{\texttt{等宽斜体}}\\
表示需要用户用自己的值或者上下文推断出的值进行替换的文本。

\begin{note}
    这个标志表示一个注意事项。
\end{note}

\section*{使用示例代码}
补充材料(示例代码,练习等)可以在\url{https://github.com/ProgrammingRust}下载。

本书旨在帮助你完成工作。一般来说,本书中提供的示例代码都可以直接在自己的编程和文章中使用。除非你需要再次分发大量本书中的代码,否则你不需要联系我们获取授权。例如,编写一个使用了书中部分代码的程序并不需要授权。售卖或者分发O'Reilly书籍中的示例代码则需要授权。通过引用本书或书中的示例代码回答问题并不需要授权。将本书中的大量代码纳入你自己的产品文档则需要授权。

我们感激,但并不要求署名。如果要署名的话,应该包含标题、作者、出版社和ISBN。例如:“Programming Rust, Second Edition by Jim Blandy, Jason Orendorff, and Leonora F.S. Tindall(O’Reilly). Copyright 2021 Jim Blandy, Leonora F.S. Tindall, and Jason Orendorff, 978-1-492-05259-3.”

如果你感觉你对示例代码的使用方式不在上述范围内,请放心联系我们\url{permissions@oreilly.com}。

\section*{O'Reilly在线学习}
\begin{note}
    40多年以来,O'Reilly Media提供技术和业务培训、知识和洞察力,以帮助公司取得成功。
\end{note}

我们独特的专家和创意网络会通过书籍、文章、会议和在线学习平台分享他们的知识。O'Reilly的在线学习平台可以让你按需访问O'Reilly及其他200多家出版社的实时培训课程、深入学习路线、交互式代码环境。更多信息请访问\url{http://oreilly.com}。

\section*{如何联系我们}
请将和本书有关的评论和问题发送给出版社:

\hangafter 0
\hangindent 3em
\noindent
\large{O'Reilly Media, Inc.}\\
\large{1005 Gravenstein Highway North}\\
\large{Sebastopol, CA 95472}\\
\large{800-998-9938 (in the United States or Canada)}\\
\large{707-829-0515 (international or local)}\\
\large{707-829-0104 (fax)}

我们有一个本书的web页面,我们在那里列出了勘误表、示例和其他附加信息。你可以通过\url{https://oreil.ly/programming-rust-2e}访问该页面。

评论或有关本书的技术问题可以发送到邮箱\url{bookquestions@oreilly.com}。

访问\url{http://www.oreilly.com}获取更多有关我们的书籍和课程的信息。

我们的Facebook:\url{http://facebook.com/oreilly}

我们的Twitter:\url{http://twitter.com/oreillymedia}

我们的YouTube:\url{http://youtube.com/oreillymedia}

\section*{致谢}
你手中的这本书从我们的官方技术评审:Brian Anderson、Matt Brubeck、J. David Eisenberg、Ryan Levick、Jack Moffitt、Carol Nichols、Erik Nordin和翻译:Hidemoto Nakada (中田 秀基) (Japanese)、Mr. Songfeng Li (Simplified Chinese)、Adam Bochenek和Krzysztof Sawka (Polish)处受益匪浅。

还有很多非官方的评审阅读了早期的草案并提供了很有价值的反馈。我们想要感谢Eddy Bruel、Nick Fitzgerald、Graydon Hoare、Michael Kelly、Jeffrey Lim、Jakob Olesen、Gian-Carlo Pascutto、Larry Rabinowitz、Jaroslav Šnajdr、Joe Walker、Yoshua Wuyts的认真评论。
Jeff Walden和NicolasPierron花费了宝贵的时间来审阅几乎整本书。就像编程一样,一本编程书籍需要高质量的bug报告才能不断成长。感谢你们。

Mozilla对Jim和Jason在此项目中的工作非常宽容,尽管这超出了我们的官方职责范围,并会和他们产生竞争。我们非常感谢Jim和Jason的领导:Dave Camp、Naveed Ihsanullah、Tom Tromey、Joe Walker的支持。他们从长远的角度看待Mozilla,我们希望这些结果证明了他们对我们的信任。

我们还想对O'Reilly里每个帮助过我们的人表达感谢,尤其是非常有耐心的编辑Jeff Bleiel和Brian MacDonald,以及我们的策划编辑Zan McQuade。

最重要的是,我们衷心感谢家人们坚定不移的爱、热情和耐心。

    \mainmatter
    \pagestyle{main}
    \chapter{系统程序员的福音}\label{ch01}
\emph{在某些场景下——例如Rust的目标场景——比竞争对手快10x或者仅仅2x是足以决定成败的事情。它决定了一个系统在市场中的命运,就像在硬件市场中一样。}
\begin{flushright}
    ——Graydon Hoare
\end{flushright}

\emph{现在所有的计算机都是并行的……并行编程才是编程。}
\begin{flushright}
    ——Michael McCool et al., Structured Parallel Programming    
\end{flushright}

\emph{民族国家的攻击者们利用TrueType解析器的漏洞来进行监视;所有的软件都对安全很敏感。}
\begin{flushright}
    ——Andy Wingo
\end{flushright}

我们选择用以上三个引言来开始本书是有原因的。但首先让我们以一个谜题开始。下面的C程序做了什么?
\begin{minted}{C}
    int main(int argc, char **argv) {
        unsigned long a[1];
        a[3] = 0x7ffff7b36cebUL;
        return 0;
    }
\end{minted}

今天早上在Jim的笔记本电脑上,上面的程序输出了:
\begin{minted}{text}
    undef: Error: .netrc file is readable by others.
    undef: Remove password or make file unreadable by others.
\end{minted}
然后它就崩溃了。如果你在自己的机器上尝试,它的行为可能会不同。这个过程中到底发生了什么呢?

这段代码是有漏洞的。数组\texttt{a}的长度为1,对\texttt{a[3]}的访问,根据C语言标准,是\emph{未定义行为}:

\emph{对于使用不可移植或错误的程序结构或错误的数据的行为,本国际标准不做任何要求}

未定义行为不仅仅会导致非预期的结果,语言标准甚至允许程序在这种情况下做\emph{任何事情}。在我们的例子中,把一个特定的值存在特定数组的第4个元素处恰巧破坏了函数的调用栈,因此导致\texttt{main}函数返回时,并没有正常的退出程序,而是跳转到了C标准库里从用户的家目录中的一个文件中读取密码的代码中。这显然是有问题的。

C和C++有几百条避免未定义行为的规则。它们大多都是一些常识:不要访问不应该访问的内存,不要让算术运算溢出,不要除以零等等。然而编译器并不强制这些规则,它没有义务去检测哪怕明目张胆的违反规则的行为。事实上,上述程序编译时不会有错误和警告。避免未定义行为的责任全部落到你——程序员身上。

根据经验,我们程序员并不能很好的识别出未定义行为。有一位Utah大学的学生,研究员Peng Li修改了C和C++的编译器来让它们在编译时报告它们正在编译的程序中是否含有会导致未定义行为的模式。他发现几乎所有的程序都有,包括那些公认的优秀项目。想要在C和C++中避免未定义行为就和仅仅知道规则就想赢得国际象棋比赛一样不切实际。

各种偶然的奇怪信息或崩溃可能只属于质量问题,但自从1988年莫里斯蠕虫病毒使用前面显示的技术的一个变种在早期互联网上从一台计算机传播到另一台计算机以来,无意中的未定义行为就成了安全漏洞的一个主要原因。

因此C和C++把程序员推到了一个很尴尬的地位:这些语言是系统编程的工业标准,但它们对程序员的要求和限制却只能保证不断出现崩溃和安全问题。回答我们的谜题只是抛出了一个更大的问题:我们不能做的更好吗?

\section{Rust替你承担责任}
我们的答案对应着我们开头的三个引言。第三个引言引用自一篇报告,这篇报告中,一个叫做Stuxnet的计算机蠕虫在2010年被发现侵入了工业界的设备,并获取了受害计算机的控制权。它只是利用了解析word文档中嵌入的TrueType字体的代码中的未定义行为,没有使用任何其他技术。这段代码的作者显然没有预料到这段代码会被以这种形式利用,这说明不仅仅只有操作系统和服务器需要担心安全问题:任何需要处理来自不受信任来源的数据的软件都可能成为受害者。

Rust语言做了一个简单的保证:如果你的代码通过了编译器的检查,那么它将不会遇到未定义行为。悬垂指针,两次释放,空指针解引用都会在编译期被捕捉到。对数组的引用通过编译期和运行期的双重检查保证安全,当索引越界时,Rust不会像不幸的C语言一样出现缓冲区溢出,而是会安全地退出程序并打印出错误消息。

Rust旨在同时实现\emph{安全}和\emph{易于使用}。为了对你的程序行为做出更强的保证,Rust对你的代码施加了比C和C++更多的限制,这些限制需要通过一些实践和经验才能习惯。但总体来看这门语言的灵活性和表达力都是很强的。Rust的应用范围之广已经证明了这一点。

根据我们的经验,在相信语言可以帮助我们捕获错误的情况下,我们将有勇气尝试更有挑战性的项目。修改复杂的大型程序的风险将会降低,因为我们不再需要关注内存管理和指针有效性的问题。调试起来也会简单得多,因为潜在的bug不会破坏不相关的代码部分。

当然,还有很多Rust也不能检测出的bug。但在实践中,没有未定义行为可以显著改善开发的现状。

\section{安全的并发编程}
在C和C++中并发是众所周知的难,开发者通常只有在已经证明了单线程代码无法达到所需性能的情况下才会考虑并发。但第二个引言则认为并行非常重要以至于现代计算机将它视为基本的操作。

事实证明,Rust中保证内存安全的限制也可以保证Rust程序中不会出现数据竞争。你可以在线程间安全的共享数据,只要它不是正在被修改。被修改的数据只能通过同步原语来访问。你可以使用所有传统的工具:互斥锁、条件变量、通道、原子量等等,Rust会通过检查确保你正确地使用它们。

这些使Rust能够充分利用现代多核机器的性能。Rust的生态还提供了普通并发原语之外的库来帮助你完成复杂的负载,包括处理器池、无锁同步机制例如Read-Copy-Update等。

\section{Rust的速度很快}
最后,对应我们的第一条引言。Rust遵循了Bjarne Stroustrup在他的文章“Abstraction and the C++ Machine Model”中提到的为C++设计的原则:

\emph{一般情况下,C++的实现遵循0开销原则:你没有用到的部分,将不会有开销。你用到的部分,你将不能找到更好的代码。}

系统编程经常需要考虑如何将机器性能发挥到极限。对于视频游戏,整个机器都需要投入工作来为玩家创造出最好的体验。对于网页浏览,浏览器的性能制约了内容发布者可以做的事情的上限,在机器本身的限制范围内,浏览器需要将尽可能多的内存和处理器资源留给内容本身。同样的原则也适用于操作系统:内核需要把机器的资源尽可能多的留给用户程序,而不是被它们自身消耗。

但当我们说Rust很“快”的时候,到底是什么意思?一个人可以用任何通用语言写出非常慢的代码。更准确地说,如果你已经准备好认真设计你的程序来最大限度的利用底层机器的性能,那么Rust可以支撑你实现目标。这门语言的效率很高,并且能给予你控制使用多少内存和CPU资源的能力。

\section{Rust使协作变得更简单}
我们在标题中隐藏了第4条引言:“系统程序员的福音”。这是在暗示Rust对代码共享和重用的支持。

Rust的包管理器和构建工具Cargo,使用户可以很容易地使用其他用户发布在Rust的公开仓库crates.io上的库。你只需要简单地在一个文件中加上库的名字和版本号,cargo将会自动下载该库和它的依赖,并把它们链接在一起。你可以将Rust的Cargo视为NPM或者RubyGems一类的东西,只不过还同时强调完善的版本控制和可复制的构建。有很多流行的Rust库可以提供从序列化到HTTP客户端和服务器再到现代图形API等几乎任何功能。

进一步讲,这门语言本身就被设计为支持协作:Rust的trait和泛型让你能创建出拥有灵活接口的库,它们可以在很多不同的上下文中工作。Rust的标准库也提供了一组核心的基础类型,为常见的情况建立了共享的约定,使不同的库可以更容易地协同使用。

下一章旨在更具体地说明我们在这一章中提出的观点,我们通过几个小的Rust程序作为示例来展示这门语言的强大之处。
    \chapter{Rust概览}\label{ch02}
Rust给像本书一样的书籍的作者提出了一个挑战:赋予这门语言特色的并不是可以在第一页就展示出来的某些惊人的特性,而是如何设计这门语言来让它的各个部分可以无缝的协同工作,最终达到我们在上一章提到的目标:安全、高性能的系统级编程。这门语言的每一部分都在其他所有部分中得到了最好的证明。

因此,相比于一次着眼于一种语言特性,我们选择了几个简单但却完整的程序作为概览,每一个程序都会涉及到一些语言特性:

\begin{itemize}
    \item 作为热身,我们准备了一个简单的计算命令行参数的程序,以及相应的单元测试。这个程序展示了Rust的核心类型,并引入了\emph{trait}。
    \item 接下来,我们构建了一个web服务器。我们将会使用一个第三方库来处理HTTP的细节,并引入字符串处理、闭包、错误处理。
    \item 我们的第三个程序绘制了一个漂亮的图形,讲计算分布到多个线程来提高速度。这一部分包括一个泛型函数的示例,阐明了怎么处理类似于一个像素的概念,并展示了Rust对并发的支持。
    \item 最后,我们展示了一个使用正则表达式处理文件的健壮的命令行程序。这个程序展示了Rust标准库中处理文件的设施,和最常用的第三方正则表达式库。
\end{itemize}

Rust保证在对代码的性能影响最小的情况下防止未定义行为,这一保证影响了整个Rust中每个部分的设计,从标准的数据结构例如vector和string到Rust程序员使用第三方库的方式都受此影响。这些具体的细节书中都会提到,但是现在,我们想向你展示Rust是一门强大且有趣的语言。

当然,首先你要在你的计算机上安装Rust。

\section{rustup和Cargo}
安装Rust的最佳方式是使用\texttt{rustup}。访问\url{https://rustup.rs}并按照说明进行操作。

或者,你可以访问\href{https://www.rust-lang.org/}{Rust网站}来获取预构建好的Linux、macOS、Windows上的包。一些操作系统发行版里也包含Rust。我们推荐\texttt{rustup},因为它是专用于管理Rust安装的工具,就像Ruby的RVM和Node的NVM一样。例如,当一个新版本的Rust发布时,你只需要输入\texttt{rustup update}就可以完成更新。

在任何情况下,完成了安装之后,你应该可以通过命令行访问以下三条新命令:
\begin{minted}{text}
    $ cargo --version
    cargo 1.49.0 (d00d64df9 2020-12-05)
    $ rustc --version
    rustc 1.49.0 (e1884a8e3 2020-12-29)
    $ rustdoc --version
    rustdoc 1.49.0 (e1884a8e3 2020-12-29)
\end{minted}

这里,\texttt{\$}是命令提示符。在Windows上,可能是\texttt{C:\textbackslash>}或者别的类似的。这里我们运行了安装的三条命令,查询它们的版本。接下来让我们依次讲解每一个命令:
\begin{itemize}
    \item \texttt{cargo}是rust的编译管理器、包管理器和通用的工具。你可以使用Cargo来新建项目、构建并运行程序、管理所有代码中依赖的外部库。
    \item \texttt{rustc}是Rust的编译器。通常我们使用Cargo来调用编译器,但有时也需要直接运行它。
    \item \texttt{rustdoc}是Rust的文档工具。如果你在源代码中按照文档注释的格式写了文档,那么\texttt{rustdoc}可以通过它们构建出漂亮的HTML文档。和\texttt{rustc}一样,我们通常用Cargo来调用\texttt{rustdoc}。
\end{itemize}

方便起见,Cargo可以为我们创建新的Rust包,并设置好一些标准元数据:
\begin{minted}{text}
    $ cargo new hello
        Created binary (application) `hello` package
\end{minted}

这个命令创建了一个叫做\texttt{hello}的新的包目录,并准备好构建一个可执行程序。

进入包的顶级目录并查看:
\begin{minted}{text}
    $ cd hello
    $ ls -la
    total 24
    drwxrwxr-x.  4 jimb jimb 4096 Sep 22 21:09 .
    drwx------. 62 jimb jimb 4096 Sep 22 21:09 ..
    drwxrwxr-x.  6 jimb jimb 4096 Sep 22 21:09 .git
    -rw-rw-r--.  1 jimb jimb    7 Sep 22 21:09 .gitignore
    -rw-rw-r--.  1 jimb jimb   88 Sep 22 21:09 Cargo.toml
    drwxrwxr-x.  2 jimb jimb 4096 Sep 22 21:09 src
\end{minted}

我们可以看到Cargo创建了一个文件\texttt{Cargo.toml}来保存包的元数据。此时这个文件里还没有太多内容:
\begin{minted}{toml}
    [package]
    name = "hello"
    version = "0.1.0"
    authors = ["You <you@example.com>"]
    edition = "2018"

    # See more keys and their definitions at
    # https://doc.rust-lang.org/cargo/reference/manifest.html

    [dependencies]
\end{minted}

如果我们的程序中需要依赖的库,我们可以在这个文件中添加它们,Cargo将会负责下载、构建和更新这些库。我们将在\hyperref[ch08]{第8章}中详细讲述\texttt{Cargo.toml}文件。

Cargo已经为我们的包初始化好了\texttt{git}版本控制系统,创建了一个\texttt{.git}元数据目录和一个\texttt{.gitignore}文件。你可以通过向\texttt{cargo new}命令传递\texttt{--vcs none}参数来跳过这一步。

\texttt{src}子目录包含了实际的Rust代码:
\begin{minted}{text}
    $ cd src
    $ ls -l
    total 4
    -rw-rw-r--.  1 jimb jimb 45 Sep 22 21:09 main.rs
\end{minted}

看起来Cargo好像已经替我们写好了程序。\texttt{main.rs}中包含以下文本:
\begin{minted}{Rust}
    fn main() {
        println!("Hello, world!");
    }
\end{minted}

在Rust中,你甚至不需要编写自己的“Hello, World!”程序,这是新的Rust程序的模板:两个文件,总共13行。

我们可以从包中的任何目录调用\texttt{cargo run}命令来构建并运行我们的程序:
\begin{minted}{text}
    $ cargo run
        Compiling hello v0.1.0 (/home/jimb/rust/hello)
        Finished dev [unoptimized + debuginfo] target(s) in 0.28s
        Running `/home/jimb/rust/hello/target/debug/hello`
    Hello, world!
\end{minted}

这里,Cargo调用了Rust的编译器\texttt{rustc},然后运行了它生成的可执行文件。Cargo把可执行文件放在了顶层目录的\texttt{target}子目录下:
\begin{minted}{text}
    $ ls -l ../target/debug
    total 580
    drwxrwxr-x. 2 jimb jimb   4096 Sep 22 21:37 build
    drwxrwxr-x. 2 jimb jimb   4096 Sep 22 21:37 deps
    drwxrwxr-x. 2 jimb jimb   4096 Sep 22 21:37 examples
    -rwxrwxr-x. 1 jimb jimb 576632 Sep 22 21:37 hello
    -rw-rw-r--. 1 jimb jimb    198 Sep 22 21:37 hello.d
    drwxrwxr-x. 2 jimb jimb     68 Sep 22 21:37 incremental
    $ ../target/debug/hello
    Hello, world!
\end{minted}

如果需要的话,Cargo可以为我们清理生成的文件:
\begin{minted}{text}
    $ cargo clean
    $ ../target/debug/hello
    bash: ../target/debug/hello: No such file or directory
\end{minted}

\section{Rust函数}
Rust的语法借鉴自其他语言。如果你熟悉C、C++、Java或者JavaScript,你可以很快找到自己的方式来理解Rust的程序结构。这里有一个使用\href{https://en.wikipedia.org/wiki/Euclidean_algorithm}{欧几里得算法}计算两个整数的最大公约数的函数。你可以把它添加到\texttt{src/main.rs}的最后:
\begin{minted}{Rust}
    fn gcd(mut n: u64, mut m: u64) -> u64 {
        assert!(n != 0 && m != 0);
        while m != 0 {
            if m < n {
                let t = m;
                m = n;
                n = t;
            }
            m = m % n;
        }
        n
    }
\end{minted}

\texttt{fn}关键字(读作“fun”)创建了一个函数。这里,我们定义了一个叫\texttt{gcd}的函数,它有两个参数\texttt{m}和\texttt{n},类型都是\texttt{u64},也就是64位无符号整数。\texttt{->}词元指明了返回值类型:我们的函数返回一个\texttt{u64}类型的值。四个空格缩进是Rust的标准风格。

Rust的整数类型的名字代表了它们的大小和符号性:\texttt{i32}是有符号32位整数;\texttt{u8}是无符号8位整数(用于“字节”值)等等。\texttt{isize}和\texttt{usize}类型分别代表可以存下一个指针的有符号和无符号整数,在32位平台上它们就是32位,在64位平台上就是64位。Rust还有两种浮点数类型:\texttt{f32}和\texttt{f64},分别是IEEE标准的单精度和双精度浮点数类型,类似于C和C++中的\texttt{float}和\texttt{double}。

默认情况下,当变量初始化后,它的值就不能再被改变,但通过在参数\texttt{m}和\texttt{n}前加上\texttt{mut}关键字(读作“mute”,\emph{mutable}的缩写)就可以在函数体中对它们进行赋值。在实践中,大多数变量都不会被重新赋值,在阅读代码时\texttt{mut}关键字将是一个有用的提示。

函数体中首先调用了\texttt{assert!}宏,确保两个参数都不是0。\texttt{!}字符标志着这是宏调用,而不是函数调用。类似于C和C++中的\texttt{assert}宏,Rust中的\texttt{assert!}宏也会检查参数是否为真,如果不为真则中断程序,并输出一条有用的信息,其中包括断言失败的源码位置。这种终止的方式被称为\emph{panic}。和C和C++中断言可以被跳过不同,Rust总是检查断言,不管程序怎么编译。还有一个\texttt{debug\_assert!}宏,当程序被编译为release模式时会被跳过。

我们函数的主体是一个包含一条\texttt{if}语句和一条赋值语句的\texttt{while}循环。和C和C++不同,Rust的条件表达式不需要括号,但紧随其后的控制流语句需要花括号。

\texttt{let}语句声明了一个局部变量,比如函数中的\texttt{t}。我们不需要写出\texttt{t}的类型,因为Rust可以通过使用这个值的方式来推断它的类型。在我们的函数中,\texttt{t}只有和\texttt{m}、\texttt{n}相匹配,是\texttt{u64}类型时才可以正常运行。Rust只在函数体内推断类型:你必须写出函数参数和返回值的类型,就像我们所做的那样。如果你想指明\texttt{t}的类型,你可以写:
\begin{minted}{Rust}
    let t: u64 = m;
\end{minted}

Rsut有\texttt{return}语句,但是\texttt{gcd}函数并不需要。如果一个函数体以一个\emph{没有}分号结尾的表达式结尾,那么这个表达式的值就是函数的返回值。事实上,任何一个花括号包围的语法块都可以作为一个表达式。例如,这里有一个表达式打印出一条消息,然后返回\texttt{x.cos()}作为它的值:
\begin{minted}{Rust}
    {
        println!("evaluating cos x");
        x.cos()
    }
\end{minted}

在Rust中当控制流到达函数底部时利用这种形式返回值是一种很典型的做法,只有当在函数的中途显式地返回时才会使用\texttt{return}语句。

\section{编写并运行单元测试}
Rust语言内建有对测试的支持。为了测试我们的\texttt{gcd}函数,我们可以在\texttt{src/main.rs}的最后添加下列代码:
\begin{minted}{Rust}
    #[test]
    fn test_gcd() {
        assert_eq!(gcd(14, 15), 1);

        assert_eq!(gcd(2 * 3 * 5 * 11 * 17,
                       3 * 7 * 11 * 13 * 19),
                   3 * 11);
    }
\end{minted}

这里我们定义了一个叫\texttt{test\_gcd}的函数,它调用了\texttt{gcd}函数并检查返回值是否正确。喊函数上方的\texttt{\#[test]}标记\texttt{test\_gcd}是一个测试函数,这种函数在正常编译时会被跳过,但在使用\texttt{cargo test}命令时会被编译并自动调用。我们可以在整个源码树的任何位置定义测试函数,\texttt{cargo test}会自动收集它们并运行。

\texttt{\#[test]}标记是\emph{属性}的是一个示例。属性是一种为函数和其他声明标记额外信息的开放式系统,类似于C++和C\#中的属性,或者Java中的注解。它们被用来控制编译器警告和代码风格检查、条件编译(类似于C和C++中的\texttt{\#ifdef})、告诉Rust怎么和其它语言编写的代码交互等。随着继续深入我们将会看到更多使用属性的例子。

把\texttt{gcd}和\texttt{test\_gcd}函数的定义添加到\texttt{hello}包里之后,我们可以在包内的某个目录下按照如下方式运行测试:
\begin{minted}{text}
    $ cargo test
        Compiling hello v0.1.0 (/home/jimb/rust/hello)
         Finished test [unoptimized + debuginfo] target(s) in 0.35s
          Running /home/jimb/rust/hello/target/debug/deps/hello-2375a82d9e9673d7

    running 1 test
    test test_gcd ... ok
    
    test result: ok. 1 passed; 0 failed; 0 ignored; 0 measured; 0 filtered out
\end{minted}

\section{处理命令行参数}
为了让我们的函数能获取一些作为命令行参数传入的数字并打印出他们的最大公约数,我们可以把\texttt{src/main.rs}中\texttt{main}函数的代码替换为如下内容:
\begin{minted}{Rust}
    use std::str::FromStr;
    use std::env;

    fn main() {
        let mut numbers = Vec::new();

        for arg in env::args().skip(1) {
            numbers.push(u64::from_str(&arg)
                         .expect("error parsing argument"));
        }

        if numbers.len() == 0 {
            eprintln!("Usage: gcd NUMBER ...");
            std::process::exit(1);
        }

        let mut d = numbers[0];
        for m in &numbers[1..] {
            d = gcd(d, *m);
        }

        println!("The greatest common divisor of {:?} is {}",   numbers, d);
    }
\end{minted}

这是一个很大的代码块,让我们一步步来理解它:
\begin{minted}{Rust}
    use std::str::FromStr;
    use std::env;
\end{minted}

第一个\texttt{use}声明引入了标准库中的\texttt{FromStr} \emph{trait}。一个\texttt{trait}就是一些可以被实现的方法的集合。任何实现了\texttt{FromStr} trait的类型都有一个\texttt{from\_str}方法,这个方法尝试把一个字符串解析为该类型。\texttt{u64}类型实现了\texttt{FromStr},因此我们将调用\texttt{u64::from\_str}来解析命令行参数。尽管我们在程序中并没有使用到\texttt{FromStr}这个名字,但为了使用这个trait的方法,必须将它引入作用域。我们将会在第\hyperref[ch11]{第11章}讲解trait。

第二个\texttt{use}声明引入了\texttt{std::env}模块,它提供了一些和运行环境进行交互的函数和类型,包括\texttt{args}函数,它可以让我们获取到程序的命令行参数。

接下来移步到程序的\texttt{main}函数:
\begin{minted}{Rust}
    fn main() {
\end{minted}

我们的\texttt{main}函数不返回值,因此我们可以省略\texttt{->}和返回类型。

\begin{minted}{Rust}
    let mut numbers = Vec::new();
\end{minted}

我们声明了一个可变的局部变量\texttt{numbers},并将它初始化为一个空的向量。\texttt{Vec}是Rust的可变长的向量类型,类似于C++的\texttt{std::vector}、Python的\texttt{list}、或者JavaScript的\texttt{array}。即使vector被设计为动态伸缩,我们仍然需要将变量标记为\texttt{mut},这样Rust才允许我们向它的末尾添加元素。

\texttt{numbers}的类型是\texttt{Vec<u64>},一个\texttt{u64}的vector,但和之前一样,我们不需要写出类型。Rust将会为我们推断出它的类型,在这里我们把\texttt{u64}类型的值添加到了vector里,而且我们之后还把这个vector的元素传给了\texttt{gcd}函数,\texttt{gcd}函数的参数只能是\texttt{u64}。

\begin{minted}{Rust}
    for arg in env::args().skip(1) {
\end{minted}

这里我们使用了一个\texttt{for}循环来处理命令行参数,将每个参数命名为\texttt{arg}变量,然后执行循环体。

\texttt{std::env}模块的\texttt{args}函数返回一个\emph{迭代器},迭代器可以惰性产生每一个值,并指示我们何时迭代结束。迭代器在\texttt{Rust}中无处不在,标准库中还包含其他的迭代器例如产生vector中的每个元素、产生文件的每一行、产生通道收到的每一条消息、以及其他几乎所有可以循环处理的东西。Rust的迭代器非常高效:编译器通常能将它们翻译为和手写循环一样的代码。我们将会在\hyperref[ch15]{第15章}介绍该怎么使用它并给出一些示例。

除了和\texttt{for}循环一起使用之外,迭代器还有很多可以直接使用的方法。例如,\texttt{args}方法返回的迭代器的第一个值总是正在运行的程序名。我们想要跳过它,因此我们调用了迭代器的\texttt{skip}方法来生成一个省略了第一个值的新迭代器。

\begin{minted}{Rust}
    numbers.push(u64.from_str(&arg)
                 .expect("error parsing argument"));
\end{minted}

这里我们调用了\texttt{u64::from\_str}来尝试将命令行参数解析为64位整数。和通过\texttt{u64}类型的值调用的方法不同,\texttt{u64::from\_str}是一个和\texttt{u64}类型的值关联的方法,类似于C++和Java中的静态方法。\texttt{from\_str}函数不直接返回一个\texttt{u64}类型的值,而是返回一个\texttt{Result}类型的值来表示解析是否成功。一个\texttt{Rusult}类型的值有两种可能:

\begin{itemize}
    \item 一个写作\texttt{Ok(v)},表示解析成功,\texttt{v}就是解析出的值。
    \item 另一个写作\texttt{Err(e)},表示解析失败,\texttt{e}是解释失败的原因。
\end{itemize}

任何可能会失败的函数,例如输入输出或其他和操作系统交互的函数,都会返回\texttt{Result}值,\texttt{Ok}时会携带成功的结果——读写的字节数、打开的文件等等——\texttt{Err}时会携带错误码来指示错误的原因。和大多数现代编程语言不同,Rust没有异常:所有的错误都通过\texttt{Rust}或者panic来处理,这会在\hyperref[ch07]{第7章}中介绍。

我们使用Rust的\texttt{expect}方法来检查解析是否成功。如果结果是\texttt{Err(e)},\texttt{expect}会打印出包含\texttt{e}的描述错误的消息。如果结果是\texttt{Ok(v)},\texttt{expect}会简单的返回\texttt{v},之后我们才能把它添加到vector的尾部。

\begin{minted}{Rust}
    if numbers.len() == 0 {
        eprintln!("Usage: gcd NUMBER ...");
        std::process::exit(1);
    }
\end{minted}

空的数字集合没有最大公约数,因此我们检查vector是否不为空,如果为空就退出程序。我们使用\texttt{eprintln!}宏来把错误信息写入到标准错误输出流。

\begin{minted}{Rust}
    let mut d = numbers[0];
    for m in &numbers[1..] {
        d = gcd(d, *m);
    }
\end{minted}

这一个循环使用了变量\texttt{d},将它更新为目前的最大公约数。和之前一样,我们将\texttt{d}标记为可变的,因此我们在循环里给它赋值。

这个\texttt{for}循环有两个特别的地方。一个是\texttt{for m in \&numbers[1..];},操作符\texttt{\&}是什么意思?另一个是\texttt{gcd(d, *m);},\texttt{*m}里的\texttt{*}又是什么意思?事实上这两个细节是互补的。

到目前为止,我们的代码只操作过像整数这类固定内存大小的值。但是现在我们要迭代一个vector,它的值可能是任意大小——有可能非常大。在处理这类值时Rust是很谨慎的:它想让程序员自己控制对内存的消耗、明确每个值的生命周期,同时确保当内存不再被需要时立即释放内存。

因此当我们在迭代时,我们想告诉Rust这个vector的\emph{所有权}仍然属于\texttt{numbers},我们只是\emph{借用}它的值来进行循环。\texttt{\&numbers[1..]}中的\texttt{\&}运算符借用了vector中从第二个元素开始到最后一个元素的引用。\texttt{for}循环迭代引用的那些元素,每次迭代中用\texttt{m}借用每一个元素。\texttt{*m}中的\texttt{*}运算符\emph{解引用}了\texttt{m},返回了它所指向的值,也就是我们传递给\texttt{gcd}的第二个值。最后,因为\texttt{numbers}拥有vector的所有权,当\texttt{numbers}离开\texttt{main}的作用域时Rust会自动释放它的内存。

Rust对所有权和引用的规则时Rust的内存管理和安全并发的关键。我们将在\hyperref[ch04]{第4章}讨论他们,并在\hyperref[ch05]{第5章}讨论他们的伙伴。

要想舒服的使用Rust,你必须习惯这些规则,但在这篇概览中,你只需要知道\texttt{\&x}是借用\texttt{x}的引用,\texttt{*r}返回引用\texttt{r}指向的值。

继续我们的程序:
\begin{minted}{Rust}
    println!("The greatest common divisor of {:?} is {}", numbers, d);
\end{minted}

迭代完\texttt{numbers}的元素之后,程序把结果打印到标准输出流。\texttt{println!}宏接收一个模板字符串,用剩余参数替换掉模板字符串里的\texttt{\{...\}},并把结果写入到标准输出流。

与C和C++中的\texttt{main}函数成功执行结束时要返回0、执行失败时返回非0不同,Rust假设不管\texttt{main}返回什么值都代表成功执行结束。只有当显式调用\texttt{expect}或\texttt{std::process::exit}之类的才会导致程序以错误的状态码终止。

\texttt{cargo run}命令允许我们向程序传递参数,因此我们可以测试我们的命令行程序:
\begin{minted}{text}
    $ cargo run 42 56
       Compiling hello v0.1.0 (/home/jimb/rust/hello)
        Finished dev [unoptimized + debuginfo] target(s) in 0.22s
         Running `/home/jimb/rust/hello/target/debug/hello 42 56`
    The greatest common divisor of [42, 56] is 14
    $ cargo run 799459 28823 27347
        Finished dev [unoptimized + debuginfo] target(s) in 0.02s
         Running `/home/jimb/rust/hello/target/debug/hello 799459 28823 27347`
    The greatest common divisor of [799459, 28823, 27347] is 41
    $ cargo run 83
        Finished dev [unoptimized + debuginfo] target(s) in 0.02s
         Running `/home/jimb/rust/hello/target/debug/hello 83`
    The greatest common divisor of [83] is 83
    $ cargo run
        Finished dev [unoptimized + debuginfo] target(s) in 0.02s
         Running `/home/jimb/rust/hello/target/debug/hello`
    Usage: gcd NUMBER ...
\end{minted}

我们在这一节中用到了Rust标准库中的一小部分特性。如果你对其他的部分很好奇,我们强烈建议你去尝试Rust的在线文档。它的搜索功能很有用,甚至还包括到源代码的链接。当你安装Rust时\texttt{rustup}命令会自动在你的计算机上安装一份文档的拷贝。你可以在Rust的\href{https://www.rust-lang.org/learn}{网站}上查阅标准库的文档,或者通过如下命令在你的浏览器中查阅:
\begin{minted}{text}
    $ rustup doc --std
\end{minted}

\section{提供web页面}
Rust的一个强项就是发布在网站\href{https://crates.io}{crates.io}上的可以自由使用的包。\texttt{cargo}命令让你可以方便的使用crates.io的包:它将会按需下载正确的版本,构建并在需要时更新它。一个Rust包,无论是库还是可执行文件,都被称为一个\emph{crate}。Cargo和crates.io都是因为这个术语而得名。

为了展示它的工作方式,我们将会使用\texttt{actix-web}网络框架crate,\texttt{serde}序列化crate和其它它们依赖的crate来构建一个简单的web服务器。如\hyperref[f2-1]{图2-1}所示,我们的网站将会提示用户输入两个数字,然后计算它们的最大公约数。
\begin{figure}[htbp]
    \centering
    \includegraphics[width=0.8\textwidth]{../img/f2-1.png}
    \caption{提供计算最大公约数功能的web页面}
    \label{f2-1}
\end{figure}

首先,我们需要使用Cargo创建一个新的包,名称为\texttt{actix-gcd}:
\begin{minted}{text}
    $ cargo new actix-gcd
        Created binary (application) `actix-gcd' package
    $ cd actix-gcd
\end{minted}

然后,我们将编辑项目中的\texttt{Cargo.toml}文件来列举出我们需要的包,它的内容如下所示:
\begin{minted}{toml}
    [package]
    name = "actix-gcd"
    version = "0.1.0"
    authors = ["You <you@example.com>"]
    edition = "2018"

    # See more keys and their definitions at
    # https://doc.rust-lang.org/cargo/reference/manifest.html

    [dependencies]
    actix-web = "1.0.8"
    serde = { version = "1.0", features = ["derive"] }
\end{minted}

\texttt{Cargo.toml}中\texttt{dependencies}节的每一行都有一个cartes.io上的crate的名字和需要使用的版本。在这个例子中,我们需要\texttt{actix-web} crate的\texttt{1.0.8}版本和\texttt{serde} crate的\texttt{1.0}版本。crates.io上这两个crate可能还有更新的版本,但是通过指定我们测试成功的特定版本,可以保证即使这两个crate发布了新的版本,代码仍然可以正常工作。我们将会在\hyperref[ch08]{第8章}中详细的讨论版本管理。

crate有一些可选的特性:这些特性是有些用户可能用不到、但仍然需要包含在crate中的部分接口或实现。\texttt{serde}提供了一个非常简洁的方式来处理web表单中的数据,但是根据\texttt{serde}的文档,只有当我们选择了这个crate的\texttt{derive}特性,才可以使用这种方式,因此我们在\texttt{Cargo.toml}文件中制定了这个特性。

注意我们只需要指明那些我们直接使用的crate,\texttt{cargo}会自动下载它们依赖的其他crate。

在我们的第一个版本中,我们将保持web服务器的简洁:它将只提供一个页面,提示用户输入要计算的数字。将\texttt{actix-gcd/src/main.rs}中的内容替换为如下:
\begin{minted}{Rust}
    use actix_web::{web, App, HttpResponse, HttpServer};

    fn main() {
        let server = HttpServer::new( || {
            App::new()
                .route("/", web::get().to(get_index))
        });

        println!("Servering on http://localhost:3000...");
        server
            .bind("127.0.0.1:3000").expect("error binding server to     address")
            .run().expect("error running server");

    }

    fn get_index() -> HttpResponse {
        HttpResponse::Ok()
            .content_type("text/html")
            .body(
                r#"
                    <title>GCD Calculator</title>
                    <form action="/gcd" method="post">
                    <input type="text" name="n"/>
                    <input type="text" name="m"/>
                    <button type="submit">Compute GCD</button>
                    </form>
                "#
            )
    }
\end{minted}

我们以一条\texttt{use}声明开始,引入一些\texttt{actix-web}的定义。当我们写下\texttt{use actix\_web::\{...\}}时,花括号里的每一个名称都被导入到作用域中,这样我们就可以直接使用\texttt{HttpResponse},而不用每次都写出全名\texttt{actix\_web::HttpResponse}。(我们稍后才会使用\texttt{serde} crate)

我们的\texttt{main}函数很简单:它首先调用\texttt{HttpServer::new}来创建一个服务器,这个服务器会响应对\texttt{"/"}路径的get方法;然后打印出一条消息;最后让服务器监听本地机器上的3000 TCP端口。

我们传递给\texttt{HttpServer::new}的参数是Rust的\emph{闭包}表达式\texttt{|| \{ App::new() ... \}}。闭包是一种可以被当作函数来调用的值。这里定义的闭包没有参数,但如果需要参数的话,要写在\texttt{||}之间。\texttt{\{ ... \}}是闭包的函数体。当我们启动服务器时,Actix会启动一个线程池来处理到达的请求。每一个线程都会调用我们的闭包来获取一个\texttt{App}的拷贝,\texttt{App}的值将告诉它如何路由和处理请求。

闭包会调用\texttt{App::new}来创建一个新的、空的\texttt{App}并调用它的\texttt{route}方法来添加一个路径\texttt{"/"}的路由。用\texttt{web::get().to(get\_index)}为这个路由添加处理函数,作用是遇到HTTP \texttt{GET}请求时调用函数\texttt{get\_index}。\texttt{route}方法会返回调用它的\texttt{App}自身,并增加新的路由。因为闭包的结尾处没有分号,因此\texttt{App}就是闭包的返回值,\texttt{HttpServer}线程将会使用它。

\texttt{get\_index}函数构建了一个\texttt{HttpResponse}类型的值作为HTTP \texttt{GET /}请求的响应。\texttt{HttpResponse::Ok()}代表HTTP \texttt{200 OK}状态,表示请求被成功处理。我们还调用了它的\texttt{content\_type}和\texttt{body}方法来填充相应的细节;每一个调用都会返回调用它们的\texttt{HttpResponse}。最后,\texttt{body}方法的返回值作为\texttt{get\_index}的返回值。

因为相应的文本中包含很多双引号,所以我们使用了Rust的raw string语法:字母\texttt{r}、0个或多个井号(\texttt{\#}字符)、一个双引号,string的内容、最后以另一个双引号加上和开头处相同数量的井号结尾。raw string中出现的任何字符都不会被转义,包括双引号;事实上,没有转移的序列例如\texttt{\textbackslash"}也可以被识别。我们可以通过增多井号的数量来保证终止的标记不会出现在字符串内容中。

编写完\texttt{main.rs}之后,我们可以使用\texttt{carto run}命令来运行它:它会自动获取所需的crate、编译它们、编译我们自己的程序、并链接在一起、然后启动它:
\begin{minted}{text}
    $ cargo run
        Updating crates.io index
     Downloading crates ...
      Downloaded serde v1.0.100
      Downloaded actix-web v1.0.8
      Downloaded serde_derive v1.0.100
    ...
      Compiling serde_json v1.0.40
      Compiling actix-router v0.1.5
      Compiling actix-http v0.2.10
      Compiling awc v0.2.7
      Compiling actix-web v1.0.8
      Compiling gcd v0.1.0 (/home/jimb/rust/actix-gcd)
       Finished dev [unoptimized + debuginfo] target(s) in 1m 24s
        Running `/home/jimb/rust/actix-gcd/target/debug/actix-gcd`
    Serving on http://localhost:3000...
\end{minted}

到目前为止,我们可以可以访问给定的URL并看到如\hyperref[f2-1]{图2-1}所示的页面。

不幸的是,点击GCD并不会做任何事,而且会把我们的浏览器导航到一个空页面。接下来让我们来修复它,通过向我们的\texttt{App}添加另一个路由来处理我们表单的\texttt{POST}请求的响应。

终于到了使用我们在\texttt{Cargo.toml}中列出的\texttt{serde} crate的时候了:它提供了一种帮助我们处理表单数据的简单方法。首先,我们需要在\texttt{src/main.rs}中添加如下\texttt{use}声明:
\begin{minted}{Rust}
    use serde::Deserialie;
\end{minted}

Rust程序员通常会把\texttt{use}声明集中在文件开始处,但这并不是必须的:Rust允许\texttt{use}声明以任何顺序出现,只要它们出现在正确的嵌套层级。

接下来让我们定义一个Rust结构体类型来表示我们希望从表单中接收到的数据:
\begin{minted}{Rust}
    #[derive(Deserialize)]
    struct GcdParameters {
        n: u64,
        m: u64,
    }
\end{minted}

这里定义了一个新的类型叫做\texttt{GcdParameters},它有两个字段\texttt{n}和\texttt{m},都是\texttt{u64}类型,和\texttt{gcd}函数的参数类型保持一致。

\texttt{struct}定义上方的注解是一个属性,类似于我们之前用来标记为测试函数时使用的\texttt{\#[test]}属性。在一个类型定义前加上\texttt{\#[derive(Deserialize)]}属性可以告诉\texttt{serde} crate在程序编译时自动为该类型生成代码来把HTML \texttt{POST}请求的表单数据转为该类型的值。事实上,这个属性可以让你从几乎所有结构化的数据:JSON、YAML、TOML或其他文本或二进制格式中解析出一个\texttt{GcdParameters}类型的值。\texttt{serde} crate还提供一个\texttt{Serialize}属性来做相反的事,即把Rust值写成结构化的格式。

有了上面的定义,我们可以简单的写出我们的处理函数:
\begin{minted}{Rust}
    fn post_gcd(form: web::Form<GcdParameters>) -> HttpResponse {
        if form.n == 0 || form.m == 0 {
            return HttpResponse::BadRequest()
                .content_type("text/html")
                .body("Computing the GCD with zero is boring.");
        }

        let response = 
            format!("The greatest common divisor of the numbers {} and {} \
                    is <b>{}</b>\n",
                    form.n, form.m, gcd(form.n, form.m));
        
        HttpResponse::Ok()
            .content_type("text/html")
            .body(response)
    }
\end{minted}

作为Actix请求的处理函数,它的参数的类型必须是Actix知道怎么从中提取出HTTP请求的类型。我们的\texttt{post\_gcd}函数有一个参数\texttt{form},它的类型是\texttt{web::Form<GcdParameters>}。当且仅当\texttt{T}可以从HTML的\texttt{POST}表单数据反序列化出来时,Actix才知道怎么\texttt{web::Form<T>}类型中提取出值。因为我们在\texttt{GcdParameters}类型的定义前加上了\texttt{\#[derive(Deserialize)]}属性,所以Actix可以从表单数据中反序列化出它,因此请求的处理函数可以使用\texttt{web::Form<GcdParameters>}作为参数。这些类型和函数之间的关系都是在编译期处理的,如果你用了Actix不知道该如何处理的类型作为参数的类型,Rust编译器将会立刻告诉你这个错误。

再看\texttt{post\_gcd}的实现:如果有参数的值为0,这个函数会返回一个HTTP \texttt{401 BAD REQUEST}错误,因为这种情况下我们的\texttt{gcd}函数会panic。然后,它使用\texttt{format!}宏创建了一个响应。\texttt{format!}宏类似于\texttt{println!}宏,只不过它不会把字符串写入到标准输出,而是会返回字符串。当响应的文本就绪后,\texttt{post\_gcd}用一个HTTP \texttt{200 OK}响应来包装它,设置好content type之后,就把它返回到请求方。

我们还要把\texttt{post\_gcd}注册为表单的处理函数。我们要把\texttt{main}函数替换为如下内容:
\begin{minted}{Rust}
    fn main() {
        let server = HttpServer::new(|| {
            App::new()
                .route("/", web::get().to(get_index))
                .route("/gcd", web::post().to(post_gcd))
        });

        println!("Servering on http://localhost:3000...");
        server
            .bind("127.0.0.1:3000").expect("error binding server to address")
            .run().expect("error running server");
    }
\end{minted}

唯一的变化在于多了一个\texttt{route}的调用,把\texttt{web::post().to(post\_gcd)}注册为路径\texttt{"/gcd"}的handler。

最后剩余的部分是我们之前编写的\texttt{gcd}函数,要把它添加到\texttt{actix-gcd/src/main.rs}文件中。完成这些之后,你可以停止之前运行的服务器并重新启动程序:
\begin{minted}{text}
    $ cargo run
    Compiling actix-gcd v0.1.0 (/home/jimb/rust/actix-gcd)
     Finished dev [unoptimized + debuginfo] target(s) in 0.0 secs
      Running `target/debug/actix-gcd`
    Serving on http://localhost:3000...
\end{minted}

这一次,通过访问\emph{http://localhost:3000},输入一些数字,然后点击Compute GCD按钮,你应该能看到如下的结果(\hyperref[f2-2]{图2-2})。
\begin{figure}[htbp]
    \centering
    \includegraphics[width=0.8\textwidth]{../img/f2-2.png}
    \caption{显示计算GCD结果的web页面}
    \label{f2-2}
\end{figure}

\section{并发}
Rust的另一项长处是对并发编程的支持。Rust中用于避免内存安全问题的规则同样可以保证线程之间在没有数据竞争的情况下共享数据。例如:
\begin{itemize}
    \item 如果你使用自旋锁来同步需要修改共享数据结构的线程,Rust保证只有当你持有锁的情况下才可以访问数据,并且当你处理完数据后自动释放锁。在C和C++中,自旋锁和要保护的数据之间的关系一般都写在注释里。
    \item 如果你想在几个线程中共享只读数据,Rust保证你不可能意外修改数据。在C和C++中,类型系统也可以帮我们做到这一点,但很容易出错。
    \item 如果你把一个数据结构的所有权从一个线程转移到另一个线程,Rust保证你确实已经失去了对它的访问权。在C和C++中,需要由你自己来保证发送线程不会再次访问该数据。如果你没有正确做到这些,那么结果将会却决于处理器的缓存和你最近写入了多少内存。
\end{itemize}

在这一节中,我们将带领你编写你的第二个多线程程序。

你已经编写过第一个多线程程序了:你用来实现最大公约数服务器的Actix web框架使用了线程池来运行请求的处理函数。如果服务器同时收到很多请求,它会立刻在若干个线程里运行\texttt{get\_form}和\texttt{post\_gcd}函数。这可能让我们有些震惊,因为当我们编写那些函数时我们完全没有并发的概念。

不过Rust保证这么做是安全的,不管你的服务器变得多复杂:只要你的程序能够编译,它就能免于数据竞争。所有的Rust函数都是线程安全的。

这一节的程序将绘制曼德勃罗集,它是一种通过迭代一个简单的复数函数得到的分形。绘制曼德勃罗集经常被称为\emph{embarrassingly parallel}算法,因为线程之间的通信太过简单;我们将会在\hyperref[ch19]{第19章}中讲述更为复杂的模式,但这个例子已经可以展示出一些核心的部分。

开始之前,我们要创建一个新的Rust项目:
\begin{minted}{text}
    $ cargo new mandelbrot
         Created binary (application) `mandelbrot` package
    $ cd mandelbrot
\end{minted}

所有的代码都会添加到\texttt{mandelbrot/src/main.rs}里,我们将会把一些依赖添加到\texttt{mandelbrot/Cargo.toml}。

在开始实现并行的曼德勃罗集之前,我们需要先描述一下我们准备实现的计算过程。

\subsection{曼德勃罗集到底是什么}
了解这一点可以帮我们在阅读代码时更加清楚它要做什么,因此首先让我们先探讨一些纯数学只是。我们将以一个简单的例子开始,并逐渐添加复杂的细节,直到我们讲到曼德勃罗集的核心。

首先这里有一个无限循环,用Rust的语法来写的话就是一个\texttt{loop}语句:
\begin{minted}{Rust}
    fn square_loop(mut x: f64) {
        loop {
            x = x * x;
        }
    }
\end{minted}

在现实中,Rust可能会看出\texttt{x}的值从来没有被使用过,因此并不执行计算。但一开始,首先让我们假设代码按照我们所写的运行。\texttt{x}的值将会发生什么变化?任何小于1的数平方都会变得更小,因此它会接近于0;1的平方还是1;大于1的数平方会变得更大,因此它会接近无限大;负数的平方将会使它变为整数,然后它的变化就和前面说的一样(\hyperref[f2-3]{图2-3})。
\begin{figure}[htbp]
    \centering
    \includegraphics[width=0.8\textwidth]{../img/f2-3.png}
    \caption{重复平方一个数的结果}
    \label{f2-3}
\end{figure}

因此根据传递给\texttt{square\_loop}的值不同,\texttt{x}可能会保持0或者1不变,也可能接近0或者接近无限大。

现在让我们考虑一个稍有不同的循环:
\begin{minted}{Rust}
    fn square_add_loop(c: f64) {
        let mut x = 0.;
        loop {
            x = x * x + c;
        }
    }
\end{minted}

这一次,\texttt{x}从0开始,并且每次迭代时平方之后再加上\texttt{c}。这导致我们更难看出\texttt{x}会怎么变化,但一些实验表明如果\texttt{c}大于\texttt{0.25}或者小于\texttt{-2.0},那么\texttt{x}将会变得接近无穷大;否则它会保持在接近0的某个区间。

更进一步,如果不用\texttt{f64}类型的值,考虑使用复数来执行相同的循环。crates.io上的\texttt{num} crate提供了一个我们可以使用的复数类型,因此我们必须在我们程序的\texttt{Cargo.toml}文件的\texttt{[dependencies]}节中添加一行对\texttt{num}的引用。这是到目前为止这个文件里的全部内容(稍后我们还会添加一些内容):
\begin{minted}{toml}
    [package]
    name = "mandelbrot"
    version = "0.1.0"
    authors = ["You <you@example>"]
    edition = "2018"

    # See more keys and their definitions at
    # https://doc.rust-lang.org/cargo/reference/manifest.html

    [dependencies]
    num = "0.4"
\end{minted}

现在我们可以写出循环的倒数第二个版本:
\begin{minted}{Rust}
    use num::Complex;

    fn complex_square_add_loop(c: Complex<f64>) {
        let mut z = Complex { re: 0.0, im: 0.0};
        loop {
            z = z * z + c;
        }
    }
\end{minted}

用\texttt{z}表示复数是一种习惯,因此我们重命名了变量。表达式\texttt{Complex \{ re: 0.0, im: 0.0 \}}创建了一个\texttt{num} crate的\texttt{Complex}类型的复数0值。\texttt{Complex}时一个Rust的结构体类型,定义类似于如下:
\begin{minted}{Rust}
    struct Complex<T> {
        /// Real portion of the complex number
        re: T,

        /// Imaginary portion of the complex number
        im: T,
    }
\end{minted}

上面的代码定义了一个叫\texttt{Complex}的结构体,有两个字段:\texttt{re}和\texttt{im}。\texttt{Complex}是一个\emph{泛型}结构体:你可以将类型名后的\texttt{<T>}理解为“任意类型\texttt{T}”。例如,\texttt{Complex<f64>}是一个\texttt{re}和\texttt{im}字段都是\texttt{f64}类型的复数,\texttt{Complex<f32>}则是32位浮点数,等等。有了这个定义之后,类似于\texttt{Complex \{ re: 0.24, im: 0.3 \}}这样的表达式将会产生一个\texttt{re}字段初始化为0.24、\texttt{im}字段初始化为0.3的\texttt{Complex}类型的值。



    \chapter{基本类型}\label{ch03}
\emph{这个世界上有很多很多不同类型的书,这是一件好事。但也有很多很多不同类型的人,每个人都想读到一些不同的东西。}
\begin{flushright}
    ——Lemony Snicket
\end{flushright}

在很大程度上,Rust语言是围绕它的类型来设计的。它对高性能代码的支持源于让开发者选择不同情况下最合适的数据表示,并在简单性和成本之间取得适当的平衡。Rust的内存和线程安全也依赖于类型系统的健全性,Rust的灵活性则来自于它的泛型和trait。

这一章将介绍Rust的基本类型。这些源码级别的类型都有对应的成本和性能可预测的机器级的组件。尽管Rust并不保证它会完全按照你的要求精确的表示数据,但只有当它是一个可靠的改进时它才会违背你的要求。

与JavaScript或Python这种动态类型语言相比,Rust要求你事先就进行更多规划。你必须写出函数参数和返回值、结构体字段、以及一些其他结构的类型。然而,Rust的两个特性使得这比你想象中的要简单很多:

\begin{itemize}
    \item 有了你指明的类型,Rust的\emph{类型推导}将会为你推导出剩余的大部分类型。在实践中,通常只有一个类型能够满足给定的变量或表达式。在这种情况下,Rust允许你留空,或者说\emph{省略}这个类型。例如,你可以像下面这样写出一个函数里的所有类型:
          \begin{minted}{Rust}
    fn build_vector() -> Vec<i16> {
        let mut v: Vec<i16> = Vec::<i16>::new();
        v.push(10i16);
        v.push(20i16);
        v
    }
    \end{minted}
          但这非常杂乱和重复。给定了函数的返回值之后,很明显\texttt{v}必须是\texttt{Vec<i16>}类型:一个16位有符号整数的vector,没有其他的类型可以满足语义。并且据此可以推出vector的每个元素必须是\texttt{i16}类型。这就是Rust的类型推导适用的场景,所以你可以改为:
          \begin{minted}{Rust}
    fn build_vector() -> Vec<i16> {
        let mut v = Vec::new();
        v.push(10);
        v.push(20);
        v
    }
    \end{minted}
          这两个定义是完全等价的,Rust将会生成完全相同的机器代码。类型推导可以回馈一部分动态类型语言的可读性,并且仍能在编译时捕捉到类型错误。
    \item 函数可以是\emph{泛型}的:一个函数可以同时处理很多不同类型的值。

          在Python和JavaScript中,所有的函数都很自然的是泛型的:一个函数可以操作任何类型的值,只要这个类型有函数体中需要的属性和方法。(这种特性通常被称作\emph{鸭子类型}:如果它像鸭子一样叫,那它就是一只鸭子。)但正是这种灵活性也导致这些语言很难检测出类型错误,在这些语言里测试通常是唯一一种捕捉类型错误的方式。Rust的泛型函数给予了这门语言某种程度上和动态类型同样的灵活性,并且仍能在编译期捕获所有的类型错误。

          除了灵活性之外,泛型函数和非泛型的函数一样高效。例如,为每个整数类型都编写一个\texttt{sum}函数与编写一个处理所有整数类型的泛型\texttt{sum}函数相比,并没有性能上的优势。我们将在\hyperref[ch11]{第11章}种详细讨论泛型函数。
\end{itemize}

这一章的剩余部分将会自上而下的覆盖Rust的类型,从最简单的数字类型例如证书和浮点数到持有多个值的复合类型:box、tuple、数组和字符串。

这里有一个Rust中类型的汇总。\hyperref[t3-1]{表3-1}显示了Rust的原始类型,包括一些来自标准库的基本类型,和一些用户自定义类型的示例。

\begin{longtable}{p{0.25\textwidth}p{0.4\textwidth}p{0.25\textwidth}}
    \caption{Rust中的类型示例}
    \label{t3-1}\\
    \hline
    \textbf{类型}   & \textbf{描述}    & \textbf{值}    \\
    \hline
    \texttt{i8, i16, i32, i64, i128, u8, u16, u32, u64, u128}    & 指定位数的有符号和无符号整数 & \texttt{42, -5i8, 0x400u16, 0o100i16, 20\_922\_789\_888\_000u64, b'*'(u8字节字面量)}    \\
    \rowcolor{tablecolor}
    \texttt{isize, usize}   & 有符号和无符号整数,和机器里的一个指针一样大(32位或64位)   & \texttt{137, -0b0101\_0010isize, 0xffff\_fc00usize} \\
    \texttt{f32, f64}       & IEEE浮点数,单精度和双精度                                & \texttt{1.61803, 3.14f32, 6.0221e23f64} \\
    \rowcolor{tablecolor}
    \texttt{bool}           & 布尔值            & \texttt{true, false} \\
    \texttt{char}           & Unicode字符,32位 & \texttt{'*', '\textbackslash n', '字', '\textbackslash x7f', '\textbackslash u\{CA0\}'} \\
    \rowcolor{tablecolor}
    \texttt{(char, u8, i32)}                        & Tuple:把类型混合在一起   & \texttt{('\%', 0x7f, -1)} \\
    \texttt{()}                                     & “单元值”(空tuple)      & \texttt{()} \\
    \rowcolor{tablecolor}
    \texttt{struct S \{ x: f32, y: f32 \}}          & 命名字段结构体            & \texttt{S \{ x: 120.0, y: 209.0 \}} \\
    \texttt{struct T (i32, char);}                  & 元组结构体                & \texttt{T(120, 'X')} \\
    \rowcolor{tablecolor}
    \texttt{struct E;}                              & 元组结构体,无字段        & \texttt{E} \\
    \texttt{enum Attend \{ OnTime, Late(u32) \}}    & 枚举,代数数据类型        & \texttt{Attend::Late(5), Attend::OnTime} \\
    \rowcolor{tablecolor}
    \texttt{Box<Attend}                             & Box:持有一个堆上的值的指针   & \texttt{Box::new(Late(15))} \\
    \texttt{\&i32, \&mut i32}                       & 共享和可变引用:生命周期不能超过所引用对象的无所有权的指针 & \texttt{\&s.y, \&mut v} \\
    \rowcolor{tablecolor}
    \texttt{String}                                 & UTF-8字符串,动态大小         & \texttt{"ラーメン: ramen"\newline.to\_string()} \\
    \texttt{\&str}                                  & \texttt{str}的引用:指向UTF-8字符串的无所有权的指针 & \texttt{"そば: soba", \&s[0..12]} \\
    \rowcolor{tablecolor}
    \texttt{[f64; 4], [u8; 256]}                    & 固定长度的数组,所有元素的类型都必须相同   & \texttt{[1.0, 0.0, 0.0, 1.0], [b' '; 256]} \\
    \texttt{Vec<f64>}                               & 可变长度的vector,所有元素的类型都必须相同 & \texttt{vec![0.367, 2.718, 7.389]} \\
    \rowcolor{tablecolor}
    \texttt{\&[u8], \&mut [u8]}                     & 切片的引用:指向数组或vector的一部分,包含指针和长度 & \texttt{\&v[10..20], \&mut a[..]} \\
    \texttt{Option<\&str>}      & 可选值:\texttt{None}(无值)或\texttt{Some(v)}(有值,值为\texttt{v})   & \texttt{Some("Dr.", None)} \\
    \rowcolor{tablecolor}
    \texttt{Result<u64, Error>} & 可能会失败的操作的结果:成功时是\texttt{Ok(v)},失败时是\texttt{Err(e)} & \texttt{Ok(4096), Err(Error::last\_os\_error())} \\
    \texttt{\&dyn Any, \&mut dyn Read}  & trait对象:指向一个实现了给定方法的任何值 & \texttt{value as \&dyn Any, \&mut file as \&mut dyn Read} \\
    \rowcolor{tablecolor}
    \texttt{fn(\&str) -> bool}          & 函数指针      & \texttt{str::is\_empty}           \\
    (闭包类型)                         & 闭包         & \texttt{|a, b| \{ a*a + b*b \}}    \\
\end{longtable}

这些类型中的大部分都会在这一章中介绍,除了下面这些:
\begin{itemize}
    \item 我们将在\hyperref[ch09]{第9章}中单独介绍\texttt{struct}类型。
    \item 我们将在\hyperref[ch10]{第10章}中单独介绍枚举类型。
    \item 我们将在\hyperref[ch11]{第11章}中介绍trait对象。
    \item 我们将在这里介绍\texttt{String}和\texttt{\&str}的基础,但在\hyperref[ch17]{第17章}中介绍更多细节。
    \item 我们将在\hyperref[ch14]{第14章}介绍函数和闭包类型。
\end{itemize}

\section{固定位数的数字类型}
Rust类型系统的基础是一组固定宽度的数字类型的集合,这些类型和现代处理器中的硬件类型相匹配。

固定宽度的数字类型可能会溢出或失去精度,但它们适用于大多数的类型,并且比任意精度的整数和精确小数快几千倍。如果你需要那些类型的数字,可以在\texttt{num} crate找到相应的支持。

Rust的数字类型的名称遵循通用的模式,宽度加上表示的含义(\hyperref[f3-2]{表3-2})。
\begin{table}[htbp]
    \centering
    \caption{Rust的数字类型}
    \label{f3-2}
    \begin{tabular}{llll}
        \hline
        \textbf{大小(比特数)}   & \textbf{无符号整数}   & \textbf{有符号整数}   & \textbf{浮点数}   \\
        \hline
        \texttt{8}  & \texttt{u8}   & \texttt{i8}   &              \\
        \rowcolor{tablecolor} 
        \texttt{16} & \texttt{u16}  & \texttt{i16}  &              \\
        \texttt{32} & \texttt{u32}  & \texttt{i32}  & \texttt{f32} \\
        \rowcolor{tablecolor} 
        \texttt{64} & \texttt{u64}  & \texttt{i64}  & \texttt{f64} \\
        \texttt{128}& \texttt{u128} & \texttt{i128} &              \\
        \rowcolor{tablecolor} 
        机器字      & \texttt{usize} & \texttt{isize} & \\
    \end{tabular}
\end{table}

这里,\emph{机器字}是运行代码的机器上的一个指针的大小,32位或者64位。

\subsection{整数类型}

Rust的无符号整数使用全部的范围来表示正数和0(\hyperref[t3-3]{表3-3})。
\begin{table}[htbp]
    \centering
    \caption{Rust无符号整数类型}
    \label{t3-3}
    \begin{tabular}{ll}
        \hline
        \textbf{类型}   &   \textbf{范围}                   \\
        \hline
        \texttt{u8}     & 0到$2^{8}-1$(0到255)            \\
        \rowcolor{tablecolor} 
        \texttt{u16}    & 0到$2^{16}-1$(0到65,535)        \\
        \texttt{u32}    & 0到$2^{32}-1$(0到4,294,967,295) \\
        \rowcolor{tablecolor} 
        \texttt{u64}    & 0到$2^{64}-1$(0到18,446,744,073,709,551,615或1万8千亿)  \\
        \texttt{u128}   & 0到$2^{128}-1$(0到大约$3.4*10^{38}$)                    \\
        \rowcolor{tablecolor} 
        \texttt{usize}  & 0到$2^{32}-1$或$2^{64}-1$         \\
    \end{tabular}
\end{table}

Rust的有符号整数使用两种互补的表示方法,使用和无符号类型相对应的位模式来表示一个包含正数和负数的范围(\hyperref[t3-4]{表3-4})。
\begin{table}[htbp]
    \centering
    \caption{Rust的有符号整数类型}
    \label{t3-4}
    \begin{tabular}{ll}
        \hline
        \textbf{类型}   &   \textbf{范围}   \\
        \hline
        \texttt{i8}     & $-2^{7}$到$2^{7}-1$(-128到127)   \\
        \rowcolor{tablecolor}
        \texttt{i16}    & $-2^{15}$到$2^{15}-1$(-32,768到32,767)  \\
        \texttt{i32}    & $-2^{31}$到$2^{31}-1$(-2,147,483,648到2,147,483,647)    \\
        \rowcolor{tablecolor}
        \texttt{i64}    & $-2^{63}$到$2^{63}-1$(-9,223,372,036,854,775,808到9,223,372,036,854,775,807)    \\
        \texttt{i128}   & $-2^{127}$到$2^{127-1}$(大约$-1.7\times10^{38}$到$+1.7\times10^{38}$) \\
        \rowcolor{tablecolor}
        \texttt{isize}  & $-2^{31}$到$2^{31}-1$,或者$-2^{63}$到$2^{63}-1$  \\
    \end{tabular}
\end{table}

Rust使用\texttt{u8}类型来表示一个字节的值。例如,从二进制文件或者套接字读取数据就会返回\texttt{u8}类型的数据流。

与C和C++不同,Rust区分了字符和数字类型:\texttt{char}不是\texttt{u8},也不是\texttt{u32}(尽管它是32位)。我们将会在“\hyperref[char]{字符}”这一节介绍Rust的\texttt{char}类型。

\texttt{usize}和\texttt{isize}类似于C和C++中的\texttt{size\_t}和\texttt{ptrdiff\_t}类型。它们的位数和目标机器上地址空间的位数相同:在32位架构上就是32位,在64位架构上就是64位。Rust要求数组索引为\texttt{usize}类型的值。数组或vector或其他任何含有多个元素的数据结构的长度都是\texttt{usize}类型。

Rust中的整数字面量可以有一个后缀来指示类型:\texttt{42u8}是一个\texttt{u8}类型的值,\texttt{1729isize}是一个\texttt{isize}类型的值。如果一个整数字面量没有类型后缀,Rust将会延迟决定它的类型,直到可以从它的使用中推断出它的类型:存储到一个已知类型的变量中、作为参数传递给一个参数类型已知的函数、和一个已知类型的值比较、以及类似的情况。如果到最后还是有很多类型可以满足,此时如果\texttt{i32}是其中一种可能,Rust将推断它为\texttt{i32}类型。否则,Rust会报歧义错误。

前缀\texttt{0x, 0o, 0b}分别表示十六进制、八进制、二进制字面量。

为了让长数字更更读,你可以在数字中间插入下划线。例如,你可以把最大的\texttt{u32}值写作\texttt{4\_294\_967\_295}。下划线放置的位置并不重要,所以你可以每四位插入一个下划线来把十六进制和二进制数字分组,例如\texttt{0xffff\_ffff}或者在最后插入下划线分隔类型后缀,例如\texttt{127\_u8}。\hyperref[t3-5]{表3-5}给出了一些整数字面量的例子。
\begin{table}[htbp]
    \centering
    \caption{整数字面量的例子}
    \label{t3-5}
    \begin{tabular}{lll}
        \hline
        \textbf{字面量} & \textbf{类型} & \textbf{十进制值} \\
        \hline
        \texttt{116i8}          & \texttt{i8}       &   116 \\
        \rowcolor{tablecolor}
        \texttt{0xcafeu32}      & \texttt{u32}      &   51966 \\
        \texttt{0b0010\_1010}   & 推断              &   42 \\
        \rowcolor{tablecolor}
        \texttt{0o106}          & 推断              &   70 \\
    \end{tabular}
\end{table}

尽管数值类型和\texttt{char}类型是不同的,Rust确实提供了\emph{字节字面量}:很像字符字面量的\texttt{u8}值:\texttt{b'X'}代表字符\texttt{X}的ASCII码值,但是是\texttt{u8}类型的值。例如,因为\texttt{A}的ASCII码值是65,字面量\texttt{b'A'}和\texttt{65u8}是等价的。只有ASCII字符可以出现在字节字面量中。

这里有一些不能用单个字符表示的字符,因为它们要么会导致歧异要么很难看出来。\hyperref[t3-6]{表3-6}中的字符只能用反斜杠转移的方式写出来。
\begin{table}[htbp]
    \centering
    \caption{需要转义的字符}
    \label{t3-6}
    \begin{tabular}{lll}
        \hline
        \textbf{字符}   &   \textbf{字节字面量} & \textbf{等价的数字值} \\
        \hline
        单引号,'   &   \texttt{b'\textbackslash''}      & 39u8 \\
        \rowcolor{tablecolor}
        反斜杠,\textbackslash &    \texttt{b'\textbackslash\textbackslash'} & 92u8 \\
        换行        &    \texttt{b'\textbackslash n'}    & 10u8 \\
        \rowcolor{tablecolor}
        回车        &   \texttt{b'\textbackslash r'}     & 13u8 \\
        制表符      &   \texttt{b'\textbackslash t'}     & 9u8 \\
    \end{tabular}
\end{table}

对于那些难以写出或看出的字符,你可以用它们的十六进制码代替。一个字节字面量的形式是\texttt{b'\textbackslash xHH'},其中\texttt{HH}是两个十六进制的数字,代表值是\texttt{HH}的字节。例如,你可以将ASCII的“escape”字符的字节字面量写作\texttt{b'\textbackslash x1b'},因为“escape”的ASCII码是27,也就是16进制的1B。因为字节字面量只是\texttt{u8}类型值的另一种表示方式,考虑使用数字字面量可能可读性会更强:只有当你想表示ASCII码时\texttt{b'\textbackslash x1b'}才会比\texttt{27}更有意义。

你可以将一种整数类型转换为另一种整数类型。我们将会在“\hyperref[cast]{类型转换}”这一节中介绍转换的原理,这里有一些例子:
\begin{minted}{Rust}
    assert_eq!(   10_i8  as u16,    10_u16); // in range
    assert_eq!( 2525_u16 as i16,  2525_i16); // in range

    assert_eq!(   -1_i16 as i32,    -1_i32); // 符号扩展
    assert_eq!(65535_u16 as i32, 65535_i32); // 0扩展

    // 转换一个超出目标类型范围的值
    // 等价于原值对2^N取模
    // N是目标类型的位数
    // 这有时也被称为“截断”
    assert_eq!( 1000_i16 as  u8,    232_u8);
    assert_eq!(65535_u32 as i16,     -1_i16);

    assert_eq!(   -1_i8  as u8,     255_u8);
    assert_eq!(  255_u8  as i8,      -1_i8);
\end{minted}

标准库提供一些整数的方法来进行操作。例如:
\begin{minted}{Rust}
    assert_eq!(2_u16.pow(4), 16);               // 求指数幂
    assert_eq!((-4_i32).abs(), 4);              // 求绝对值
    assert_eq!(0b101101_u8.count_ones(), 4);    // 位计数
\end{minted}

你可以在在线文档中找到这些。但是注意,文档中\texttt{i32}(原始类型)和模块导入的类型(搜索\texttt{std::i32})有不同的单独页面。

在实际编码时,你不需要像我们在这里一样写出类型后缀,因为上下文会自动推断出类型。当推断不出来时,错误信息可能会让你很惊讶。例如,下面的代码不能编译:
\begin{minted}{Rust}
    println!("{}", (-4).abs());
\end{minted}

Rust报错:
\begin{minted}{text}
    error: can't call method `abs` on ambiguous numeric type `{integer}`
\end{minted}

这可能有点迷惑:所有的整数类型都有\texttt{abs}方法,所以问题在哪呢?从技术角度来说,Rust需要在调用某个类型的方法之前知道这个值的精确类型。只有当所有的方法调用都被解析之后仍然存在歧义才会使用默认的\texttt{i32}类型,而在这里,在解析\texttt{abs}方法时就需要知道\texttt{-4}的类型,默认推导为\texttt{i32}的规则在此时不能生效。解决方法是指明类型,要么加上类型后缀,要么使用类型特定的函数:
\begin{minted}{Rust}
    println!("{}", (-4_i32).abs());
    println!("{}", i32::abs(-4));
\end{minted}

注意函数调用的优先级高于一元前缀运算符,所以当对负数调用方法时一定要小心。如果这个地方第一个表达式里\texttt{-4\_i32}两侧没有括号,\texttt{-4\_i32.abs()}将会对\texttt{4}调用\texttt{abs}方法,返回正数\texttt{4},然后求负数返回\texttt{-4}。

\subsection{Checked、Wrapping、Saturating、Overflowing算术}

当整数运算溢出时,如果是在debug模式下Rust会panic。在release模式下,运算结果会\emph{回环}:它会返回正确的值对结果类型能表示的范围取余之后的结果。(这两种情况下,溢出都不像在C和C++中一样是未定义行为)。

例如,下面的代码在debug模式下会panic:
\begin{minted}{Rust}
    let mut i = 1;
    loop {
        i *= 10;    // panic: 尝试乘到溢出
                    // (但只有在debug模式会panic!)
    }
\end{minted}

在release模式下,溢出时乘法会回环成负数,然后循环会无限执行。

如果默认行为不是你希望的结果,整数类型提供了一个方法让你指定想要做什么。例如,下面的代码在任何构建模式下都会panic:
\begin{minted}{Rust}
    let mut i: i32 = 1;
    loop {
        // panic: 乘法溢出(在任何构建模式下)
        i = i.checked_mul(10).expect("multiplication overflowed");
    }
\end{minted}

这些整数运算的方法可以被分为四个通用的类别:
\begin{itemize}
    \item \emph{Checked}操作返回一个结果的\texttt{Option}值:如果运算结果可以被结果类型正确表示就返回\texttt{Some(v)},否则返回\texttt{None}。例如:
    \begin{minted}{Rust}
    // 10和20的结果可以用u8表示。
    assert_eq!(10_u8).checked_add(20), Some(30));

    // 不幸的是,100和200的和不能用u8表示。
    assert_eq!(100_u8).checked_add(200), None);

    // 求和,如果溢出就panic。
    let sum = x.checked_add(y).unwrap();

    // 奇怪的是,在一种特定情况下,有符号除法也可能会导致溢出。
    // 一个有符号整数能表示-2^(n-1),但不能表示2^(n-1)。
    assert_eq!((-128_i8).checked_div(-1), None);
    \end{minted}

    \item \emph{Wrapping}操作返回正确的值对结果类型能表示的范围的余数:
    \begin{minted}{Rust}
    // 第一个积可以用u16来表示。
    // 第二个不能,因此我们得到250000对2^16取模。
    assert_eq!(100_u16.wrapping_mul(200), 20000);
    assert_eq!(500_u16.wrapping_mul(500), 53392);

    // 有符号数的操作可能会回环成负数。
    assert_eq!(500_i16.wrapping_mul(500), -12144);

    // 在移位操作中,移动的位数会回环到该类型的位数之内
    // 因此对16位的数字移动17位等于移动1位
    assert_eq!(5_i16.wrapping_shl(17), 10);
    \end{minted}
    正如解释的那样,这就是release模式下算术操作的行为。使用这种写法的好处是在所有的构建模式下代码的行为都一致。

    \item \emph{Saturating}操作会返回最接近正确结果的表示。换句话说,结果被“截断”到这个类型能表示的最大或最小值:
    \begin{minted}{Rust}
    assert_eq!(32760_i16.saturating_add(10), 32767);
    assert_eq!((-32760_i16).saturating_sub(10), -32768);
    \end{minted}
    没有饱和除法、取余、位移操作。

    \item \emph{Overflowing}操作返回一个tuple\texttt{(reulst, overflowed)},其中\texttt{result}是回环版本的方法返回的结果,而\texttt{overflowed}是一个指示是否发生溢出的\texttt{bool}值:
    \begin{minted}{Rust}
    assert_eq!(255_u8.overflowing_sub(2), (253, false));
    assert_eq!(255_u8.overflowing_add(2), (1, true));
    \end{minted}
    \texttt{overflowing\_shl}和\texttt{overflowing\_shr}稍微有些偏离这个模式:只有当移位距离恰好等于类型的位宽度时\texttt{overflowed}才为true。实际的移位距离等于要求的距离对位宽度取余后的结果:
    \begin{minted}{Rust}
    // 对`u16`来说移动17位太多了,17对16取余等于1。
    assert_eq!(5_u16.overflowing_shl(17), (10, true));
    \end{minted}
\end{itemize}

\hyperref[t3-7]{表3-7}中列出了以\texttt{checked\_},\texttt{wrapping\_},\texttt{saturating\_},\texttt{overflowing\_}为前缀的方法。
\begin{table}[htbp]
    \centering
    \caption{操作的名称}
    \label{t3-7}
    \begin{tabular}{lll}
        \hline
        \textbf{操作}   & \textbf{名称后缀} &   示例 \\
        \hline
        加法    &   \texttt{add}    & \texttt{100\_i8.checked\_add(27) == Some(127)}    \\
        \rowcolor{tablecolor}
        减法    &   \texttt{sub}    & \texttt{10\_u8.checked\_sub(11) == None}      \\
        乘法    &   \texttt{mul}    & \texttt{128\_u8.saturating\_mul(3) == 255}    \\
        \rowcolor{tablecolor}
        除法    &   \texttt{div}    & \texttt{64\_u16.wrapping\_div(8) == 8}    \\
        取余    &   \texttt{rem}    & \texttt{(-32768\_i16).wrapping\_rem(-1) == 0} \\
        \rowcolor{tablecolor}
        负数    &   \texttt{neg}    & \texttt{(-128\_i8).checked\_neg() == None} \\
        绝对值  &   \texttt{abs}    & \texttt{(-32768\_i16).wrapping\_abs() == -32768} \\
        \rowcolor{tablecolor}
        指数    &   \texttt{pow}    & \texttt{3\_u8.checked\_pow(4) == Some(81)} \\
        左移    &   \texttt{shl}    & \texttt{10\_u32.wrapping\_shl(34) == 40}  \\
        \rowcolor{tablecolor}
        右移    &   \texttt{shr}    & \texttt{40\_u64.wrapping\_shr(66) == 10}  \\
    \end{tabular}
\end{table}

\subsection{浮点数}
Rust提供IEEE的单精度和双精度浮点数。这两个类型还包括正无穷、负无穷、正0、负0和\emph{非数}值。(\hyperref[t3-8]{表3-8})

\begin{table}[htbp]
    \centering
    \caption{IEEE单精度和双精度浮点数类型}
    \label{t3-8}
    \begin{tabular}{lll}
        \hline
        \textbf{类型}   & \textbf{精度} &   \textbf{范围}   \\
        \hline
        \texttt{f32}    & IEEE单精度浮点数(至少6位十进制数字)    & 大约$-3.4\times10^{38}$到$+3.4\times10^{38}$    \\
        \rowcolor{tablecolor}
        \texttt{f64}    & IEEE双精度浮点数(至少15位十进制数字)    & 大约$-1.8\times10^{308}$到$1.8\times10^{308}$   \\
    \end{tabular}
\end{table}

Rust的\texttt{f32}和\texttt{f64}分别对应C和C++(如果实现支持IEEE浮点数的话)以及Java(总是使用IEEE浮点数)里的\texttt{float}和\texttt{double}类型。

浮点数字面量的一般形式如\hyperref[f3-1]{图3-1}。
\begin{figure}[htbp]
    \centering
    \includegraphics[width=0.8\textwidth]{../img/f3-1.png}
    \caption{浮点数字面量}
    \label{f3-1}
\end{figure}

整数部分之后的部分都是可选的,但小数部分、指数、或者类型后缀至少需要有一个,才能和整数字面量区分开。小数部分可以只有一个单独的小数点,因此\texttt{5.}是一个有效的浮点数。

如果一个浮点数字面量缺少类型后置,和处理整数一样,Rust会检查上下文来查看这个值是如何使用的。如果最后它发现两种浮点数类型都可以满足语义,那么它会默认选择\texttt{f64}。

为了实现类型推导,Rust把整数字面量和浮点数字面量区分为不同的种类:它从来不会把一个浮点数类型推断为整数类型,反之亦然。\hyperref[t3-9]{表3-9}展示了一下浮点数字面量的例子。

\begin{table}[htbp]
    \centering
    \caption{浮点数字面量的示例}
    \label{t3-9}
    \begin{tabular}{lll}
        \hline
        \textbf{字面量} & \textbf{类型} & \textbf{数值} \\
        \hline
        \texttt{-1.5625}    & 自动推断  & $-1\frac{9}{16}$  \\
        \rowcolor{tablecolor}
        \texttt{2.}         & 自动推断  & 2 \\
        \texttt{0.25}       & 自动推断  & $\frac{1}{4}$   \\
        \rowcolor{tablecolor}
        \texttt{1e4}        & 自动推断  & 10,000    \\
        \texttt{40f32}      & \texttt{f32}  & 40    \\
        \rowcolor{tablecolor}
        \texttt{9.109\_383\_56e-31f64} & \texttt{f64} & 大约是$9.10938356\times10^{-31}$ \\
    \end{tabular}
\end{table}

\texttt{f32}和\texttt{f64}类型还关联了IEEE要求的特殊常量值例如\texttt{INFINITY}、\texttt{NEG\_INFINITY}(负无穷)、\texttt{NAN}(非数值)、\texttt{MIN}和\texttt{MAX}(最小和最大的有限值):
\begin{minted}{Rust}
    assert!((-1. / f32::INFINITY).is_sign_negative());
    assert_eq!(-f32::MIN, f32::MAX);
\end{minted}
\texttt{f32}和\texttt{f64}类型提供了完整的数值计算的方法;例如,\texttt{2f64.sqrt()}是2的平凡根。还有一些示例:
\begin{minted}{Rust}
    assert_eq!(5f32.sqrt() * 5f32.sqrt(), 5.);  // 精确的5.0
    assert_eq!((-1.01f64).floor(), -2.0);
\end{minted}

再重复一次,方法调用的优先级高于前缀运算符,因此对负数调用方法时确保要用括号括起来。

\texttt{std::f32::consts}和\texttt{std::f64::consts}模块提供了常用的数学常数,例如\texttt{E}、\texttt{PI}、2的平方根。

当查找文档时,记得既有类型的文档,名称叫“\texttt{f32}(primitive type)”和“\texttt{f64}(primitive type)”,又有模块的文档,名称叫\texttt{std::f32}和\texttt{std::f64}。

和整数一样,在实际编码时通常你不需要写出浮点数字面量的类型后缀,但如果你要写,那么只需要指明变量和函数其中一个类型即可:
\begin{minted}{Rust}
    println!("{}", (2.0_f64).sqrt());
    println!("{}", f64::sqrt(2.0));
\end{minted}
和C和C++不同,Rust中几乎没有隐式类型转换。如果一个函数接收\texttt{f64}类型的参数,传递\texttt{i32}的值作为参数将是一个错误。事实上,Rust甚至不允许从\texttt{i16}到\texttt{i32}这样的隐式转换,尽管每一个\texttt{i16}值也都是一个合法的\texttt{i32}值。但你总是可以使用\texttt{as}运算符来进行\texttt{显式}转换:\texttt{i as f64},或者\texttt{x as i32}。

缺少隐式类型转换导致Rust的表达式可能会比C和C++中类似的表达式更加冗长。然而,隐式整数转换经常导致bug和安全漏洞。尤其是用来表示内存中某个东西的长度的整数,可能会导致意外的溢出。在我们的实践中,在Rust中显式写出类型转换可以提醒我们可能忽略的问题。

我们会在“\hyperref[cast]{类型转换}”一节中介绍转换的原理。

\section{布尔类型}

Rust的布尔类型\texttt{bool},只有两个值:\texttt{true}和\texttt{false}。比较运算符例如\texttt{==}和\texttt{<}会产生\texttt{bool}类型的结果:\texttt{2 < 5}的结果是\texttt{true}。

许多语言都很宽容,允许在需要布尔值的上下文中使用其他类型:C和C++隐式把字符、整数、浮点数和指针转换为布尔值,因此它们可以直接用作\texttt{if}或\texttt{while}语句的条件。Python还允许string、list、字典、甚至集合用作布尔值,如果不为空时视为true。然而Rust非常严格:像\texttt{if}和\texttt{while}这样的控制流的条件必须是\texttt{bool}表达式,短路求职运算符\texttt{\&\&}和\texttt{||}也是这样。你必须写\texttt{if x != 0 \{ ... \}},而不能写\texttt{if x \{ ... \}}。

Rust的\texttt{as}运算符可以把\texttt{bool}值转换为整数值:
\begin{minted}{Rust}
    assert_eq!(false as i32, 0);
    assert_eq!(true  as i32, 1);
\end{minted}

然而,\texttt{as}不能反过来把整数值转换为\texttt{bool}值。你必须显式写出比较运算例如\texttt{x != 0}。

尽管\texttt{bool}类型只需要单个比特来表示,Rust还是使用整个字节来表示\texttt{bool},因此你可以创建指向它的指针。

\section{字符}\label{char}
Rust的字符类型\texttt{char}代表一个单独的Unicode字符,是一个32位的值。

Rust使用\texttt{char}表示单个字符,但使用UTF-8编码字符串和文本流。因此,\texttt{String}表示它的文本是一个UTF-8字节序列,而不是字符的数组。

字符串字面量是被单括号包围的单个字符,例如\texttt{'8'}或\texttt{'!'}。你可以使用Unicode范围内的任何字符:\texttt{'錆'}是一个\texttt{char}字面量代表日语汉字中的\emph{sabi}(rust)。

和字节字面量一样,一些字符需要反斜杠转义(\hyperref[t3-10]{表3-10})。
\begin{table}[htbp]
    \centering
    \caption{需要反斜杠转义的字符}
    \label{t3-10}
    \begin{tabular}{lll}
        \hline
        \textbf{字符}   &   \textbf{Rust字符字面量} \\
        \hline
        单引号,'   &   \texttt{b'\textbackslash''} \\
        \rowcolor{tablecolor}
        反斜杠,\textbackslash &    \texttt{b'\textbackslash\textbackslash'} \\
        换行        &    \texttt{b'\textbackslash n'} \\
        \rowcolor{tablecolor}
        回车        &   \texttt{b'\textbackslash r'} \\
        制表符      &   \texttt{b'\textbackslash t'} \\
    \end{tabular}
\end{table}

如果你喜欢的话,你可以以十六进制的方式写出一个字符的Unicode编码:
\begin{itemize}
    \item 如果字符的码点在U+0000道U+007F之间(可以据此判断是否在ASCII字符集之中),那么你可以将字符写作\texttt{'\textbackslash xHH'},\texttt{HH}是一个两位的十六进制数字。例如,字符字面量\texttt{'*'}和\texttt{'\textbackslash x2A'}是等价的,因为字符\texttt{*}的码点是42,十六进制是2A。
    \item 你可以用\texttt{'\textbackslash u{HHHHHH}'}形式写出任何Unicode字符,\texttt{HHHHHH}是一个最长6位的十六进制数字,可以用下划线分隔。\footnote{译者注:此处原文中给了一个例子,但译者不知道该怎么打出卡纳达语里的字符,复制粘贴也不行,就省略了这个例子。}
\end{itemize}

一个\texttt{char}总是存储一个在0x0000到0xD7FF或0xE000到0x10FFFF之间的Unicode码点。一个\texttt{char}绝不会在两个范围之间(即0xD800到0xDFFF),也不会超出Unicode的编码空间(即大于0x10FFFF)。Rust使用类型系统和动态检查来确保\texttt{char}值总是在允许的范围内。

Rust永远不会进行\texttt{char}和其他任何类型之间的隐式转换。你可以使用\texttt{as}转换运算符来把\texttt{char}转换为整数类型,对于小于32位的类型,字符值的高位会被截断:\footnote{译者注:这个例子中也省略了卡纳达语字符相关的内容}
\begin{minted}{Rust}
    assert_eq!('*' as i32, 42);
\end{minted}

另外,\texttt{u8}是唯一可以用\texttt{as}运算符转换成\texttt{char}的整数类型:Rust只会对开销很低并且不可能失败的转换使用\texttt{as}运算符,但任何\texttt{u8}之外的整数类型都包含不是有效的Unicode码点的值,因此这些转换需要运行时检查。所以,标准库提供了函数\texttt{std::char::from\_u32}接受任何\texttt{u32}值,并返回\texttt{Option<char>}:如果\texttt{u32}的值不是合法的Unicode码点,\texttt{from\_u32}会返回\texttt{None};否则,它会返回\texttt{Some(c)},\texttt{c}就是作为转换结果的\texttt{char}。

标准库为字符类型提供了一些有用的方法,你可以在文档中的“\texttt{char}(primitive type)”和模块“\texttt{std::char}”的页面中查找这些方法。例如:
\begin{minted}{Rust}
    assert_eq!('*'.is_alphabetic(), false);
    assert_eq!('β'.is_alphabetic(), true);
    assert_eq!('8'.to\_digit(10), Some(8));
    assert_eq!(std::char::from_digit(2, 10), Some('2'));
\end{minted}

当然,单个字符显然没有字符串和文本流有趣。我们将会在“\hyperref[string]{字符串类型}”中介绍Rust的标准\texttt{String}类型和常用的文本处理操作。

\section{元组}
\emph{元组}是两个、或三个、四个、五个、……不同类型的值的组合。你可以将元组看作被逗号分隔和括号包围的元素序列。例如,\texttt{("Brazil", 1985)}是一个元组,它的第一个元素是一个静态分配的字符串,第二个元素是一个整数,它的类型是\texttt{(\&str, i32)}。给定一个元组值\texttt{t},你可以通过\texttt{t.0}、\texttt{t.1}、……来访问它的元素。

某种程度上,元组类似于数组:这两个类型都代表一系列有固定顺序的值。许多编程语言合并或结合了这两种概念,但在Rust中,它们是完全独立的。一方面,元组的每个元素可以拥有不同的类型,而数组的所有元素必须有相同的类型。另外,元组只允许常数索引,例如\texttt{t.4}。你不可能写\texttt{t.i}或者\texttt{t[i]}来获取第i个元素。

Rust代码中经常使用元组类型来返回多个值。例如,字符串切片中的\texttt{split\_at}方法:用于将一个字符串切分为两半并返回的函数,被声明为类似如下形式:
\begin{minted}{Rust}
    fn split_at(&self, mid: usize) -> (&str, &str);
\end{minted}

返回类型\texttt{(\&str, \&str)}是两个字符串切片组成的元组。你可以使用模式匹配语法来把返回的元素赋值给不同的变量:
\begin{minted}{Rust}
    let text = "I see the eigenvalue in thine eye";
    let (head, tail) = text.split_at(21);
    assert_eq!(head, "I see the eigenvalue ");
    assert_eq!(tail, "in thine eye");
\end{minted}

这比下边的等价代码可读性更强:
\begin{minted}{Rust}
    let text = "I see the eigenvalue in thine eye";
    let temp = text.split_at(21);
    let head = temp.0;
    let tail = temp.1;
    assert_eq!(head, "I see the eigenvalue ");
    assert_eq!(tail, "in thine eye");
\end{minted}

你也可以将元组视为一种极简的结构体类型。例如,在\hyperref[ch02]{第2章}的曼德勃罗集程序中,我们需要向函数传递要绘制的图片的宽度和高度。我们可以声明一个有\texttt{width}和\texttt{height}成员的结构体,但这么简单的事没有必要搞得这么复杂,因此我们用了一个元组:
\begin{minted}{Rust}
    /// 写入缓冲区`pixels`,它的大小由`bounds`给出,
    /// 写入的文件名是`filename`。
    fn write_image(filename: &str, pixels: &[u8], bounds: (usize, usize))
        -> Result<(), std::io::Error>
    { ... }
\end{minted}

参数\texttt{bounds}的类型是\texttt{(usize, usize)},一个有两个\texttt{usize}值得元组。诚然,我们可以直接使用单独的\texttt{width}和\texttt{height}参数,生成的机器代码也会完全相同。这么写的目的只是想表明,我们把图片的大小看成一个值,而不是两个值,使用元组的写法可以清晰的表现出这一点。

元组的另一个常见用法是0元组\texttt{()}。这通常被称为\emph{单元类型}因为它只有一个取值,也写作\texttt{()}。Rust在上下文要求某种类型,但没有有意义的值要传递的情况下使用单元类型。

例如,一个不返回值的函数的返回类型是\texttt{()}。标准库中的\texttt{std::mem::swap}函数没有有意义的返回值;它只是交换两个参数的值。\texttt{std::mem::swap}的声明如下:
\begin{minted}{Rust}
    fn swap<T>(x: &mut T, y: &mut T);
\end{minted}
\texttt{<T>}意味着\texttt{swap}是\emph{泛型}的:你可以将它用于任何类型\texttt{T}的引用。但签名中省略了\texttt{swap}的返回类型,这实际上是返回单元类型的缩写:
\begin{minted}{Rust}
    fn swap<T>(x: &mut T, y: &mut T) -> ();
\end{minted}

与此类似,我们之前提到的\texttt{write\_image}例子中返回值类型是\texttt{Result<(), std::io::Error>},这意味着如果出错时函数返回\texttt{std::io::Error}类型的值,如果成功时返回无值。

如果你想的话,你可以在元组的最后一个元素之后加上一个逗号:类型\texttt{(\&str, i32,)}和\texttt{(\&str, i32)}是等价的,\texttt{("Brazil", 1985,)}和\texttt{("Brazil", 1985)}也是。Rust允许任何逗号分隔的值列表最后加上一个额外的逗号:函数参数、数组、结构体和枚举定义,等等。这对人类来说可能看起来很奇怪,但当需要在最后添加或删除条目时会变得方便一些。

为了一致性,还有只包含单个值的元组。字面量\texttt{("lonely hearts",)}是一个包含单个字符串的元组,它的类型是\texttt{(\&str,)}。这里,最后的逗号是必须的,这是为了和单纯的用括号把表达式括起来相区分。

\section{指针类型}
Rust有几个表示内存地址的类型。

这是Rust和其他大多数有垃圾回收语言的不同之处。在Java中,如果\texttt{class Rectangle}包含一个字段\texttt{Vector2D upperLeft},那么\texttt{upperLeft}实际上是对另一个单独创建的\texttt{Vector2D}对象的引用。在Java中一个对象从来不会真的包含其他对象。

Rust里则不同。Rust被设计为最少分配内存。默认情况下Rust里是值嵌套的,值\texttt{((0, 0), (1440, 900))}被存储为四个相邻的整数。如果你把它赋值给一个局部便变量,那么你将得到一个4个证书大小的局部变量,不会在堆上分配任何内存。

这有助于提高内存效率,但它会导致当Rust程序需要指向其他值的指针时,必须显式使用指针类型。好消息是safe Rust里的指针类型有一些约束来保证不会出现未定义行为,因此Rust中的指针比C++中的更容易正确使用得多。

我们将会在这里讨论三种指针类型:引用、box、unsafe指针。

\subsection{引用}

一个\texttt{\&String}(读作“ref String”)类型的值是一个\texttt{String}值的引用,一个\texttt{\&i32}类型的值是一个\texttt{i32}值的引用,等等。

把引用想象成Rust的基本指针类型可以让我们更容易理解。在运行时,一个\texttt{i32}的引用是一个单独的机器字,里面存储的是指向的\texttt{i32}值的地址,可能在栈上也可能在堆上。表达式\texttt{\&x}产生一个\texttt{x}的引用;在Rust的术语中,我们称它\emph{借用了\texttt{x}的引用}。给定引用\texttt{r},表达式\texttt{*r}就是\texttt{r}指向的值。这些类似于C和C++中的\texttt{\&}和\texttt{*}运算符。类似于C指针,当引用离开作用域时并不会自动释放任何资源。

然而和C指针不同的是,Rust的引用永远不为空:在safe Rust中没有任何办法产生一个空引用。而且和C指针不同,Rust通过追踪值的所有权和生命周期,在编译期就可以杜绝悬垂指针、double free和指针失效的情况。

Rust的指针有以下两种类型:
\begin{flushleft}
    \emph{\texttt{\&T}}
\end{flushleft}

\hangafter 0
\hangindent 2em
\noindent
一个不可变的共享引用。你可以同时拥有同一个值的多个共享的引用,但它们都是只读的:修改它们指向的值是禁止的,就像C中的\texttt{const T*}一样。

\begin{flushleft}
    \emph{\texttt{\&mut T}}
\end{flushleft}

\hangafter 0
\hangindent 2em
\noindent
一个可变的、独占的应用。你可以读写它指向的值,类似于C中的\texttt{T*}。但只要这个引用存在,你不能再持有任何这个值的其他任何类型的引用。事实上,这时候你唯一可以访问这个值的方法就是通过这个可变引用。

Rust使用这种方式来区分共享和可变的引用,以此来强制执行“单个写者或多个读者”规则:要么你可以读写值,要么它只能被任何数量的读者共享。这种分隔由编译器检查来强制执行,它是Rust安全保证的核心。\hyperref[ch05]{第5章}解释了Rust中使用安全引用的规则。

\subsection{Box}

最简单的在堆上分配内存的方式是使用\texttt{Box::new}:
\begin{minted}{Rust}
    let t = (12, "eggs");
    let b = Box::new(t);    // 在堆上分配一个元组
\end{minted}

\texttt{t}的类型是\texttt{(i32, \&str)},因此\texttt{b}的类型是\texttt{Box<(i32, \&str)>}。调用\texttt{Box::new}会在堆上分配足够的内存来存储元组。当\texttt{b}离开作用域时,内存会被立即释放,除非\texttt{b}被\texttt{move}了——例如被返回了。move对Rust处理堆上分配的值的方式至关重要,我们将在\hyperref[ch04]{第4章}详解介绍这一点。

\subsection{原始指针}

Rust也有原始指针类型\texttt{*mut T}和\texttt{*const T}。原始指针类似于C++中的指针。使用原始指针是不安全的,因为Rust无法追踪它指向的到底是什么。例如,原始指针可能是空、或者可能指向被释放的内存、或者现在指向一个和之前不同类型的值。所有C++中经典的指针错误都有可能。

然而,你只能在\texttt{unsafe}块中解引用原始指针。\texttt{unsafe}块是Rust中的可选机制,它的安全性取决于你自己。如果你的代码没有\texttt{unsafe}块(或者有\texttt{unsafe}块但里面的代码都是完全正确的),那么整本书中强调的安全性保证都适用。细节见\hyperref[ch22]{第22章}。


    \chapter{所有权与move}\label{ch04}

当提到内存管理,我们希望编程语言能够具备以下两个特点:
\begin{itemize}
    \item 我们希望内存能在我们想要释放的时候被及时释放。这样我们可以控制程序的内存消耗。
    \item 我们永远不希望使用一个指向已经被释放的对象的指针。这会导致未定义行为,进而导致崩溃和安全漏洞。
\end{itemize}

但这两点看起来似乎是相互矛盾的:释放一个还有指针指向的对象的内存必定会导致悬垂指针。几乎所有的主流编程语言都属于两个阵营之一,取决于它们放弃了哪一点:
\begin{itemize}
    \item “安全优先”的阵营使用垃圾回收来管理内存,自动释放那些没有指针指向的对象。这种做法通过将对象一直保持到没有指针指向来避免悬垂指针。几乎所有的现代语言都落入了这个阵营,包括Python、JavaScript、Ruby、Java、C\#、Haskell。

    但依赖垃圾回收意味着放弃控制对象被回收的精确时间。通常来说,垃圾收集器令人讨厌,并且理解为什么内存没有如你所料的被释放可能会是一个挑战。

    \item “控制优先”的阵营让你自己负责释放内存。你的程序的内存消耗完全控制在你手中,但如何避免悬垂指针成了你最大的问题。C和C++是这个阵营里仅有的主流语言。

    如果你从没犯过错,那说明你很厉害。但证据表明,你最终还是会犯错。指针的错误使用一直都是那些被报导的安全问题的罪魁祸首。
\end{itemize}

Rust旨在同时保证安全和性能,因此这两种阵营都是不可接受的。但如果兼顾两者很简单的话,早就有人做出来了。要想兼顾两者,必须从根本上作出改变。

Rust以一种令人惊讶的方式打破了这个死锁:严格限制程序使用指针的方法。这一章和接下来将专注于解释这些限制和为什么它们能解决问题。目前,只能说你常用的一些程序结构可能不符合这些规则,你可能需要寻找替代方案。但这些限制的最终效果是给这种混乱带来了足够的秩序,以允许Rust在编译期检查你的程序是否能避免内存安全错误:悬垂指针、两次释放、使用未初始化的内存等。在运行时,你的指针只是简单的地址,就像在C和C++中一样。不同的是你的代码已经被证明是安全的。

这些规则也为Rust实现安全的并发编程奠定了基础。使用Rust精心设计的线程原语,保证内存安全的规则也可以用于保证你的代码可以避免数据竞争。Rust程序中的一个bug不可能导致一个线程破坏另一个线程的数据进而导致在不相干的地方出现很难复现的错误。多线程代码中的不确定行为被那些专为它设计的特性——自选锁、消息通道、院子类型等完全隔离,不会出现在正常的内存访问中。C和C++中的多线程代码臭名昭著,但Rust漂亮的解决了它。

即使有这些限制,你会发现它仍然可以足够灵活地处理几乎所有的任务,而它可以消除内存管理和并发bug的优势将证明你需要改变你需要对自己的风格进行调整。这是Rust最大的赌注,也是它的核心和成功之处。这本书的作者们看好Rust,正是因为我们在C和C++方面有丰富的经验。对我们来说,遵守Rust的规则太容易了。\footnote{译者注:此处原文:For us, Rust's deal is a no-brainer.}

Rust的规则可能和你在其他编程语言中看到的不同。了解怎么和它们一起工作并利用它们的优势,在我们看来是学习Rust的核心挑战。在这一章中,我们将首先展示相同的潜在问题如何在其他语言中导致问题,以此来深入了解Rust规则背后的逻辑和意图。之后,我们将详细解释Rust的规则、从概念和机制层面探究所有权的含义、如何在各种场景下追踪所有权的变化、以及一些为了提供更大的灵活性而打破这些规则的类型。

\section{所有权}

如果你读过C或C++的代码,你可能会看到过有注释说一个某个类的实例\emph{拥有}某些它指向的其他对象。这一般意味着有所有权的对将决定何时释放被拥有的对象:当拥有者被销毁时,它会销毁所有它拥有的对象。

例如,假设你写了如下C++代码:
\begin{minted}{Rust}
    std::string s = "frayed knot";
\end{minted}

字符串\texttt{s}在内存中的表示通常如\hyperref[f4-1][图4-1]所示:
\begin{figure}
    \centering
    \includegraphics[width=0.8\textwidth]{../img/f4-1.png}
    \caption{一个栈上的C++ \texttt{std::string},指向它在堆上分配的内存}
    \label{f4-1}
\end{figure}

这里,实际上\texttt{std::string}对象本身总是只有3个字长,包括一个指向堆上分配的缓冲区的指针、缓冲区的最大容量(也就是在不重新分配缓冲区的情况下,能存储的最大文本长度),和已经持有的文本的长度。这些都是\texttt{std::string}的私有字段,使用者不能访问。

一个\texttt{std::string}拥有它的缓冲区,当程序销毁string时,它的析构函数会释放缓冲区。以前,一些C++库在多个\texttt{std::string}值之间共享单个缓冲区,使用一个引用计数来决定缓冲区什么时候应该被释放。较新版本的C++标准有效地排除了这种表示,所有现代的C++库都是用上图中的方式。

在这些情况下,人们普遍认为,尽管其他代码创建被拥有内存的指针是没问题的,但代码有责任确保在所有者决定销毁它拥有的对象之前所有的指针都已消失。你可以创建一个指向\texttt{std::string}的缓冲区的指针,但当string被销毁后,你的指针就无效了,你必须自己保证不再使用它。拥有者决定所拥有对象的生命周期,所有其他的对象必须尊重它的决定。

我们在这里使用\texttt{std::string}做为例子展示了C++中的所有权是什么样子的:它只是一个标准库普遍遵守的规范,然而即使语言鼓励你也遵守相似的实践,但如何设计你自己的类型最终还是取决于你。

然而在Rust中,所有权的概念被内建在语言之中,并且通过编译期检查确保强制执行。每一个值都只有一个决定它生命周期的所有者。当所有者被释放——Rust中的术语叫\emph{dropped}——它拥有的值也会被dropped。这些规则意味着你可以通过检查代码很容易的查明某个值的生命周期,并给你系统级语言应有的控制生命周期的能力。

一个变量拥有它的值。当控制流离开了变量声明的语法快,变量会被drop,因此它的值也会随之一起drop。例如:
\begin{minted}{Rust}
    fn print_padovan() {
        let mut padovan = vec![1,1,1];  // 在这里分配
        for i in 3..10 {
            let next = padovan[i-3] + padovan[i-2];
            padovan.push(next);
        }
        println!("P(1..10) = {:?}", padovan);
    }
\end{minted}

变量\texttt{padovan}的类型是\texttt{Vec<i32>},一个32位整数的vector。在内存中,\texttt{padovan}看起来将类似于\hyperref[f4-2]{图4-2}。

\begin{figure}[htbp]
    \centering
    \includegraphics[width=0.8\textwidth]{../img/f4-2.png}
    \caption{栈上的\texttt{Vec<i32>},指向它在堆上的缓冲区}
    \label{f4-2}
\end{figure}

这和我们之前展示的C++的\texttt{std::string}非常像,除了缓冲区里的元素是32位整数,而不是字符。注意存储\texttt{padovan}的指针、容量和长度的字都在\texttt{print\_padovan}函数的栈帧中,只有vector的缓冲区是在堆上分配的。

和之前展示的string \texttt{s}一样,vector拥有它用来存储元素的缓冲区。当变量\texttt{padovan}在函数结尾处离开作用域时,程序会drop这个vector。因为vector拥有它的缓冲区,缓冲区也会随之drop。

Rust的\texttt{Box}类型是另一个所有权的例子。一个\texttt{Box<T>}是一个指向存储在堆上的类型\texttt{T}的值的指针,调用\texttt{Box::new(v)}会在堆上分配一些空间,把值\texttt{v}移动进去,然后返回一个\texttt{Box}指向堆上的空间。因为一个\texttt{Box}拥有它所指向的空间,当\texttt{Box}被drop的时候,堆上的空间也会被释放。

例如,你可以像这样在堆上分配一个元组:
\begin{minted}{Rust}
    {
        let point = Box::new((0.625, 0.5));     // point在这里分配
        let label = format!("{:?}", point);     // label在这里分配
        assert_eq!(label, "(0.625, 0.5)");
    }                                           // point和label都在这里drop
\end{minted}

当程序调用\texttt{Box::new}时,它会在堆上为一个有两个\texttt{f64}值的元组分配空间,把它的参数\texttt{(0.625, 0.5)}移动进去,然后返回一个指向它的指针。当控制流到达\texttt{assert\_eq!}的调用时,栈帧如\hyperref[f4-3]{图4-3}所示。

\begin{figure}[htbp]
    \centering
    \includegraphics[width=0.8\textwidth]{../img/f4-3.png}
    \caption{两个本地变量,每个都拥有堆上的一块内存}
    \label{f4-3}
\end{figure}

栈帧本身存储了变量\texttt{point}和\texttt{label},每一个变量都指向自己拥有的堆上的内存。当它们drop时,它们拥有的内存也随之释放。

类似于变量拥有它们的值一样,结构体拥有它们的字段,元组、数组、vector拥有它们的元素。

\begin{minted}{Rust}
    struct Person { name: String, birth: i32 }
    let mut composers = Vec::new();
    composers.push(Person { name: "Palestrina".to_string(),
                            birth: 1525 });
    composers.push(Person { name: "Dowland".to_string(),
                            birth: 1563 });
    composers.push(Person { name: "Lully".to_string(),
                            birth: 1632 });
    for composer in &composers {
        println!("{}, born {}", composer.name, composer.birth);
    }
\end{minted}

这里,\texttt{composers}是一个\texttt{Vec<Person>},一个结构体的vector,每个结构体有一个字符串和数字。在内存中,\texttt{composers}的最终结果如\hyperref[f4-4]{图4-4}所示。

\begin{figure}[htbp]
    \centering
    \includegraphics[width=0.8\textwidth]{../img/f4-4.png}
    \caption{一个更复杂的所有权树}
    \label{f4-4}
\end{figure}

这里有很多的所有权关系,但每一个都很直观:\texttt{composers}拥有一个vector,vector拥有它的元素,每一个元素是一个\texttt{Person}结构体;每个结构体拥有它的字段;其中的字符串字段拥有它的文本。当控制流离开了\texttt{composers}声明的作用域,程序会drop它的值,同时drop它拥有的所有内容。如果这里还有其他类型的集合,例如\texttt{HashMap}、\texttt{BTreeSet},那么过程也是一样的。

到这里,让我们退后一步并思考我们到目前为止展示的所有权关系。每个值都只有一个所有者,这样很容易决定什么时候drop这个值。但单个值可能拥有很多其他值:例如,vector \texttt{composers}拥有它的所有元素。这些元素也可能反过来拥有其他值:\texttt{composers}的每个元素拥有一个字符串,字符串又拥有它的文本。

所有者和它们拥有的值组成了\emph{树}:你的拥有者是你的父结点,你拥有的值是你的孩子结点。每棵树的根结点是一个变量;当这个变量离开作用域时,整个树都会随之销毁。我们可以在\texttt{composers}的图中看到这样一棵所有权的树:它不是搜索树数据结构意义上的“树”、也不是DOM元素组成的HTML文档树。相反,我们有一个由混合类型构建的树,Rust的单一所有者规则禁止任何可能使布局变得比树更复杂的连接操作。Rust程序中的每个值都是树中的一个结点,树的根就是变量。

Rust程序通常完全不会像C和C++程序中使用\texttt{free}和\texttt{delete}一样显式drop值。Rust中drop值的方式是将它从所有权树移除:当离开作用域时、或者从vector中删除元素时、或者类似的情况。这时,Rust保证值会和它拥有的所有值一起被drop掉。

在某种意义上,Rust不如其他语言强大:每个其他的编程语言都允许你在对象之间构建任意的关系图,这些对象以你认为合适的方式互相指向。但正因为Rust不够强大,所以它才可以对你的程序进行更强大的分析。Rust的安全保证可以实现的原因就是你的代码中可能出现的所有权关系更加容易处理。这是我们之前提到的Rust的“激进赌注”的一部分:Rust声称,在实践中,解决问题时通常有足够的灵活性来保证至少有一些完美的解决方案可以在语言强加的限制范围内实现。

也就是说,我们到目前为止解释的所有权的概念太过死板以至于很难使用。Rust在以下几个方面扩展了这个简单的想法:
\begin{itemize}
    \item 你可以将值从一个所有者移动到另一个所有者。这允许你构建、更改、拆除所有权树。
    \item 很简单的类型例如整数、浮点数和字符被所有权规则排除在外。它们被称为\texttt{Copy}类型。
    \item 标准库提供了引用计数的指针类型\texttt{Rc}和\texttt{Arc},它们允许值在一定的限制下可以有多个所有者。
    \item 你可以“借用一个值的引用”,引用是生命周期受限的非占有的指针。
\end{itemize}

这些策略中的每一条都改善了所有权模型的灵活性,同时仍然坚持Rust的承诺。我们将依次介绍它们,引用将在下一章介绍。

\section{move}
在Rust里对于大多数类型,赋值给变量、把值传给函数、或者从函数返回值并不会拷贝这个值:它们只会\emph{move}它。源对象放弃了值的所有权,把所有权转移给了目的对象,同时源对象变为未初始化的状态;此时目的对象控制值的生命周期。Rust程序一次一个值、一次一个move的构建和拆除复杂的结构。

你可能会很惊讶Rust改变了这些基础操作的含义。确实赋值操作很早之前就已经有了明确的含义。然而,如果你仔细观察过不同的语言是怎么处理赋值操作的,你就会发现不同语言的处理方式有很大差别。这种差别也让我们能更容易的看出Rust的选择的含义和结果。

考虑下面的Python代码:
\begin{minted}{python}
    s = ['udon', 'ramen', 'soba']
    t = s
    u = s
\end{minted}

每一个Python对象都有一个引用计数,用来追踪当前有多少个值指向它。因此在对\texttt{s}赋值以后,程序的状态如\hyperref[f4-5]{图4-5}所示(注意有一些内容被省略了)。

\begin{figure}[htbp]
    \centering
    \includegraphics[width=0.9\textwidth]{../img/f4-5.png}
    \caption{Python如何在内存中表示一个字符串的列表}
    \label{f4-5}
\end{figure}

因为只有\texttt{s}指向列表,所以列表的引用计数是1;因为列表是唯一指向那些字符串的对象,所以每个字符串的引用计数也是1。

当程序执行到\texttt{t}和\texttt{u}的赋值时会发生什么?Python把赋值操作简单实现为让目标变量也指向源变量指向的对象,然后增加对象的引用计数。因此,这段程序的最终状态如\hyperref[f4-6]{图4-6}所示:
\begin{figure}[htbp]
    \centering
    \includegraphics[width=0.8\textwidth]{../img/f4-6.png}
    \caption{在Python里把\texttt{s}赋值给\texttt{t}和\texttt{u}的结果}
    \label{f4-6}
\end{figure}

Python拷贝了\texttt{s}的指针,并赋给了\texttt{t}和\texttt{u},然后把列表的引用计数更新为3。Python中的赋值开销很低,但因为它创建了新的指向对象的引用,我们必须维护引用计数来知道我们什么时候可以释放值。

现在考虑下面类似的C++代码:
\begin{minted}{C++}
    using namespace std;
    vector<string> s = { "udon", "ramen", "soba" };
    vector<string> t = s;
    vector<string> u = s;
\end{minted}

一开始\texttt{s}的值在内存中如\hyperref[f4-7]{图4-7}所示。

\begin{figure}[htbp]
    \centering
    \includegraphics[width=0.8\textwidth]{../img/f4-7.png}
    \caption{C++里一个字符串的vector在内存中的表示}
    \label{f4-7}
\end{figure}

当把\texttt{s}赋值给\texttt{t}和\texttt{u}时会发生什么呢?在C++里赋值一个\texttt{std::vector}会产生一份这个vector的拷贝;\texttt{std::string}的行为类似。因此当程序到达末尾时,它实际上有3个vector和9个字符串(\hyperref[f4-8]{图4-8})。

\begin{figure}[htbp]
    \centering
    \includegraphics[width=0.8\textwidth]{../img/f4-8.png}
    \caption{在C++里把\texttt{s}赋值给\texttt{t}和\texttt{u}的结果}
    \label{f4-8}
\end{figure}

根据值的不同,C++里的赋值可能会消耗任意数量的内存和处理器时间。然而,它的优势是,程序可以很容易的决定何时释放这些内存:当变量离开作用域时,所有这里分配的内存都会被自动释放。

某种意义上,C++和Python选择了相反的策略:Python里赋值操作开销很小,但引用计数(通用一点的说法,垃圾回收)开销很大。C++保持了内存的所有权都很清楚,但赋值时会执行对象的深拷贝导致开销很大。C++程序员通常不太热衷于这种选择:深拷贝可能开销很大,通常会有更好的替代方法。

因此Rust的中的类似程序会怎么做呢?代码如下:
\begin{minted}{Rust}
    let s = vec!["udon".to_string(), "ramen".to_string, "soba".to_string()];
    let t = s;
    let u = s;
\end{minted}

类似于C和C++,Rust把字符串字面量例如\texttt{"udon"}存储在只读内存中,因此,为了更清楚地与C++和Python的例子进行对比,我们调用了\texttt{to\_string}来获得在堆上分配的\texttt{String}值。

在\texttt{s}的初始化之后,因为Rust和C++有相似的vector和string表示,所以看起来和C++中的情况很像(\hyperref[f4-9]{图4-9})。

\begin{figure}[htbp]
    \centering
    \includegraphics[width=0.8\textwidth]{../img/f4-9.png}
    \caption{Rust中一个字符串的vector在内存中的表示}
    \label{f4-9}
\end{figure}

但回想一下,Rust里大多数类型的赋值操作都是把值从源对象\emph{移动}到目的对象,然后源对象变为未初始化的状态。因此\texttt{t}初始化完之后,程序的内存状态如\hyperref[f4-10]{图4-10}所示。

\begin{figure}[htbp]
    \centering
    \includegraphics[width=0.8\textwidth]{../img/f4-10.png}
    \caption{Rust里把\texttt{s}赋值给\texttt{t}之后的结果}
    \label{f4-10}
\end{figure}

这里发生了什么?赋值语句\texttt{let t = s;}把vector的三个字段从\texttt{s}移动到了\texttt{t};现在\texttt{t}拥有了这个vector。vector的元素则仍待在原来的位置,string的位置也没有发生变化。每一个值都只有一个所有者,尽管所有者已经变了。不需要调整引用计数,并且编译器现在把\texttt{s}视作未初始化的状态。

因此当我们到达\texttt{let u = s;}时会发生什么呢?者将会把\texttt{s}的值赋给\texttt{u}。Rust禁止使用未初始化的值,所以编译器会报如下错误:
\begin{minted}{text}
    error[E0382]: use of moved value: `s`
      |
    7 |     let s = vec!["udon".to_string(), "ramen".to_string(), "soba".to_string()];
      |         - move occurs because `s` has type `Vec<String>`,
      |           which does not implement the `Copy` trait
    8 |     let t = s;
      |             - value moved here
    9 |     let u = s;
      |             ^ value used after move
\end{minted}

考虑Rust在这里使用move的结果。类似于Python,赋值操作开销很小,程序简单的把vector的三个字长的头部从一个点移动到了另一个点。但和C++类似,所有权总是很清晰:程序不需要引用计数或者垃圾回收来判断什么时候释放vector的元素和string的内容。

你为此付出的代价是如果你想要拷贝你必须显式写出。如果你想要最后和C++程序一样的状态,也就是每个变量都有独立的拷贝,那你必须调用vector的\texttt{clone}方法,它会对vector和它的元素执行深拷贝:
\begin{minted}{Rust}
    let s = vec!["udon".to_string(), "ramen".to_string(), "soba".to_string()];
    let t = s.clone();
    let u = s.clone();
\end{minted}

你也可以通过Rust引用计数指针类型复现Python代码的行为,我们将在“\hyperref[rc]{Rc和Arc:共享所有权}”这一节中简要介绍这一点。

\subsection{更多move的操作}

在我们上面展示的初始化例子中,都是在使用\texttt{let}语句引入变量的同时把值赋给它们。复制给一个变量将与此有细微的不同,如果你把值移动进一个已经被初始化的变量,Rust会drop变量之前的值。例如:
\begin{minted}{Rust}
    let mut s = "Govinda".to_string();
    s = "Siddhartha".to_string();   // 值"Govinda"在这里drop
\end{minted}

在这段代码中,当程序把\texttt{"Siddhartha"}赋值给\texttt{s}时,它之前的值\texttt{"Govinda"}首先被drop掉。但考虑下面的代码:
\begin{minted}{Rust}
    let mut s = "Govinda".to_string();
    let t = s;
    s = "Siddhartha".to_string();   // 这里不会drop任何内容
\end{minted}

这一次,\texttt{t}拿走了\texttt{s}中原本的字符串的所有权,因此当我们给\texttt{s}赋值时,它是未初始化的。在这种场景下,不会发生drop。

我们在这里使用初始化和赋值的例子是因为它们足够简单,但Rust在几乎场景下都使用move。向函数传参会把所有权移动给函数的参数;从函数返回值会把所有权移动给调用者;创建一个元组会把值移动给元组,等等。

你现在可能对我们之前章节给出的例子中到底发生了什么有了更深入的理解。例如,当我们构建作曲家的vector时,我们写了:
\begin{minted}{Rust}
    struct Person { name: String, birth: i32 }

    let mut composers = Vec::new();
    composers.push(Person { name: "Palestrina".to_string(), 
                            birth: 1525 });
\end{minted}

这段代码展示了除了初始化和赋值之外,move发生的几个场景:
\begin{flushleft}
    \emph{从函数返回值}
\end{flushleft}

\hangafter 0
\hangindent 2em
\noindent
\texttt{Vec::new()}的调用会创建一个新的vector并返回,返回的并不是指向vector的指针,而是vector本身:它的所有权从\texttt{Vec::new}移动到了变量\texttt{composers}。类似的,\texttt{to\_string}调用返回了一个新的\texttt{String}实例。

\begin{flushleft}
    \emph{构造新的值}
\end{flushleft}

\hangafter 0
\hangindent 2em
\noindent
新的\texttt{Person}结构体的\texttt{name}字段被\texttt{to\_string}的返回值初始化。结构体获得了这个字符串的所有权。

\begin{flushleft}
    \emph{向函数传递值}
\end{flushleft}

\hangafter 0
\hangindent 2em
\noindent
整个\texttt{Person}结构体,而不是指向它的指针,被传递给vector的\texttt{push}方法,这个方法将值移动到了结构体的尾部。vector获得了\texttt{Person}的所有权,因此也变成了name \texttt{String}的间接所有者。

像这样移动值可能听起来并不是很高效,但有两件事需要记住。第一,move只作用于恰当的值,而不作用于它们拥有的堆存储。对于vector和string来说,\emph{恰当的值}是它们三个字长的头部,潜在的很多元素的数组和文本缓冲区仍然停留在堆中原本的位置。第二,Rust编译器的代码生成部分擅长“看穿”所有这些move;在实践中,机器码通常会直接把值存储到它属于的位置。

\subsection{move和控制流}
之前的例子中的控制流都很简单,move会如何影响更复杂的代码呢?通用的原则是,如果一个变量的值被移动走并且从此之后没有再被赋予一个新的值,那么它被认为是未初始化的。例如,如果一个变量在\texttt{if}表达式的条件判断之后还是有值的,那我们在两个分支中都可以使用它:
\begin{minted}{Rust}
    let x = vec![10, 20, 30];
    if c {
        f(x);   // ... 在这里移动x的值是ok的
    } else {
        g(x);   // ... 在这里移动x的值也是ok的
    }
    h(x);   // 错误:如何任何一个分支使用了x,那么x在此处将是未初始化的
\end{minted}

出于类似的原因,在循环里移动一个变量的值是禁止的:
\begin{minted}{Rust}
    let x = vec![10, 20, 30];
    while f() {
        g(x);   // 错误:x会在第一次迭代时被移动
                // 第二次迭代时就是未初始化状态 
    }
\end{minted}

也就是说,我们需要在每次迭代里都重新赋予它一个新值:
\begin{minted}{Rust}
    let mut x = vec![10, 20, 30];
    while f() {
        g(x);       // 移动x的值
        x = h();    // 给x一个新值
    }
\end{minted}




















    \chapter{引用}\label{ch05}

\emph{Libraries cannot provide new inabilities.}

\begin{flushright}
——Mark Miller
\end{flushright}

我们至今为止见过的所有指针类型——简单的\texttt{Box<T>}堆指针、\texttt{String}和\texttt{Vec}内部的指针都拥有值:当所有者被drop时,指针指向的值也会随之消失。Rust还有非拥有指针类型称为\emph{引用},引用对指向的值的生命周期没有影响。

事实上,正相反,引用绝不应该比它们指向的值活的更长。你必须在你的代码中明确表明,引用的生命寿命比它指向的值更短。为了强调这一点,Rust将创建某个值的引用称为\emph{借用}值:你最终必须把你借走的还给它的所有者。

如果你在读到“你必须在你的代码中明确表明”时感到一丝怀疑,那说明你很优秀。引用自身并没有什么特殊的——本质上,它们只是地址。但保证它们安全的规则是Rust独有的,你以前不可能看到过类似的。尽管这些规则是Rust里最难掌握的部分,但它们能防止的经典的、日常的bug的范围之广令人惊讶,它们对多线程的影响也正在显现。这也是Rust的赌注。

这一章中,我们将讨论Rust中的引用如何工作,展示引用、函数和自定义类型如何包含生命周期信息来保证它们被安全使用,阐释它怎么能在编译期、不引入运行时开销的同时防止常见的bug。

\section{值的引用}

举个例子,假设我们要为文艺复兴时期优秀的艺术家和他们的著名作品建一个表格。Rust的标准库包含一个哈希表类型,所以我们可以像这样定义我们的类型:
\begin{minted}{Rust}
    use std::collections::HashMap;

    type Table = HashMap<String, Vec<String>>;
\end{minted}

换句话说,这是一个把\texttt{String}值映射到\texttt{Vec<String>}值的哈希表,它把艺术家的名字关联到它们的作品的名字。你可以使用\texttt{for}循环来第迭代\texttt{HashMap}的条目,因此我们可以写一个函数打印出一个\texttt{Table}:
\begin{minted}{Rust}
    fn show(table: Table) {
        for (artist, works) in table {
            println!("works by {}:", artist);
            for work in works {
                println!("  {}", work);
            }
        }
    }
\end{minted}

构造和打印表格都很直观:
\begin{minted}{Rust}
    fn main() {
        let mut table = Table::new();
        table.insert("Gesualdo".to_string(),
                     vec!["many madrigals".to_string(),
                          "Tenebrae Responsoria".to_string()]);
        table.insert("Caravaggio".to_string(),
                     vec!["The Musicians".to_string(),
                          "The Calling of St. Matthew".to_string()]);
        table.insert("Cellini".to_string(),
                     vec!["Perseus with the head of Medusa".to_string(),
                          "a salt cellar".to_string()]);
        show(table);
    }
\end{minted}

它也能正常工作:
\begin{minted}{text}
    $ cargo run
         Running `/home/jimb/rust/book/fragments/target/debug/fragments`
    works by Gesualdo:
      many madrigals
      Tenebrae Responsoria
    works by Cellini:
      Perseus with the head of Medusa
      a salt cellar
    works by Caravaggio:
      The Musicians
      The Calling of St. Matthew
\end{minted}

但如果你阅读过上一章中有关move的小节,你就会发现\texttt{show}的定义有一些问题。首先,\texttt{HashMap}不是\texttt{Copy}类型——它不可能是,因为它持有动态分配的表格。因此当程序调用\texttt{show(table)}时,整个结构都被移动到函数里,变量\texttt{table}将变为未初始化。(迭代它时没有特定的顺序,你可能会得到一个不同的顺序,不用担心)如果调用者代码尝试继续使用\texttt{table},它会遇到问题:
\begin{minted}{Rust}
    ...
    show(table);
    assert_eq!(table["Gesualdo"][0], "many madrigals");
\end{minted}

Rust会报错\texttt{table}不再可用:
\begin{minted}{text}
    error: borrow of moved value: `table`
       |
    20 |     let mut table = Table::new();
       |         --------- move occurs because `table` has type
       |                   `HashMap<String, Vec<String>>`,
       |                   which does not implement the `Copy` trait
    ...
    31 |     show(table);
       |          ----- value moved here
    32 |     assert_eq!(table["Gesualdo"][0], "many madrigals");
       |                ^^^^^ value borrowed here after move
\end{minted}

事实上,如果我们仔细查看\texttt{show}的定义,会发现外层的\texttt{for}循环获取了哈希表的所有权然后完全消费了它,内层的\texttt{for}循环对每一个vector做了同样的事(我们之前已经在“liberté, égalité, fraternité”的例子中见过这种行为了)。因为move语义,我们仅仅是为了打印它就已经完全销毁了整个结构体。感谢你,Rust!

正确的处理方式是使用引用。引用让你可以访问一个值,同时不影响它的所有权。引用有两种:
\begin{itemize}
    \item \emph{共享引用}让你能读取但不能修改被引用的值。然而,你可以同时持有多个共享引用。表达式\texttt{\&e}返回一个指向\texttt{e}的值的共享引用,如果\texttt{e}的类型是\texttt{T},那么\texttt{\&e}的类型就是\texttt{\&T},读作“ref \texttt{T}”。共享引用是\texttt{Copy}类型。
    \item 如果你有一个值的\emph{可变引用},你可以读取和修改这个值。然而,你不能同时再有任何其他有效的引用。表达式\texttt{\&mut e}返回一个指向\texttt{e}的值的可变引用,它的类型是\texttt{\&mut T},读作“ref mute \texttt{T}”。可变引用不是\texttt{Copy}类型。
\end{itemize}





















    \chapter{表达式}\label{ch06}

\emph{LISP programmers know the value of everything, but the cost of nothing}

\begin{flushright}
    ——Alan Perlis, epigram \#55
\end{flushright}

在这一章中,我们将介绍Rust的\emph{表达式},它是构成Rust函数体和大部分Rust代码的构建块。Rust中大部分都是表达式。在这一章中,我们将探索表达式的力量以及如何克服它的局限。我们还将介绍控制流,它在Rust中完全是以表达式为基础的,最后还要介绍Rust中的基本运算符如何单独和组合工作。

还有一些从技术角度应该划入这一类的概念,例如闭包和迭代器,因为足够重要因此我们之后会用单独的章节介绍它们。现在,我们希望能用尽可能少的页数介绍尽可能多的语法。

\section{表达式语言}

Rust表面上看上去像C家族的语言,但这其实是一个误解。在C语言中,\emph{表达式}和\emph{语句}之间有很大的不同。表达式是一些像这样的代码:
\begin{minted}{C}
    5 * (fahr-32) / 9
\end{minted}
而语句则是像这样的:
\begin{minted}{C}
    for (; begin != end; ++begin) {
        if (*begin == target)
            break;
    }
\end{minted}
表达式有值,但语句没有。

Rust是一种\emph{表达式语言}。这意味着它遵循了起源于Lisp的传统,也就是表达式负责完成所有工作。

在C中,\texttt{if}和\texttt{switch}是语句。它们并不产生值,也不能被用在表达式中间。在Rust中,\texttt{if}和\texttt{match}\emph{可以}产生值。我们已经在\hyperref[ch02]{第2章}中看到过一个产生数字值的\texttt{match}表达式:
\begin{minted}{Rust}
    pixels[r * bounds.0 + c] =
        match escapes(Complex { re: point.0, im: point.1 }, 255) {
            None => 0,
            Some(count) => 255 - count as u8
        };
\end{minted}

一个\texttt{if}表达式可以用于初始化一个变量:
\begin{minted}{Rust}
    let status =
        if cpu.temperature <= MAX_TEMP {
            HttpStatus::Ok
        } else {
            HttpStatus::ServerError  // server melted
        };
\end{minted}

一个\texttt{match}表达式可以被用作函数参数或宏的参数:
\begin{minted}{Rust}
    println!("Inside the vat, you see {}.",
        match vat.contents {
            Some(brain) => brain.desc(),
            None => "nothing of interest"
        });
\end{minted}

这解释了Rust为什么没有C的三元运算符\texttt{(expr1 ? expr2 : expr3))}。在C中,它是一种类似\texttt{if}语句的表达式。在Rust中这种写法是多余的,因为\texttt{if}表达式可以同时实现这两种功能。

C中的大部分控制流工具都是语句,在Rust中则全是表达式。

\section{优先级和结合性}

\hyperref[t6-1]{表6-1}总结了Rust的表达式语法。我们将在这一章中介绍所有这些表达式。运算符按照优先级从高到低的顺序列出。(类似于大多数编程语言,Rust使用\emph{运算符优先级}来决定当表达式中含有多个运算符时的运算顺序。例如,在表达式\texttt{limit < 2 * broom.size + 1}中,\texttt{.}运算符优先级最高,因此会先访问字段。)

\begin{longtable}{p{0.25\textwidth}p{0.35\textwidth}p{0.3\textwidth}}
    \caption{表达式}
    \label{t6-1} \\
    \hline
    \textbf{表达式类型} & \textbf{示例} & \textbf{相关trait} \\
    \hline
    数组字面量      & \texttt{[1, 2, 3]}         & \\
    重复数组字面量  & \texttt{[0; 50]}           & \\
    元组            & \texttt{(6, "crullers")}  & \\
    \cline{1-2}
    组合            & \texttt{(2 + 2)}             & \\
    块              & \texttt{\{ f(); g() \}}      & \\
    控制流表达式     & \texttt{if ok \{ f() \}}     & \\
                    & \texttt{if ok \{ 1 \} else \{ 0 \}}                   & \\
                    & \texttt{if let Some(x) = f() \{ x \} else \{ 0 \}}    & \\
                    & \texttt{match x \{ None => 0, \_ => 1 \}}             & \\
                    & \texttt{for v in e \{ f(v); \}}                       & \texttt{\hyperref[iter]{std::iter::IntoIterator}} \\
                    & \texttt{while ok \{ ok = f(); \}}                     & \\
                    & \texttt{while let Some(x) = it.next() \{ f(x); \}}    & \\
                    & \texttt{loop \{ next\_event(); \}}                    & \\
                    & \texttt{break}                  & \\
                    & \texttt{continue}               & \\
                    & \texttt{return 0}               & \\
    宏调用          & \texttt{println!("ok")}         & \\
    路径            & \texttt{std::f64::consts::PI}   & \\
    \cline{1-2}
    结构体字面量     & \texttt{Point \{x: 0, y: 0\}}     & \\
    \cline{1-2}
    元组字段访问    & \texttt{pair.0}   & \texttt{\hyperref[deref]{Deref}, \hyperref[deref]{DerefMut}} \\
    结构体字段访问  & \texttt{point.x}  & \texttt{\hyperref[deref]{Deref}, \hyperref[deref]{DerefMut}} \\
    方法调用       & \texttt{point.translate(50, 50)} & \texttt{\hyperref[deref]{Deref}, \hyperref[deref]{DerefMut}} \\
    函数调用       & \texttt{stdin()}   & \texttt{\hyperref[fn]{Fn(Arg0, ...) -> T}, \hyperref[fn]{FnMut(Arg0, ...) -> T}, \hyperref[fn]{FnOnce(Arg0, ...) -> T}} \\
    索引            & \texttt{arr[0]}   & \texttt{\hyperref[index]{Index}, \hyperref[index]{IndexMut}, \hyperref[deref]{Deref}, \hyperref[deref]{DerefMut}} \\
    \cline{1-2}
    错误检查        & \texttt{create\_dir("tmp")?}   & \\
    \cline{1-2}
    逻辑/位 NOT     & \texttt{!ok}  & \texttt{\hyperref[unop]{Not}} \\
    负             & \texttt{-num}  & \texttt{\hyperref[unop]{Neg}} \\
    解引用          & \texttt{*ptr} & \texttt{\hyperref[deref]{Deref}, \hyperref[deref]{DerefMut}} \\
    借用            & \texttt{\&val}    & \\
    \cline{1-2}
    类型转换    & \texttt{x as u32} & \\
    \cline{1-2}
    乘法        & \texttt{n * 2}    & \texttt{\hyperref[biop]{Mul}} \\
    除法        & \texttt{n / 2}    & \texttt{\hyperref[biop]{Div}} \\
    余数(取模) & \texttt{n \% 2}   & \texttt{\hyperref[biop]{Rem}} \\
    \hline
    加法        & \texttt{n + 1}    & \texttt{\hyperref[biop]{Add}} \\
    减法        & \texttt{n - 1}    & \texttt{\hyperref[biop]{Sub}} \\
    \hline
    左移        & \texttt{n << 1}   & \texttt{\hyperref[biop]{Shl}} \\
    右移        & \texttt{n >> 1}   & \texttt{\hyperref[biop]{Shr}} \\
    \hline
    位与        & \texttt{n \& 1}   & \texttt{\hyperref[biop]{BitAnd}} \\
    \hline
    位异或      & \texttt{n \^{} 1} & \texttt{\hyperref[biop]{BitXor}} \\
    \hline
    位或        & \texttt{n | 1}    & \texttt{\hyperref[biop]{BitOr}}  \\
    \hline
    小于        & \texttt{n < 1}    & \texttt{\hyperref[cmp]{std::cmp::PartialOrd}} \\
    小于等于    & \texttt{n <= 1}   & \texttt{\hyperref[cmp]{std::cmp::PartialOrd}} \\
    大于        & \texttt{n > 1}    & \texttt{\hyperref[cmp]{std::cmp::PartialOrd}} \\
    大于等于    & \texttt{n >= 1}   & \texttt{\hyperref[cmp]{std::cmp::PartialOrd}} \\
    等于        & \texttt{n == 1}   & \texttt{\hyperref[equal]{std::cmp::PartialEq}} \\
    不等于      & \texttt{n != 1}   & \texttt{\hyperref[equal]{std::cmp::PartialEq}} \\   
    \hline
    逻辑与      & \texttt{x.ok \&\& y.ok}       & \\
    \cline{1-2}
    逻辑或      & \texttt{x.ok || backup.ok}    & \\
    \cline{1-2}
    左闭右开区间 & \texttt{start .. stop}   & \\
    左闭右闭区间 & \texttt{start ..= stop}  & \\
    \cline{1-2}
    赋值        & \texttt{x = val}  & \\
    复合赋值    & \texttt{x *= 1}   & \texttt{\hyperref[assign]{MulAssign}} \\
                & \texttt{x /= 1}   & \texttt{\hyperref[assign]{DivAssign}} \\
                & \texttt{x \%= 1}  & \texttt{\hyperref[assign]{RemAssign}} \\
                & \texttt{x += 1}   & \texttt{\hyperref[assign]{AddAssign}} \\
                & \texttt{x -= 1}   & \texttt{\hyperref[assign]{SubAssign}} \\
                & \texttt{x <<= 1}  & \texttt{\hyperref[assign]{ShlAssign}} \\
                & \texttt{x >>= 1}  & \texttt{\hyperref[assign]{ShrAssign}} \\
                & \texttt{x \&= 1}  & \texttt{\hyperref[assign]{BitAndAssign}} \\
                & \texttt{x \^{}= 1}& \texttt{\hyperref[assign]{BitXorAssign}} \\
                & \texttt{x |= 1}   & \texttt{\hyperref[assign]{BitOrAssign}} \\
    \cline{1-2}
    闭包        & \texttt{|x, y| x + y} & \\
\end{longtable}

所有可以链式使用的运算符都是左结合的。也就是说,一条运算链例如\texttt{a - b - c}被组合为\texttt{(a - b) - c},而不是\texttt{a - (b - c)}。这些运算符可以被任意组合:
\begin{minted}{text}
    * / % + - << >> & ^ | && || as
\end{minted}
比较运算符、赋值运算符、范围运算符\texttt{.. }和\texttt{..=}不能被链式使用。

\section{块和分号}

块是最通用的表达式。一个块产生一个值,可以被用于任何需要一个值的地方:\begin{minted}{Rust}
    let display_name = match post.author() {
        Some(author) => author.name(),
        None => {
            let network_info = post.get_network_metadata()?;
            let ip = network_info.client_address();
            ip.to_string()
        }
    };
\end{minted}
\texttt{Some(author) =>}之后的代码是简单的表达式\texttt{author.name()},\texttt{None =>}之后的代码则是一个块表达式。对Rust来种,两种表达式没有区别。块表达式的值是它的最后一条表达式的值,也就是\texttt{ip.to\_string()}。

注意\texttt{ip.to\_string()}后面没有分号。Rust中的大部分代码行都以分号或者花括号结尾,类似于C和Java。如果一个块看起来像C代码一样在所有的表达式后边都有分号,那它的行为就和C块一样,它的值将是\texttt{()}。正如我们在\hyperref[ch02]{第2章}提到的,当你省略了块中最后一个表达式后边的分号,那么块的值将是最后一个表达式的值,而不是通常的\texttt{()}。

在一些语言中,尤其是Javascript,你可以省略分号,语言会自动为你添加上——这样除了方便一点,并没有任何区别。然而在Rust中,分号通常是有实际意义的:
\begin{minted}{Rust}
    let msg = {
        // let语句:总是需要分号
        let dandelion_control = puffball.open();

        // 表达式 + 分号:方法被调用,返回值被丢弃
        dandelion_control.release_all_seeds(launch_codes);

        // 没有分号的表达式:方法被调用,
        // 返回值被存储到 `msg`
        dandelion_control.get_status()
    };
\end{minted}

语句块可以包含声明最后还能产生一个值的能力是一个很有用的特性,而且可以很快习惯。它的一个缺陷是如果你偶然忘记了分号会导致一条错误信息:
\begin{minted}{Rust}
    ...
    if preferences.changed() {
        page.compute_size()  // oops, 缺少分号
    }
\end{minted}

如果你在C或者Java程序中犯了这种错误,编译器会简单地直接指出你少写了一个分号。然而这是Rust的报错:
\begin{minted}{text}
    error[E0308]: mismatched types
    22 |         page.compute_size()  // oops, missing semicolon
       |         ^^^^^^^^^^^^^^^^^^^- help: try adding a semicolon `;`
       |         |
       |         expected (), found tuple
       |
       = note: expected unit type `()`
                  found tuple `(u32, u32)`
\end{minted}

在缺少分号的情况下,块的值将是\texttt{page.compute\_size()}返回的值,但一个没有\texttt{else}分支的\texttt{if}语句必须总是返回\texttt{()}。幸运的是,Rust知道这种类型的错误并建议加上分号。

\section{声明}

除了表达式和分号之外,一个块中可能包含任意数量的声明。最常见的情况是\texttt{let}声明,它用来声明局部变量:
\begin{minted}{Rust}
    let name: type = expr;
\end{minted}

类型和初始值是可选的,分号是必须的。

一个\texttt{let}声明可以在不初始化的情况下声明一个变量。这有时很有用,因为有时候一个变量需要在控制流的中途初始化:
\begin{minted}{Rust}
    let name;
    if user.has_nickname() {
        name = user.nickname();
    } else {
        name = generate_unique_name();
        user.register(&name);
    }
\end{minted}

局部变量\texttt{name}有两种不同的初始化路径,但两条路径上它都只会被初始化一次,所以\texttt{name}不需要声明为\texttt{mut}。

在变量初始化之前使用它会导致错误(这和使用被移动的值的错误紧密相关,Rust希望你只在变量的值存在时使用它们!)。

你有时可能会看到代码似乎重新声明一个已经存在的变量,例如:
\begin{minted}{Rust}
    for line in file.lines() {
        let line = line?;
        ...
    }
\end{minted}

\texttt{let}声明创建了一个新的、类型不同的、第二个变量\texttt{line}。第一个\texttt{line}的类型是\texttt{Result<String, io::Error>},第二个\texttt{line}是一个\texttt{String}。第二个声明在块的剩余部分会取代第一个。这被称为\emph{遮蔽(shadowing)},在Rust程序中非常常见。上面的代码等价于:
\begin{minted}{Rust}
    for line_result in file.lines() {
        let line = line_result?;
        ...
    }
\end{minted}

在这本书中,我们将坚持在这种场景中使用\texttt{\_result}后缀,来保证变量的名字不同。

一个块还可以包含\emph{item declarations}。一个item是一个可以出现在全局或模块中的声明,例如\texttt{fn}、\texttt{struct}、\texttt{use}。

后面的章节将会详细介绍item。现在,\texttt{fn}足够作为一个例子了。任何块都可以包含\texttt{fn}声明:
\begin{minted}{Rust}
    use std::io;
    use std::cmp::Ordering;

    fn show_files() -> io::Result<()> {
        let mut v = vec![];
        ...
        fn cmp_by_timestamp_then_name(a: &FileInfo, b: &FileInfo) -> Ordering {
            a.timestamp.cmp(&b.timestamp)   // 首先,比较时间戳 
                .reverse()                  // 最新的文件优先
                .then(a.path.cmp(&b.path))  // 比较路径
        }

        v.sort_by(cmp_by_timestamp_then_name);
        ...
    }
\end{minted}

当一个\texttt{fn}在块内声明的时候,它的作用域是整个块,它可以在整个块内\emph{使用}。但是一个嵌套的\texttt{fn}不能访问外围作用域的局部变量和参数。例如,函数\texttt{cmp\_by\_timestamp\_then\_name}不能使用\texttt{v}。(Rust还有闭包,闭包可以使用外层作用域的变量,见\hyperref[ch14]{第14章}。)

一个块甚至可以包含整个模块。这听起来可能有些多余了:我们真的需要把语言的\emph{每一部分}嵌套在其他部分中的能力吗?——但程序员(尤其是使用宏的程序员)可以找到语言提供的每一个正交碎片的用法。

\section{if与match}

\section{类型转换}\label{cast}

    \chapter{错误处理}\label{ch07}

\emph{I knew if I stayed around long enough, something like this would happen.}

\begin{flushright}
    ——George Bernard Shaw on dying
\end{flushright}

Rust的错误处理不同寻常,无法用很短的一个章节来介绍它。其实它里面并没有什么困难的概念,只有一些可能对你来说可能很新的概念。这一章将覆盖Rust中两种不同的错误处理:panic和\texttt{Result}。

一般的错误使用\texttt{Result}类型来处理,\texttt{Result}通常代表程序之外的东西引起的问题,例如错误的输入、网络中断、权限问题等。这种情况的出现不由我们决定,即使是一个完全没有bug的程序也可能随时遇到它们。这一章中的大部分内容都是在讨论这种错误。我们将首先介绍panic,因为它比较简单。

panic是另一种错误,一种\emph{永远不应该发生}的错误。

\section{panic}

当程序遇到一些由程序自身的bug导致的非常糟糕的事情时它会panic。例如:
\begin{itemize}
    \item 数组访问越界
    \item 整数除以0
    \item 在值为\texttt{Err}的\texttt{Result}上调用\texttt{.expect()}方法
    \item 断言失败
\end{itemize}

(还有一个宏\texttt{panic!()},用于当你的代码自己发现了错误,想要直接触发panic的情况。\texttt{panic!()}接受可选的\texttt{println!()}-风格的参数,用于构建错误信息。)

这些条件的共同之处在于——它们都是程序员的错。一条好的经验法则是:“不要panic”。

但是我们都有犯错误的时候。当这些不该发生的错误发生了的时候,该怎么办?值得注意的是,Rust给了你一个选择:Rust可以展开堆栈或者中止进程。栈展开是默认行为。

\subsection{栈展开}
当海盗们瓜分抢来的战利品时,船长将得到一半的战利品。普通的船员们均分剩下的一半。(海盗们讨厌分数,因此如果均分时不能除尽,结果会向下取整,余数将分给船上的鹦鹉。)
\begin{minted}{Rust}
    fn pirate_share(total: u64, crew_size: usize) -> u64 {
        let half = total / 2;
        half / crew_size as u64
    }
\end{minted}

这段代码也许可以工作几个世纪,直到有一天船长是抢劫之后唯一的幸存者。如果我们传递的\texttt{crew\_size}为0,它将会除以0。在C++中,这将是未定义行为。在Rust中,它会触发panic,panic通常会按照如下方式继续:
\begin{itemize}
    \item 打印一条错误消息到终端:
    \begin{minted}{text}
    thread 'main' panicked at 'attempt to divide by zero',
    pirates.rs:3780
    note: Run with `RUST_BACKTRACE=1` for a backtrace.
    \end{minted}

    如果你设置了\texttt{RUST\_BACKTRACE}环境变量,Rust还会打印出此时的堆栈信息。

    \item 堆栈被展开。这和C++中的异常处理很像。
    
    任何当前函数内的临时值、局部变量、或者参数都会按照与它们创建时相反的顺序被drop掉。

    drop一个值意味着清理它:函数使用过的任何\texttt{String}或\texttt{Vec}都会被释放,任何打开的\texttt{File}都会被关闭,等等。用户自定义的\texttt{drop}方法也会被调用,见“\hyperref[drop]{Drop}”一节。在\texttt{pirate\_share()}的例子中,没有要清理的内容。

    一旦当前的函数调用被清理完毕,我们会移动到它的调用者,以同样的方式drop它的变量和参数。然后我们移动到\emph{那个}函数的调用者,以此类推。

    \item 最后,线程退出。如果panic的线程是主线程,整个进程会退出(退出代码不为0)。
\end{itemize}

对这种有序的处理,也许\emph{panic}是一个有误导性的名字。panic并不是崩溃,也不是未定义行为,它更类似于Java中的\texttt{RuntimeException}或C++中的\texttt{std::logic\_error}。它的行为都是良定义的,它只是不应该发生。

panic是安全的。它不违背Rust中的任何安全规则,即使你设法在一个标准库的方法中引起panic,它也用于不会导致悬垂指针或者初始化到一半的值。关键在于Rust在任何错误的事情发生之前就捕捉到了无效的数组访问或者类似的情况。如果继续下去将是不安全的,所以Rust会展开堆栈。但进程的其他部分可以继续运行。

panic是以线程为单位。一个线程可以panic,而其他线程继续处理它们的业务。在\hyperref[ch19]{第19章}中,我们会展示一个父线程怎么查明一个子线程是否panic并优雅地处理错误。

还有一种方式\emph{捕获}栈展开,允许线程存活并继续运行。标准库函数\texttt{std::panic::catch\_unwind()}可以做到这一点。我们不会解释如何使用它,但Rust的测试工具使用了这个机制,用于在测试时断言失败的情况下恢复执行(当编写可以在C或C++中调用的Rust代码时这也是必须的,因为在非Rust代码中的栈展开是未定义行为,见\hyperref[ch22]{第22章})。

理想情况下,我们希望没有bug并且永远不会panic的代码。但没有完美的事物,你可以使用线程和\texttt{catch\_unwind()}来处理panic,让你的程序更加健壮。一个重要的警告是这些工具只会捕获展开堆栈的panic。不是所有的panic都以这种方式处理。

\subsection{中止}


\section{Result}

\subsection{捕捉错误}

\subsection{Result类型别名}

\subsection{打印错误}

\subsection{传播错误}\label{properror}
    \chapter{crate与模块}\label{ch08}

\emph{This is one note in a Rust theme: systems programmers can have nice things.}

\begin{flushright}
    ——Robert O'Callahan, “\href{https://robert.ocallahan.org/2016/08/random-thoughts-on-rust-cratesio-and.html}{Random Thoughts on Rust: crates.io and IDEs}”
\end{flushright}

假设你在编写一个仿真蕨类植物从细胞开始生长的程序。你的程序就像蕨类一样,一开始非常简单,可能所有代码都在单个文件里——就像一个孢子。随着它逐渐成长,它开始逐渐建立起内部的结构,不同的片段负责不同的功能。它将分裂为多个文件,可能覆盖整个目录树。随着时间的推移,它可能会成为整个软件生态系统的重要组成部分 。对于任何成长到不仅仅是几个数据结构和几百行代码的程序,都必须要对代码进行组织。

这一章将会介绍Rust中用于组织程序的特性:crate和模块。我们还会介绍Rust crate的结构和分发相关的话题,包括如何编写文档和测试Rust代码,如何禁用不需要的编译器警告,如何使用Cargo来管理项目依赖和版本,如何在Rust的公开crate仓库:crates.io上发布开源的库,crate的版本如何演变,等等。我们将使用蕨类仿真程序作为我们的例子。

\section{Crate}

Rust程序由\emph{crates}组成。每一个crate都是一个完整的、一体的单元:一个库或可执行文件的所有代码、加上相关的测试、示例、工具、配置、以及一些其他东西。为了编写你自己的蕨类模拟器,你可能需要使用和3D图形、生物信息学、并行计算等相关的第三方库。这些库就像箱子一样(见\hyperref[f8-1]{图8-1})。

\begin{figure}[htbp]
    \centering
    \includegraphics[width=0.9\textwidth]{../img/f8-1.png}
    \caption{一个crate和它的依赖}
    \label{f8-1}
\end{figure}

查看crate是什么以及它们是如何工作的最简单方法就是使用带有\texttt{--verbose}参数的\texttt{cargo build}来构建一个有一些依赖的程序。我们用\hyperref[mandelbrot]{一个并发的曼德勃罗集}作为示例。结果如下所示:
\begin{minted}{text}
    $ cd mandelbrot
    $ cargo clean   # delete previously compiled code
    $ cargo build --verbose
        Updating registry `https://github.com/rust-lang/crates.io-index`
     Downloading autocfg v1.0.0
     Downloading semver-parser v0.7.0
     Downloading gif v0.9.0
     Downloading png v0.7.0
    
    ... (downloading and compiling many more crates)

        Compiling jpeg-decoder v0.1.18
          Running `rustc
             --crate-name jpeg_decoder
             --crate-type lib
             ...
             --extern byteorder=.../libbyteorder-29efdd0b59c6f920.rmeta
             ...
        Compiling image v0.13.0
          Running `rustc
             --crate--name image
             --crate-type lib
             ...
             --extern byteorder=.../libbyteorder-29efdd0b59c6f920.rmeta
             --extern gif=.../libgif-a7006d35f1b58972.rmeta
             --extern jpeg_decoder=.../libjped_decoder-5c10558d0d57d300.rmeta
        Compiling mandelbrot v0.1.0 (/tmp/rustbook-test-files/mandelbrot)
          Running `rustc
             --edition=2018
             --crate-name mandelbrot
             --crate-type bin
             ...
             --extern crossbeam=.../libcrossbeam-f87b4b3d3284acc2.rlib
             --extern image=.../libimage-b5737c12bd641c43.rlib
             --extern num=.../libnum-1974e9a1dc582ba7.rlib -C link-arg=-fuse-ld=lld`
         Finished dev [unoptimized + debuginfo] target(s) in 16.94s
\end{minted}

我们重新格式化了\texttt{rustc}的命令行来改善可读性,并且删掉了很多和我们的讨论无关的编译器选项,用省略号(\ldots)代替了它们。

你可能还记得,当我们完成曼德勃罗集程序时,它的\emph{main.rs}包含几个引入其它crate的\\
\texttt{use}声明:
\begin{minted}{Rust}
    use num::Complex;
    // ...
    use image::ColorType;
    use image::png::PNGEncoder;
\end{minted}

我们还在\emph{Cargo.toml}中指定了每个crate的版本:
\begin{minted}{toml}
    [dependencies]
    num = "0.4"
    image = "0.13"
    crossbeam = "0.8"
\end{minted}

这里的\emph{依赖}指这个程序使用的其它crate,也就是我们依赖的代码。我们可以在\href{https://crates.io}{crates.io}中找到这些crate,那是Rust社区用于存放开源的crate的网站。例如,我们可以访问crates.io并搜索图片库来找到\texttt{image}库。crates.io上的每个crate的页面上会显示它的\emph{README.md}文件和到文档和源代码的链接,还有一行配置例如\texttt{image = "0.13"},你可以复制这一行并添加到你的\emph{Crago.toml}中。这里显示的版本号直接用了我们在编写这个程序时这三个包的最新版本。

Cargo的输出说明了这些信息是如何被使用的。当我们运行\texttt{cargo build}时,Cargo会首先从crates.io下载这些crate的指定版本的源码。然后,它读取那些crate的\emph{Cargo.toml}文件,下载\emph{它们}的依赖,然后递归操作。例如,\texttt{image} crate的0.13.0版本的源代码中包含一个\emph{Cargo.toml}文件,内容如下:
\begin{minted}{toml}
    [dependencies]
    byteorder = "1.0.0"
    num-iter = "0.1.32"
    num-rational = "0.1.32"
    num-traits = "0.1.32"
    enum_primitive = "0.1.0"
\end{minted}

看到这些内容,Cargo知道在它可以使用\texttt{image}之前,它必须先拉取这些crate。我们称它们为\texttt{mandelbrot}的\emph{间接(transitive)}依赖。所有这些依赖的集合告诉了Cargo需要知道的有关如何构建和构建顺序的一切信息,它被称为crate的\emph{依赖图}。Cargo自动处理依赖图和间接依赖的能力是程序员们付出时间和努力的一大胜利。

当获得了源代码之后,Cargo会编译所有的crate。它会运行Rust的编译器\texttt{rustc},一次编译依赖图中的一个crate。当编译这些库时,Cargo会使用\texttt{--crate-type lib}选项。这告诉\texttt{rustc}不要寻找\texttt{main()}函数,而是产生一个包含编译过代码的\emph{.rlib}文件,这个文件可以被用于创建可执行文件和其他\emph{.rlib}文件。

当编译程序时,Cargo会使用\texttt{--crate-type bin},编译的结果将是一个目标平台的二进制可执行文件:例如在Windows上就是\emph{mandelbrot.exe}。

对于每一个\texttt{rustc}命令,Cargo都会传递\texttt{--extern}选项,给出crate用到的每一个库的名称。这样,当\texttt{rustc}看到一行类似于\texttt{use image::png::PNGEncoder}的代码时,它可以分辨出\texttt{image}是另一个crate的名字,而且Cargo传递的选项让它知道该从哪里寻找编译好的crate。Rust的编译器需要访问这些\emph{.rlib}文件,因为它们包含编译好的库中的代码。Rust将会将代码静态链接到最终的可执行文件中。\emph{.rlib}还包含类型信息,因此Rust可以通过检查确保我们在代码中使用的库的特性确实存在而且被正确使用。它还包含一份crate的public内联函数、泛型、宏、特性的拷贝,这些东西只有当Rust看到我们如何使用它们时才可以将它们编译为机器代码。

\texttt{cargo build}支持各种选项,其中的大部分都超出了本书的范围,不过我们在这里会提到其中一个:\texttt{cargo build --release}会生成优化后的构建。Release构建运行得更快,但需要更长的时间来编译,而且它们不检查整数溢出、跳过\texttt{debug\_assert!()}断言,并且它们在panic时生成的堆栈追踪通常不太可靠。

\subsection{版本}

Rust有极强的兼容性保证。任何在Rust 1.0中能编译的代码必须在Rust 1.50或者1.900(如果发布了的话)中也能编译。


但有时社区会遇到一些令人信服的扩展语言的建议,这可能会导致旧代码不能再编译。例如,经过了多次讨论之后,Rust确定了一种支持异步编程的语法,将标识符\texttt{async}和\texttt{await}重新用作关键字(见\hyperref[ch20]{第20章})。但这项语言的改变可能会导致使用\texttt{async}或者\texttt{await}作为变量名的代码不能再编译。

为了在不破坏这些现有代码的前提下演变,Rust使用了\emph{版本}。Rust的2015版本和Rust 1.0兼容。2018版本将\texttt{async}和\texttt{await}改为关键字、精简了模块系统、还引入了一些和2015版本不兼容的其它语言更改。每个crate在\emph{Cargo.toml}文件中的\texttt{package}节中用一行类似如下的说明指定Rust的版本:
\begin{minted}{Rust}
    edition = "2018"
\end{minted}

如果缺少这个关键字,将会假设使用2015版本,因此旧的crate完全不需要做任何更改。但如果你想使用异步函数或者新的模块系统,你需要确保\emph{Cargo.toml}中有\texttt{edition = "2018"}(或者可能更新的版本)。

Rust保证编译器将总是接受语言的所有版本,并且程序可以自由混合使用不同版本编写的crate。即使一个2015版本的crate依赖一个2018版本的crate也没有问题。换句话说,一个crate的版本只影响它的代码是如何被构建的,版本的区别只体现在代码编译的时候。这意味着没有必要更新旧的版本来适配现代Rust的生态。类似的,也没有必要将crate保持在旧版本来避免影响到它的用户。你只需要在想使用新的语言特性时更改自己代码中的版本。

版本并不是每年都会更新,只有当Rust项目觉得有必要出新版本的时候才会更新。例如,没有2020版本。把\texttt{edition}设置为\texttt{"2020"}将会导致错误。\href{https://doc.rust-lang.org/stable/edition-guide}{Rust版本指南}介绍了每一个版本中的变化,并提供了版本系统的背景知识。

使用最新版本几乎总是一个好主意,尤其是新编写代码时。\texttt{cargo new}会默认创建最新版本的项目。本书中将始终使用2018版本。

如果你有一个用更旧版本的Rust编写的crate,\texttt{cargo fix}命令也许可以帮你自动把代码更新到更新的版本。Rust版本指南详细解释了\texttt{cargo fix}命令。

\subsection{构建配置}
有几个\emph{Cargo.toml}中的配置选项可以影响到\texttt{cargo}生成的\texttt{rustc}命令行(\hyperref[t8-1]{表8-1})。

\begin{table}[htbp]
    \centering
    \caption{Cargo.toml配置节}
    \label{t8-1}
    \begin{tabular}{ll}
        \hline
        \textbf{命令行}     & \textbf{使用到的Cargo.toml节}   \\
        \hline
        \texttt{cargo build} & \texttt{[profile.dev]}   \\
        \rowcolor{tablecolor}
        \texttt{cargo build --release} & \texttt{[profile.release]} \\
        \texttt{cargo test}  & \texttt{[profile.test]}  \\
    \end{tabular}
\end{table}

通常默认的行为就足够了,但我们会发现一个例外是你想使用一个profiler——一个用于测量程序使用CPU时间情况的工具。为了从profiler获取最准确的数据,你将同时需要优化(通常只在release构建中可用)和调试符号(通常只在debug构建中可用)。为了同时启用两者,在\emph{Cargo.toml}中添加:
\begin{minted}{toml}
    [profile.release]
    debug = true    # 允许在release构建中启用调试符号
\end{minted}

\texttt{debug}设置控制是否给\texttt{rustc}传递\texttt{-g}选项。有了这个配置,当你输入\\
\texttt{cargo build --release}时,你将会得到一个带有调试符号的二进制文件。优化的设置将不会被影响。

\href{https://doc.rust-lang.org/cargo/reference/manifest.html}{Cargo文档}中列出了很多其他你可以在\emph{Cargo.toml}中调整的设置。

\section{模块}

如果说crate决定了项目之间的代码共享,那么\emph{模块}则决定了项目\emph{内部}的代码组织。它们扮演了Rust中的命名空间——一种包含函数、类型、常量等内容的容器,这些模块组成了你的Rust程序或库。一个模块看起来类似于这样:

\begin{minted}{Rust}
    mod spores {
        use cells::{Cell, Gene};

        /// 成熟蕨类植物产生的细胞。它会随着风飘散,
        /// 这也是蕨类生命周期的一部分。一个孢子会成长为一个原叶体——
        /// 一个宽达5mm的完整的独立有机体。它会产生受精卵,
        /// 这些受精卵会成长为新的蕨类植物(植物的性别很复杂)。
        pub struct Spore {
            ...
        }

        /// 模拟通过减数分裂产生孢子的过程。
        pub fn produce_spore(factory: &mut Sporangium) -> Spore {
            ...
        }

        // 提取一个孢子中的基因。
        pub(crate) fn genes(spore: &Spore) -> Vec<Gene> {
            ...
        }

        /// 混合基因为减数分裂做准备(分裂间期的一部分)。
        fn recombine(parent: &mut Cell) {
            ...
        }

        ...
    }
\end{minted}

模块是\emph{item}的集合,\texttt{item}是命名的特性,例如本例中的\texttt{Spore}结构体和两个函数。\texttt{pub}关键字将item设为公有的,因此可以从模块外部访问。

把函数标记为\texttt{pub(crate)},意味着它在这个crate中任何地方都可以访问,但不作为外部接口的一部分公开。它不能被其他crate使用,也不会在crate的文档中显示。

任何没有被标记为\texttt{pub}的都是私有的,只能在定义它的模块和子模块中使用:
\begin{minted}{Rust}
    let s = spores::produce_spore(&mut factory);    // ok
    
    spores::recombine(&mut cell);   // 错误:`recombine`是私有的
\end{minted}

将item标记为\emph{pub}通常称为“导出”这个item。

这一节的剩余部分将覆盖使用模块所需要了解的细节:
\begin{itemize}
    \item 我们会展示如果需要的话怎么嵌套模块和把它们分布在不同的文件和目录中。
    \item 我们会解释Rust从其他模块中引用item的路径语法,并展示怎么导入item,这样就不需要每次都写出完整的路径。
    \item 我们会接触Rust对结构体字段的细粒度控制。
    \item 我们会介绍\emph{prelude}模块,它通过收集几乎所有用户都会用到的常见导入来减少重复的导入。
    \item 我们会展示\emph{常量}和\emph{静态量},这是两种为了清晰和一致性而设计的定义命名变量的方式。
\end{itemize}

\subsection{嵌套模块}

模块可以嵌套,事实上一个模块只是一些子模块的集合的情况是很常见的:
\begin{minted}{Rust}
    mod plant_structures {
        pub mod roots {
            ...
        }
        pub mod stems {
            ...
        }
        pub mod leaves {
            ...
        }
    }
\end{minted}

如果你想要让嵌套模块中的一个item对其他crate可见,那需要保证将它\emph{和所有嵌套包含它的模块}标记为public。否则你会看到一个类似这样的警告:
\begin{minted}{text}
    warning: function is never used: `is_square`
      --> src/crates_unused_items.rs:23:9
       |
    23 | /         pub fn is_square(root: &Root) -> bool {
    24 | |             root.cross_section_shape().is_square()
    25 | |         }
       | |_________^
       |
\end{minted}

可能这个函数这时确实是死代码。但如果你是想将它用在其他crate中,Rust会让你明白它实际上并不可见。你需要保证嵌套包含它的模块也都被标记为\texttt{pub}。

也可以声明\texttt{pub(super)},让一个item只在父模块中可见。\texttt{pub(in <path>)}可以让它在一个指定的父模块和其后代中可见。这在深层嵌套的模块中很有用:
\begin{minted}{Rust}
    mod plant_structures {
        pub mod roots {
            pub mod products {
                pub(in crate::plant_structures::roots) struct Cytokinin {
                    ...
                }
            }

            use products::Cytokinin;    // ok: 在`roots`模块中
        }

        use roots::products::Cytokinin; // error: `Cytokinin`是私有的
    }

    // error: `Cytokinin`是私有的
    use plant_structures::roots::products::Cytokinin;
\end{minted}

通过这种方式,我们可以写出一个有数量庞大的代码和完整的模块层次结构的程序,不管这些模块的关系如何,我们都可以将整个程序写在单个文件里。

但实际上以这种方式来工作非常的痛苦,因此还有另一种方案。

\subsection{单独文件中的模块}\label{ModInFile}
一个模块还可以这么写:
\begin{minted}{Rust}
    mod spores;
\end{minted}

之前,我们还在花括号中包含了\texttt{spores}模块的主体。这里,我们通过这种方式告诉Rust编译器\texttt{spores}模块在一个单独的叫\emph{spores.rs}的文件里:
\begin{minted}{Rust}
    // spores.rs

    /// 成熟蕨类植物产生的一个细胞...
    pub struct Spore {
        ...
    }

    /// 模拟减数分裂产生孢子的过程。
    pub fn produce_spore(factory: &mut Sporangium) -> Spore {
        ...
    }

    /// 从一个孢子中提取基因。
    pub(crate) fn genes(spore: &Spore) -> Vec<Gene> {
        ...
    }

    /// 混合基因为减数分裂做准备(分裂间期的一部分)。
    fn recombine(parent: &mut Cell) {
        ...
    }
\end{minted}

\emph{spores.rs}只包含组成模块的item。它并不需要任何说明来表明它是一个模块。

这个\texttt{spores}模块和我们在上一节中展示的版本的\emph{唯一}不同就是代码的位置。有关公有性和私有性的规则和之前完全相同。Rust从来不会单独编译模块,即使它们在单独的文件里。当你构建一个Rust的crate的时候,你总是要重新编译它里面所有的模块。

一个模块也可以有自己的目录。当Rust看到\texttt{mod spores;}时,它会检查\emph{spores.rs}和\\
\emph{spores/mod.rs},如果这两个文件都不存在或者都存在就会报错。本例中因为\texttt{spores}模块没有任何子模块,所以我们使用了\emph{spores.rs}。但考虑一下我们之前写的\texttt{plant\_structures}模块。如果我们决定将那个模块和它的三个子模块分割在单独的文件中,最终的项目看起来就是这样:
\begin{minted}{text}
    fern_sim/
    |-- Cargo.toml
    |-- src/
        |-- main.rs
        |-- spores.rs
        |-- plant_structures/
            |-- mod.rs
            |-- leaves.rs
            |-- roots.rs
            |-- stems.rs
\end{minted}

在\emph{main.rs}中,我们声明了\texttt{plant\_structures}模块:
\begin{minted}{Rust}
    pub mod plant_structures;
\end{minted}

这会导致Rust去加载\emph{plant\_structures/mod.rs},这个文件里又声明了三个子模块:
\begin{minted}{Rust}
    // 在plant_structures/mod.rs中
    pub mod roots;
    pub mod stems;
    pub mod leaves;
\end{minted}

这三个模块的内容都被存储在单独的文件中,分别命名为\emph{leaves.rs}、\emph{roots.rs}、\emph{stems.rs},和\emph{mod.rs}一样在\texttt{plant\_structures}目录下。

使用同名的文件和目录来组成模块也是可行的。例如,如果\texttt{stems}需要包含两个分别叫\texttt{xylem}和\texttt{phloem}的模块,我们可以选择将\texttt{stems}保留在\emph{plant\_structures/stems.rs}中,然后添加一个新的\texttt{stems}目录:
\begin{minted}{text}
    fern_sim/
    |-- Cargo.toml
    |-- src/
        |-- main.rs
        |-- spores.rs
        |-- plant_structures/
            |-- mod.rs
            |-- leaves.rs
            |-- roots.rs
            |-- stems/
            |   |-- phloem.rs
            |   |-- xylem.rs
            |
            |-- stems.rs
\end{minted}

然后在\emph{stems.rs}中声明这两个新的子模块:
\begin{minted}{Rust}
    // 在plant_structures/stems.rs中
    pub mod xylem;
    pub mod phloem;
\end{minted}

这三种方式——模块在自己单独的文件中、模块在自己同名的目录中的\emph{mod.rs}中,模块在自己单独的文件中并有一个同名的目录包含子模块——给了模块系统足够的灵活性来支撑你需要的任何程序结构。

\subsection{路径和导入}
\texttt{::}运算符用于访问模块中的特性。你的项目中任何地方的代码都可以通过路径来引用任何标准库的特性:
\begin{minted}{Rust}
    if s1 > s2 {
        std::mem::swap(&mut s1, &mut s2);
    }
\end{minted}

\texttt{std}是标准库的名称。路径\texttt{std}指向标准库的顶层模块。\texttt{std::mem}是在标准库中的一个子模块,\texttt{std::mem::swap}是\texttt{std::mem}模块中的一个public函数。

你可以始终用这种方式编写所有的代码,每当你需要圆或字典时都写出\\
\texttt{std::f64::consts::PI}和\texttt{std::collections::HashMap::new},但这样太过繁琐,而且难以阅读。替代方案是把一些特性\emph{导入}用到它们的模块:
\begin{minted}{Rust}
    use std::mem;

    if s1 > s2 {
        mem::swap(&mut s1, &mut s2);
    }
\end{minted}

\texttt{use}声明会导致\texttt{mem}变为\texttt{std::mem}在整个块或模块中的一个局部别名。

你也可以写\texttt{std::mem::swap}来导入\texttt{swap}函数本身,而不是\texttt{mem}模块。然而,我们之前的方式被认为是最佳的风格:引入类型、trait和模块(例如\texttt{std::mem})然后使用相对路径访问函数、常量和其他成员。

可以一次导入若干个名字:
\begin{minted}{Rust}
    use std::collections::{HashMap, HashSet};   // 导入两个
    
    use std::fs::{self, File};  // 导入`std::fs`和`std::fs::File`
    
    use std::io::prelude::*;    // 导入所有内容
\end{minted}

也可以写出所有的单独导入:
\begin{minted}{Rust}
    use std::collections::HashMap;
    use std::collections::HashSet;

    use std::fs;
    use std::fs::File;

    // std::io::prelude中的所有public item:
    use std::io::prelude::Read;
    use std::io::prelude::Write;
    use std::io::prelude::BufRead;
    use std::io::prelude::Seek;
\end{minted}

你可以使用\texttt{as}导入一个item并同时给它一个不同的局部名称:
\begin{minted}{Rust}
    use std::io::Result as IOResult;

    // 返回类型等价于`std::io::Result<()>`
    fn save_spore(spore: &Spore) -> IOResult<()>
    ...
\end{minted}

模块并不会\emph{自动}从父模块中继承名称。例如,假设我们的\emph{proteins/mod.rs}有如下内容:
\begin{minted}{Rust}
    // proteins/mod.rs
    pub enum AminoAcid { ... }
    pub mod synthesis;
\end{minted}

那么\emph{synthesis.rs}里的代码并不会自动导入类型\texttt{AminoAcid}:
\begin{minted}{Rust}
    // proteins/synthesis.rs
    pub fn synthesis(seq: &[AminoAcid]) // 错误:找不到类型`AminoAcid`
        ...
\end{minted}

每一个模块都会以空白的状态开始,必须导入它使用的名称:
\begin{minted}{Rust}
    // proteins/synthesis.rs
    use super::AminoAcid;   // 显式地从父模块中导入
    pub fn synthesize(seq: &[AminoAcid]) // ok
        ...
\end{minted}

默认情况下,路径是相对于当前模块的:
\begin{minted}{Rust}
    // 在proteins/mod.rs中

    // 从子模块中导入
    use synthesis::synthesize;
\end{minted}

\texttt{self}也是当前模块的同义词,因此我们可以写:
\begin{minted}{Rust}
    // 在proteins/mod.rs中

    // 导入一个枚举中的名字
    // 这样我们可以用`Lys`来表示赖氨酸,而不是`AminoAcid::Lys`
    use self::AminoAcid::*;
\end{minted}

或者简写为:
\begin{minted}{Rust}
    // 在proteins/mod.rs中

    use AminoAcid::*;
\end{minted}

(这里的\texttt{AminoAcid}的例子,有些违背我们之前提到的只导入类型、trait和模块的风格。如果我们的程序包含很长的氨基酸序列,那么根据奥威尔的第六原则:“Break any of these rules sooner than say anything outright barbarous.”(绝不要用粗俗语言,为此可以打破上面任一规则。)这么做也是有道理的。)

路径中的\texttt{super}和\texttt{crate}关键字有特殊的含义:\texttt{super}指代父模块,\texttt{crate}指代包含当前模块的crate。

使用相对于crate根的路径而不是相对于当前模块的路径可以使在项目中移动代码变得更容易,因为这样的话就算当前模块的路径变了,导入也不会出错。例如,我们可以使用\texttt{crate}来写\emph{synthesis.rs}:
\begin{minted}{Rust}
    // proteins/synthesis.rs
    use crate::proteins::AminoAcid; // 显式的相对于crate根的导入

    pub fn synthesize(seq: &[AminoAcid]) // ok
        ...
\end{minted}

子模块可以通过\texttt{use super::*}访问父模块中的私有item。

如果你有一个模块和当前正在使用的某一个模块同名,那么引用它们的时候就要小心了。例如,如果你的程序在\emph{Cargo.toml}列出了\texttt{image} crate依赖,同时还有一个模块叫\texttt{image},那么以\texttt{image}开头的路径将会导致歧义:
\begin{minted}{Rust}
    mod image {
        pub struct Sampler {
            ...
        }
    }

    // 错误:这是指向`image`模块,还是`image` crate?
    use image::Pixels;
\end{minted}

即使\texttt{image}模块没有\texttt{Pixels}类型,这个歧义也会被认为是错误:如果之后又添加了\texttt{Pixels}的定义,那么可能会改变路径指向的内容,这可能会令人迷惑。

为了解决歧义,Rust有一种特殊的路径称为\emph{绝对路径},它们以\texttt{::}开头,将总是指向一个外部的crate。为了指向\texttt{image} crate中的\texttt{Pixels}类型,你可以写:
\begin{minted}{Rust}
    use ::image::Pixels;    // `image` crate的`Pixels`
\end{minted}

为了指向你自己的模块中的`Sampler`类型,你可以写:
\begin{minted}{Rust}
    use self::image::Sampler;   // `image`模块的`Sampler`
\end{minted}

模块和文件的概念并不相同,但模块和Unix文件系统中的文件和目录存在自然的类比关系。\texttt{use}关键字创建别名,就像\texttt{ln}命令创建链接。路径类似于文件名,有绝对路径和相对路径两种形式。\texttt{self}和\texttt{super}类似于\texttt{.}和\texttt{..}这两个特殊的目录。

\subsection{标准prelude}
我们之前说每个模块都以“空白的状态”开始,但事实上并不是\emph{完全}空白的状态。

其中一点是,标准库\texttt{std}被自动链接到每个项目。这意味着你总是可以使用\texttt{std::whatever}\\
这种方式来引用\texttt{std}里的item,例如\texttt{std::mem::swap()}。另外,一些特殊的常见名称,例如\texttt{Vec}和\texttt{Result}也被包含在\emph{标准prelude}中并且被自动导入。具体的行为就好像是包括根模块在内的每个模块都以如下导入开始:
\begin{minted}{Rust}
    use std::prelude::v1::*;
\end{minted}

标注的prelude包含一些通用的trait和类型。

在\hyperref[ch02]{第2章}中时,我们提到了库有时会提供叫\texttt{prelude}的模块。但\texttt{std::prelude::v1}是唯一一个自动导入的。将一个模块命名为\texttt{prelude}只是一个惯例,告诉用户他应该导入\texttt{*}。

\subsection{pub use声明}
即使\texttt{use}声明只是别名,它们也可以是公有的:
\begin{minted}{Rust}
    // in plant_structures/mod.rs
    ...
    pub use self::leaves::Leaf;
    pub use self::roots::Root;
\end{minted}

这意味着\texttt{Leaf}和\texttt{Root}是\texttt{plant\_structures}模块里的public item。它们实际上只是\\
\texttt{plant\_structures::leaves::Leaf}和\texttt{plant\_structures::roots::Root}的别名。

标准的prelude就是以一系列\texttt{pub}导入的方式实现的。

\subsection{pub结构体字段}
模块可以包含用户用\texttt{struct}关键字定义的自定义结构体类型。我们将在\hyperref[ch09]{第9章}中介绍细节,但这是一个了解模块如何和结构体字段的可见性交互的好时机。

一个简单的结构体类似这样:
\begin{minted}{Rust}
    pub struct Fern {
        pub roots: RootSet,
        pub stems: StemSet
    }
\end{minted}

一个结构体的所有字段,包括私有字段,都可以在定义结构体的整个模块及其子模块中访问,在模块之外,只有公有的字段才可以被访问。

事实证明通过模块来实现访问控制,而不是像Java和C++那样通过类来实现,对于程序设计将是很大的帮助。不仅能减少重复“getter”和“setter”方法,还能消除类似C++中\texttt{friend}声明的需求。一个模块中可以定义几个紧密结合的类型,例如\texttt{frond::LeafMap}和\\
\texttt{frond::LeafMapIter},让它们能按需互相访问彼此的私有字段,同时对程序中的其他部分隐藏实现的细节。

\subsection{静态量和常量}\label{static}

除了函数、类型和嵌套模块之外,模块里还可以定义\emph{常量}和\emph{静态量}。

\texttt{const}关键字定义常量,语法类似于\texttt{let},除了它必须被标记为\texttt{pub}以及要显式写出类型。还有,为了方便,常量一般都用\texttt{UPPERCASE\_NAMES}:
\begin{minted}{Rust}
    pub const ROOM_TEMPERATURE: f64 = 20.0;     // 摄氏度
\end{minted}

\texttt{static}关键字定义静态量,和\texttt{const}的用法几乎一样:
\begin{minted}{Rust}
    pub static ROOM_TEMPERATURE: f64 = 68.0;    // 华氏温度
\end{minted}

常量有些类似于C++中的\texttt{\#define},值被编译进代码中每一个使用它的地方。一个静态量就是一个在程序开始之前就初始化并持续到程序退出的变量。在代码中使用常量表示幻数和字符串。使用静态量表示更大规模的数据,或者用于需要借用全局常量的引用时。

没有\texttt{mut}常量。静态量可以被标记为\texttt{mut},但就像我们在\hyperref[ch05]{第5章}中讨论的一样,Rust没有任何方法强制实现\texttt{mut}静态量的独占访问。因此,这种变量天生线程不安全,safe代码完全不能使用它们:
\begin{minted}{Rust}
    static mut PACKETS_SERVED: usize = 0;

    println!("{} served", PACKETS_SERVED);  // 错误:使用了可变的静态量
\end{minted}

Rust不鼓励全局可变的状态。关于替代方案的讨论,见\nameref{globalvar}一节。

\section{将程序变为库}

当你的蕨类模拟器完成之后,你发现你需要不止一个程序。假设你还有一个命令行程序运行这个模拟器并把结果保存到文件中。现在,你想要编写其他程序来对保存的结果进行科学分析、实时渲染成长中的植物、渲染逼真的植物等等。所有这些程序都要共享基本的蕨类模拟器的代码。你需要创建一个库。

第一步是将你现有的项目分解为两部分:一个库crate和一个可执行文件。前者包含所有的共享代码,后者包含只有命令行程序才需要的代码。

为了展示怎么做到这一点,让我们给出一个简单粗暴的示例程序:
\begin{minted}{Rust}
    struct Fern {
        size: f64,
        growth_rate: f64
    }

    impl Fern {
        /// 模拟一个蕨类植物一天地成长
        fn grow(&mut self) {
            self.size *= 1.0 + self.growth_rate;
        }
    }

    /// 运行蕨类模拟器模拟几天的变化
    fn run_simulation(fern: &mut Fern, days: usize) {
        for _ in 0 .. days {
            fern.grow();
        }
    }

    fn main() {
        let mut fern = Fern {
            size: 1.0,
            growth_rate: 0.001
        };
        run_simulation(&mut fern, 1000);
        println!("final fern size: {}", fern.size);
    }
\end{minted}

我们假设这个程序有一个普通的\emph{Cargo.toml}文件:
\begin{minted}{toml}
    [package]
    name = "fern_sim"
    version = "0.1.0"
    authors = ["You <you@example.com>"]
    edition = "2018"
\end{minted}

将这个程序变为库是很简单的。只需要如下步骤:
\begin{enumerate}
    \item 将文件\emph{src/main.rs}重命名为\emph{src/lib.rs}
    \item 给\emph{src/lib.rs}里将作为库的公开特性的item添加\texttt{pub}关键字
    \item 暂时把\texttt{main}函数移动到一个别的临时文件中。我们将很快回来处理它。
\end{enumerate}

最后的\emph{src/lib.rs}文件看起来像这样:
\begin{minted}{Rust}
    pub struct Fern {
        pub size: f64,
        pub growth_rate: f64
    }

    impl Fern {
        /// 模拟一个蕨类植物一天地成长
        pub fn grow(&mut self) {
            self.size *= 1.0 + self.growth_rate;
        }
    }

    /// 运行蕨类模拟器模拟几天的变化
    pub fn run_simulation(fern: &mut Fern, days: usize) {
        for _ in 0 .. days {
            fern.grow();
        }
    }
\end{minted}

注意我们不需要更改\emph{Cargo.toml}中的任何内容。因为我们的最精简的\emph{Cargo.toml}会保持Cargo的默认行为。默认情况下,\texttt{cargo build}会查看源代码目录下的文件然后判断要构建什么。当它看到\emph{src/lib.rs}文件,它就知道要构建一个库。

\emph{src/lib.rs}里的代码组成了库的\emph{根模块}。其他使用我们库的crate可以访问根模块里的公有item。

\section{src/bin目录}\label{src/bin}

想让原本的\texttt{fern\_sim}程序再次工作也非常简单:Cargo有一些内建的支持,可以让较小的程序和库存在同一个crate中。

事实上,Cargo自身就是用这种方式编写的。它的大部分代码都在一个Rust库里。我们在本书中至今为止用过的\texttt{cargo}命令行程序只是一个简单的包装程序,它会调用库来完成真正的工作。库和命令行程序\href{https://github.com/rust-lang/cargo}{在同一个源代码仓库中}。

我们也可以把程序和库放在同一个crate中。将以下代码放入\emph{src/bin/efern.rs}的文件中:
\begin{minted}{Rust}
    use fern_sim::{Fern, run_simulation};

    fn main() {
        let mut fern = Fern {
            size: 1.0,
            growth_rate: 0.001
        };
        run_simulation(&mut fern, 1000);
        println!("final fern size: {}", fern.size);
    }
\end{minted}

这个\texttt{main}函数就是我们之前放在一边的那个。我们添加了\texttt{use}声明来使用\texttt{fern\_sim} crate里的一些item:\texttt{Fern}和\texttt{run\_simulation}。也就是说,我们像使用库一样使用了那个crate。

因为我们把这个文件放在了\emph{src/bin}里,所以下次运行\texttt{cargo build}时,Cargo将会同时编译\texttt{fren\_sim}库和这个程序。我们可以使用\texttt{cargo run --bin efern}来运行\texttt{efern}程序。可以通过使用\texttt{--verbose}参数来查看Cargo运行的命令,大概是这样:
\begin{minted}{text}
    $ cargo build --verbose
       Compiling fern_sim v0.1.0 (file:///.../fern_sim)
         Running `rustc src/lib.rs --crate-name fern_sim --crate-type lib ...`
         Running `rustc src/bin/efern.rs --crate-name efern --crate-type bin ...`
    $ cargo run --bin efern --verbose
           Fresh fern_sim v0.1.0 (file:///.../fern_sim)
         Running `target/debug/efern`
    final fern size: 2.7169239322355985
\end{minted}

我们仍然不需要修改\emph{Cargo.toml}中的任何内容,Cargo的默认行为是查看源文件然后进行判断。它会自动把\emph{src/bin}里的\emph{.rs}文件当作额外的程序来构建。

我们也可以在\emph{src/bin}目录下通过子目录来构建更大的程序。假设我们想要提供另一个程序在屏幕上绘制蕨类植物,但绘制代码比较多而且是模块化的,因此它应该有自己的文件。我们可以将第二个程序放在单独的子目录中:
\begin{minted}{text}
    fern_sim/
    |-- Cargo.toml
    |-- src/
        |-- bin/
            |-- efern.rs
            |-- draw_fern/
                |-- main.rs
                |-- draw.rs
\end{minted}

把更大的可执行程序存放在自己的目录中的优势是不会打乱库代码和\texttt{src/bin}目录。

当然,既然现在\texttt{fern\_sim}是一个库,那么我们还有另一种选择。我们可以把这个程序放入单独的项目中,在一个完全独立的目录中,在它的\emph{Cargo.toml}中列出\texttt{fern\_sim}作为依赖:
\begin{minted}{toml}
    [dependencies]
    fern_sim = { path = "../fern_sim" }
\end{minted}

可能这正是你之后编写其他蕨类模拟程序时采用的方法。\emph{src/bin}目录只适合像\texttt{efern}和\\
\texttt{draw\_fern}这样的简单程序。

\section{属性}\label{attribute}

Rust程序中的任何item都可以用\emph{属性}修饰。属性是Rust中向编译器传递指令和建议的语法。例如,假设你得到了如下警告:
\begin{minted}{text}
    libgit2.rs: warning: type `git_revspec` should have a camel case name
        such as `GitRevspec`, #[warn(non_camel_case_types)] on by default
\end{minted}

但是你选择这个名字是有原因的,你并不希望Rust发出警告。你可以通过给类型加上\texttt{\#[allow]}属性来禁用警告:
\begin{minted}{Rust}
    #[allow(non_camel_case_types)]
    pub struct git_revspec {
        ...
    }
\end{minted}

条件编译是另一个通过属性来实现的特性,这个属性叫\texttt{\#[cfg]}:
\begin{minted}{Rust}
    // 只有当我们在构建Android应用时才在项目中包含这个模块
    #[cfg(target_os = "android")]
    mod mobile;
\end{minted}

\texttt{\#[cfg]}的完整语法见\href{https://doc.rust-lang.org/reference/conditional-compilation.html}{Rust参考手册},\hyperref[t8-2]{表8-2}中列出了最常用的选项。

\begin{table}[htbp]
    \centering
    \caption{最常用的\texttt{\#[cfg]}选项}
    \label{t8-2}
    \begin{tabular}{p{0.2\textwidth}p{0.7\textwidth}}
        \hline
        \textbf{\texttt{\#[cfg]}选项}   & \textbf{编译的条件} \\
        \hline
        \texttt{test}   & 测试模式时(用\texttt{cargo test}或者\texttt{rustc --test}编译时) \\
        \rowcolor{tablecolor}
        \texttt{debug\_assertions} & 调试断言开始时(通常在非优化构建中) \\
        \texttt{unix}   & 为Unix系统编译时,包括macOS \\
        \rowcolor{tablecolor}
        \texttt{windows}& 为Windows编译时 \\
        \texttt{target\_pointer\_width = "64"} & 目标平台是64位时。其他可能的值是\texttt{"32"} \\
        \rowcolor{tablecolor}
        \texttt{target\_arch = "x86\_64"} & x86-64架构特定。其他值:\texttt{"x86", "arm", "aarch64", "powerpc", "powerpc64", "mips"} \\
        \texttt{target\_os = "macos"} & macOS特定。其他值:\texttt{"windows", "ios", "android", "linux", "freebsd", "openbsd", "netbsd", "dragonfly"} \\
        \rowcolor{tablecolor}
        \texttt{feature = "robots"} & 启用用户自定义特性\texttt{"robots"}时(用\texttt{cargo build --feature robots}或者\texttt{rustc --cfg feature='"robots"'})。特性在\href{https://doc.rust-lang.org/cargo/reference/manifest.html}{\emph{Cargo.toml}的\texttt{[features]}节中声明}。 \\
        \texttt{not}(\emph(A)) & \emph{A}不满足时。为了让一个函数有两种不同的实现,可以将其中一个标记为\texttt{\#[cfg(X)]},另一个表记为\texttt{\#[cfg(not(X))]}。 \\
        \rowcolor{tablecolor}
        \texttt{all}(\emph{A}, \emph{B}) & 当\emph{A}和\emph{B}都满足时(等价于\texttt{\&\&})。 \\
        \texttt{any}(\emph{A}, \emph{B}) & 当\emph{A}或\emph{B}满足时(等价于\texttt{||})。 \\
    \end{tabular}
\end{table}

有时,我们需要关心函数的内联展开,这是一种我们通常乐意让编译器实现的一种优化。我们可以使用\texttt{\#[inline]}属性来实现内联:
\begin{minted}{Rust}
    /// 调整相互渗透下,两个相邻细胞中的各种离子的浓度
    #[inline]
    fn do_osmosis(c1: &mut Cell, c2: &mut Cell) {
        ...
    }
\end{minted}

在一种特殊情况下,如果没有\texttt{\#[inline]}一定\emph{不会}发生内联。就是当一个crate中的一个函数或方法在另一个crate中被调用时,除非它是泛型的(有泛型参数)或者被显式标记为\texttt{\#[inline]},否则它一定不会被内联展开。

另外,编译器只是把\texttt{\#[inline]}当作一个建议。Rust也支持\texttt{\#[inline(always)]}来要求一个函数必须在每一次被调用时内联展开;和\texttt{\#[inline(never)]}来要求一个函数永远不会被内联展开。

有一些属性,例如\texttt{\#[cfg]}和\texttt{\#[allow]},可以被用于整个模块,然后作用于模块里的所有东西。而另一些,例如\texttt{\#[test]}和\texttt{\#[inline]}只能用于单独的item。正如你可能期待的一样,每个属性都是定制的,并且有自己的一组受支持的参数。Rust的参考手册详细列出了\href{https://doc.rust-lang.org/reference/attributes.html}{所有支持的属性}。

为了将一个属性用于整个crate,可以在\emph{main.rs}或者\emph{lib.rs}文件的最前边添加属性,并用\texttt{\#!}来代替\texttt{\#},像这样:
\begin{minted}{Rust}
    // libgit2_sys/lib.rs
    #![allow(non_camel_case_types)]

    pub struct git_revspec {
        ...
    }

    pub struct git_error {
        ...
    }
\end{minted}

\texttt{\#!}告诉Rust将一个属性附加到作用域里的每一个item,而不是下一个item:在这个例子中,\texttt{\#![allow]}属性作用于整个\texttt{libgit2\_sys} crate,而不仅仅是\texttt{struct git\_revspec}。

\texttt{\#!}还可以在函数体内、结构体定义内等地方使用,但它通常被用在一个文件的开头,来将一个属性附加到整个模块或crate。一些属性总是使用\texttt{\#!}语法,因为它们只能被用于整个crate。

例如,\texttt{\#![feature]}属性可以打开Rust语言和库里的\emph{unstable}特性。这些特性都是实验性的,因此可能会有bug或者可能会在将来修改或者删除。例如,正如下面那段代码一样,Rust对追踪宏展开例如\texttt{assert!}有实验性的支持,但是因为这个支持是实验性的,因此你必须安装nightly版本的Rust并且显式声明你要使用宏追踪的特性:
\begin{minted}{Rust}
    #![feature(trace_macros)]
    
    fn main() {
        // 我想知道assert_eq!实际上到底被替换成了什么代码
        trace_macros!(true);
        assert_eq!(10*10*10 + 9*9*9, 12*12*12 + 1*1*1);
        trace_macros!(false);
    }
\end{minted}

随着时间的推移,Rust队伍有时会\emph{标准化}一个实验特性,这样它就会成为语言标准的一部分。然后\texttt{\#![feature]}特性就会变得多余,Rust会生成一个警告建议你删除它。

\section{测试和文档}

正如我们在\nameref{test}一节中看到的一样,Rust中内建了一个简单的单元测试框架。测试是用\texttt{\#[test]}属性标记的普通函数:
\begin{minted}{Rust}
    #[test]
    fn math_works() {
        let x: i32 = 1;
        assert!(x.is_positive());
        assert_eq!(x + 1, 2);
    }
\end{minted}

\texttt{cargo test}会运行项目中的所有测试:
\begin{minted}{text}
    $ cargo test
       Compiling math_test v0.1.0 (file:///.../math_test)
         Running target/release/math_test-e31ed91ae51ebf22
    
    running 1 test
    test math_works ... ok

    test result: ok. 1 passed; 0 failed; 0 ignored; 0 measuerd; 0 filtered out
\end{minted}
(你也可能会看到一些有关“doc-tests”的输出,我们将在稍后讨论它。)

测试的行为和你的crate是可执行程序还是库没有关系。你可以通过向Cargo传递参数来运行指定的测试:\texttt{cargo test math}会运行名称中包含\texttt{math}的所有测试。

测试通常使用Rust标准库里的\texttt{assert!}和\texttt{assert\_eq!}宏。若\texttt{expr}为真,\texttt{assert!(expr)}断言会成功;否则它会panic,并导致测试失败。\texttt{assert\_eq!(v1, v2)}类似于\texttt{assert!(v1 == v2)},除了当断言失败时,它会同时打印出两个操作数的值。

也可以在普通的代码中使用这两个宏,但注意即使是release构建\texttt{assert!}和\texttt{assert\_eq!}也会被包含进去。使用\texttt{debug\_assert!}和\texttt{debug\_assert\_dq!}来编写只在debug构建时才会检查的断言。

为了测试错误的情况,需要向测试添加\texttt{\#[should\_panic]}属性:
\begin{minted}{Rust}
    /// 正如我们在上一章中说的一样,
    /// 只有当除0会导致panic时这个测试才能通过
    #[test]
    #[allow(unconditional_panic, unused_must_use)]
    #[should_panic(expected="divide by zero")]
    fn test_divide_by_zero_error() {
        1 / 0;  // 应该panic!
    }
\end{minted}

在这个例子中,我们需要添加\texttt{allow}属性来告诉编译器允许我们写一些可以静态证明一定会panic的代码,然后进行除法并丢弃结果值。因为通常情况下,编译器会阻止这种愚蠢的代码。

你也可以在测试中返回\texttt{Result<(), E>}。只要错误类型实现了\texttt{Debug} trait(这是通常的情况),你就可以使用\texttt{?}略过\texttt{Ok}的情况返回一个\texttt{Result}:
\begin{minted}{Rust}
    use std::num::ParseIntError;

    /// 只有当"1024"是个有效的数字时这个测试才能通过
    #[test]
    fn main() -> Result<(), ParseIntError> {
        i32::from_str_radix("1024", 10)?;
        Ok(())
    }
\end{minted}

用\texttt{\#[test]}标记的函数会条件编译。普通的\texttt{cargo build}或者\texttt{cargo build --release}会跳过测试代码。但当你运行\texttt{cargo test}时,Cargo会编译程序两次:一次是用普通方式编译,另一次是在启用测试和测试工具的情况下编译。这意味着如果需要的话,你的单元测试可以和要测试的代码放在一起,这样可以访问内部的实现细节,并且没有运行时开销。然而,这可能会导致一些警告。例如:
\begin{minted}{Rust}
    fn roughly_equal(a: f64, b: f64) -> bool {
        (a - b).abs() < 1e-6
    }

    #[test]
    fn trig_works() {
        use std::f64::const::PI;
        assert!(roughly_equal(PI.sin(), 0.0));
    }
\end{minted}

在省略测试代码的构建中,\texttt{roughly\_equal}可能未被使用,Rust会警告:
\begin{minted}{text}
    $ cargo build
       Compiling math_test v0.1.0 (file:///.../math_test)
    warning: function is never used: `roughly_equal`
     --> src/crates_unused_testing_function.rs:7:1
      |
    7 | / fn roughly_equal(a: f64, b: f64) -> bool {
    8 | |     (a - b).abs() < 1e-6    
    9 | | }
      | |_^
      |
       = note: #[warn(dead_code)] on by default
\end{minted}

因此,当你的测试足够复杂需要支持的代码的时候,将它们放在一个\texttt{tests}模块中并使用\texttt{\#[cfg]}属性将整个模块声明为测试模式特定:
\begin{minted}{Rust}
    #[cfg(test)]    // 只有在测试时才包含这个模块
    mod tests {
        fn roughly_equal(a: f64, b: f64) -> bool {
            (a - b).abs() < 1e-6
        }

        #[test]
        fn trig_works() {
            use std::f64::const::PI;
            assert!(roughly_equal(PI.sin(), 0.0));
        }
    }
\end{minted}

Rust的测试工具默认会使用多线程同时运行多个测试,这是你的Rust代码线程安全的一个附带的好处。为了禁用并行测试,运行\texttt{cargo test -- --test-threads 1}。(第一个\texttt{--}确保\texttt{cargo test}会把\texttt{--test-threads}选项传递给测试的可执行程序)如果想运行单个测试,运行\texttt{cargo test testname}。这意味着从技术上讲,我们在\hyperref[ch02]{第2章}中展示的曼德勃罗程序并不是那一章中的第二个多线程程序,而是第三个!\nameref{test}一节中运行的\texttt{cargo test}才是第一个。

通常来说,测试工具只会显示失败的测试的输出。为了显示通过的测试的输出,可以运行\texttt{cargo test -- --no-capture}。

\subsection{集成测试}
你的蕨类模拟器的规模仍然在增长。你决定将所有主要的代码放进一个库里,这样可以被很多可执行程序使用。模拟最终使用\emph{fern\_sim.rlib}库的用户使用它的方式对它进行一些测试将是一个好主意。另外,你还有一些首先从二进制文件中加载保存的模拟器的测试,将这些很大的测试文件保存在\texttt{src}目录中是很尴尬的。集成测试可以帮忙解决这两个问题。

集成测试是\texttt{src}目录下的\emph{tests}目录中的\emph{.rs}文件。当你运行\texttt{cargo test}时,Cargo会将每一个集成测试编译成独立的crate,然后和你的库以及Rust的测试工具链接。这里有一个例子:
\begin{minted}{Rust}
    // tests/unfurl.rs - Fiddleheads unfurl in sunlight

    use fern_sim::Terrarium;
    use std::time::Duration;

    #[test]
    fn test_fiddlehead_unfurling() {
        let mut world = Terrarium::load("tests/unfurl_files/fiddlehead.tm");
        assert!(world.fern(0).is_furled());
        let one_hour = Duration::from_secs(60 * 60);
        world.apply_sunlight(one_hour);
        assert!(world.fern(0).is_fully_unfurled());
    }
\end{minted}

集成测试之所以有价值,部分原因是它们从crate外部看待你的crate,就像一个用户一样。它们测试crate的public API。

\texttt{cargo test}会同时运行单元测试和集成测试。为了只运行特定文件中的集成测试——例如\emph{tests/unfurl.rs}——使用命令\texttt{cargo test --test unfurl}。

\subsection{文档}
命令\texttt{cargo doc}为你的库生成HTML文档:
\begin{minted}{text}
    $ cargo doc --no-deps --open
     Documenting fern_sim v0.1.0 (file:///.../fern_sim)
\end{minted}

\texttt{--no-deps}选项告诉Cargo只为\texttt{fern\_sim}自身生成文档,不要为它依赖的crate生成文档。

\texttt{--open}选项告诉Cargo稍后在你的浏览器中打开文档。

结果如\hyperref[f8-2]{图8-2}所示。Cargo会把新的文档文件保存在\emph{target/doc}中,起始页面是\\
\emph{target/doc/fern\_sim/index.html}。

\begin{figure}[htbp]
    \centering
    \includegraphics[width=0.9\textwidth]{../img/f8-2.png}
    \caption{\texttt{rustdoc}生成的文档示例}
    \label{f8-2}
\end{figure}

文档从你的库中的\texttt{pub}特性生成,加上你为它们编写的任何\emph{文档注释}。我们在这一章中已经看到过一些文档注释了。它们看起来像普通的注释:
\begin{minted}{Rust}
    /// 模拟减数分裂产生孢子的过程。
    pub fn produce_spore(factory: &mut Sporangium) -> Spore {
        ...
    }
\end{minted}
但当Rust看到它们是以三个斜杠开头时,它会把它们当作\texttt{\#[doc]}属性。Rust将上面的示例看作和下面的代码等价:
\begin{minted}{Rust}
    #[doc = "模拟减数分裂产生孢子的过程。"]
    pub fn produce_spore(factory: &mut Sporangium) -> Spore {
        ...
    }
\end{minted}

当你编译一个库或可执行程序时,这些属性并不会改变任何东西,但当你生成文档时,public特性的文档注释将会被包含在输出中。

同样的,以\texttt{//!}开头的注释被当作\texttt{\#![doc]}属性,这个属性会被附加到作用域中,通常是一个模块或者crate。例如,你的\emph{fern\_sim/src/lib.rs}文件看起来可能像这样:
\begin{minted}{Rust}
    //! 从独立的细胞开始,
    //! 模拟蕨类植物的生长。
\end{minted}

文档注释的内容被当做Markdown,一种简单HTML格式的速记符号。星号可以用作\texttt{*\emph{italics}*}和\texttt{**\textbf{bold type}**},空行被当作段落结束,等等。你也可以包含HTML标签,它们会被逐字符拷贝到格式化文档。

Rust中的文档注释的一个特殊特性是Markdown链接可以使用Rust的item路径而不是相对URL来指明它们引用的item,例如\texttt{leaves::Leaf}。Cargo将会查找路径指向的内容,并创建指向正确文档页面中正确位置的链接。例如,这段代码生成的文档会链接到\texttt{VascularPath}、\texttt{Leaf}和\texttt{Root}的页面:
\begin{minted}{Rust}
    /// 创建并返回一个 [`VascularPath`] ,
    /// 它代表营养从给定的 [`Root`][r] 传输到给定的 [`Leaf`](leaves::Leaf) 的路径。
    ///
    /// [r]: roots::Root
    pub fn trace_path(leaf: &leaves::Leaf, root: &roots::Root) -> VascularPath {
        ...
    }
\end{minted}

你还可以添加搜索别名来使用内建的搜索特性让item更容易被找到。在crate的文档中搜索“path”或者“route”将会导向\texttt{VascularPath}:
\begin{minted}{Rust}
    #[doc(alias = "route")]
    pub struct VascularPath {
        ...
    }
\end{minted}

你可以使用\texttt{`backticks`}在文本中间创建代码片段。输出时,这些片段将会用等宽字体显示。更大的代码示例可以通过缩进四个空格来显示:
\begin{minted}{Rust}
    /// 文档注释中的一个代码块
    ///
    ///    if samples::everything().works() {
    ///        println!("ok");
    ///    }
\end{minted}
你也可以使用Markdown-fenced代码块,效果和上面一样:
\begin{minted}{Rust}
    /// 另一个片段,但是写法不同:
    /// 
    /// ```
    /// if samples::everything().works() {
    ///     println!("ok");
    /// }
    /// ```
\end{minted}

不管你用哪种方式,当你在文档注释中包含代码块的时候会发生一件有趣的事:Rust会自动把它当做一个测试。

\subsection{文档测试}
当你运行Rust库crate中的测试时,Rust会通过检查确保文档中出现的所有代码都能运行并正常工作。它将文档注释中出现的所有代码作为单独的可执行程序crate编译,然后和库链接并运行。

这里有一个文档测试的例子。运行\texttt{cargo new --lib ranges}(\texttt{--lib}参数告诉Cargo我们要创建一个库crate,而不是可执行程序crate)并将下列代码加到\emph{ranges/src/lib.rs}中:
\begin{minted}{Rust}
    use std::ops::Range;

    /// 如果两个范围重叠就返回true
    ///
    ///     assert_eq!(ranges::overlap(0..7, 3..10), true);
    ///     assert_eq!(ranges::overlap(1..5, 101..105), false);
    ///
    /// 如果有一个范围是空的,视为不重叠。
    ///
    ///     assert_eq!(ranges::overlap(0..0, 0..10), false);
    ///
    pub fn overlap(r1: Range<usize>, r2:Range<usize>) -> bool {
        r1.start < r1.end && r2.start < r2.end &&
            r1.start < r2.end && r2.start < r1.end
    }
\end{minted}

文档注释中的这两个小代码块将会出现在\texttt{cargo doc}生成的文档中,如\hyperref[f8-3]{图8-3}所示。

\begin{figure}[htbp]
    \centering
    \includegraphics[width=0.8\textwidth]{../img/f8-3.png}
    \caption{文档注释会展示一些文档测试}
    \label{f8-3}
\end{figure}

它们也会成为两个单独的测试:
\begin{minted}{text}
    $ cargo test
       Compiling ranges v0.1.0 (file:///.../ranges)
    ...
       Doc-tests ranges

    running 2 tests
    test overlap_0 ... ok
    test overlap_1 ... ok

    test result: ok. 2 passed; 0 failed; 0 ignored; 0 measuerd; 0 filtered out
\end{minted}

如果你向Cargo传递\texttt{--verbose}参数,你将会看到它用\texttt{rustdoc --test}来运行这两个测试。\texttt{rustdoc}会把每个示例代码存储在单独的文件中,加上少量的样本代码来生成两个程序。这是第一个:
\begin{minted}{Rust}
    use ranges;
    fn main() {
        assert_eq!(ranges::overlap(0..7, 3..10), true);
        assert_eq!(ranges::overlap(1..5, 101..105), false);
    }
\end{minted}

这是第二个:
\begin{minted}{Rust}
    use ranges;
    fn main() {
        assert_eq!(ranges::overlap(0..0, 0..10), false);
    }
\end{minted}
如果程序能成功编译和运行,那么就可以通过测试。

这两段示例代码都包含了断言,但那只是因为在这种情况下,断言适用于编写文档。文档测试背后的目的不是让你把所有测试都放在注释里,而是让你写出最可能的用法的示例,然后Rust确保你文档中的代码示例确实可以编译并运行。

通常一个最小化的能工作的示例还包含一些细节,例如导入或初始化代码。这些是让代码能编译所必需的,但并不重要,不应该在文档中显示。为了隐藏代码示例中的某一行,可以在行的开始加上\texttt{\#}和一个空格:
\begin{minted}{Rust}
    /// 开启光照,并运行模拟器一段时间
    ///
    ///     # use fern_sim::Terrarium;
    ///     # use std::time::Duration;
    ///     # let mut tm = Terrarium::new();
    ///     tm.apply_sunlight(Duration::from_secs(60));
    ///
    pub fn apply_sunlight(&mut self, time: Duration) {
        ...
    }
\end{minted}

有时在文档中展示一个包含\texttt{main}函数的完整的示例程序会很有帮助。如果那些细节代码已经出现在了你的代码示例中,那你显然不会再希望\texttt{cargo}自动添加它们,因为这样会导致无法编译。因此\texttt{rustdoc}将所有包含字符串\texttt{fn main}的代码块视为完整的程序,不会再添加任何东西。

可以为特定的代码块禁用测试。为了告诉Rust编译你的示例,但并不真的运行它,可以使用带有\texttt{no\_run}注解的fenced代码块:
\begin{minted}{Rust}
    /// 将所有的本地培养皿上传到在线的gallery。
    ///
    /// ```no_run
    /// let mut session = fern_sim::connect();
    /// session.upload_all();
    /// ```
    pub fn upload_all(&mut self) {
        ...
    }
\end{minted}

如果甚至不希望代码编译,可以将\texttt{no\_run}替换为\texttt{ignore}。使用\texttt{ignore}标记的块将不会出现在\texttt{cargo test}的输出中,但\texttt{no\_run}标记的块会出现在测试的输出中,如果能编译就能通过测试。如果代码块完全不是Rust代码,那么使用语言的名字来标记。例如\texttt{c++}或\texttt{sh},或者普通文本时用\texttt{text}。\texttt{rustdoc}并不知道几百种编程语言的名字,因此,它将所有不认识的注解都当作非Rust的代码块,这样会禁用代码高亮和文档测试。

\section{指定依赖}\label{depend}

我们已经看到过一种方式来告诉Cargo你的项目依赖的crate和版本号。
\begin{minted}{toml}
    image = "0.6.1"
\end{minted}

有若干种方式可以指定依赖,并且你可能想问一些类似使用哪个版本的细节问题,因此值得花几页篇幅讨论这个问题。

首先,假设你想使用没有在crates.io上发布的依赖。一种方式是指明Git仓库的URL和修订号:
\begin{minted}{toml}
    image = { git = "https://github.com/Piston/image.git", rev = "528f19c" }
\end{minted}

这个crate是GitHub上的开源仓库,但你也可以指明一个在你自己公司内网里的私有Git仓库。正如这里展示的一样,你可以指定要使用的特定的\texttt{rev}、\texttt{tag}或者\texttt{branch}。(这三种方式都可以告诉Git要checkout到源代码的哪个版本。)

另一种方式是指定一个含有crate源码的目录:
\begin{minted}{toml}
    image = { path = "vendor/image" }
\end{minted}

当你的团队使用单个版本控制仓库来包含若干个甚至整个依赖图的所有crate时,这种方式会很方便。每个crate都可以使用相对路径指定它的依赖。

这种细粒度的依赖控制方式是一种很强大的功能。如果你发现任何开源的crate都不完全符合你的要求,你可以fork它:点击GitHub上的Fork按钮,然后更改\emph{Cargo.toml}中的一行。这样你的下一次\texttt{cargo build}命令将会使用你fork的版本而不是官方的版本。

\subsection{版本}
当你在\emph{Cargo.toml}中写下类似\texttt{image = "0.13.0"}这样的内容时,Cargo会以比较宽松的方式解释它。它会使用和0.13.0版本兼容的最新版本。

兼容性规则改编自\href{https://semver.org/}{语义版本号}。

\begin{itemize}
    \item 0.0开头的版本号太过原始,因此Cargo假设它和任意版本都不兼容。
    \item 一个以0.\emph{x}开头的版本号,其中\emph{x}是非零数字,将会被认为和其他0.\emph{x}系列的版本兼容。\footnote{译者注:即\emph{0.x.a}和\emph{0.x.b}兼容,但\emph{0.x.a}和\emph{0.y.a}不兼容。}我们指定了\texttt{image}的版本是0.6.1,但Cargo有可能会使用0.6.3的版本。(这并不是语义版本号标准对0.\emph{x}版本号的说明,但这个规则被证明很有用,因此需要保留。)
    \item 一旦一个项目到达了1.0,则只有主版本号会打破兼容性。因此如果你要求2.0.1版本,Cargo可能会使用2.17.99版本,但不会是3.0。
\end{itemize}

版本号默认是弹性的,因为如果不这样,那么使用哪个版本的问题将会被过度约束。假设库\texttt{libA}使用了\texttt{num = "0.1.31"},而库\texttt{libB}使用了\texttt{num = "0.1.29"}。如果版本号要求精确匹配,将没有任何项目可以同时使用这两个库。允许Cargo使用任何兼容的版本是更加实用的默认配置。

不同的项目对依赖和版本有不同的要求。你可以使用运算符指定一个精确的版本号或者一个版本范围,如\hyperref[t8-3]{表8-3}所示。

\begin{table}[htbp]
    \centering
    \caption{在Cargo.toml文件中指明版本}
    \label{t8-3}
    \begin{tabular}{p{0.2\textwidth}p{0.7\textwidth}}
        \hline
        \textbf{Cargo.toml文件} & \textbf{含义} \\
        \hline
        \texttt{iamge = "=0.10.0"}  & 只使用精确的0.10.0版本 \\
        \rowcolor{tablecolor}
        \texttt{image = ">=1.0.5"}  & 使用1.0.5或者更高的版本(即使是2.9,如果可用的话) \\
        \texttt{image = ">1.0.5 <1.1.9"} & 使用比1.0.5高,但比1.1.9低的版本 \\
        \rowcolor{tablecolor}
        \texttt{image = "<=2.7.10"} & 使用任何小于等于2.7.10的版本 \\
    \end{tabular}
\end{table}

另一种你偶尔可能会见到的版本指定方式是通配符\texttt{*},它告诉Cargo可以使用任何版本。除非有别的\emph{Cargo.toml}文件中指定了更加具体的版本约束,否则Cargo将会使用最新的可用版本。\href{https://doc.rust-lang.org/cargo/reference/specifying-dependencies.html}{doc.crates.io上的Cargo文档}更加详细地介绍了版本指定的内容。

注意兼容性规则意味着不能纯粹出于营销原因选择版本号。它们必须有真实的含义。这是crate的维护者和用户之间的约定。如果你正在维护一个1.7版本的crate,并且你决定删除一个函数或者做出其他不能完全向后兼容的更改,你必须将版本号提升到2.0。如果你把版本号更新到1.8,那么你就等于是在宣称新版本和1.7版本兼容,你的用户可能会发现他们构建时出现错误。

\subsection{Cargo.lock}
\emph{Cargo.toml}中的版本号故意设计的比较灵活,然而我们并不需要每次构建时都让Cargo把库更新到最新版本。想象一下你正在一个紧张的调试会话中,突然一次\texttt{cargo build}把库更新到了最新的版本。这可能会造成很大的破坏,在调试途中任何的改变都会导致问题。事实上,任何时候更新库都不是一个好时机。

Cargo也有一个内建的机制来解决这个问题。当你第一次构建项目时,Cargo会输出一个\emph{Cargo.lock}文件记录下它使用的每一个crate的精确版本。之后的构建将会查询这个文件并继续使用相同的版本。只有当你告诉Cargo更新版本时它才会这么做,你可以手动更改\emph{Cargo.toml}文件中的版本号或者运行\texttt{cargo update}:
\begin{minted}{text}
    $ cargo update
        Updating registry `https://github.com/rust-lang/crates.io-index`
        Updating libc v0.2.7 -> v0.2.11
        Updating png v0.4.2 -> v0.4.3
\end{minted}

\texttt{cargo update}只会更新到和你在\emph{Cargo.toml}中指定的版本兼容的最新版本。如够你指定了\texttt{image = "0.6.1"},然后你想更新到0.10.0,那你必须在\emph{Cargo.toml}中更改版本号。当你下一次构建时,Cargo会更新到\texttt{image}库的新版本,然后把新的版本号存储在\emph{Cargo.lock}中。

上面的示例展示了Cargo更新了两个crates.io上的crate。存储在Git中的依赖也会发生类似的事情。假设我们的\emph{Cargo.toml}文件包含如下内容:
\begin{minted}{toml}
    image = { git = "https://github.com/Piston/image.git", branch = "master" }
\end{minted}

如果已经有了\emph{Cargo.lock}文件,那么\texttt{cargo build}将不会从Git仓库pull新的更改。它会读取\emph{Cargo.lock}并且每次都使用和上次相同的版本。但\texttt{cargo update}将会pull最新的\texttt{master}分支,因此下一次构建将会使用最新的版本。

\emph{Cargo.lock}是自动生成的,通常你不需要手动编辑它。尽管如此,如果你的项目是可执行程序,你应该把\emph{Cargo.lock}提交到版本控制系统。这样,任何构建你的项目的人都会使用完全相同的版本。你的\emph{Cargo.lock}文件的历史将会记录依赖的更新。

如果你的项目是普通的Rust库,不要提交\emph{Cargo.lock}。你的库的下游用户将会有自己的包含整个依赖图的版本信息的\emph{Cargo.lock}文件,他们会忽略你的库中的\emph{Cargo.lock}文件。在少数情况下,你的项目可能是一个共享库(即输出是\emph{.dll}、\emph{.dylib}、\emph{.so}文件),这时将不会有这种下游的用户,因此你应该提交\emph{Cargo.lock}。

\emph{Cargo.toml}的弹性版本声明让你可以更容易地在自己的项目中使用Rust库,并能最大化库之间的兼容性。\emph{Cargo.lock}的记录可以保持一致性,在不同机器之间复现相同的构建。它们通过协作能有效地帮助你避免依赖地狱。

\section{把crate发布到crates.io}

你已经决定了将你的蕨类模拟器库作为开源软件发布。祝贺你!这个过程非常的简单。

首先,确保Cargo可以为你打包crate。

\begin{minted}{text}
    $ cargo package
    warning: manifest has no description, license, license-file, documentation,
    homepage or repository. See http://doc.crates.io/manifest.html#package-metadata
    for more info.
       Packaging fern_sim v0.1.0 (file:///.../fern_sim)
       Verifying fern_sim v0.1.0 (file:///.../fern_sim)
       Compiling fern_sim v0.1.0 (file:///.../fern_sim/target/package/fern_sim-0.1.0)
\end{minted}

\texttt{cargo package}命令会产生一个文件(这里是\emph{target/package/fern\_sim-0.1.0.crate}),这个文件包含你库里的所有源文件,包括\emph{Cargo.toml}。这是你将要上传到crates.io与全世界分享的文件。(你可以使用\texttt{cargo package --list}来查看包含了哪些文件。)Cargo会进行二次检查,像你的最终用户一样从\emph{.crate}文件构建你的库,来确保它能正常工作。

Cargo警告说\emph{Cargo.toml}的\texttt{[package]}节缺少了一些对下游用户来说很重要的信息,例如你分发代码所依据的许可证。警告中的URL是非常优秀的资源,因此我们不会在这里详细解释所有的字段。简单来说,你可以通过在\emph{Cargo.toml}中添加几行内容来修复警告:
\begin{minted}{toml}
    [package]
    name = "fern_sim"
    version = "0.1.0"
    edition = "2018"
    authors = ["You <you@example.com>"]
    license = "MIT"
    homepage = "https://fernsim.example.com/"
    repository = "https://gitlair.com/sporeador/fern_sim"
    documentation = "http://fernsim.example.com/docs"
    description = """
    Fern simulation, from the cellular level up.
    """
\end{minted}

\begin{note}
    一旦你在crates.io上发布了这个crate,任何下载了你的crate的人都可以看到\emph{Cargo.toml}文件。因此如果\texttt{authors}字段包含了一个你想保持私有的电子邮箱地址,现在是时候修改它了。
\end{note}

这个阶段有时会出现的另一个问题是你的\emph{Cargo.toml}文件中可能通过\texttt{path}指定了别的crate的位置,正如\nameref{depend}中展示的一样:
\begin{minted}{toml}
    image = { path = "vendor/image" }
\end{minted}

对你和你的团队来说,这可能能正常工作。但显然当其他人下载了\texttt{fern\_sim}库之后,他们的计算机上不会有和你的计算机上一样的文件和目录。因此Cargo会\emph{忽略}自动下载的库中的\texttt{path}字段,这可能会导致构建错误。然而解决方案也是很直观的:如果你打算在crates.io上发布你的库,那么它的依赖也应该发布在crates.io上。将\texttt{path}替换为版本号来指定依赖:
\begin{minted}{toml}
    image = "0.13.0"
\end{minted}

如果你喜欢的话,你可以既指定一个\texttt{path}用于你自己本地的构建,同时为其它用户指定一个\texttt{version}:
\begin{minted}{toml}
    image = { path = "vendor/image", version = "0.13.0" }
\end{minted}

当然,这种情况下保持两种版本的同步是你自己的责任。

最后,在发布一个crate之前,你需要登录进crates.io并获取一个API key。这一步也很直观:当你在crates.io上创建了账号之后,你的“Account Settings”页面将会显示一个\texttt{cargo login}命令,像这样:
\begin{minted}{text}
    $ cargo login 5j0dV54BjlXBpUUbfIj7G9DvNl1vsWW1
\end{minted}

Cargo会把key保存在一个配置文件中,API key应该像密码一样保持私密。因此最好只在你自己的计算机上运行这个命令。

这些都完成之后,最后的步骤是运行\texttt{cargo publish}:
\begin{minted}{text}
    $ cargo publish
        Updating registry `https://github.com/rust-lang/crates.io-index`
       Uploading fern_sim v0.1.0 (file:///.../fern_sim)
\end{minted}

完成之后,你的库就加入了crates.io上的其他数以千计的crate中。

\section{工作空间}

随着项目的持续增长,你最终会编写很多crate。它们在一个单独的源代码仓库中逐个排列:
\begin{minted}{text}
    fernsoft/
    |-- .git/...
    |-- fern_sim/
    |   |-- Cargo.toml
    |   |-- Cargo.lock
    |   |-- src/...
    |   |-- target/...
    |
    |-- fern_img/
    |   |-- Cargo.toml
    |   |-- Cargo.lock
    |   |-- src/...
    |   |-- target/...
    |
    |-- fern_video/
        |-- Cargo.toml
        |-- Cargo.lock
        |-- src/...
        |-- target/...
\end{minted}

Cargo工作的方式是每一个crate都有自己的构建目录\texttt{target},每一个\texttt{target}目录都包含了这个crate的所有依赖的单独的构建。这些构建是完全独立的。即使两个crate有相同的依赖,它们也不会共享任何编译生成的代码。这是一种浪费。

你可以通过使用Cargo\emph{工作空间}来节省编译时间和磁盘空间,一些crate可以共享一个公共的构建目录和\emph{Cargo.lock}文件。

你需要做的所有事就是在仓库的根目录创建一个\emph{Cargo.toml}文件,然后添加几行内容:
\begin{minted}{toml}
    [workspace]
    members = ["fern_sim", "fern_img", "fern_video"]
\end{minted}
这里\texttt{fern\_sim}等是包含你的crate的子目录的名称。然后删除这些子目录里的\emph{Cargo.lock}文件和\texttt{target}目录。

修改完之后,在任何crate中运行\texttt{cargo build}将会自动创建并使用根目录下的共享构建目录(这个例子中是\texttt{fernsoft/target})。命令\texttt{cargo build --workspace}会构建当前工作空间内的所有crate。\texttt{cargo test}和\texttt{cargo doc}也接受\texttt{--workspace}选项。

\section{更多有趣的事}

如果你还不满意,Rus社区还为你准备了更多东西:
\begin{itemize}
    \item 当你在\href{https://crates.io}{crates.io}上发布了开源的crate之后,你的文档将会被自动呈现并托管在\emph{doc.rs}上,这要感谢Onur Aslan。
    \item 如果你的项目是在GitHub上,Travis CI可以在你每一次提交时构建并测试你的代码。它非常容易部署,详情见\href{https://travis-ci.org}{travis-ci.org}。如果你已经对Travis很熟悉了,这个\emph{.travis.yml}文件可以帮助你上手:
    \begin{minted}{yaml}
    language: rust
    rust:
      - stable
    \end{minted}
    \item 你可以从你的crate的顶层文档注释生成一个\emph{README.md}文件。这个特性由Livio Ribeiro以第三方Cargo插件的形式提供。运行\texttt{cargo install cargo-readme}来安装这个插件,运行\texttt{cargo readme --help}来学习如何使用它。
\end{itemize}

我们可以继续了。

Rust是一门新语言,但它设计的目的之一是支持大型的、复杂的项目。它有强大的工具和活跃的社区。系统程序员将迎来春天。

    \chapter{结构体}\label{ch09}

\section{使用impl定义方法}\label{method}

\section{内部可变性}\label{intermut}

    \chapter{枚举与模式}\label{ch10}

\emph{Surprising how much computer stuff makes sense viewed as tragic deprivation of sum types (cf. deprivation of lambdas).}

\begin{flushright}
    ——Graydon Hoare
\end{flushright}

这一章的第一个话题将是一个古老的、强有力的、可以帮你在短期内完成很多工作的、并且在许多语言中以不同的名字广为人知的特性。但它并不是魔鬼。而是一种用户自定义的数据类型,它是ML和Haskell程序员们熟知的和类型、也是互斥的联合、还是代数数据类型。在Rust中,它们被称为\emph{枚举(enumerations)},或者简写为\emph{enum}。和魔鬼不同的是,它们非常安全、索取的代价也很小。

C++和C\#都有枚举,你可以使用它们来定义自己的类型,这种类型的取值范围是一些命名常量的集合。例如,你可能定义过一个叫\texttt{Color}的类型,取值范围为\texttt{Red}、\texttt{Orange}、\texttt{Yellow}等等。这种枚举在Rust中也能工作,但Rust进一步扩展了枚举。一个Rust枚举可以包含数据,包括多种不同类型的数据。例如,Rust的\texttt{Result<String, io::Error}类型是一个枚举;这样一个值要么是一个包含\texttt{String}的\texttt{Ok}值要么是一个包含\texttt{io::Error}的\texttt{Err}值。这就超出了C++和C\#中枚举的能力。它更像C中的\texttt{union}——但和联合不同的是,Rust的枚举是类型安全的。

枚举适用于一个值有多种可能的情况。使用它们的“代价”是你必须使用模式匹配来安全地访问数据,这也是我们这一章中的第二个话题。

如果你使用过Python的解包或者JavaScript中的解构,那你可能觉得模式也很熟悉,但Rust同样扩展了模式。Rust的模式有点像匹配数据的正则表达式。它们被用来测试一个值是否具有特定的期望的形态。它们可以一次从结构体或这元组中提取出多个字段存入局部变量。并且和正则表达式类似,它们很简洁,通常只用单行代码就能完成任务。

这一章将以枚举的基础开始,展示数据怎么被关联到枚举选项以及枚举是怎么存储在内存中的。然后我们会展示Rust的模式和\texttt{match}表达式如何简洁地指定基于枚举、结构体、数组、切片的逻辑。模式也可以包含引用、move和\texttt{if}条件,这让它们的功能更加强大。

\section{枚举}\label{enum}

简单的C风格枚举非常直观:
\begin{minted}{Rust}
    enum Ordering {
        Less,
        Equal,
        Greater,
    }
\end{minted}

这里声明了一个有三个可能的值的\texttt{Ordering}类型,这些值被称为\emph{variant}或者\emph{constructor}:\texttt{Ordering::Less}、\texttt{Ordering::Equal}、\texttt{Ordering::Greater}。这个枚举是标准库的一部分,因此Rust代码可以导入它:
\begin{minted}{Rust}
    use std::cmp::Ordering;

    fn compare(n: i32, m: i32) -> Ordering {
        if n < m {
            Ordering::Less
        } else if n > m {
            Ordering::Greater
        } else {
            Ordering::Equal
        }
    }
\end{minted}

或者它的所有constructor:
\begin{minted}{Rust}
    use std::cmp::Ordering::{self, *};  // `*`意思是导入所有的子item

    fn compare(n: i32, m: i32) -> Ordering {
        if n < m {
            Less
        } else if n > m {
            Greater
        } else {
            Equal
        }
    }
\end{minted}

导入constructor之后,我们可以写\texttt{Less}来代替\texttt{Ordering::Less}等,但因为这样不够明显,因此一般认为\emph{不要}导入它们式更好的风格,除非它能是你的代码的可读性更强。

为了导入一个在当前模块中声明的枚举的constructor,可以使用\texttt{self}:
\begin{minted}{Rust}
    enum Pet {
        Orca,
        Giraffe,
        ...
    }

    use self::Pet::*;
\end{minted}

在内存中,C风格的枚举值被存储为整数。有时告诉Rust使用哪些整数会很有用:
\begin{minted}{Rust}
    enum HttpStatus {
        Ok = 200,
        NotModified = 304,
        NotFound = 404,
        ...
    }
\end{minted}

否则,Rust会从0开始自动分配值。

默认情况下,Rust用能容纳所有值的最小的内建整数类型来存储C风格枚举。大多数情况下都是一个单独的字节:
\begin{minted}{Rust}
    use std::mem::size_of;
    assert_eq!(size_of::<Ordering>(), 1);
    assert_eq!(size_of::<HttpStatus>(), 2); // 404不能存储在u8中
\end{minted}

你可以通过添加\texttt{\#[repr]}属性来覆盖Rust选择的内存表示方式。更多的细节见“\nameref{repr}”。

将C风格的枚举转换为整数是允许的:
\begin{minted}{Rust}
    assert_eq!(HttpStatus::Ok as i32, 200);
\end{minted}

然而,反过来把整数转换为枚举是不允许的。和C和C++不同,Rust保证枚举的值只能是\texttt{enum}生命中列出的值之一。未经检查的从整数类型到枚举类型的转换会打破这种保证,所以它是不允许的。你可以写出你自己的带检查的版本:
\begin{minted}{Rust}
    fn http_status_from_u32(n: u32) -> Option<HttpStatus> {
        match n {
            200 => Some(HttpStatus::Ok),
            304 => Some(HttpStatus::NotModified),
            404 => Some(HttpStatus::NotFound),
            ...
            _ => None,
        }
    }
\end{minted}

或者使用\href{https://crates.io/crates/enum_primitive}{\texttt{enum\_primitive}} crate。它包含一个宏可以为你自动生成这种类型的转换代码。

和结构体一样,编译器也可以为你自动生成类似\texttt{==}运算符这样的特性,但你需要显式地要求这样:
\begin{minted}{Rust}
    #[derive(Copy, Clone, Debug, PartialEq, Eq)]
    enum TimeUnit {
        Seconds, Minutes, Hours, Days, Months, Years,
    }
\end{minted}

枚举也和结构体一样可以拥有方法:
\begin{minted}{Rust}
    impl TimeUnit {
        /// 返回该时间单位的复数名词。
        fn plural(self) -> &'static str {
            match self {
                TimeUnit::Seconds => "seconds",
                TimeUnit::Minutes => "minutes",
                TimeUnit::Hours => "hours",
                TimeUnit::Days => "days",
                TimeUnit::Months => "months",
                TimeUnit::Years => "years",
            }
        }

        /// 返回该时间单位的单数名词。
        fn singular(self) -> &'static str {
            self.plural().trim_end_matches('s')
        }
    }
\end{minted}

C风格的枚举就这么多内容了。Rust中最有趣的一类枚举是那些带有数据的枚举。我们将展示这些枚举如何存储在内存中、如何通过添加类型参数将它们变为泛型的,以及如何通过枚举构建复杂的数据结构。

\subsection{带有数据的枚举}

一些程序总是需要显示完整的日期和时间,并且精确到毫秒。但对于大多数程序,显示大概的时间范围会更加友好,例如“两个月以前”。我们可以用之前定义的枚举编写一个新的枚举来实现这一点:
\begin{minted}{Rust}
    /// 一个故意舍入的时间戳,因此我们的程序会显示“6个月以前”
    /// 而不是“February 9, 2016, at 9:49 AM”。
    #[derive(Copy, Clone, Debug, PartialEq)]
    enum RoughTime {
        InThePast(TimeUnit, u32),
        JustNow,
        InTheFuture(TimeUnit, u32),
    }
\end{minted}

这个枚举中的两个variant,即\texttt{InThePast}和\texttt{InTheFuture}都有参数。这些被称为\emph{tuple variant}。就像类元组结构体一样,它们的constructor是创建新的\texttt{RoughTime}值的函数:
\begin{minted}{Rust}
    let four_score_and_seven_years_ago =
        RoughTime::InThePast(TimeUnit::Years, 4 * 20 + 7);

    let three_hours_from_now =
        RoughTime::InTheFuture(TimeUnit::Hours, 3);
\end{minted}

枚举也可以有\emph{struct variant},它们和普通的结构体一样拥有命名字段:
\begin{minted}{Rust}
    enum Shape {
        Sphere { center: Point3d, radius: f32 },
        Cuboid { corner1: Point3d, corner2: Point3d },
    }

    let unit_sphere = Shape::Sphere {
        center: ORIGIN,
        radius: 1.0,
    };
\end{minted}

总的来说,Rust有三种枚举variant,分别对应我们在上一章中展示的三种结构体。没有数据的variant对应类单元结构体。元组variant对应类元组结构体。结构体variant对应有花括号和命名字段的结构体。一个枚举可以同时有这三种variant:
\begin{minted}{Rust}
    enum RelationshipStatus {
        Single,
        InARelationship,
        ItsComplicated(Option<String>),
        ItsExtremelyComplicated {
            car: DifferentialEquation, 
            cdr: EarlyModernistPoem,
        },
    }
\end{minted}

所有种类的constructor都和枚举自身有相同的可见性。

\subsection{内存中的枚举}

在内存中,带有数据的枚举被存储为一个很小的整数\emph{标签(tag)},加上一块足够存储所有variant中最大的那个的内存。标签字段是Rust内部要使用的,它表示是哪一个constructor创建了这个值,进而得知这个值有哪些字段。

在Rust 1.50中,\texttt{RoughTime}存储为8个字节,如\hyperref[f10-1]{图10-1}所示。

\begin{figure}[htbp]
    \centering
    \includegraphics[width=0.9\textwidth]{../img/10-1.png}
    \caption{内存中的\texttt{RoughTime}值}
    \label{f10-1}
\end{figure}

对于枚举的布局Rust不做任何保证。然而,为了给将来的优化留下余地,在一些情况下它可能会用比图中所示更加高效的方式包装一个枚举。例如,一些泛型结构体可以不用标签存储,我们稍后会讲到它。

\subsection{使用枚举实现富数据结构}

枚举在实现树形结构时也很有用。例如,假设一个Rust程序要处理任意的Json数据。在内存中,任何Json文档都可以被表示为一个这种Rust类型的值:
\begin{minted}{Rust}
    use std::collections::HashMap;

    enum Json {
        Null,
        Boolean(bool),
        Number(f64),
        String(String),
        Array(Vec<Json>),
        Object(Box<HashMap<String, Json>>),
    }
\end{minted}

与Rust代码相比,用英文来解释这个数据结构也不会再有太大的改进了。JSON标准定义了可以出现在JSON文档中的数据类型:\texttt{null}、布尔值、数字、字符串、JSON值的数组、以及带有字符串键和JSON值的对象。这个\texttt{Json}枚举简单地列出了这些类型。

这并不是一个假想的例子。你可以在\texttt{serde\_json} crate中找到一个非常相似的枚举,它是一个用于Rust结构体序列化的库,也是crates.io上下载次数最多的crate之一。

用于表示\texttt{Object}的\texttt{HashMap}外层的\texttt{Box}只是为了让\texttt{Json}值更加紧凑。在内存中,\texttt{Json}类型的值将占据4个机器字。\texttt{String}和\texttt{Vec}都是3个字,Rust会再添加一个字节的标签,再加上对齐所以总共是4个字。\texttt{Null}和\texttt{Boolean}值没有足够的数据利用全部的空间,但所有的\texttt{Json}值大小必须相同,因此这时多余的空间就被浪费了。\hyperref[f10-2]{图10-2}展示了一些示例的\texttt{Json}值在内存中的实际视图。

\begin{figure}[htbp]
    \centering
    \includegraphics[width=0.8\textwidth]{../img/f10-2.png}
    \caption{内存中的\texttt{Json}值}
    \label{f10-2}
\end{figure}

一个\texttt{HashMap}会更大如果我们一定要在每一个\texttt{Json}值中给它留出空间,它们将会变得更大,也就是8个字。但\texttt{Box<HashMap>}是单个字:它只是一个指向堆上分配的数据的指针。我们甚至可以通过装箱更多的字段来让\texttt{Json}变得更加紧凑。

这里优秀的地方在于,我们如此简单的就完成了这一切。如果是在C++中,可能要写一个这样的一个类才行:
\begin{minted}{Rust}
    class JSON {
    private:
        enum Tag {
            Null, Boolean, Number, String, Array, Object
        };
        union Data {
            bool boolean;
            double number;
            shared_ptr<string> str;
            shared_ptr<vector<JSON>> array;
            shared_ptr<unordered_map<string, JSON>> object;

            Data() {}
            ~Data() {}
            ...
        };
        
        Tag tag;
        Data data;
    
    public:
        bool is_null() const { return tag == Null; }
        bool is_boolean const { return tag == Boolean; }
        bool get_boolean() const {
            assert(is_boolean());
            return data.boolean;
        }
        void set_boolean(bool value) {
            this->~JSON();  // 清除string/array/object值
            tag = Boolean;
            data.boolean = value;
        }
        ...
    };
\end{minted}

30行代码,我们才刚刚开始。这个类还需要构造函数、析构函数、一个赋值运算符。另一种方案是通过继承,首先创建一个基类\texttt{JSON}和它的子类\texttt{JSONBoolean}、\texttt{JSONString}等等。无论哪种方式,等到完成之后,我们的C++ JSON库都要有一堆代码了。其他程序员需要花费不少精力来阅读和使用它。而Rust的整个枚举只需要8行代码。

    \chapter{trait与泛型}\label{ch11}

\emph{[A] computer scientist tends to be able to deal with nonuniform structures—case 1, case 2, case 3—while a mathematician will tend to want one unifying axiom that governs an entire system.}

\begin{flushright}
    ——Donald Knuth
\end{flushright}

编程界中最伟大的发现之一就是可以编写处理多种不同类型的代码,\emph{即使是还没有定义出来的类型也可以}。这里有两个例子:
\begin{itemize}
    \item \texttt{Vec<T>}是泛型的:你可以创建一个任意类型的vector,包括你自己定义的类型,即使\texttt{Vec}的作者完全不知道这个类型。
    \item 很多类型都有\texttt{.write()}方法,包括\texttt{File}和\texttt{TcpStream}。你的代码可以通过引用获取一个writer(任意的writer),并向它写入数据。你的代码不需要关心那个writer到底是什么类型。然后,如果有人添加了一个新的writer类型,你的代码将会自动支持它。
\end{itemize}

当然,这并不是什么新鲜的功能。它被称为\emph{多态(polymorphism)},是20世纪70年代很热门的新的编程语言技术。但现在它已经非常普遍了。Rust使用两个相关的特性来支持多态:trait和泛型。很多程序员可能已经很熟悉这两个概念了,但Rust采用了一种受Haskell的typeclass启发的新方法。

\emph{trait}是Rust中的接口或抽象基类。首先,它们看起来很像Java或C\#中的接口。用于写入字节的trait叫做\texttt{std::io::Write},它在标准库中的定义看起来像这样:
\begin{minted}{Rust}
    trait Write {
        fn write(&mut self, buf: &[u8]) -> Result<usize>;
        fn flush(&mut self) -> Result<()>;

        fn write_all(&mut self, buf: &[u8]) -> Result<()> { ... }
        ...
    }
\end{minted}

这个trait提供了几个方法,我们只展示了前三个。

标准类型\texttt{File}和\texttt{TcpStream}都实现了\texttt{std::io::Write}。\texttt{Vec<u8>}也是。这三个类型都提供\texttt{.write()}、\texttt{.flush()}等方法。使用writer的代码不需要关心它的类型,像这样:
\begin{minted}{Rust}
    use std::io::Write;

    fn say_hello(out: &mut dyn Write) -> std::io::Result<()> {
        out.write_all(b"hello world\n")?;
        out.flush()
    }
\end{minted}

\texttt{out}的类型是\texttt{\&mut dyn Write},意思是“任何实现了\texttt{Write} trait的值的可变引用”。我们可以把任何这样的值的可变引用传递给\texttt{say\_hello}:
\begin{minted}{Rust}
    use std::fs::File;
    let mut local_file = File::create("hello.txt")?;
    say_hello(&mut local_file)?;    // 可以工作

    let mut bytes = vec![];
    say_hello(&mut bytes)?;         // 也可以工作
    assert_eq!(bytes, b"hello world\n");
\end{minted}

这一章首先展示trait怎么使用、怎么工作、怎么定义自己的trait。但trait的用途比我们目前提到的更多。我们将使用它们给现有类型添加扩展的方法,甚至像\texttt{str}和\texttt{bool}这种内建类型也可以。我们将会解释为什么给一个类型添加trait不会消耗多余的内存,以及如何在没有虚方法开销的情况下使用trait。我们将看到一些Rust提供的用于操作符重载和其他特性的语言内建的trait。我们还将介绍\texttt{Self}类型、关联函数、关联类型。Rust从Haskell中提取了这三个特性,它们可以优雅地解决其他语言中需要通过变通的方法或者hack才能解决的问题。 

\emph{泛型}是Rust中另一种形式的多态。类似于C++的模板,一个泛型函数或类型可以用于多种不同的类型:
\begin{minted}{Rust}
    /// 给定两个值,找出较小的那个
    fn min<T: Ord>(value1: T, value2: T) -> T {
        if value1 <= value2 {
            value1
        } else {
            value2
        }
    }
\end{minted}

这个函数中的\texttt{<T: Ord>}意味着\texttt{min}可以用于任何实现了\texttt{Ord} trait的类型\texttt{T}——也就是,任何有序的类型。这样的一个要求被称为\emph{约束(bound)},因为它列举出了类型\texttt{T}需要满足的限制。编译器会为你实际使用的每一个类型\texttt{T}生成自定义的机器代码。

泛型和trait紧密相关:泛型函数在约束中使用trait来表明它可以用于哪些类型的参数。所以我们还会讨论\texttt{\&mut dyn Write}和\texttt{<T: Write>}有哪些相似和不同之处,以及如何在这种两种使用trait的方式中选择。

\section{使用trait}

一个trait就是一个给定的类型可能支持也可能不支持的特性。通常,一个trait代表一种能力:一个类型可以做的事情。
\begin{itemize}
    \item 一个实现了\texttt{std::io::Write}的值可以写入字节。
    \item 一个实现了\texttt{std::iter::Iterator}的值可以产生值的序列。
    \item 一个实现了\texttt{std::clone::Clone}的值可以产生自身在内存中的克隆。
    \item 一个实现了\texttt{std::fmt::Debug}可以使用\texttt{println!()}的\texttt{\{:?\}}格式说明符进行打印。
\end{itemize}

这4个trait都是Rust标准库的一部分,有很多标准类型都实现了它们。例如:
\begin{itemize}
    \item \texttt{std::fs::File}实现了\texttt{Write} trait,它把字节写入到本地文件。\texttt{std::net::TcpStream}写入到网络连接。\texttt{Vec<u8>}也实现了\texttt{Write}。在字节vector上调用\texttt{.write()}会往尾部添加数据。
    \item \texttt{Range<i32>}(\texttt{0..10}的类型)实现了\texttt{Iterator} trait,一些和切片、哈希表等相关联的迭代器类型也实现了这个trait。
    \item 大多数标准库类型实现了\texttt{Clone}。一些例外主要是像\texttt{TcpStream}这样的不仅仅表示内存中的数据的类型。
    \item 大多数标准库类型支持\texttt{Debug}。
\end{itemize}

有关trait方法有一个不寻常的规则:trait自身必须在作用域里。否则,所有它的方法都会被隐藏:
\begin{minted}{Rust}
    let mut buf: Vec<u8> = vec![];
    buf.write_all(b"hello")?;   // 错误:没有叫`write_all`的方法
\end{minted}

这种情况下,编译器会打印出友好的错误消息建议你添加\texttt{std::io::Write},然后确实能修复这个问题:
\begin{minted}{Rust}
    use std::io::Write;

    let mut buf: Vec<u8> = vec![];
    buf.write_all(b"hello")?;   // ok
\end{minted}

Rust会有这个规则是因为,正如我们稍后会在本章中看到的,你可以使用trait来给任意类型添加新的方法——即使是标准库的类型例如\texttt{u32}和\texttt{str}。第三方的crate也可以做同样的事情。显然,这会导致名称冲突!但因为Rust让你自己导入你需要使用的trait,所以crate可以轻松地利用这种强大的功能。要想导致冲突,你需要导入两个trait,这两个trait要给同一个类型添加相同名称的方法。这在实践中是很少见的。(如果你确实陷入了冲突中,你可以使用本章稍后会介绍的\nameref{fullymethod}来指明你想要使用哪一个。)

\texttt{Clone}和\texttt{Iterator}的方法不需要特殊的导入是因为它们默认总是在作用域里,它们是标准prelude的一部分:Rust会自动导入每个模块中的名称。事实上,prelude就是一个精心挑选的trait的集合。我们将在\hyperref[ch13]{第13章}中介绍更多有关它们的内容。

C++和C\#程序员可能已经注意到了trait方法很像虚方法。然而,类似上面的函数调用速度很快,与任何其他方法调用一样快。简单来说,这里面并没有多态性。显然\texttt{buf}是一个vector,不是一个文件或者网络连接,所以编译器可以简单地生成一个\texttt{Vec<u8>::write()}的调用。它甚至可以内联这个方法。(C++和C\#通常也会这样,尽管子类化的可能性有时会排除这一点。)只有通过\texttt{\&mut dyn Write}的调用才会有动态分发的开销,这种调用也被称为虚方法调用,类型里的\texttt{dyn}关键字暗示了这一点。\texttt{dyn Write}被称为\emph{trait对象(trait object)};我们将会在接下来的小节中看到trait对象的技术细节,以及它们与泛型函数的比较。

\subsection{trait对象}\label{traitobject}
在Rust中有两种使用trait来编写多态代码的方式:trait对象和泛型。我们将会首先介绍trait对象,在下一节中介绍泛型。

Rust不允许\texttt{dyn Write}类型的变量:
\begin{minted}{Rust}
    use std::io::Write;

    let mut buf: Vec<u8> = vec![];
    let writer: dyn Write = buf; // 错误:`Write`并没有固定的大小
\end{minted}

一个变量的大小必须在编译期时已知,然而实现了\texttt{Write}的类型可以是任何大小。

如果你来自C\#或者Java的话可能会感觉很惊讶,但原因其实很简单。在Java中,一个\texttt{OutputStream}(Java中类似\texttt{std::io::Write}的标准接口)类型的变量是一个任何实现了\texttt{OutputStream}的对象的引用。它是一个引用的事实不言而喻,C\#以及其他大多数语言中的接口也是一样。

我们在Rust中想要的也是一样的,但是在Rust中引用是显式的:
\begin{minted}{Rust}
    let mut buf: Vec<u8> = vec![];
    let writer: &mut dyn Write = &mut buf;  // ok
\end{minted}

一个trait类型的引用,例如\texttt{writer},被称为一个\emph{trait对象}。和其他引用一样,一个trait对象指向某个值、它有生命周期、它可以是可变的或者是共享的。

让一个trait对象与众不同的是Rust在编译期通常不知道被引用值的类型是什么。因此一个trait对象包括一点额外的有关被引用值的类型信息。类型信息被严格限制为只有Rust自己可以在幕后使用:当你调用\texttt{writer.write(data)}时,Rust需要这个类型信息来依据\texttt{*writer}的类型动态调用正确的\texttt{write}方法。你不能直接查询类型信息,Rust也不支持将trait对象\texttt{\&mut dyn Write}向下转换回精确的类型例如\texttt{Vec<u8>}。

\subsubsection{trait对象的布局}
在内存中,一个trait对象是一个胖指针,由指向值的指针加上一个指向表示该值类型的表的指针组成。因此每一个trait对象要占两个机器字,如\hyperref[f11-1]{图11-1}所示。

\begin{figure}[htbp]
    \centering
    \includegraphics[width=0.9\textwidth]{../img/f11-1.png}
    \caption{内存中的trait对象}
    \label{f11-1}
\end{figure}

C++也有这种运行时的类型信息。它被称为\emph{虚表}或者\emph{vtable}。在Rust中和在C++中一样,vtable只会在编译期生成一次,然后被所有相同类型的对象共享。\hyperref[f11-1]{图11-1}中较深颜色的阴影显示的内容,包括vtable,都是Rust的私有实现。这些字段和数据结构你不能直接访问。当你调用trait对象的方法时语言本身会自动使用vtable来决定要调用哪个实现。

熟练的C++程序员可能会注意到Rust和C++采取的内存策略有些不同。在C++中,虚表指针或者称为\emph{vptr}被存储为结构体的一部分,而Rust使用胖指针来代替。结构体本身不包含任何自身字段之外的东西。这样,一个结构体可以实现一大堆trait而不需要包含一大堆vptr。即使像\texttt{i32}这样的大小还不足以容纳一个vptr的类型,也可以实现trait。

当需要时Rust会自动把普通引用转换为trait对象。这就是为什么我们能在这个例子中直接把\texttt{\&mut local\_file}传递给\texttt{say\_hello}:
\begin{minted}{Rust}
    let mut local_file = File::create("hello.txt")?;
    say_hello(&mut local_file)?;
\end{minted}

\texttt{\&mut local\_file}的类型是\texttt{\&mut File},而\texttt{say\_hello}的参数类型是\texttt{\&mut dyn Write}。因为\texttt{File}是一种writer,所以Rust允许这种普通引用到trait对象的转换。

同样的,Rust也乐于把\texttt{Box<File>}转换成\texttt{Box<dyn Write>},它拥有一个在堆上的writer:
\begin{minted}{Rust}
    let w: Box<dyn Write> = Box::new(local_file);
\end{minted}

\texttt{Box<dyn Write>}和\texttt{\&mut dyn Write}一样是一个胖指针:它包含writer自身的地址和\\
vtable的地址。其他指针类型例如\texttt{Rc<dyn Write>}也一样。

这种转换是唯一创建trait对象的方法。编译器做的工作其实很简单,当转换发生时,Rust知道被引用值的真正类型(这个例子中是\texttt{File}),因此它只是加上了正确的vtable的地址、把普通指针变成了胖指针。

\subsection{泛型函数和类型参数}
在这一章的开始处,我们展示了\texttt{say\_hello()}函数,它以trait对象为参数。让我们把这个函数重写为泛型函数:
\begin{minted}{Rust}
    fn say_hello<W: Write>(out: &mut W) -> std::io::Result<()> {
        out.write_all(b"hello world\n")?;
        out.flush()
    }
\end{minted}

只有类型签名改变了:
\begin{minted}{Rust}
    fn say_hello(out: &mut dyn Write)   // 普通函数

    fn say_hello<W: Write>(out: &mut W) // 泛型函数
\end{minted}

让函数变为泛型函数的正是\texttt{<W: Write>}短语,它是一个\emph{类型参数}。它意味着在整个函数体内,\texttt{W}代表任何实现了\texttt{Write} trait的类型。按照习惯,类型参数通常是大写字母。

类型\texttt{W}到底是什么取决于泛型函数如何被调用:
\begin{minted}{Rust}
    say_hello(&mut local_file)?;    // 调用say_hello::<File>
    say_hello(&mut bytes)?;         // 调用say_hello::<Vec<u8>>
\end{minted}

当你把\texttt{\&mut local\_file}传递给泛型的\texttt{say\_hello()}函数时,你实际是在调用\\
\texttt{say\_hello::<File>()}。Rust会为这个函数生成机器码,机器码里还会调用\texttt{File::write\_all()}\\
和\texttt{File::flush()}。当你传递\texttt{\&mut bytes}时,你实际是在调用\texttt{say\_hello::<Vec<u8>>()}。Rust会为这个版本的函数生成单独的机器码,然后调用相应的\texttt{Vec<u8>}的方法。在这两种情况下,Rust都从参数的类型推导出类型\texttt{W},这个过程被称为\emph{单态化(monomorphization)},编译器会自动进行处理。

你也可以指明类型参数:
\begin{minted}{Rust}
    say_hello::<File>(&mut local_file)?;
\end{minted}

很少情况下才需要显式写出参数,因为Rust通常可以通过参数推断出类型参数。这里,\texttt{say\_hello}泛型函数期望一个\texttt{\&mut W}参数,而我们传入了一个\texttt{\&mut File},因此Rust推断出\texttt{W = File}。

如果你正在调用的泛型函数并没有足以推断出参数的线索,你需要显式地指明:
\begin{minted}{Rust}
    // 调用一个没有参数的泛型方法collect<C>()
    let v1 = (0 .. 1000).collect();     // 错误:不能推断出类型
    let v2 = (0 .. 1000).collect::<Vec<i32>>(); // ok
\end{minted}

有时我们需要一个类型参数可以支持多种功能。例如,如果我们想打印出一个vector中出现次数最多的10个值,我们需要这些值可以打印:
\begin{minted}{Rust}
    use std::fmt::Debug;

    fn top_ten<T: Debug>(values: &Vec<T>) { ... }
\end{minted}

但这还不够。我们怎么判断哪个值是出现次数最多的?通常的办法是把每个值当作键存入一个哈希表。这意味着这些值需要支持\texttt{Hash}和\texttt{Eq}操作。\texttt{T}的约束还必须包括\texttt{Debug}。这种情况下应该使用的语法是\texttt{+}号:
\begin{minted}{Rust}
    use std::hash::Hash;
    use std::fmt::Debug;

    fn top_ten<T: Debug + Hash + Eq>(values: &Vec<T>) { ... }
\end{minted}

一些类型实现了\texttt{Debug}、一些实现了\texttt{Hash}、一些支持\texttt{Eq},还有少数类型例如\texttt{u32}和\texttt{String},实现了这三个trait,如\hyperref[f11-2]{图11-2}所示。

\begin{figure}[htbp]
    \centering
    \includegraphics[width=0.8\textwidth]{../img/f11-2.png}
    \caption{trait作为类型的集合}
    \label{f11-2}
\end{figure}

也可以不给类型参数指定任何约束,但这样的话你几乎不能对它进行任何操作。你只能移动它、将它放在box或vector里。

泛型函数可以有多个类型参数:
\begin{minted}{Rust}
    /// 在一个大规模的分区数据集上进行查询。
    /// 见<http://research.google.com/archive/mapreduce.html>。
    fn run_query<M: Mapper + Serialize, R: Reducer + Serialize>(
        data: &DataSet, map: M, reduce: R) -> Results
    { ... }
\end{minted}

正如这个例子展示的一样,约束可能太长以至于很难阅读。Rust提供了使用关键字\texttt{where}的替代语法:
\begin{minted}{Rust}
    fn run_query<M, R>(data: &DataSet, map: M, reduce: R) -> Results
        where M: Mapper + Serialize,
              R: Reducer + Serialize
    { ... }
\end{minted}

类型参数\texttt{M}和\texttt{R}仍然在前边声明,但约束被移动到单独的行。这种\texttt{where}语法可以用于泛型结构体、泛型枚举、类型别名以及方法——任何允许约束的地方。

当然,\texttt{where}语法的一个替代是保持简单:寻找一种不需要大量使用泛型的方法来编写程序。

\nameref{RefAsArg}介绍了生命周期的语法。一个泛型函数可以同时有生命周期参数和类型参数。生命周期参数在前:
\begin{minted}{Rust}
    /// 返回`candidates`中距离`target`最近的点的引用。
    fn nearest<'t, 'c, P>(target: &'t P, candidates: &'c [P]) -> &'c P
        where P: MeasureDistance
    {
        ...
    }
\end{minted}

这个函数有两个参数:\texttt{target}和\texttt{candidates}。它们都是引用,但我们给了它们不同的生命周期\texttt{'t}和\texttt{'c}(正如在\nameref{DistLife}中讨论的那样)。这个函数可以用于任何实现了\texttt{MeasureDistance} trait的类型\texttt{P},因此我们可以在一个程序中用\texttt{Point2d}值调用它,而在另一个程序中用\texttt{Point3d}值调用它。

生命周期绝不会影响到机器码。两个\texttt{P}的类型相同但生命周期不同的\texttt{nearest()}的调用,将会调用同一个编译好的函数。只有不同的类型才会导致Rust编译一个泛型函数的多个拷贝。

当然,函数并不是Rust中唯一的泛型代码:
\begin{itemize}
    \item 我们已经在\nameref{GenStruct}和\nameref{GenEnum}中介绍过泛型类型了。
    \item 一个单独的方法也可以是泛型的,就算定义它的类型不是泛型的:
    \begin{minted}{Rust}
    impl PancakeStack {
        fn push<T: Topping>(&mut self, goop: T) -> PancakeResult<()> {
            goop.pour(&self);
            self.absorb_topping(goop)
        }
    }
    \end{minted}
    \item 类型别名也可以是泛型的:
    \begin{minted}{Rust}
    type PancakeResult<T> = Result<T, PancakeError>;
    \end{minted}
    \item 我们将在本章稍后介绍泛型trait。
\end{itemize}

所有这一节中介绍的特性——约束、\texttt{where}子句、生命周期参数等——可以被用于所有泛型item,而不仅仅是函数。

\subsection{选择哪一种}\label{WhichToUse}
选择trait对象还是泛型代码是一件很微妙的事情。因为它们都基于trait,有很多相似之处。

任何当你需要一个混合类型的值的集合的情况下trait对象都是正确的选择。从技术上讲创建泛型的沙拉是可行的:
\begin{minted}{Rust}
    trait Vegetable {
        ...
    }

    struct Salad<V: Vegetable> {
        veggies: Vec<V>
    }
\end{minted}

然而,这是一个非常糟糕的设计。每一个这样的沙拉都全部是由单一类型的蔬菜组成的。不是所有人都适合这么做,本书的作者之一曾经为一个\texttt{Salad<IcebergLettuce>}支付了\$14美元,并且直到现在也没有忘记那次经历。

然而我们怎么构建一个更好的沙拉呢?因为\texttt{Vegetable}值可能是不同大小的,我们不能要求Rust创建一个\texttt{Vec<dyn Vegetable>}:
\begin{minted}{Rust}
    struct Salad {
        veggies: Vec<dyn Vegetable> // 错误:`dyn Vegetable`并
                                    // 没有固定大小
    }
\end{minted}

trait对象就是解决方案:
\begin{minted}{Rust}
    struct Salad {
        veggies: Vec<Box<dyn Vegetable>>
    }
\end{minted}

每一个\texttt{Box<dyn Vegetable>}可以持有任何类型的蔬菜,但box自身的大小是固定的——两个指针——因此可以存储在vector中。除了在食物里放盒子这个不幸的比喻之外,它确实就是我们需要的。它也同样适用于绘图应用中的形状、游戏中的怪物、网络路由器中的可插拔路由算法等等。

另一个使用trait对象的可能的原因是减小编译出的代码的体积。Rust可能需要编译一个泛型函数很多次,因为它要为每一个用到的类型都编译一次。这可能导致生成的二进制文件很大,这种现象在C++圈子里称为\emph{代码膨胀(code bloat)}。近年来内存越来越充裕,因此我们中的大多数人可以忽略代码的体积,但确实还有一些受限制的环境。

除了涉及到沙拉或者资源受限的环境之外,泛型与trait对象相比有三个优势。因此在Rust中泛型是更加普遍的选择。

第一个优势是速度。注意泛型函数签名中没有\texttt{dyn}关键字。因为你在编译期指明了确切的类型,不管是显式还是通过类型推导,编译器都知道实际上调用了哪个\texttt{write}。没有使用\texttt{dyn}关键字是因为没有trait对象——因此也没有涉及动态分发。

引言中展示的泛型\texttt{min()}函数就和我们单独编写\texttt{min\_u8}、\texttt{min\_i64}、\texttt{min\_string}等函数一样快。编译器还可以像其他函数一样内联它,因此在release构建中,一个对\texttt{min::<i32>}的调用可能只有两三条指令。对于常量的调用,例如\texttt{min(5, 3)}可能会更快:Rust可以在编译期对它进行求值,因此不会有任何运行时开销。

或者考虑这个泛型函数调用:
\begin{minted}{Rust}
    let mut sink = std::io::sink();
    say_hello(&mut sink);
\end{minted}

\texttt{std::io::sink()}返回一个\texttt{Sink}类型的writer,它会偷偷丢弃掉所有写入的字节。

当Rust为此生成机器码的时候,它可以产生先调用\texttt{Sink::write\_all}、再检查错误、最后调用\texttt{Sink::flush}的代码。这正是泛型函数体的内容。

或者,Rust可以查看那些方法,然后意识到下列情况:
\begin{itemize}
    \item \texttt{Sink::write\_all()}什么也不做。
    \item \texttt{Sink:flush()}什么也不做。
    \item 两个方法都不可能返回错误。
\end{itemize}

简单来说,Rust拥有所有优化掉这个函数调用所需的信息。

相比与trait对象的行为,Rust直到运行时才能知道一个trait对象指向的值到底是什么类型。因此即使你传递了一个\texttt{Sink},虚方法的调用开销和检查错误的开销仍然不可避免。

泛型的第二个优势是有的trait不支持trait对象。trait只支持一部分特性,例如关联函数只能使用泛型,这样就完全排除了trait对象。当我们讲到这些特性时会指出它们。

泛型的第三个优势是可以很容易地一次给泛型类型参数添加多个trait约束,例如我们的\texttt{top\_ten}函数就要求它的参数\texttt{T}要实现\texttt{Debug + Hash + Eq}。trait对象不能这么做:Rust不支持类似\texttt{\&mut (dyn Debug + Hash + Eq)}这样的类型。(你可以用本章中稍后会讲到的\hyperref[subtrait]{子trait}来实现类似的功能,但这样有点复杂。)

\section{定义和实现trait}

定义一个trait很简单,只需要给出名字和trait方法的签名类型。假设我们在编写一个游戏,我们可能会定义像这样的trait:
\begin{minted}{Rust}
    /// 一个角色、物品、风景等
    /// 任何可以显示在屏幕上的游戏世界的物体。
    trait Visible {
        /// 在给定的画布上渲染这个对象。
        fn draw(&self, canvas: &mut Canvas);

        /// 如果点击(x, y)会选中这个对象就返回true。
        fn hit_test(&self, x: i32, y: i32) -> bool;
    }
\end{minted}

为了实现一个trait,需要使用语法\texttt{impl TraitName for Type}:
\begin{minted}{Rust}
    impl Visible for Broom {
        fn draw(&self, canvas: &mut Canvas) {
            for y in self.y - self.height -1 .. self.y {
                canvas.write_at(self.x, y, '|');
            }
            canvas.write_at(self.x, self.y , 'M);
        }

        fn hit_test(&self, x: i32, y: i32) -> bool {
            self.x == x
            && self.y - self.height - 1 <= y
            && y <= self.y
        }
    }
\end{minted}

注意\texttt{impl}包含了一份\texttt{Visible} trait中每个方法的实现,除此之外没有别的内容。在trait \texttt{impl}中定义的任何东西都必须是trait的特性。如果我们想要添加一个\texttt{Broom::draw()}的帮助函数,我们必须在单独的\texttt{impl}块中定义它:

\begin{minted}{Rust}
    impl Broom {
        /// 下面的Broom::draw()用到的帮助函数。
        fn broomstick_range(&self) -> Range<i32> {
            self.y - self.height - 1 .. self.y
        }
    }
\end{minted}

这些帮助函数可以在trait \texttt{impl}块中使用:
\begin{minted}{Rust}
    impl Visible for Broom {
        fn draw(&self, canvas: &mut Canvas) {
            for y in self.broomstick_range() {
                ...
            }
            ...
        }
        ...
    }
\end{minted}

\subsection{默认方法}
我们之前讨论的\texttt{Sink} writer可以用少数几行代码实现。首先,我们定义如下类型:
\begin{minted}{Rust}
    /// 一个忽略写入数据的writer
    pub struct Sink;
\end{minted}

\texttt{Sink}是一个空结构体,因为我们不需要在里面存储任何数据。接下来,我们为\texttt{Sink}提供了一份\texttt{Write} trait的实现:
\begin{minted}{Rust}
    use std::io::{Write, Result};

    impl Write for Sink {
        fn write(&mut self, buf: &[u8]) -> Result<usize> {
            // 假装成功写入了整个缓冲区
            Ok(buf.len())
        }

        fn flush(&mut self) -> Result<()> {
            Ok(())
        }
    }
\end{minted}

到目前为止,这和\texttt{Visible} trait很像。但是我们展示过\texttt{Write} trait还有一个\texttt{write\_all}方法:
\begin{minted}{Rust}
    let mut out = Sink;
    out.write_all(b"hello world\n")?;
\end{minted}

为什么Rust允许我们\texttt{impl Write for Sink}时不定义\texttt{write\_all}方法?答案就是标准库中\texttt{Write} trait的定义中包含了一个\texttt{write\_all}的\emph{默认实现}:
\begin{minted}{Rust}
    trait Write {
        fn write(&mut self, buf: &[u8]) -> Result<usize>;
        fn flush(&mut self) -> Result<()>;
        
        fn write_all(&mut self, buf: &[8]) -> Result<()> {
            let mut bytes_written = 0;
            while bytes_written < buf.len() {
                bytes_written += self.write(&buf[bytes_written..])?;
            }
            Ok(())            
        }

        ...
    }
\end{minted}

\texttt{write}和\texttt{flush}方法是每一个writer必须实现的基本方法。一个writer可能也实现了\\
\texttt{write\_all},但如果没有,将会使用我们上边展示的默认实现。

你自己的trait也可以使用相同的语法包含默认实现。

默认方法最有戏剧性的使用是在标准库的\texttt{Iterator} trait,它只有一个需要实现的方法\texttt{.next()},和一堆默认实现的方法。\hyperref[ch15]{第15章}中会解释原因。

\subsection{trait和其他人的类型}\label{OrphanRule}
Rust允许你在任意类型上实现任意trait,只要trait或者类型是在当前crate中定义的。

这意味着任何时候如果你想给任何类型添加一个方法,你都可以用trait来做到这一点:
\begin{minted}{Rust}
    trait IsEmoji {
        fn is_emoji(&self) -> bool;
    }

    /// 为内建的字符类型实现IsEmoji方法
    impl IsEmoji for char {
        fn is_emoji(&self) -> bool {
            ...
        }
    }

    assert_eq!('$'.is_emoji(), false);
\end{minted}

类似于其他trait方法,只有当\texttt{IsEmoji}在作用域中时新的\texttt{is\_emoji}方法才可见。

这个trait的唯一目的就是给现有类型\texttt{char}添加一个方法。这被称为\emph{扩展trait(extension trait)}。当然,你可以把这个trait添加给其他类型,例如\texttt{impl IsEmoji for str \{ ... \}}等。

你甚至可以使用泛型\texttt{impl}块来一次性给一整个家族的类型添加一个扩展trait。这个trait可以在任何类型上实现:
\begin{minted}{Rust}
    use std::io::{self, Write};

    /// trait for values to which you can send HTML.
    trait WriteHtml {
        fn write_html(&mut self, html: &HtmlDocument) -> io::Result<()>;
    }
\end{minted}

为所有writer实现这个trait,可以为所有Rust writer添加这个方法:
\begin{minted}{Rust}
    /// 你可以向任意std::io writer写入HTML
    impl<W: Write> WriteHtml for W {
        fn write_html(&mut self, html: &HtmlDocument) -> io::Reuslt<()> {
            ...
        }
    }
\end{minted}

\texttt{impl<W: Write> WriteHtml for W}这一行意思是“对于任何实现了\texttt{Write}的类型\texttt{W},这里有一个为\texttt{W}编写的\texttt{WriteHtml}的实现”。

\texttt{serde}库提供了一个很好的例子,它展示了可以在标准类型上实现用户自定义trait这种能力的重要作用。\texttt{serde}是一个序列化库。也就是说,你可以使用它把任何Rust数据结构写入到磁盘,并在稍后加载它们。这个库定义了一个trait \texttt{Serialize},库支持所有实现了这个trait的数据类型。因此在\texttt{serde}的源码中,为\texttt{bool, i8, i16, i32},数组和元组类型等,包括标准数据结构例如\texttt{Vec}和\texttt{HashMap}都实现了\texttt{Serialize} trait。

这样的结果是\texttt{serde}为所有这些类型添加了一个\texttt{.serialize()}方法。它可以像这样使用:
\begin{minted}{Rust}
    use serde::Serialize;
    use serde_json;

    pub fn save_configuration(config: &HashMap<String, String>) 
        -> std::io::Result<()>
    {
        // 创建一个JSON序列化器来把数据写入到文件。
        let writer = File::create(config_filename())?;
        let mut serializer = serde_json::Serializer::new(writer);

        // serde的`.serialize()`方法负责剩余的内容。
        config.serialize(&mut serializer)?;

        Ok(())
    }
\end{minted}

我们之前说过当你实现一个tarit时,trait和类型至少有一个必须是在当前crate中新定义的。这被称为\emph{孤儿规则(orphan rule)}。它帮助确保tarit的实现是唯一的。你的代码不能\texttt{impl Write for u8},因为\texttt{Write}和\texttt{u8}都是在标准库中定义的。如果Rust允许crate这么做,那么不同的crate中可能会有不同的\texttt{u8}类型的\texttt{Write} trait实现。Rust将不知道为一个方法调用选择哪种实现。

(C++也有一个类似的唯一性约束:一次定义规则。在传统的C++风格中,除了最简单的情况之外,编译器并不会强制这一点,如果你打破了这个规则会遇到未定义行为。)

\subsection{trait中的\texttt{Self}}
trait中可以将\texttt{Self}关键字用作类型。例如标准的\texttt{Clone} trait,看起来像这样(简化版):
\begin{minted}{Rust}
    pub trait Clone {
        fn clone(&self) -> Self;
        ...
    }
\end{minted}

这里使用\texttt{Self}作为返回类型意味着\texttt{x.clone()}的返回值类型和\texttt{x}的类型相同,不管\texttt{x}是什么。如果\texttt{x}是一个\texttt{String},那么\texttt{x.clone()}的类型就是\texttt{String}——不是\texttt{dyn Clone}或者别的可克隆的类型。

同样,如果我们定义了这个trait:
\begin{minted}{Rust}
    pub trait Spliceable {
        fn splice(&self, other: &Self) -> Self;
    }
\end{minted}
还有两个实现:
\begin{minted}{Rust}
    impl Spliceable for CherryTree {
        fn splice(&self, other: &Self) -> Self {
            ...
        }
    }

    impl Spliceable for Mammoth {
        fn splice(&self, other: &Self) -> Self {
            ...
        }
    }    
\end{minted}

在第一个\texttt{impl}中,\texttt{Self}就是\texttt{CherryTree}的别名;而在第二个\texttt{impl}中,它是\texttt{Mammoth}的别名。这意味着我们可以把两棵樱桃树或者两只猛犸象拼接在一起,而不能创建出樱桃树-猛犸象杂交种。\texttt{self}的类型和\texttt{other}的类型必须相同。

一个使用了\texttt{Self}类型的trait和trait对象不兼容:
\begin{minted}{Rust}
    // 错误:trait `Spliceable`不能转变为一个对象
    fn splice_anything(left: &dyn Spliceable, right: &dyn Spliceable) {
        let combo = left.splice(right);
        // ...
    }
\end{minted}

当我们在深入研究trait的高级特性时会多次看到原因。Rust拒绝这段代码是因为它没有办法对\texttt{left.splice(right)}调用进行类型检查。关键点在于trait对象的类型直到运行时才能知道。Rust没有办法在编译期知道\texttt{left}和\texttt{right}是不是相同的类型。

trait对象实际上是为最简单的trait设计的,就是那种可以用Java中的接口或者C++中的抽象基类实现的那种trait。trait的还有更多有用的高级特性,但它们不能和现有的trait对象共存。因为使用trait对象时,你会丢失Rust对程序进行类型检查时必须的类型信息。

现在,假设我们想要一个从基因上讲不可能的拼接,我们可以设计一个trait对象友好的trait:
\begin{minted}{Rust}
    pub trait MegaSpliceable {
        fn splice(&self, other: &dyn MegaSpliceable) -> Box<dyn MegaSpliceable>;
    }
\end{minted}

这个trait可以和trait对象兼容。调用\texttt{.splice()}方法时的类型检查不会有问题,因此参数\texttt{other}的类型不需要和\texttt{self}的类型相同,尽管它们的类型都是\texttt{MegaSpliceable}。

\subsection{子trait}\label{subtrait}
我们可以定义一个trait作为另一个trait的扩展:
\begin{minted}{Rust}
    /// 游戏世界中的某个生物,可能是玩家或者
    /// 小精灵、石像鬼、松鼠、食人魔等。
    trait Creature: Visible {
        fn position(&self) -> (i32, i32);
        fn facing(&self) -> Direction;
        ...
    }
\end{minted}

短语\texttt{trait Creature: Visible}意味着所有的生物都是可视的。每一个实现了\texttt{Creature}的类型都必须实现\texttt{Visible} trait:
\begin{minted}{Rust}
    impl Visible for Broom {
        ...
    }

    impl Creature for Broom {
        ...
    }
\end{minted}
我们可以以任何顺序实现这两个trait,但为一个没有实现\texttt{Visible}的类型实现\texttt{Creature}是错误的。这里,我们说\texttt{Creature}是\texttt{Visible}的一个\emph{子trait(subtrait)},而\texttt{Visible}是\texttt{Creature}的\emph{父trait(supertrait)}。

子trait类似Java或者C\#中的子接口,用户可以假定任何实现了子trait的值一定也实现了它的父trait。但在Rust中,一个子trait不会继承父trait中的相关item,如果你想调用方法的话仍然要确保每个trait都在作用域中。

事实上,Rust的子trait只是对\texttt{Self}的约束的缩写。\texttt{Creature}的定义和下面这个完全等价:
\begin{minted}{Rust}
    trait Creature where Self: Visible {
        ...
    }
\end{minted}

\subsection{类型关联函数}
在大多数面向对象语言中,接口不能包含静态方法或者构造函数,但trait可以包含类型关联函数,Rust中的关联函数类似于静态方法:
\begin{minted}{Rust}
    trait StringSet {
        /// 返回一个空的集合。
        fn new() -> Self;
        
        /// 返回一个包含`strings`中所有字符串的集合。
        fn from_slice(strings: &[&str]) -> Self;

        /// 查找集合是否包含`string`。
        fn contains(&self, string: &str) -> bool;

        /// 向集合中添加一个字符串。
        fn add(&mut self, string: &str);
    }
\end{minted}

每一个实现了\texttt{StringSet} trait的类型都必须实现这四个关联函数。前两个函数\texttt{new()}和\\
\texttt{from\_slice()},没有\texttt{self}参数。它们充当构造函数。在非泛型代码中,这些函数可以使用\texttt{::}\\
语法调用,就像其他类型关联函数一样:
\begin{minted}{Rust}
    // 创建两个impl StringSet的多态类型:
    let set1 = SortedStringSet::new();
    let set2 = HashedStringSet::new();
\end{minted}

在泛型代码中也是一样的。除了类型是一个类型变量,因此这里需要调用\texttt{S::new()}:
\begin{minted}{Rust}
    /// 返回`document`中有但`wordlist`中没有的单词的集合。
    fn unknown_words<S: StringSet>(document: &[String], wordlist: &S) -> S {
        let mut unknowns = S::new();
        for word in document {
            if !wordlist.contains(word) {
                unknowns.add(word)
            }
        }
        unknowns
    }
\end{minted}

类似Java和C\#的接口,trait对象不支持类型关联函数。如果你想使用\texttt{\&dyn StringSet} trait对象,那你必须修改trait,给那些不接受\texttt{self}参数的关联函数加上\texttt{where Self: Sized}约束:

\begin{minted}{Rust}
    trait StringSet {
        fn new() -> Self
            where Self: Sized;

        fn from_slice(strings: &[&str]) -> Self
            where Self: Sized;

        fn contains(&self, string: &str) -> bool;

        fn add(&mut self, string: &str);
    }
\end{minted}

这个约束告诉Rust trait对象不支持这个关联函数。加上之后,你可以创建\texttt{StringSet}的\\
trait对象了,但仍然不能使用\texttt{new}和\texttt{from\_slice},不过你可以使用它们调用\texttt{.contains()}和\\
\texttt{.add()}。同样的技巧也适用于其他和trait对象不兼容的方法。(从技术上解释为什么会这样是相当乏味的,因此我们不会解释。不过\texttt{Sized} trait将会在\hyperref[ch13]{第13章}介绍。)

\section{完全限定方法调用}\label{fullymethod}

目前为止我们展示过的所有调用trait方法的方式都需要Rust自动为我们填充一些缺失的东西。例如,假设你写了如下代码:
\begin{minted}{Rust}
    "hello".to_string();
\end{minted}

显然这里的\texttt{to\_string}指的是\texttt{ToString} trait的\texttt{to\_string}方法,而我们调用的是\texttt{str}类型的实现。因此这场游戏中出现了四个玩家:trait、trait方法、trait方法的实现、调用trait方法实现的值。我们不需要每次调用方法时都完全写出这四个部分是一件好事,但有些情况下你也可能会需要一种精确的方式来表达你的意思。这种情况下就要用到完全限定方法调用。

首先,要知道方法只是一种特殊的函数。这两种调用是等价的:
\begin{minted}{Rust}
    "hello".to_string()

    str::to_string("hello")
\end{minted}

第二种形式看起来很像一个关联函数的调用,即使\texttt{to\_string}方法以\texttt{self}为参数也没有问题,只会简单的传递\texttt{self}作为函数的第一个参数。

因为\texttt{to\_string}是标准的\texttt{ToString} trait的方法,所以还有两种调用方式:
\begin{minted}{Rust}
    ToString::to_string("hello")

    <str as ToString>::to_string("hello")
\end{minted}

这四种方法调用功能完全相同。通常你最可能写\texttt{value.method()}。其他的形式是\emph{限定(qualified)}方法调用。它们指明了方法关联到的类型或者trait。最后一种带尖括号的形式同时指明了类型和trait,这种形式被称为\emph{完全限定(fully qualified)}方法调用。

当你写\texttt{"hello".to\_string()}时候,使用\texttt{.}运算符,你不需要精确地说明你要调用哪个\texttt{to\_string}方法。Rust有一个依据类型、强制解引用等机制的查找算法来确定是哪个方法。使用完全限定调用,你可以精确地说明你想要调用哪个方法,这可以在一些罕见的情况下有所帮助:
\begin{itemize}
    \item 当两个方法的名称相同时。经典的例子是\texttt{Outlaw}有两个来自不同trait的\texttt{.draw()}方法,一个用于在屏幕上绘制它,另一个用于和law交互:
    \begin{minted}{Rust}
    outlaw.draw();              // error: draw on screen or draw pistol?

    Visible::draw(&outlaw);     // ok: draw on screen
    HasPistol::draw(&outlaw);   // ok: corral
    \end{minted}

    通常你可能更愿意重命名其中一个方法,但有时你不能这么做。

    \item 当\texttt{self}参数的类型不能被推断出来时:
    \begin{minted}{Rust}
    let zero = 0;   // 类型为定义:可能是`i8`,`u8`,...

    zero.abs();     // 错误:不能在有歧义的数字类型
                    // 上调用方法`abs`

    i64::abs(zero); // ok
    \end{minted}

    \item 当使用函数本身作为函数类型的值的时候:
    \begin{minted}{Rust}
    let words: Vec<String> =
        line.split_whitespace()         // 迭代器会产生&str值
            .map(ToString::to_string)   // ok
            .collect();
    \end{minted}

    \item 当在宏中调用trait方法时。我们将在\hyperref[ch21]{第21章}中解释。
\end{itemize}

完全限定语法也可以用于关联函数。在之前的小节中,我们用了\texttt{S::new()}在泛型函数中创建一个新的集合。我们还可以写成\texttt{StringSet::new()}或者\texttt{<S as StringSet>::new()}。

\section{定义类型关系的trait}

到目前为止,我们看到过的每个trait都是独立的:一个trait就是一些可以实现的方法的集合。trait也可以用于需要多个类型协同工作的场景。它们可以描述类型之间的关系:
\begin{itemize}
    \item \texttt{std::iter::Iterator} trait将迭代器类型和产生的值的类型联系在了一起。
    \item \texttt{std::ops::Mul} trait将可以做乘法的类型联系了起来。在表达式\texttt{a * b}中,值\texttt{a}和\texttt{b}可以是相同类型,也可以是不同的类型。
    \item \texttt{rand} crate包含一个代表随机数生成器的trait(\texttt{rand::Rng}),和一个代表可以被随机生成的类型的trait(\texttt{rand::Distribution})。这些trait定义了这些类型怎么协同工作。
\end{itemize}

日常编程中你可能并不需要创建这样的trait,但你会在标准库和第三方crate中看到它们。在这一节中,我们将展示这些例子是怎么实现的、根据需要介绍相关的Rust的语言特性。这里最核心的技能就是读懂trait和方法签名、并搞清楚它们到底想表达什么意思。

\subsection{关联类型(或迭代器是如何工作的)}
我们将以迭代器开始。到目前为止每一门面向对象的语言都有内建的对迭代器的支持,迭代器是表示遍历一系列值的对象。

Rust有一个标准的\texttt{Iterator} trait,它的定义如下:
\begin{minted}{Rust}
    pub trait Iterator {
        type Item;

        fn next(&mut self) -> Option<Self::Item>;
        ...
    }
\end{minted}

这个trait的第一个特性\texttt{type Item},是一个\emph{关联类型(associated tepe)}。每一个实现了\texttt{Iterator}的类型都必须指明它产生什么类型的值。

第二个特性\texttt{next()}方法,在返回值类型中使用了关联类型。\texttt{next()}返回一个\\
\texttt{Option<Self::Item>}:要么是\texttt{Some(item)},即序列中的下一个值;要么是\texttt{None},表示已经没有值了。这个类型被写作\texttt{Self::Item},而不是普通的\texttt{Item},这是因为\texttt{Item}是每一个迭代器类型的一个特性,而不是单独的类型。和往常一样,\texttt{self}和\texttt{Self}类型需要显式地出现在使用它们的字段、方法等的代码中。

这里有个示例为一个类型实现了\texttt{Iterator}:
\begin{minted}{Rust}
    // (这段代码出自std::env标准库模块)
    impl Iterator for Args {
        type Item = String;

        fn next(&mut self) -> Option<String> {
            ...
        }

        ...
    }
\end{minted}

我们在\hyperref[ch02]{第2章}中使用过标准库函数\texttt{std::env::args()}来获取命令行参数,\texttt{std::env::Args}\\
就是它返回的迭代器的类型。它产生\texttt{String}值,因此\texttt{impl}块中声明了\texttt{type Item = String;}。

泛型代码也可以使用关联类型:
\begin{minted}{Rust}
    /// 循环一个迭代器,把值存储到新的vector中。
    fn collect_into_vector<I: Iterator>(iter: I) -> Vec<I::Item> {
        let mut results = Vec::new();
        for value in iter {
            results.push(value);
        }
        results
    }
\end{minted}

在这个函数体中,Rust为我们推断出了\texttt{value}的类型,这很棒。但我们必须指明\\
\texttt{collect\_into\_vector}的返回类型,而\texttt{Item}关联类型是唯一的方法。(\texttt{Vec<I>}显然是错的:它说明函数会返回一个迭代器的vector!)

上面的代码你可能永远不会自己写出来,因为在阅读了\hyperref[ch15]{第15章}后,你就会知道迭代器已经有了一个标准方法\texttt{iter.collect()}来做这件事了。因此在继续之前让我们再看一个例子:
\begin{minted}{Rust}
    /// 打印出一个迭代器产生的所有值
    fn dump<I>(iter: I)
        where I: Iterator
    {
        for (index, value) in iter.enumerate() {
            println!("{}: {:?}", index, value); // error
        }
    }
\end{minted}

还差一点就完成了。这里只有一个问题:\texttt{value}可能不是一个可打印的类型。
\begin{minted}{text}
    error: `<I as Iterator>::Item` doesn't implement `Debug`
      |
    8 |         println!("{}: {:?}", index, value);   // error
      |                                     ^^^^^
      |                          `<I as Iterator>::Item` cannot be formatted
      |                          using `{:?}` because it doesn't implement `Debug`
      = help: the trait `Debug` is not implemented for `<I as Iterator>::Item`
      = note: required by `std::fmt::Debug::fmt`
    help: consider further restricting the associated type
      |
    5 |     where I: Iterator, <I as Iterator>::Item: Debug
      |                      ^^^^^^^^^^^^^^^^^^^^^^^^^^^^^^
\end{minted}

错误信息有一点混淆,因为Rust使用了语法\texttt{<I as Iterator>::Item},这种方式比\texttt{I::Item}\\
更加显式和详细。这是有效的Rust语法,不过你很少会需要用这种方式指明类型。

错误信息的关键是,要想让这段泛型代码能编译,我们必须确保\texttt{I::Item}实现了\texttt{Debug} trait,这个trait用于使用\texttt{\{:?\}}格式化值。正如错误信息建议的那样,我们可以通过添加一个\texttt{I::Item}的约束来解决这个问题:
\begin{minted}{Rust}
    use std::fmt::Debug;

    fn dump<I>(iter: I)
        where I: Iterator, I::Item: Debug
    {
        ...
    }
\end{minted}

或者,我们可以写“\texttt{I}必须是一个产生\texttt{String}值的迭代器”:
\begin{minted}{Rust}
    fn dump<I>(iter: I)
        where I: Iterator<Item=String>
    {
        ...
    }
\end{minted}

\texttt{Iterator<Item=String>}本身是一个trait。如果你把\texttt{Iterator}看作所有可能的迭代器类型的集合,那么\texttt{Iterator<Item=String>}就是\texttt{Iterator}的一个子集:产生\texttt{String}的迭代器类型的集合。这个语法可以用在任何需要一个trait名字的位置,包括trait对象类型:
\begin{minted}{Rust}
    fn dump(iter: &mut dyn Iterator<Item=String>) {
        for (index, s) in iter.enumerate() {
            println!("{}: {:?}", index, s);
        }
    }
\end{minted}

带有关联类型的trait,例如\texttt{iterator},和trait对象是兼容的,不过必须像这里展示的一样指明所有的关联类型才可以。否则,\texttt{s}的类型可能是任何东西,因此Rust无法对这段代码进行类型检查。

我们已经展示了很多涉及到迭代器的例子,因为目前迭代器是关联类型最突出的用途。但关联类型在任何trait需要涉及方法以外的东西的场景中都很有用:

\begin{itemize}
    \item 在一个线程池库中,一个\texttt{Task} trait表示一个工作单元,它可能有一个关联的\texttt{Output}类型。
    \item 一个\texttt{Pattern} trait表示一种搜索字符串的方式,它可能有一个关联的\texttt{Match}类型,表示字符串中和模式匹配的所有信息:
    \begin{minted}{Rust}
    trait Pattern {
        type Match;

        fn search(&self, string: &str) -> Option<Self::Match>;
    }

    /// 你可以在字符串中搜索一个特定的字符。
    impl Pattern for char {
        /// 一个`Match`只是发现字符的位置
        type Match = usize;

        fn search(&self, string: &str) -> Option<usize> {
            ...
        }
    }
    \end{minted}

    如果你熟悉正则表达式,那么很容易就能看出\texttt{impl Pattern for RegExp}将会有一个更加精密的\texttt{Match}类型,可能是一个包含匹配的开始和结尾、匹配的括号组的位置等内容的结构体。

    \item 一个用于关系型数据库的库可能有一个\texttt{DatabaseConnection} trait,它有一个关联类型表示事务、游标、预处理语句等等。
\end{itemize}

关联类型完美适用于每一个实现都有\emph{一个}特定的相关类型的情况:每一个\texttt{Task}的类型产生一个特定类型的\texttt{Output};每一个\texttt{Pattern}的类型查找一个特定的\texttt{Match}类型。然而,正如我们即将看到的一样,一些类型间的关系并不是这种模式。

\subsection{泛型trait(或运算符重载是如何工作的)}
Rust中的乘法使用了这个trait:
\begin{minted}{Rust}
    /// std::ops::Mul,用于支持乘法(`*`)的类型
    pub trait Mul<RHS> {
        /// `*`运算符产生的结果的类型
        type Output;

        /// `*`运算符用到的的方法
        fn mul(self, rhs: RHS) -> Self::Output;
    }
\end{minted}

\texttt{Mul}是一个泛型类型。类型参数\texttt{RHS}是\emph{右手边(righthand side)}的缩写。

这里的类型参数和在结构体或函数中的含义一样:\texttt{Mul}是一个泛型trait,它实例化出的\texttt{Mul<f64>}、\texttt{Mul<String>}、\texttt{Mul<Size>}等都是不同的trait,正如\texttt{min::<i32>}和\texttt{min::<String>}\\
是不同的函数、\texttt{Vec<i32>}和\texttt{Vec<String>}是不同的类型一样。

单个类型例如\texttt{WindowSize},可以同时实现\texttt{Mul<f64>}和\texttt{Mul<i32>},甚至更多。你可以将一个\texttt{WindowSize}和很多其它类型相乘。每一个实现都有它自己的关联\texttt{Output}类型。

泛型trait可以不受孤儿规则的约束:你可以为一个外部类型实现一个外部trait,只要trait的类型参数中有一个是在当前crate中定义的类型。因此,假设你自己已经定义了\texttt{WindowSize},你可以为\texttt{f64}实现\texttt{Mul<WindowSize>},即使你既没有定义\texttt{Mul}又没有定义\texttt{f64}。这些实现甚至也可以是泛型的,例如\texttt{impl<T> Mul<WindowSize> for Vec<T>}。之所以可以这样是因为在别的crate中没有任何方法可以为任何类型实现\texttt{Mul<WindowSize>},因此和你的实现之间不可能发生冲突。(我们在\nameref{OrphanRule}一节中介绍过孤儿规则。)这正是像\texttt{nalgebra}这样的crate为vector定义算术运算的方法。

之前展示的trait忽略了一个小细节。真正的\texttt{Mul} trait看起来像这样:
\begin{minted}{Rust}
    pub trait Mul<RHS=Self> {
        ...
    }
\end{minted}

语法\texttt{RHS=Self}意思是\texttt{RHS}的默认值为\texttt{Self}。如果我们写\texttt{impl Mul for Complex},而不指明\texttt{Mul}的类型参数,那么意味着\texttt{impl Mul<Complex> for Complex}。如果我们在一个约束中写\texttt{where T: Mul},那么意味着\texttt{T: Mul<T>}。

在Rust中,表达式\texttt{lhs * rhs}是\texttt{Mul::mul(lhs, rhs)}的缩写。因此在Rust中重载\texttt{*}运算符和实现\texttt{Mul} trait一样简单。我们将在下一章中展示示例。

\subsection{\texttt{impl Trait}}
你可能想象过,组合使用多种泛型类型可能会变得一团糟。例如,仅仅只使用标准库中的组合器组合几个迭代器会让你的返回类型变得眼花缭乱:
\begin{minted}{Rust}
    use std::iter;
    use std::vec::IntoIter;
    fn cyclical_zip(v: Vec<u8>, u: Vec<u8>) ->
        iter::Cycle<iter::Chain<IntoIter<u8>, IntoIter<u8>>> {
            v.into_iter().chain(u.into_iter()).cycle()
    }
\end{minted}

我们可以简单的将返回类型替换为一个triat对象:
\begin{minted}{Rust}
    fn cyclical_zip(v: Vec<u8>, u: Vec<u8>) -> Box<dyn Iterator<Item=u8>> {
        Box::new(v.into_iter().chain(u.into_iter()).cycle())
    }
\end{minted}

然而,在大多数情况下,仅仅为了避免丑陋的类型签名,就要在每一次调用这个函数时付出动态分发的开销和一次不可避免的堆分配并不是一个好的折衷。

Rust有一个专为此情形设计的特性叫做\texttt{impl Trait}。\texttt{impl Trait}允许我们“擦除”返回值的类型,只指明它实现的trait或traits,并且没有动态分发或者堆分配:
\begin{minted}{Rust}
    fn cyclical_zip(v: Vec<u8>, u: Vec<u8>) -> impl Iterator<Item=u8> {
        v.into_iter().chain(u.into_iter()).cycle()
    }
\end{minted}

现在,与指明嵌套的迭代器组合器结构体的类型相比,\texttt{cyclical\_zip}的签名只简单的说明了它返回一种产生\texttt{u8}的迭代器。返回类型表达了函数的意图,而不是实现细节。

这确实清理了代码并增强了可读性,但\texttt{impl Trait}并不只是一个方便的缩写。使用\texttt{impl Trait}意味着你可以在将来修改实际返回的类型,只要新的类型仍然实现了\\
\texttt{Iterator<Item=u8>},任何调用了这个函数的代码将仍然能不出错地继续编译。这为库的作者提供了很大的灵活性,因为只有相关的的功能被编码进类型签名。

例如,如果一个库的第一版按照上面的方法使用迭代器组合器,然后又发现了一个更好的算法,那么库的作者可以使用不同的迭代器组合器或者甚至返回一个自定义的实现了\texttt{Iterator}的类型,而库的用户可以在完全不改变代码的情况下享受性能的提升。

使用\texttt{impl Trait}类似于面向对象语言中广泛使用的工厂模式的静态分发版本,这很有诱惑力。例如,你可以定义一个这样的trait:
\begin{minted}{Rust}
    trait Shape {
        fn new() -> Self;
        fn area(&self) -> f64;
    }
\end{minted}

在为几个类型实现了它之后,你可能想根据一个运行时的值来决定使用不同的\texttt{Shape},例如一个用户输入的字符串。使用\texttt{impl Shape}作为返回类型并不可行:
\begin{minted}{Rust}
    fn make_shape(shape: &str) -> impl Shape {
        match shape {
            "circle" => Circle::new(),
            "triangle" => Triangle::new(),  // 错误:不兼容的类型
            "shape" => Rectangle::new(),
        }
    }
\end{minted}

从调用者的角度来看,像这样的函数并没有什么意义。\texttt{impl Trait}是一种静态分发的版本,因此编译器需要在编译期知道函数内返回的实际类型,这样才能在栈上分配正确数量的空间并调用正确的字段和方法。这里,它可能是\texttt{Circle}、\texttt{Triganle}或者\texttt{Rectangle},它们的空间大小都不同,而且都有不同的\texttt{area()}实现。

很重要的一点是要注意Rust不允许trait方法使用\texttt{impl Trait}作为返回类型。要想支持这一点需要对语言的类型系统进行一些改进。在这项工作完成之前,只有自由函数和关联到特定类型的函数可以使用\texttt{impl Trait}作为返回值。

\texttt{impl Trait}也可以用来在函数中接受泛型参数。例如,考虑下面的简单泛型代码:
\begin{minted}{Rust}
    fn print<T: Display>(val: T) {
        println!("{}", val);
    }
\end{minted}

它和下面的使用\texttt{impl Trait}的版本相同:
\begin{minted}{Rust}
    fn print(val: impl Display) {
        println!("{}", val);
    }
\end{minted}

这里有一个很重要的例外。使用泛型允许函数的调用者指定泛型参数的类型,例如\texttt{print::<i32>(42)},而使用\texttt{impl Trait}则不行。

每一个\texttt{impl Trait}参数都会被赋予一个自己的匿名类型参数,因此\texttt{impl Trait}局限于最简单的泛型函数中,不能表示参数的类型之间的关系。

\subsection{关联常量}
像结构体和枚举一样,trait也可以有关联常量。你可以用和结构体或枚举一样的语法给trait声明关联常量:
\begin{minted}{Rust}
    trait Greet {
        const GREETING: &'static str = "Hello";
        fn greet(&self) -> String;
    }
\end{minted}

trait中的关联常量也有特殊的作用。像关联类型和函数一样,你可以声明它们但不赋给它们值:
\begin{minted}{Rust}
    trait Float {
        const ZERO: Self;
        const ONE: Self;
    }
\end{minted}

然后,实现这些trait的类型可以定义这些值:
\begin{minted}{Rust}
    impl Float for f32 {
        const ZERO: f32 = 0.0;
        const ONE: f32 = 1.0;
    }

    impl Float for f64 {
        const ZERO: f64 = 0.0;
        const ONE: f64 = 1.0;
    }
\end{minted}

你可以编写使用这些值的泛型代码:
\begin{minted}{Rust}
    fn add_one<T: Float + Add<Output=T>>(value: T) -> T {
        value + T::ONE
    }
\end{minted}

注意关联常量不能和trait对象一起使用,因为编译器依赖实现的类型信息,才能在编译期找出正确的值。

即使是一个没有任何行为的简单trait,例如\texttt{Float},也可以给出足够的类型信息,再搭配上少数运算符,就可以实现一些非常普遍的数学函数例如斐波那契数列:
\begin{minted}{Rust}
    fn fib<T: Float + Add<Output=T>>(n: usize) -> T {
        match n {
            0 => T::ZERO,
            1 => T::ONE,
            n => fib::<T>(n - 1) + fib::<T>(n - 2)
        }
    }
\end{minted}

在上面两节中,我们已经展示了用trait描述类型间关系的不同方法。所有这些都可以避免虚方法开销和向下转换,因为它们允许Rust在编译期就知道精确的类型。

\section{逆向工程约束}\label{RevBound}

当没有单个trait可以满足你的所有需求时,编写泛型代码可能会变得非常困难。假设我们写了这个做一些计算的函数:
\begin{minted}{Rust}
    fn dot(v1: &[i64], v2: &[i64]) -> i64 {
        let mut total = 0;
        for i in 0 .. v1.len() {
            total = total + v1[i] * v2[i];
        }
        total
    }
\end{minted}

现在我们想用相同的代码来处理浮点数值。我们可能会尝试这样写:
\begin{minted}{Rust}
    fn dot<N>(v1: &[N], v2: &[N]) -> N {
        let mut total: N = 0;
        for i in 0 .. v1.len() {
            total = total + v1[i] * v2[i];
        }
        total
    }
\end{minted}

这样是行不通的:Rust会抱怨\texttt{*}和\texttt{+}的使用以及\texttt{0}的类型。我们可以用\texttt{Add}和\texttt{Mul} trait来要求\texttt{N}是一个支持\texttt{+}和\texttt{*}的类型。对于\texttt{0}的使用也要修改,因为在Rust中\texttt{0}总是整数,而相应的浮点值是\texttt{0.0}。幸运的是,那些有默认值的类型有一个标准的\texttt{Default} trait。对于数值类型,默认值总是0:
\begin{minted}{Rust}
    use std::ops::{Add, Mul};

    fn dot<N: Add + Mul + Default>(v1: &[N], v2: &[N]) -> N {
        let mut total = N::default();
        for i in 0 .. v1.len() {
            total = total + v1[i] * v2[i];
        }
        total
    }
\end{minted}

这已经接近正确答案了,但还不够:
\begin{minted}{text}
    error: mismatched types
      |
    5 | fn dot<N: Add + Mul + Default>(v1: &[N], v2: &[N]) -> N {
      |        - this type parameter
    ...
    8 |         total = total + v1[i] * v2[i];
      |                         ^^^^^^^^^^^^^ expected type parameter `N`,
      |                                       found associated type
      |
      = note: expected type parameter `N`
                found associated type `<N as Mul>::Output`
    help: consider further restricting this bound
      |
    5 | fn dot<N: Add + Mul + Default + Mul<Output = N>>(v1: &[N], v2: &[N]) -> N {
      |                               ^^^^^^^^^^^^^^^^^
\end{minted}

我们的新代码假设两个\texttt{N}类型的值相乘产生另一个\texttt{N}类型的值。这并不是绝对的,你可以重载乘法运算符来返回任何你希望的类型。我们需要一种方式来告诉Rust这个泛型函数只能用于有普通乘法的类型,这也就是\texttt{N * N}要返回\texttt{N}类型的值。错误消息中的建议\emph{几乎总是}对的:我们可以把\texttt{Mul}换成\texttt{Mul<Output=N>},然后\texttt{Add}也进行相同的替换:
\begin{minted}{Rust}
    fn dot<N: Add<Output=N> + Mul<Output=N> + Default>(v1: &[N], v2: &[N]) -> N
    {
        ...
    }
\end{minted}

这个时候,约束已经开始逐渐累积,让代码变得难以阅读。让我们把约束移动到\texttt{where}子句中:
\begin{minted}{Rust}
    fn dot<N>(v1: &[N], v2: &[N]) -> N
        where N: Add<Output=N> + Mul<Output=N> + Default
    {
        ...
    }
\end{minted}

很好。但Rust仍然会抱怨下面这行代码:
\begin{minted}{text}
    error: cannot move out of type `[N]`, a non-copy slice
      |
    8 |         total = total + v1[i] * v2[i];
      |                         ^^^^^
      |                         |
      |                         cannot move out of here
      |                         move occurs because `v1[_]` has type `N`,
      |                         which does not implement the `Copy` trait
\end{minted}

因为我们没有要求\texttt{N}是一个可拷贝的类型,Rust把\texttt{v[i]}解释为尝试把一个值移出切片,这是禁止的。但我们根本不希望修改这个切片;我们只希望拷贝这个值来进行操作。幸运的是,所有Rust的内建数值类型都实现了\texttt{Copy},因此我们可以简单地把它添加到\texttt{N}的约束中:
\begin{minted}{Rust}
    where N: Add<Output=N> + Mul<Output=N> + Default + Copy
\end{minted}

这次,代码可以编译运行了。最终的代码看起来像这样:
\begin{minted}{Rust}
    use std::ops::{Add, Mul};

    fn dot<N>(v1: &[N], v2: &[N]) -> N
        where N: Add<Output=N> + Mul<Output=N> + Default + Copy
    {
        let mut total = N::default();
        for i in 0 .. v1.len() {
            total = total + v1[i] * v2[i];
        }
        total
    }

    #[test]
    fn test_dot() {
        assert_eq!(dot(&[1, 2, 3, 4], &[1, 1, 1, 1]), 10);
        assert_eq!(dot(&[53.0, 7.0], &[1.0, 5.0]), 88.0);
    }
\end{minted}

在Rust中偶尔会发生的一种情况是:有一段时间会与编译器激烈斗争,但最后写出来的代码看起来相当不错,好像编写起来轻而易举,并且运行得很漂亮。 

我们在这里做的就是对\texttt{N}的约束进行逆向工程,让编译器来指导并检查我们的工作。这段代码写起来很麻烦是因为标准库中没有单独的\texttt{Number} trait包含我们需要的所有运算符和方法。有一个流行的开源crate叫做\texttt{num}定义了这样一个trait!我们已经知道,我们可以在\emph{Cargo.toml}中添加\texttt{num}并编写:
\begin{minted}{Rust}
    use num::Num;

    fn dot<N: Num + Copy>(v1: &[N], v2: &[N]) -> N {
        let mut total = N::zero();
        for i in 0 .. v1.len() {
            total = total + v1[i] * v2[i];
        }
        total
    }
\end{minted}

正如在面向对象语言中正确的接口让一切变得美好一样,在泛型编程中,正确的trait让一切变得美好。

为什么我们会遇到这种问题?为什么Rust的设计者不让泛型变得类似于C++的模板一样,把约束隐藏在代码中,à la “duck typing”?

Rust的方案的一个优势是泛型代码的向前兼容性。你可以修改一个公有的泛型函数或方法的实现,只要你不修改签名,就不会影响到使用它的用户。

约束的另一个优势是当你遇到编译器的错误时,至少编译器可以告诉你错误在哪。C++编译器涉及到模板的错误消息比Rust的要长很多,并且会指出很多不同行的代码,因为编译器没有办法辨别到底是谁的错误导致了这个问题:是模板、或者是它的调用者?

可能显式写出约束最重要的优势是它们就在代码和文档中。你可以在Rust中查看泛型函数的签名,然后看出它到底接受什么类型的参数。而模板则做不到这一点。在像Boost这样的C++库中为参数类型编写完整的文档的工作甚至比我们在这里经历的工作更加艰巨。Boost的开发者们并没有一个可以检查他们的工作的编译器。

\section{trait作为基础}

trait是Rust最主要的特性之一,并且有充足的理由支持这一观点。设计一个程序或者库时没有什么比设计一个好的接口更重要了。

本章是语法、规则和解释的风暴。现在我们已经铺设好基础了,可以开始讨论Rust中更多trait和泛型的用法。事实上,我们才刚刚触及皮毛。接下来的两章将介绍标准库提供的通用trait。再往后的章节介绍闭包、迭代器、输入/输出、并发。trait和泛型在这些话题中都扮演了中心的角色。

    \chapter{运算符重载}\label{ch12}
    \chapter{工具trait}\label{ch13}

\emph{Science is nothing else than the search to discover unity in the wild variety of nature—or, more exactly, in the variety of our experience. Poetry, painting, the arts are the same search, in Coleridge’s phrase, for unity in variety.}

\begin{flushright}
    ——Jacob Bronowski
\end{flushright}

这一章将介绍Rust中的“工具” trait,它们是标准库中能够显著影响到编写Rust代码的方式的trait,因此你需要熟悉它们才能写出惯用的代码并设计出你的用户会觉得是“Rustic”的crate接口。它们可以分为三大类:

\codeentry{语言扩展trait}
\hangparagraph{正如我们上一章介绍的运算符重载trait可以让你对自己的类型使用Rust的表达式运算符,还有几个其他的标准库trait充当Rust的扩展,让你可以把自己的类型更紧密地集成到语言中。这一类包括\texttt{Drop}、\texttt{Deref}和\texttt{DerefMut},以及转换用的trait \texttt{From}和\texttt{Into}。我们将在本章介绍所有这些trait。}

\codeentry{标记trait}
\hangparagraph{有几个trait通常用于约束泛型类型变量来表达一些特殊的约束。这一类包括\texttt{Sized}和\texttt{Copy}。}

\codeentry{公开的词汇表trait}
\hangparagraph{这些trait并没有神秘的编译器集成,你可以在自己的代码中定义等价的trait。但它们服务于为常见问题制定常规解决方案的重要目标。这些trait在crate和模块之间的公共接口中特别有价值:通过减少不必要的变化,它们让接口更容易理解,它们还增加了不同crate的特性可以简单地集成在一起的可能性,并且无需样板或自定义的粘合代码。这一类包括\texttt{Default}、引用借用trait \texttt{AsRef}、\texttt{AsMut}、\texttt{Borrow}、\texttt{BorrowMut},可能失败的转换 trait\texttt{TryFrom}和\texttt{TryInto},以及\texttt{ToOwned} trait,它是\texttt{Clone}的泛化。}

\hyperref[t13-1]{表13-1}是对它们的总结。

\begin{table}[htbp]
    \centering
    \caption{工具trait汇总}
    \label{t13-1}
    \begin{tabular}{p{0.2\textwidth}p{0.9\textwidth}}
        \hline
        \textbf{trait}  & \textbf{说明} \\
        \hline

        \nameref{drop}  & 析构器。当一个值被drop时Rust会自动运行的清理代码。    \\
        \rowcolor{tablecolor}
        \nameref{sized} & 标记trait,标记一个类型有一个编译期已知的固定大小,与动态大小的类型(例如切片)相反。 \\
        \nameref{clone} & 支持克隆的类型。  \\
        \rowcolor{tablecolor}
        \nameref{Copy}  & 标记trait,标记一个类型可以通过按位拷贝包含值的内存来克隆新值。   \\
        \nameref{deref} & 为智能指针类型准备的trait。   \\
        \rowcolor{tablecolor}
        \nameref{default}   & 有一个有意义的“默认值”的类型。    \\
        \nameref{asref} & 用于从一个类型的值借用另一个类型的引用的转换trait。   \\
        \rowcolor{tablecolor}
        \nameref{borrow}& 转换trait,类似于\texttt{Asref/AsMut},但额外保证一致的哈希性、顺序性和相等性。   \\
        \nameref{from}  & 用于将一个类型的值转换为另一个类型的值的转换trait。   \\
        \rowcolor{tablecolor}
        \nameref{tryfrom}   & 用于将一个类型的值转换为另一个类型的值的转换trait,用于可能失败的转换。   \\
        \nameref{toowned}   & 将一个引用转换为一个有所有权的值的转换trait。 \\
    \end{tabular}
\end{table}

还有一些其它重要的标准库trait。我们将在\hyperref[ch15]{第15章}中介绍\texttt{Iterator}和\texttt{IntoIterator}。用于计算哈希值的\texttt{Hash} trait,将在\hyperref[ch16]{第16章}中介绍。还有一对标记线程安全类型的trait,\texttt{Send}和\texttt{Sync},将在\hyperref[ch19]{第19章}中介绍。

\section{\texttt{Drop}}\label{drop}

当一个值的所有者消失时,我们说Rust \emph{drop}了这个值。drop一个值意味着释放这个值拥有的所有其他值、堆上的存储空间和系统资源。drop会在各种情况下发生:当变量离开作用域时、处于表达式语句的末尾时、截断vector时从尾部移除元素时,等等。

在大多数情况下,Rust自动为你处理drop过程。例如,假设你定义了下面的类型:
\begin{minted}{Rust}
    struct Appellation {
        name: String,
        nicknames: Vec<String>
    }
\end{minted}

一个\texttt{Appellation}拥有为字符串内容和vector的元素缓冲区分配的堆上的空间。当一个\texttt{Appellation}被drop时,Rust会清理所有这些内容,你不需要编写任何代码。然而,如果你想的话,你可以通过实现\texttt{std::ops::Drop} trait来自定义Rust如何drop你的类型的值:
\begin{minted}{Rust}
    trait Drop {
        fn drop(&mut self);
    }
\end{minted}

\texttt{Drop}的实现类似于C++中的析构函数,或者其它语言中的终结函数。当一个值被drop时,如果它实现了\texttt{std::ops::Drop},Rust会在清理它的字段或元素之前先调用它的\texttt{drop}方法。这种\texttt{drop}的隐式调用是唯一一种调用这个方法的方式,如果你尝试显式地调用这个方法,Rust会标记为错误。

因为Rust会在drop一个值的字段或方法之前先用这个值调用\texttt{Drop::drop},所以这个方法接收到的值总是保持完全初始化的状态。我们的\texttt{Appellation}类型的一个\texttt{Drop}的实现可以充分利用它的字段:
\begin{minted}{Rust}
    impl Drop for Appellation {
        fn drop(&mut self) {
            print!("Dropping {}", self.name);
            if !self.nicknames.is_empty() {
                print!(" (AKA {})", self.nicknames.join(", "));
            }
            println!("");
        }
    }
\end{minted}

有了这个实现,我们可以写出下列代码:
\begin{minted}{Rust}
    {
        let mut a = Appellation {
            name: "Zeus".to_string(),
            nicknames: vec!["cloud collector".to_string(),
                            "king of the gods".to_string()]
        };

        println!("before assignment");
        a = Appellation { name: "Hera".to_string(), nicknames: vec![] };
        println!("at end of block");
    }
\end{minted}

当我们把第二个\texttt{Appellation}赋给\texttt{a}的时候,第一个值会被drop,当我们离开\texttt{a}的作用域时,第二个值也会被drop。这段代码会打印出如下内容:
\begin{minted}{text}
    before assignment
    Dropping Zeus (AKA cloud collector, king of the gods)
    at end of block
    Dropping Hera
\end{minted}

因为我们的\texttt{Appellation}的\texttt{std::ops::Drop}实现只打印了一条消息,那么它的内存到底是怎么被精确地清理掉的?\texttt{Vec}类型也实现了\texttt{Drop},drop它的每个元素,然后释放在堆上分配的缓冲区。一个\texttt{String}在内部使用\texttt{Vec<u8>}来保存文本,因此\texttt{String}自身没有实现\texttt{Drop},它让它的\texttt{Vec}来清理字符。同样的规则也适用于\texttt{Appellation}值:当一个值被drop时,它的\texttt{Vec}的\texttt{Drop}实现负责清理每一个字符串的内容,并最终释放存储元素的缓冲区。保存\texttt{Appellation}值的内存本身也有一个拥有者,可能是一个局部变量或者一些数据结构,它们负责释放它。

如果一个变量的值被移动走,导致当它离开作用域时是未初始化的状态,那么Rust会避免drop这个变量:它里面没有值可以drop。即使按照控制流一个变量的值可能被移动走、也可能没有的情况下,这个原则也会生效。Rust会使用一个不可见的标记来追踪变量的状态,它指示变量的值是否需要被drop:
\begin{minted}{Rust}
    let p;
    {
        let q = Appellation { name: "Cardamine hirsuta".to_string(),
                              nicknames: vec!["shotweed".to_string(),
                                              "bittercress".to_string()] };
        if complicated_condition() {
            p = q;
        }
    }
    println!("Sproing! What was that?");
\end{minted}

根据\texttt{complicated\_condition}返回\texttt{true}还是\texttt{false},\texttt{p}或者\texttt{q}最后将会拥有这个\texttt{Appellation},另一个变为未初始化。这个值最终落在哪个变量里决定了它会在\texttt{println!}之前还是之后被drop。因为\texttt{q}在\texttt{println!}之前离开作用域,而\texttt{p}在之后。尽管一个值可能会被移来移去,但Rust只会drop它一次。

你通常不需要实现\texttt{std::ops::Drop},除非你想定义一个拥有一些Rust不知道的资源的类型。例如,在Unix系统上,Rust的标准库内部使用下面的类型来表示一个操作系统文件描述符:
\begin{minted}{Rust}
    struct FileDesc {
        fd: c_int,
    }
\end{minted}

\texttt{FileDesc}的\texttt{fd}字段就是当程序使用完它之后应该被关闭的文件描述符的序号。\texttt{c\_int}是\texttt{i32}的一个别名。标准库中按照如下方式为\texttt{FileDesc}实现了\texttt{Drop}:
\begin{minted}{Rust}
    impl Drop for FileDesc {
        fn drop(&mut self) {
            let _ = unsafe { libc::close(self.fd) };
        }
    }
\end{minted}

这里,\texttt{libc::close}是C库中的\texttt{close}函数的Rust名称。Rust代码只能在\texttt{unsafe}块中调用C函数,因此这里标准库使用了\texttt{unsafe}块。

如果一个类型实现了\texttt{Drop},它就不能再实现\texttt{Copy}。如果一个类型是\texttt{Copy}的,那意味着按位复制就可以创建一个新的独立拷贝。但通常在同样的数据上调用同一个\texttt{drop}方法不止一次是一个错误。

标注prelude中包含了一个drop值的函数\texttt{drop},但它的定义一点也不神奇:
\begin{minted}{Rust}
    fn drop<T>(_x: T) { }
\end{minted}

换句话说,它以值接收参数,从调用者那里获取所有权——然后什么也不做。当\texttt{\_x}离开作用域时Rust会drop它的值,正如它对其他任何变量做的一样。

\section{\texttt{Sized}}\label{sized}

\emph{固定大小的类型(sized type)}是指那些所有实例值都占用相同大小的内存空间的类型。Rust中几乎所有的类型都是固定大小的:每一个\texttt{u64}都是8字节,每一个\texttt{(f32, f32, f32)}类型的值都占12个字节。即使枚举也是固定大小的:不管它当前实际的variant是哪一个,一个枚举总是占用能存下最大的variant的空间。即使\texttt{Vec<T>}拥有一个大小可变的堆上缓冲区,\texttt{Vec}值本身的大小就是一个缓冲区的指针,加上容量,加上长度。因此\texttt{Vec<T>}是一个固定大小的类型。

所有的固定大小的类型都实现了\texttt{std::marker::Sized} trait,它没有任何方法或关联类型。Rust为所有适合的类型自动实现它,你不能自己实现它。\texttt{Sized}唯一的用途是作为类型参数的约束:一个类似\texttt{T: Sized}的约束要求\texttt{T}是一个大小在编译期已知的类型。这种类型的trait被称为\emph{标记trait(marker trait)},因为Rust语言本身使用它们来标记有特定特点的类型。

然而,Rust还有少量\emph{大小不固定的类型(unsized type)},它们的值的大小并不相同。例如,字符串切片类型\texttt{str}(注意,没有\&)就是大小不固定的。字符串\texttt{"diminutive"}和\texttt{"big"}分别是占用了10个和3个字节的\texttt{str}切片的引用。如\hyperref[f13-1]{图13-1}所示。数组切片类型例如\texttt{[T]}(再次注意,这里也没有\&)也是大小不固定的:一个共享引用例如\texttt{\&[u8]}可以指向一个任意大小的\texttt{[u8]}切片。因为\texttt{str}和\texttt{[T]}类型表示不同大小的值的集合,因此它们是大小不固定的类型。

\begin{figure}[htbp]
    \centering
    \includegraphics[width=0.8\textwidth]{../img/f13-1.png}
    \caption{指向大小不固定的值的引用}
    \label{f13-1}
\end{figure}

Rust中另一种常见的大小不固定类型是\texttt{dyn}类型,它是trait对象引用的目标。正如我们在“\nameref{traitobject}”中解释的一样,一个trait对象是一个指向实现了给定trait的值的指针。例如,类型\texttt{\&dyn std::io::Write}和\texttt{Box<dyn std::io::Write>}是指向实现了\texttt{Write} trait的值的指针。被引用的目标可能是一个文件或者网络套接字,或者是实现了\texttt{Write}的自定义类型。因为实现了\texttt{Write}的类型的集合是开放的,所以\texttt{dyn Write}也是大小不固定的类型:它的值可能有任意的大小。

Rust不能在变量中存储大小不固定的值或者将它们传递为参数。你只能通过指针例如\texttt{\&str}或\texttt{Box<dyn Write>}来处理它们,指针本身是固定大小的。正如\hyperref[f13-1]{表13-1}所示,一个指向大小不固定的值的指针总是一个\emph{胖指针(fat pointer)},占用两个字节:一个指向切片的指针加上切片的长度、一个trait对象加上一个指向方法实现的vtable的指针。

trait对象和切片的指针是对称的。在这两种情况下,都缺乏相应的类型信息:你不能在不知道\texttt{[u8]}长度的情况下索引它,也不能在不知道被指向的值的具体\texttt{Write}实现的情况下调用\texttt{Box<dyn Write>}的方法。在这两种情况下,胖指针填充了类型缺失的信息,加上了一个长度或者vtable的指针。被省略的静态信息被替换为了动态信息。

因为大小不固定的类型限制太多,所以大多数泛型类型参数都被限制为\texttt{Sized}类型。事实上,它几乎总是必须的,因此在Rust中它是默认的:如果你写\texttt{struct S<T> \{ ... \}},Rust认为你的意思是\texttt{struct S<T: Sized> \{ ... \}}。如果你不让\texttt{T}有这个约束,你必须显式写出来,即\texttt{struct S<T: ?Sized> \{ ... \}}。\texttt{?Sized}语法专门用于这种场景,含义是”不需要是\texttt{Sized}“。例如,如果你写了\texttt{struct S<T: ?Sized> \{ b: Box<T> \}},那么Rust将允许你写\texttt{S<Str>}和\texttt{S<dyn Write>},这时\texttt{b}将是一个胖指针;而\texttt{S<i32>}和\texttt{S<String>}中,\texttt{b}是一个普通指针。

抛开它们的限制不谈,大小不固定的类型让Rust的类型系统工作得更加顺畅。如果阅读标准库的文档,你偶尔会看到类型参数中的\texttt{?Sized}约束,这几乎总是意味着给定的类型只能被指向,同时允许相关的代码既能处理普通类型、又能处理切片和trait对象。当一个类型参数有\texttt{?Sized}约束时,人们通常会说它是\emph{可能大小不固定(questionably sized)}:它可能是\texttt{Sized},也可能不是。

除了切片和trait对象之外,还有另一种大小不固定的类型。一个结构体的最后一个字段(也只有最后一个字段)可能是大小不固定的,那么这个结构体本身也是大小不固定的。例如,\texttt{Rc<T>}引用计数指针内部被实现为私有类型\texttt{RcBox<T>}的指针,它存储了\texttt{T}和引用计数。这里有一个\texttt{RcBox}的简化版的定义:
\begin{minted}{Rust}
    struct RcBox<T: ?Sized> {
        ref_count: usize,
        value: T,
    }
\end{minted}

\texttt{value}字段就是\texttt{Rc<T>}引用的\texttt{T}值,\texttt{Rc<T>}解引用之后就是一个这个字段的指针。\texttt{ref\_count}字段保存引用计数。

真正的\texttt{RcBox}是标准库的实现细节,不能用于公开使用。但假设我们在处理一个上面的定义。你可以将\texttt{RcBox}和固定大小的类型一起使用,例如\texttt{RcBox<String>},结果将是一个固定大小的结构体类型。或者你可以将它和大小不固定的类型一起使用,例如\texttt{RcBox<dyn std::fmt::Display>}(其中\texttt{Display}表示可以被\texttt{println!}以及类似的宏格式化的类型),\texttt{RcBox<dyn Display>}是一个大小不固定的结构体类型。

你不能直接创建一个\texttt{RcBox<dyn Display>}值。你必须先创建一个普通的、固定大小的\texttt{RcBox},它的\texttt{value}字段的类型需要实现\texttt{Display},例如\texttt{RcBox<String>}。然后Rust允许你把一个它的引用\texttt{\&RcBox<String>}转换为胖指针引用\texttt{\&RcBox<dyn Display>}:
\begin{minted}{Rust}
    let boxed_lunch: RcBox<String> = RcBox {
        ref_count: 1,
        value: "lunch".to_string()
    };

    use std::fmt::Display;
    let boxed_displayable: &RcBox<dyn Display> = &boxed_lunch;
\end{minted}

当向函数传递参数时会隐式发生这个转换,因此你可以向接受\texttt{\&RcBox<dyn Display>}参数的函数传递一个\texttt{\&RcBox<String>}:
\begin{minted}{Rust}
    fn display(boxed: &RcBox<dyn Display>) {
        println!("For your enjoyment: {}", &boxed.value);
    }

    display(&boxed_lunch);
\end{minted}

这将会产生下列输出:
\begin{minted}{text}
    For your enjoyment: lunch
\end{minted}

\section{\texttt{Clone}}\label{clone}

\texttt{std::clone::Clone} trait用于那些可以拷贝自身的类型。\texttt{Clone}的定义如下:
\begin{minted}{Rust}
    trait Clone: Sized {
        fn clone(&self) -> Self;
        fn clone_from(&mut self, source: &Self) {
            *self = source.clone()
        }
    }
\end{minted}

\texttt{clone}方法应该构建一个\texttt{self}的独立拷贝并返回它。因为这个方法的返回类型是\texttt{Self},并且函数不能返回大小不固定的值,因此\texttt{Clone} trait扩展了\texttt{Sized} trait:它会约束实现的\texttt{Self}类型是\texttt{Sized}。

拷贝一个值通常意味着拷贝它拥有的所有内容,因此\texttt{clone}可能在时间和内存上的开销都比较大。例如,克隆一个\texttt{Vec<String>}不止要拷贝vector,还要拷贝它的每一个\texttt{String}元素。这就是为什么Rust不会自动拷贝值,而是要求你显式地调用方法来拷贝。引用计数指针例如\texttt{Rc<T>}和\texttt{Arc<T>}是例外:拷贝它们只会简单地递增引用计数并返回给你一个新的指针。

\texttt{clone\_from}方法将\texttt{self}修改为\texttt{source}的拷贝。\texttt{clone\_from}的默认实现简单地拷贝\texttt{source},然后将它移动进\texttt{*self}。这总是能正确工作,当对于某些类型,还有更快的方法达成相同的效果。例如,假设\texttt{s}和\texttt{t}都是\texttt{String}。语句\texttt{s = t.clone();}必须先克隆\texttt{t},drop掉\texttt{s}的旧值,然后将克隆的值移动进\texttt{s},因此这里面有一次堆分配和堆释放。但如果\texttt{s}原本的堆缓冲区有足够的容量存下\texttt{t}的内容,那么没有必要进行释放和分配操作:可以简单地将\texttt{t}的文本拷贝进\texttt{s}的缓冲区,然后调整它的长度。在泛型代码中,你应该使用\texttt{clone\_from},这样可以充分利用优化过的实现的优势。

如果你的类型的\texttt{Clone}实现只是简单地调用每一个字段或者元素的\texttt{clone}方法,然后利用这些克隆的值构造一个新的值,那么\texttt{clone\_from}的默认定义就已经够了,Rust将会为你实现它:只要在类型的定义上方加上\texttt{\#[derive(Clone)]}。

标准库中几乎所有应该能拷贝的类型都实现了\texttt{Clone}。基本类型例如\texttt{bool}和\texttt{i32}实现了。容器类型例如\texttt{String, Vec<T>, HashMap}也实现了。一些不应该能拷贝的类型例如\texttt{std::sync::Mutex}没有实现\texttt{Clone}。一些类型例如\texttt{std::fs::File}可以拷贝,但如果操作系统没有足够的资源那么拷贝可能会失败,这些类型也没有实现\texttt{Clone},因为\texttt{clone}不允许失败。作为替代,\texttt{std::fs::File}提供了一个\texttt{try\_clone}方法,它返回一个\texttt{std::io::Result<File>}来报告失败。

\section{\texttt{Copy}}\label{Copy}

在\hyperref[ch04]{第4章}中,我们解释过,对于大多数类型,赋值操作会移动它的值,而不是拷贝它们。移动值让我们可以更容易地追踪它们拥有的资源。但在“\nameref{copy}”中,我们指出了例外情况:不持有任何资源的简单类型可以是\texttt{Copy}类型,这种类型的赋值操作会拷贝源值,而不是移动值并把源值设为未初始化。

那个时候,我们并没有确切地说明\texttt{Copy}到底是什么,但现在我们可以告诉你:如果一个类型实现了\texttt{std::marker::Copy}标记trait,那么它就是\texttt{Copy}的,\texttt{Copy}的定义如下:
\begin{minted}{Rust}
    trait Copy: Clone { }
\end{minted}

很容易就可以为你自己的类型实现它:
\begin{minted}{Rust}
    impl Copy for MyType { }
\end{minted}

但因为\texttt{Copy}是一个有特殊含义的标记trait,所以Rust只允许可以通过逐字节的浅拷贝来拷贝自身的类型实现\texttt{Copy}。如果一个类型拥有任何其他资源,例如堆缓冲区或者操作系统句柄,那么它将不能实现\texttt{Copy}。

任何实现了\texttt{Drop} trait的类型不能实现\texttt{Copy}。Rust假定如果一个类型需要特殊的清理代码,那么它肯定也需要特殊的拷贝代码,所以不能是\texttt{Copy}。

和\texttt{Clone}一样,你可以使用\texttt{\#[derive(Copy)]}来让Rsut为你实现\texttt{Copy}。你通常能一次性看到它们两个,即\texttt{\#[derive(Copy, Clone)]}。

在将类型变为\texttt{Copy}之前请仔细思考。尽管这样做会让类型更容易使用,但却对类型本身的实现添加了很大的限制。而且隐式的拷贝可能会有很大的开销。我们已经在“\nameref{copy}”中详细解释了这些因素。

\section{\texttt{Deref}与\texttt{DerefMut}}\label{deref}

\section{\texttt{Default}}\label{default}

\section{\texttt{AsRef}与\texttt{AsMut}}\label{asref}

\section{\texttt{Borrow}与\texttt{BorrowMut}}\label{borrow}

\section{\texttt{From}与\texttt{Into}}\label{from}

\section{\texttt{TryFrom}与\texttt{TryInto}}\label{tryfrom}

\section{\texttt{ToOwned}}\label{toowned}

    \chapter{闭包}\label{ch14}
    \chapter{迭代器}\label{ch15}

\emph{It was the end of a very long day.}

\begin{flushright}
    ——Phil
\end{flushright}

一个\emph{迭代器(iterator)}可以产生一个指的序列,通常会使用一个循环来进行处理。Rust的标准库提供了遍历vector、字符串、哈希表和其他集合的迭代器,以及从一个输入流中产生若干行文本的迭代器、到达网络服务器的连接的迭代器、通过通道从其他线程接收到的值的迭代器,等等。当然,你可以实现自己的迭代器。Rust的\texttt{for}循环提供了一种自然地使用迭代器的语法,但迭代器自身也提供了丰富的方法集合用于映射、过滤、连接、收集等用途。

Rust的迭代器灵活、表达力强、高效。考虑下面的函数,它返回前\texttt{n}个正数的和(通常也被称为\emph{第n个三角数(nth triangle number)}:
\begin{minted}{Rust}
    fn triangle(n: i32) -> i32 {
        let mut sum = 0;
        for i in 1..=n {
            sum += i;
        }
        sum
    }
\end{minted}

表达式\texttt{1..=n}是一个\texttt{RangeInclusive<i32>}值。一个\texttt{RangeInclusive<i32>}是一个产生从起点到终点的所有整数的迭代器(包含起点和终点),因此你可以将它用作\texttt{for}循环的操作数来求\texttt{1}到\texttt{n}的和。

但迭代器也有一个\texttt{fold}方法,你可以使用它实现如下的等价定义:
\begin{minted}{Rust}
    fn triangle(n: i32) -> i32 {
        (1..=n).fold(0, |sum, item| sum + item)
    }
\end{minted}

以\texttt{0}作为起始的总和,\texttt{fold}会获取\texttt{1..=n}产生的每个值,然后用总和和产生的值调用闭包\texttt{|sum, item| sum + item},每一次闭包的返回值就是新的总和。它最后返回的值就是\texttt{fold}自身返回的值——在这个例子中,就是整个序列的总和。如果你习惯使用\texttt{for}和\texttt{while}循环,那么这看起来会有些奇怪,但一旦你习惯了它,\texttt{fold}就是一个可读性强而简洁的替代方案。

这种写法是函数式编程语言的标准写法,这使得表达式有更强的表现力。但Rsut的迭代器是精心设计的,为了保证编译器可以把它们翻译成优秀的机器代码。在release构建模式下构建上面第二个定义时,Rust知道\texttt{fold}的定义,并且把它内联进\texttt{triangle}。然后,闭包\texttt{|sum, item| sum + item}也会被内联。最后,Rust会检查组合之后的代码,然后发现有一种更简单的方法计算从1到\texttt{n}的和:和总是等于\texttt{n * (n+1) / 2}。Rust会把\texttt{triangle}的整个函数体,包括循环、闭包等所有内容,变成一次乘法指令和一些其他的位运算。

这个例子恰巧可以转换成简单的算术,但在更复杂的使用中迭代器也可以表现的很好。它们是Rust提供灵活抽象的同时只有很小甚至没有开销的另一个例子。

在本章中,我们将会解释:
\begin{itemize}
    \item \texttt{Iterator}和\texttt{IntoIterator} trait,它们是Rust迭代器的基础
    \item 经典迭代器管道的三个阶段:从初始的值创建一个迭代器;通过选择或处理值将一种迭代器变成另一种;消耗迭代器产生的值
    \item 如何为自己的类型实现迭代器
\end{itemize}

迭代器有很多方法,所以一旦你了解了大概的思路,就可以跳过一节。但迭代器在Rust的习惯用法中非常普遍,熟悉这些随附的工具对掌握这门语言至关重要。

\section{\texttt{Iterator}与\texttt{IntoIterator trait}}\label{iter}

一个迭代器是任何实现了\texttt{std::iter::Iterator} trait的类型:
\begin{minted}{Rust}
    trait Iterator {
        type Item;
        fn next(&mut self) -> Option<Self::Item>;
        ... // 很多默认方法
    }
\end{minted}

\texttt{Item}是迭代器产生的值的类型。\texttt{next}方法可能返回\texttt{Some(v)},其中\texttt{v}是迭代器的下一个值;或者返回\texttt{None},表示已经到达序列的终点。这里我们省略了\texttt{Iterator}的很多默认方法;我们将在本章的剩余部分分别介绍它们。

如果有一种自然的方法从一个类型上迭代,那么这个类型可以实现\texttt{std::iter::IntoIterator},它的\texttt{into\_iter}方法获取一个值并返回一个迭代它的迭代器:
\begin{minted}{Rust}
    trait IntoIterator where Self::IntoIter: Iterator<Item=Self::Item> {
        type Item;
        type IntoIter: Iterator;
        fn into_iter(self) -> Self::IntoIter;
    }
\end{minted}

\texttt{IntoIter}是迭代器自身的类型,\texttt{Item}是它产生的值的类型。我们称所有实现了\texttt{IntoIterator}的类型为\emph{可迭代对象(iterable)},因为你可以迭代它。

Rust的\texttt{for}循环将这些部分漂亮地组合在一起。为了迭代一个迭代器的元素,你可以写:
\begin{minted}{Rust}
    println!("There's:");
    let v = vec!["antimony", "arsenic", "alumium", "selenium"];

    for element in &v {
        println!("{}", element);
    }
\end{minted}

在底层,每一个\texttt{for}循环只是\texttt{IntoIterator}和\texttt{Iterator}的方法调用的缩写:
\begin{minted}{Rust}
    let mut iterator = (&v).into_iter();
    while let Some(element) = iterator.next() {
        println!("{}", element);
    }
\end{minted}

\texttt{for}循环使用了\texttt{IntoIterator::into\_iter}来把操作数\texttt{\&v}转换成一个迭代器,然后重复调用\texttt{Iterator::next}。每一次返回\texttt{Some(element)}时,\texttt{for}循环会执行循环体;如果它返回\texttt{None},循环会终止。

考虑这个例子,其中有一些迭代器的术语:
\begin{itemize}
    \item 正如我们所说,\emph{迭代器(iterator)}是任何实现了\texttt{Iterator}的类型。
    \item \emph{可迭代对象(iterable)}是任何实现了\texttt{IntoIterator}的类型:你可以通过调用它的\texttt{into\_iter}方法获得一个迭代它的迭代器。这个例子中vector的引用\texttt{\&v}就是可迭代对象。
    \item 一个迭代器\emph{产生(produce)}值。
    \item 迭代器产生的值是\emph{item}。这里,item是\texttt{"antimony", "arsenic},等等。
    \item 接受迭代器产生的item的代码是\emph{消费者(consumer)}。这个例子中,\texttt{for}循环就是消费者。
\end{itemize}

尽管\texttt{for}循环总是调用操作数的\texttt{into\_iter},你也可以直接向\texttt{for}循环传递迭代器;例如,当你在\texttt{Range}上循环时就是这种情况。所有的迭代器都会自动实现\texttt{IntoIterator},它们的\texttt{into\_iter}方法简单地返回迭代器自身。

如果在迭代器返回了\texttt{None}之后,你再调用它的\texttt{next}方法,那么\texttt{Iterator} trait并没有指定这种情况下该怎么做。大多数会再次返回\texttt{None},但不是所有。(如果这导致了问题,“\nameref{fuse}”中介绍的\texttt{fuse}适配器可能会有帮助。)

\section{创建迭代器}
Rust标准库文档中详细解释了每种类型提供哪些种类的迭代器,但标准库提供了一些通用的约定来帮助你找到需要的迭代器。

\subsection{\texttt{iter}和\texttt{iter\_mut}方法}
大多数集合类型提供\texttt{iter}和\texttt{iter\_mut}方法,它们返回一个迭代器,迭代器会产生每一个item的共享引用或可变引用。数组切片例如\texttt{\&[T]}和\texttt{\&mut [T]}也有\texttt{iter}和\texttt{iter\_mut}方法。除了使用\texttt{for}循环自动处理之外,这些方法是最常用的获得迭代器的方法:
\begin{minted}{Rust}
    let v = vec![4, 20, 12, 8, 6];
    let mut iterator = v.iter();
    assert_eq!(iterator.next(), Some(&4));
    assert_eq!(iterator.next(), Some(&20));
    assert_eq!(iterator.next(), Some(&12));
    assert_eq!(iterator.next(), Some(&8));
    assert_eq!(iterator.next(), Some(&6));
    assert_eq!(iterator.next(), None);
\end{minted}

这个迭代器的item类型是\texttt{\&i32}:每一次调用\texttt{next}都会产生下一个元素的引用,直到到达vector的终点。

每一个类型都可以实现\texttt{iter}和\texttt{iter\_mut},不管它们实现的方式是什么。\texttt{std::path::Path}的\texttt{iter}返回的迭代器一次产生路径的一段:
\begin{minted}{Rust}
    use std::ffi::OsStr;
    use std::path::Path;

    let path = Path::new("C:/Users/JimB/Downloads/Fedora.iso");
    let mut iterator = path.iter();
    assert_eq!(iterator.next(), Some(OsStr::new("C:")));
    assert_eq!(iterator.next(), Some(OsStr::new("Users")));
    assert_eq!(iterator.next(), Some(OsStr::new("JimB")));
    ...
\end{minted}

这个迭代器的item类型是\texttt{\&std::ffi::OsStr},它是操作系统调用接受的一种字符串类型的引用切片。

如果某个类型有不止一种迭代方式,那么这个类型通常为每种遍历方式提供特定的方法,因为这时普通的\texttt{iter}方法将会导致歧义。例如,\texttt{\&str}字符串切片类型没有\texttt{iter}方法。作为替代,假设\texttt{s}是\texttt{\&str},那么\texttt{s.bytes()}返回一个产生\texttt{s}的每个字节的迭代器,而\texttt{s.chars()}会以UTF-8编码解析它的内容,然后产生每一个Unicode字符。

\subsection{\texttt{IntoIterator}实现}
当一个类型实现了\texttt{IntoIterator}之后,你可以自己调用它的\texttt{into\_iter}方法,正如\texttt{for}循环做的一样:
\begin{minted}{Rust}
    // 你通常应该使用HashSet,但它的迭代顺序是不确定的,
    // 因此这个例子中BTreeSet会工作得更好。
    use std::collections::BTreeSet;
    let mut favorites = BTreeSet::new();
    favorites.insert("Lucy in the Sky With Diamonds".to_string());
    favorites.insert("Liebesträume No. 3".to_string());

    let mut it = favorites.into_iter();
    assert_eq!(it.next(), Some("Liebesträume No. 3".to_string()));
    assert_eq!(it.next(), Some("Lucy in the Sky With Diamonds".to_string()));
    assert_eq!(it.next(), None);
\end{minted}

大多数集合实际上都提供了好几个\texttt{IntoIterator}的实现,分别是为共享引用(\texttt{\&T})、可变引用(\texttt{\&mut T})、移动(\texttt{T})提供的实现:
\begin{itemize}
    \item 给定一个集合的\emph{共享引用(shared reference)},\texttt{into\_iter}返回一个产生item的共享引用的迭代器。例如,在上面的代码中,\texttt{(\&favorites).into\_iter()}将会返回一个\texttt{Item}类型是\texttt{\&String}的迭代器。
    \item 给定一个集合的\emph{可变引用(mutable reference)},\texttt{into\_iter}返回一个产生item的可变引用的迭代器。例如,如果\texttt{vector}是\texttt{Vec<String>},那么\texttt{(\&mut vector).into\_iter()}将返回一个\texttt{Item}类型是\texttt{\&mut String}的迭代器。
    \item 当集合\emph{以值}传递时,\texttt{into\_iter}返回一个获取集合所有权并返回item自身的迭代器;item的所有权从集合移动到消费者,原来的集合在这个过程中被消耗。例如,上面代码中的\texttt{favorites.into\_iter()}会返回一个产生每个字符串值的迭代器;消费者会接受每个字符串的所有权。当迭代器被drop时,\texttt{BTreeSet}中剩余的所有元素也都会被drop,并且集合会变为未初始化。
\end{itemize}

因为一个\texttt{for}循环会对操作数调用\texttt{IntoIterator::into\_iter},这三种实现会导致有下面三种迭代方式:迭代集合的共享引用、迭代集合的可变引用、或者消耗集合并获取它的元素的所有权:
\begin{minted}{Rust}
    for element in &collection { ... }
    for element in &mut collection { ... }
    for element in collection { ... }
\end{minted}

这三种写法会调用上面列出的\texttt{IntoIterator}实现之一。

并不是每个类型都提供了全部这三种实现。例如,\texttt{HashSet}、\texttt{BTreeSet}、\texttt{BinaryHeap}没有实现共享引用的\texttt{IntoIterator},因为修改它们的元素可能会破坏类型的不变量:修改后的值可能会有不同的哈希值、或者和它的邻居的顺序关系会改变,因此修改元素会导致它们北方在错误的地方。其他的类型支持可变性,但只支持部分。例如,\texttt{HashMap}和\texttt{BTreeMap}产生表项的value的可变引用,以及key的共享引用,原因和上面类似。

一般的准则是迭代应该高效和可预测,因此Rust不提供开销很大或者可能展现出令人惊讶的行为的实现(例如,重新哈希被修改的\texttt{HashSet}条目并因此导致之后的迭代中可能再次遇到它们)。

切片实现了三种\texttt{IntoIterator}变体中的两个;因为它们并不拥有自己引用的元素,因此没有“以值”的实现。作为代替,\texttt{\&[T]}和\texttt{\&mut [T]}的\texttt{into\_iter}返回一个产生共享引用和可变引用的迭代器。如果你把底层切片类型\texttt{[T]}想象成一种集合,那么它就落入了之前的模式。

你可能已经注意到前两种\texttt{IntoIterator}的变体产生共享和可变的引用,这和调用\texttt{iter}或者\texttt{iter\_mut}是等价的。为什么Rust同时提供两者?

\texttt{IntoIterator}让\texttt{for}循环能正常工作,因此它显然是必要的。但当你不使用\texttt{for}循环时,使用\texttt{favorites.iter()}比\texttt{(\&favorites).into\_iter()}更加清晰。你可能会频繁需要以共享引用迭代,因此\texttt{iter}和\texttt{iter\_mut}也很有用。

\texttt{IntoIterator}在泛型代码中也很有用:你可以使用一个约束例如\texttt{T: IntoIterator}来限制类型参数\texttt{T}必须是可以迭代的类型。或者,你可以写\texttt{T: IntoIterator<Item=U>}来进一步要求迭代会产生\texttt{U}类型的值。例如,这个函数打印出任何item可以用\texttt{"{:?}"}格式打印的可迭代对象:
\begin{minted}{Rust}
    use std::fmt::Debug;

    fn dump<T, U>(t: T)
        where T: IntoIterator<Item=U>,
              U: Debug
    {
        for u in t {
            println!("{:?}", u);
        }
    }
\end{minted}
你不能在这个泛型函数中使用\texttt{iter}或者\texttt{iter\_mut},因为它们不是任何trait的方法:大多数可迭代类型只是恰好有这两个方法。

\subsection{\texttt{from\_fn}和\texttt{successors}}

一个简单而通用的产生一个值序列的方式是提供一个返回它们的闭包。

给定一个返回\texttt{Option<T>}的函数,\texttt{std::iter::from\_fn}返回一个迭代器,它简单地调用那个函数来产生item。例如:
\begin{minted}{Rust}
    use rand::random;   // 在Cargo.toml中添加依赖:rand = "0.7"
    use std::iter::from_fn;
    // 产生1000个随机数,在[0, 1]之间均匀分布。
    // (这并不是你想在`rand_distr` crate中找到的分布,
    // 但你可以很容易地自己实现它)
    let lengths: Vec<f64> =
        from_fn(|| Some((random::<f64>() - random::<f64>()).abs()))
        .take(1000)
        .collect();
\end{minted}

这里调用了\texttt{from\_fn}来制作一个产生随机数的迭代器。因为这个迭代器总是返回\texttt{Some},因此这个序列永远不会终止,但我们调用了\texttt{take(1000)}来限制只要前1000个元素。然后\texttt{collect}从最后的迭代器构建一个vector。这是一种高效地构建初始化的vector的方式。我们将在本章稍后的“\nameref{BuildColl}”中介绍为什么。

如果每一个item都依赖上一个,那么\texttt{std::iter::successors}函数可以漂亮地工作。你需要提供一个初始item,和一个获取上一个item并返回一个下一个item的\texttt{Option}。如果返回\texttt{None},那么迭代终止。例如,这里有另一种编写\hyperref[ch02]{第2章}中的曼德勃罗集绘制器的\texttt{escape\_time}函数的方法:
\begin{minted}{Rust}
    use num::Complex;
    use std::iter::successors;

    fn escape_time(c: Complex<f64>, limit: usize) -> Option<usize> {
        let zero = Complex { re: 0.0, im: 0.0 };
        successors(Some(zero), |&z| { Some(z * z + c) })
            .take(limit)
            .enumerate()
            .find(|(_i, z)| z.norm_sqr() > 4.0)
            .map(|(i, _z)| i)
    }
\end{minted}

从zero开始,\texttt{successors}调用通过重复平方再加上参数\texttt{c}来产生一个复平面上点的序列。当绘制曼德勃罗集时,我们希望知道这个序列会一直在原点附近还是远离原点。\texttt{take(limit)}调用设置了序列长度的限制,\texttt{enumerate}为每一个点加上一个序号、把每个点\texttt{z}变为元组\texttt{(i, z)}。然后我们使用\texttt{find}来查找第一个离远点足够远可以逃离的点。如果存在这样的点,\texttt{find}方法返回一个\texttt{Option::Some((i, z))},否则返回\texttt{None}。\texttt{Option::map}的调用会把\texttt{Some((i, z))}变为\texttt{Some(i)},但不会改变\texttt{None}:这正是我们想要的返回值。

\texttt{from\_fn}和\texttt{successors}都接受\texttt{FnMut}闭包,因此你的闭包可以捕获并修改作用域中的变量。例如,这个\texttt{fibonacci}函数使用一个\texttt{move}闭包来捕获一个变量并使用它作为运行状态:
\begin{minted}{Rust}
    fn fibonacci() -> impl Iterator<Item=usize> {
        let mut state = (0, 1);
        std::iter::from_fn(move || {
            state = (state.1, state.0 + state.1);
            Some(state.0)
        })
    }

    assert_eq!(fibonacci().take(8).collect::<Vec<_>>(),
               vec![1, 1, 2, 3, 5, 8, 13, 21]);
\end{minted}

注意:\texttt{from\_fn}和\texttt{successors}方法非常灵活,你可以通过传递闭包来达到你想要的行为,并将很多迭代器的使用变为一次对其中一个的调用。但这样做会忽略迭代器提供的表明数据流动和使用标准名称用于通用模式的能力。在你使用这两个函数之前请确保你已经熟悉了本章中的其他迭代器方法,它们通常是更好的完成工的方式。

\subsection{\texttt{drain}方法}
很多集合类型提供一个\texttt{drain}方法来获取集合的可变引用,并返回一个迭代器把每个元素的所有权传递给消费者。然而,和\texttt{into\_iter()}以值获取集合并消耗它不同,\texttt{drain}借用一个集合的可变引用,并且当迭代器被drop时,它会移除集合中剩余的所有元素,让集合变为空。

对于可以用范围索引的类型,例如\texttt{String}、vector、\texttt{VecDeque},\texttt{drain}方法获取一个要移除的元素的范围,而不是消耗整个序列:
\begin{minted}{Rust}
    use std::iter::FromIterator;

    let mut outer = "Earth".to_string();
    let inner = String::from_iter(outer.dran(1..4));

    assert_eq!(outer, "Eh");
    assert_eq!(inner, "art");
\end{minted}

如果你确实要消耗整个序列,使用整个范围\texttt{..}作为参数。

\subsection{其他迭代器源}
上面的几节基本都是关于像vector和\texttt{HashMap}这样的集合类型的,但标准库中还有很多其他类型支持迭代。\autoref{t15-1}总结了一些有趣的类型,但还有更多没有列出。我们将在专门介绍特定类型的章节(即\hyperref[ch16]{第16章}、\hyperref[ch17]{第17章}、\hyperref[ch18]{第18章})中详细介绍其中的一些方法。

\begin{longtable}{p{0.22\textwidth}p{0.23\textwidth}p{0.45\textwidth}}
    \caption{标准库中的其他迭代器}
    \label{t15-1}\\
    \hline
    \textbf{类型或trait} & \textbf{表达式} & \textbf{注意} \\
    \hline
    \multirow{2}{*}{\texttt{std::ops::Range}} & \texttt{1..10} & 端点必须是整数才能迭代。包括起点但不包括终点。 \\
    & \texttt{(1..10).step\_by(2)} \cellcolor{tablecolor} & 产生1,3,5,7,9。 \cellcolor{tablecolor} \\
    \hline
    \texttt{std::ops::RangeFrom} & \texttt{1..} & 无限迭代。起点必须是整数。当值到达了这种类型的极限时可能会panic或者溢出。 \\
    \hline
    \rowcolor{tablecolor}
    \texttt{std::ops:: RangeInclusive} & \texttt{1..=10} & 类似\texttt{Range},但包括终点值。 \\
    \hline
    \texttt{Option<T>} & \texttt{Some(10).iter()} & 类似于一个长度为0(\texttt{None})或1的vector(\texttt{Some(v)})。 \\
    \hline
    \rowcolor{tablecolor}
    \texttt{Result<T, E>} & \texttt{Ok("blah").iter()} & 类似于\texttt{Option},产生\texttt{Ok}值。 \\
    \hline
    \multirow{7}{*}{\texttt{Vec<T>, \&[T]}} & \texttt{v.windows(16)} & 从左到右产生重叠的、连续的给定长度的切片。 \\
    & \texttt{v.chunks(16)} \cellcolor{tablecolor} & 从左到右产生非重叠的、连续的给定长度的切片。 \cellcolor{tablecolor} \\
    & \texttt{v.chunks\_mut(1024)} & 类似\texttt{chunks},不过切片是可变的。 \\
    & \texttt{v.split(|byte| byte \& 1 != 0)} \cellcolor{tablecolor} & 产生被满足条件的元素分隔的切片。 \cellcolor{tablecolor} \\
    & \texttt{v.split\_mut(...)} & 同上,但产生可变切片。 \\
    & \texttt{v.rsplit(...)} \cellcolor{tablecolor} & 类似\texttt{split},但从右向左产生切片。 \cellcolor{tablecolor} \\
    & \texttt{v.splitn(n, ...)} & 类似\texttt{split},但最多产生\texttt{n}个切片。 \\
    \hline
    \multirow{5}{*}{\texttt{String, \&str}} & \texttt{s.bytes()} \cellcolor{tablecolor} & 产生UTF-8字符串的字节。 \cellcolor{tablecolor} \\
    & \texttt{s.chars()} & 产生UTF-8字符串的\texttt{char}。 \\
    & \texttt{s.split\_whitespace()} \cellcolor{tablecolor} & 以空格分隔字符串,产生非空字符们的切片。 \cellcolor{tablecolor} \\
    & \texttt{s.lines()} & 产生字符串的每一行的切片。 \\
    & \texttt{s.split('/')} \cellcolor{tablecolor} & 用给定的模式分隔字符串,产生每两个匹配之间的内容的切片。模式可以是字符、字符串或者闭包。\cellcolor    {tablecolor} \\
    \hline
    \multirow{5}{*}{\shortstack[l]{\texttt{std::collections::}\\\texttt{HashMap, std::}\\\texttt{collections::BTreeMap}}} & \texttt{s.matches(char:: is\_numeric)} & 产生匹配给定模式的切片。 \\
    & \texttt{map.keys(), map.values()} \cellcolor{tablecolor} & 产生map的key或value的共享引用。 \cellcolor{tablecolor} \\
    & \texttt{map.values\_mut()} & 产生条目的value的可变引用。 \\
    \hline
    \multirow{3}{*}{\shortstack[l]{\texttt{std::collections::}\\\texttt{HashSet, std::}\\\texttt{collections::BTreeSet}}} & \texttt{set1.union(set2)} \cellcolor{tablecolor} & 产生\texttt{set1}和\texttt{set2}的并集的元素的共享引用。 \cellcolor{tablecolor} \\
    & \texttt{set1.intersection(set2)} & 产生\texttt{set1}和\texttt{set2}的交集的元素的共享引用。 \\
    & & \\
    \hline
    \rowcolor{tablecolor}
    \texttt{std::sync::mpsc:: Receiver} & \texttt{rev.iter()} & 产生另一个线程通过相应的\texttt{Sender}发送的值。 \\
    \hline
    \multirow{2}{*}{\texttt{std::io::Read}} & \texttt{stream.bytes()} & 产生来自I/O流的字节。 \\
    & \texttt{stream.chars()} \cellcolor{tablecolor} & 以UTF-8解析流,产生\texttt{char}。 \cellcolor{tablecolor} \\
    \hline
    \multirow{2}{*}{\texttt{std::io::BufRead}} & \texttt{bufstream.lines()} & 以UTF-8解析流,产生\texttt{String}。 \\
    & \texttt{bufstream.split(0)} \cellcolor{tablecolor} & 用给定的字节切分流,产生\texttt{Vec<u8>}缓冲区。 \cellcolor{tablecolor} \\
    \hline
    \texttt{std::fs::ReadDir} & \texttt{std::fs::read\_dir(path)} & 产生目录项。 \\
    \hline
    \rowcolor{tablecolor}
    \texttt{std::net::TcpListener} & \texttt{listener.incoming()} & 产生到来的网络连接。 \\
    \hline
    \multirow{3}{*}{自由函数} & \texttt{std::iter::empty()} & 立即返回\texttt{None}。 \\
    & \texttt{std::iter::once(5)} \cellcolor{tablecolor} & 产生给定值然后结束。 \cellcolor{tablecolor} \\
    & \texttt{std::iter::repeat("\#9")} & 永远产生给定值。 \\
    \hline
\end{longtable}

\section{迭代器适配器}

一旦你得到了一个迭代器,\texttt{Iterator} trait还提供了广泛的\emph{适配器方法(adapter method)},或者简称为\emph{适配器(adapter)},它们消耗一个迭代器然后构建一个新的迭代器。为了展示适配器如何工作,我们将从两个最流行的适配器\texttt{map}和\texttt{filter}开始。然后我们会介绍其他的适配器,它们包括几乎所有你能想到的把一个序列的值变成另一个序列的方法:截断、跳过、组合、反向、连接、重复,等等。

\subsection{\texttt{map}的\texttt{filter}}
\texttt{Iterator} trait的\texttt{map}适配器让你通过对每一个item应用一个闭包来产生新迭代器。\texttt{filter}迭代器让你通过一个闭包决定保留哪些item丢弃哪些item,以此过滤迭代器中的某些item。

例如,假设你在迭代文本的每一行,并且想省略每一行的前导和尾部的空格。标准库的\texttt{str::trim}方法排除一个\texttt{\&str}中的前导和尾部空格,返回一个新的新的借用\texttt{\&str}。你可以使用\texttt{map}适配器来对迭代器返回的每一行应用\texttt{str::trim}:
\begin{minted}{Rust}
    let text = "  ponies \n   giraffes\niguanas  \nsquid".to_string();
    let v: Vec<&str> = text.lines()
        .map(str::trim)
        .collect();
    assert_eq!(v, ["ponies", "giraffes", "iguanas", "squid"]);
\end{minted}

\texttt{text.lines()}调用返回一个产生每一行的迭代器。对迭代器调用\texttt{map}返回第二个迭代器,它会对每一行调用\texttt{str::trim},然后将结果作为产生的item。最后,\texttt{collect}把所有item聚集成一个vector。

当然,\texttt{map}返回的迭代器,本身也可以继续适配。如果你想从结果中排除“iguanas”,你可以像下面这样写:
\begin{minted}{Rust}
    let text = "  ponies \n   giraffes\niguanas  \nsquid".to_string();
    let v: Vec<&str> = text.lines()
        .map(str::trim)
        .filter(|s| *s != "iguanas")
        .collect();
    assert_eq!(v, ["ponies", "giraffes", "squid"]);
\end{minted}

这里\texttt{filter}返回第三个迭代器,只有当\texttt{map}返回的迭代器产生的item调用闭包\texttt{|s| *s != "iguanas"}后返回\texttt{true}时,第三个迭代器才会产生这个item。一个这样的迭代器适配器链就像Unix shell中的管道:每一个适配器都有单个功能,很容易就能看清楚值的序列是如何从左到右转换的。

这两个适配器的签名如下:
\begin{minted}{Rust}
    fn map<B, F>(self, f: ) -> impl Iterator<Item=B>
        where Self: Sized, F: FnMut(Self::Item) -> B;

    fn filter<P>(self, predicate: P) -> impl Iterator<Item=Self::Item>
        where Self: Sized, P: FnMut(&Self::Item) -> bool;
\end{minted}

在标准库中,\texttt{map}和\texttt{filter}实际上返回指定的不透明\texttt{struct}类型,分别是\texttt{std::iter::Map}和\texttt{std::iter::Filter}。然而,它们的名字提供的信息量很少,所以在本书中,我们将用\texttt{-> impl Iterator<Item=...>}来代替,因为它们能告诉我们我们实际想要知道的信息:这个方法返回一个产生给定类型的item的\texttt{Iterator}。

因为大多数适配器以值获取\texttt{self},所以它们需要\texttt{Self}是\texttt{Sized}(大多数迭代器都是)。

\texttt{map}迭代器会依次把所有item以值传递给闭包,然后把结果返回给消费者。\texttt{filter}迭代器以共享引用把所有的item传递给闭包,保留选中的item的所有权,然后把它们传递给消费者。这就是为什么上面的例子要先解引用\texttt{s}再和\texttt{"iguanas"}比较:\texttt{filter}迭代器的item类型是\texttt{\&str},所以闭包参数的类型是\texttt{\&\&str}。

有关迭代器适配器有两个重要的点。

首先,在一个迭代器上调用适配器并不会消耗任何item,它只会返回一个新的迭代器,这个迭代器按需处理第一个迭代器产生的item来产生自己的item。在一个适配器链中,唯一会消耗item的方式就是对最后的迭代器调用\texttt{next}。

因此在我们之前的例子中,\texttt{text.lines()}方法调用本身并不从字符串解析行,它只是返回一个迭代器,只有当需要的时候这个迭代器\emph{才会}解析行。类似的,\texttt{map}和\texttt{filter}只是返回需要时\emph{才会}映射或过滤的新迭代器。在最后一个\texttt{collect}开始对\texttt{filter}迭代器调用\texttt{next}之前,将不会有任何计算发生。

当你的适配器有副作用时这一点尤其重要。例如,下面的代码什么也不打印:
\begin{minted}{Rust}
    ["earth", "water", "air", "fire"]
        .iter().map(|elt| println!("{}", elt));
\end{minted}

\texttt{iter}调用返回一个迭代数组元素的迭代器,\texttt{map}调用返回第二个迭代器,第二个迭代器对第一个迭代器产生的每个值调用闭包。但如果整个链中没有要求产生值的操作,那么将不会有\texttt{next}方法被调用。事实上,Rust会警告你这种情况:
\begin{minted}{text}
    warning: unused `std::iter::Map` that must be used
      |
    7 | /     ["earth", "water", "air", "fire"]
    8 | |          .iter().map(|elt| println!("{}", elt));
      | |________________________________________________^
      |
      = note: iterators are lazy and do nothing unless consumed
\end{minted}

错误消息中的术语“lazy”并不是贬义词;它只是对任何直到需要时才进行计算的机制的一种称呼。迭代器应该做最少的必要的工作来满足\texttt{next}调用是Rust的习惯;在这个例子中,并没有\texttt{next}调用,因此不会有任何计算发生。

第二个重要的点是迭代器适配器是0成本抽象。因为\texttt{map}、\texttt{filter}以及它们的同伴都是泛型的,将它们用于迭代器会生成特定迭代器类型的代码。这意味着Rust有足够的信息把每一个迭代器\texttt{next}方法内联到消费者中,然后把整个操作作为一个单元翻译为机器码。因此我们上面展示的\texttt{lines/map/filter}迭代器链和你手写的代码一样高效:
\begin{minted}{Rust}
    for line in text.lines() {
        let line = line.trim();
        if line != "iguanas" {
            v.push(line);
        }
    }
\end{minted}

这一节剩余的部分将介绍\texttt{Iterator} trait可用的适配器。

\subsection{\texttt{filter\_map}和\texttt{flat\_map}}
\texttt{map}适配器适用于一个输入item产生一个输出item的情况。但如果你想删除迭代中的某些item而不是处理它们,或者想将一个item替换成0个或更多的item时该怎么做呢?\texttt{filter\_map}和\texttt{flat\_map}适配器赋予了你这种灵活性。

\texttt{filter\_map}适配器类似于\texttt{map},除了它的闭包要么将一个item转换成一个新的item(和\texttt{map}一样),要么从迭代中丢弃这个item。因此,它有些像\texttt{filter}和\texttt{map}的结合。它的签名如下:
\begin{minted}{Rust}
    fn filter_map<B, F>(self, f: F) -> impl Iterator<Item=B>
        where Self: Sized, F: FnMut(Self::Item) -> Option<B>;
\end{minted}

除了闭包返回\texttt{Option<B>}之外,而不是\texttt{B}之外,它和\texttt{map}的签名是一样的。当闭包返回\texttt{None}时,这个item会从迭代器中丢弃;当它返回\texttt{Some(b)}时,\texttt{b}就是\texttt{filter\_map}迭代器产生的下一个item。

例如,假设你想扫描一个字符串中空格分隔的单词,找到其中可以被解析为数字的并处理它,然后丢弃其他单词。那你可以写:
\begin{minted}{Rust}
    use std::str::FromStr;

    let text = "1\nfrond .25 289\n3.1415 estuary\n");
    for number in text
        .split_whitespace()
        .filter_map(|w| f64::from_str(w).ok())
    {
        println!("{:4.2}", number.sqrt());
    }
\end{minted}
打印结果如下:
\begin{minted}{text}
    1.00
    0.50
    17.00
    1.77
\end{minted}

传给\texttt{filter\_map}的闭包尝试对每一个空格分隔的切片调用\texttt{f64::from\_str}。这会返回一个\texttt{Result<f64, ParseFloatError>},它的\texttt{.ok()}返回一个\texttt{Option<f64>}:解析错误变为\texttt{None},成功的解析会变为\texttt{Some(v)}。\texttt{filter\_map}迭代器丢弃所有的\texttt{None}值,然后对每一个\texttt{Some(v)}产生值\texttt{v}。

但为什么要将\texttt{map}和\texttt{filter}融合成这样的单个操作,而不是直接使用两个适配器?\texttt{filter\_map}适配器适用于刚刚展示过的这种情况,即只有实际尝试处理过item才知道应不应该包含这个item的情况。你可以只用\texttt{filter}和\texttt{map}做到同样的事情,但这样会很笨拙:
\begin{minted}{Rust}
    text.split_whitespace()
        .map(|w| f64::from_str(w))
        .filter(|r| r.is_ok())
        .map(|r| r.unwrap())
\end{minted}

你可以认为\texttt{flat\_map}适配器和\texttt{map}、\texttt{filter\_map}是同一类的,区别在于现在闭包不是只能返回一个item(\texttt{map})或者0或1个item(\texttt{filter\_map}),而是可以返回任意数量的item。\texttt{flat\_map}迭代器产生闭包返回的序列的串联。

\texttt{flat\_map}的签名如下:
\begin{minted}{Rust}
    fn flat_map<U, F>(self, f: F) -> impl Iterator<Item=U::Item>
        where F: FnMut(Self::Item) -> U, U: IntoIterator;
\end{minted}
传给\texttt{flat\_map}的闭包必须返回一个可迭代对象,但任何类型的可迭代对象都可以。\footnote{事实上,因为\texttt{Option}也是一个可迭代对象,行为就像一个有0个或者1个item的序列。所以假设\texttt{closure}返回一个\texttt{Option<T>},那么\texttt{iterator.filter\_map(closure)}等价于\texttt{iterator.flat\_map(closure)}。}

例如,假设我们有一个把国家映射到主要城市的表。给定一个构架的列表,那我们怎么遍历它们的主要城市?
\begin{minted}{Rust}
    use std::collections::HashMap;

    let mut major_cities = HashMap::new();
    major_cities.insert("Japan", vec!["Tokyo", "Kyoto"]);
    major_cities.insert("The United States", vec!["Portland", "Nashville"]);
    major_cities.insert(""Brazil", vec!["São Paulo", "Brasilia"]);
    major_cities.insert("Kenya", vec!["Nairobi", "Mombasa"]);
    major_cities.insert("The Netherlands", vec!["Amsterdam", "Utrecht"]);

    let countries = ["Japan", "Brazil", "Kenya"];

    for &city in countries.iter().flat_map(|country| &major_cities[country]) {
        println!("{}", city);
    }
\end{minted}
这会打印出下列内容:
\begin{minted}{text}
    Tokyo
    Kyoto
    São Paulo
    Brasilia
    Nairobi
    Mombasa
\end{minted}

这段代码的意思是,对于每一个国家,我们都获取它的城市的vector,然后将所有vector连接成单个序列,然后打印出来。

但记住迭代是惰性的:只有当\texttt{for}循环调用了\texttt{flat\_map}迭代器的\texttt{next}方法时才会开始计算。完全连接的序列从来不会在内存中构造。实际上,这里只有一个小的状态机,对于每一个城市迭代器,一次打印一个item,直到耗尽,然后为下一个国家产生一个新的城市迭代器。效果就类似于嵌套的循环,但被打包用作迭代器。

\subsection{\texttt{flatten}}
\texttt{flatten}适配器把迭代器的item连接起来,假设每一个item都是可迭代对象:
\begin{minted}{Rust}
    use std::collections::BTreeMap;

    // 把城市映射到公园的表:每一个value都是一个vector。
    let mut parks = BTreeMap::new();
    parks.insert("Portland",  vec!["Mt. Tabor Park", "Forest Park"]);
    parks.insert("Kyoto",     vec!["Tadasu-no-Mori Forest", "Maruyama Koen"]);
    parks.insert("Nashville", vec!["Percy Warner Park", "Dargon Park"]);

    // 构建一个所有公园的vector。`values`返回一个产生vector的迭代器,
    // 然后`flatten`按顺序产生每一个vector的元素。
    let all_parks: Vec<_> = parks.values().flatten().cloned().collect();

    assert_eq!(all_parks,
               vec!["Tadasu-no-Mori Forest", "Maruyama Koen", "Percy Warner Park", 
                    "Dragon Park", "Mt. Tabor Park", "Forest Park"]);
\end{minted}

“flatten”这个名字来自于想象把一个两层的结构压扁成一层的结构:\texttt{BTreeMap}和它的\texttt{Vec}的元素被压成一个产生所有元素的迭代器。

\texttt{flatten}的签名如下:
\begin{minted}{Rust}
    fn flatten(self) -> impl Iterator<Item=Self::Item::Item>
        where Self::Item: IntoIterator;
\end{minted}

换句话说,迭代器的item自身必须实现了\texttt{IntoIterator},这样它才是一个高效的序列的序列。\texttt{flatten}方法返回一个这些序列连接之后的迭代器。当然,这都是惰性完成的,只有当我们迭代完了一个序列才会从\texttt{self}产生一个新的item。

\text{flatten}方法还有一些令人惊讶的用法。如果你有一个\texttt{Vec<Option<...>>}并且你想只迭代其中的\texttt{Some}值,那么\texttt{flatten}可以漂亮地工作:
\begin{minted}{Rust}
    assert_eq!(vec![None, Some("day"), None, Some("one")]
               .into_iter()
               .flatten()
               .collect::<Vec<_>>(),
               vec!["day", "one"]);
\end{minted}

这种方式可以工作是因为\texttt{Option}自身实现了\texttt{IntoIterator},代表一个有0或1个元素的序列。\texttt{None}元素对迭代过程没有贡献,而每一个\texttt{Some}元素贡献一个值。类似的,你可以使用\texttt{flatten}来迭代\texttt{Option<Vec<...>>}:\texttt{None}和空vector的行为一样。

\texttt{Result}也实现了\texttt{IntoIterator},\texttt{Err}时代表一个空的序列,因此对一个产生\texttt{Result}值的迭代器调用\texttt{flatten}可以高效地排除所有\texttt{Err},产生一个解包之后的成功值的序列。我们不推荐在代码中忽略错误,但当用户知道自己在做什么时这是一个巧妙的技巧。

当你需要\texttt{flatten}时你可能会发现你真正需要的是\texttt{flat\_map}。例如,标准库的\texttt{str::to\_uppercase}方法把一个字符串转换成大写,工作方式类似于下面的代码:
\begin{minted}{Rust}
    fn to_uppercase(&self) -> String {
        self.chars()
            .map(char::to_uppercase)
            .flatten() // 有更好的方式
            .collect()
    }
\end{minted}

这里必须使用\texttt{flatten}的原因是\texttt{ch.to\_uppercase()}并不是返回单个字符,而是返回一个可能产生一个或更多字符的迭代器。将每一个字符映射到大写形式会返回一个产生字符迭代器的迭代器,\texttt{flatten}将它们拼接在一起,因此我们最后可以调用\texttt{collect}把它们转换为一个\texttt{String}。

但这种\texttt{map}和\texttt{flatten}的组合使用如此普遍,以至于\texttt{Iterator}提供了\texttt{flat\_map}适配器来处理这种情况。(事实上,\texttt{flat\_map}比\texttt{flatten}更先加入标准库。)因此上面的代码可以写成:
\begin{minted}{Rust}
    fn to_uppercase(&self) -> String {
        self.chars()
            .flat_map(char::to_upeprcase)
            .collect()
    }
\end{minted}




\subsection{\texttt{fuse}}\label{fuse}

\section{消耗迭代器}

\subsection{构建集合:\texttt{collect}和\texttt{FromIterator}}\label{BuildColl}

\section{实现自己的迭代器}
    \chapter{集合}\label{ch16}

\emph{We all behave like Maxwell’s demon. Organisms organize. In everyday experience lies the reason sober physicists across two centuries kept this cartoon fantasy alive. We sort the mail, build sand castles, solve jigsaw puzzles, separate wheat from chaff, rearrange chess pieces, collect stamps, alphabetize books, create symmetry, compose sonnets and sonatas, and put our rooms in order, and all this we do requires no great energy, as long as we can apply intelligence.}

\begin{flushright}
    ——James Gleick, The Information: A History, a Theory, a Flood
\end{flushright}

Rust标准库里包含几种\emph{集合(collection)},它们是在内存中存储数据的泛型类型。我们已经在本书的很多地方使用过集合,例如\texttt{Vec}和\texttt{HashMap}。在本章中,我们将详细介绍这两种类型的方法,以及其他六种标准集合。但在我们开始之前,让我们先讨论一下Rust的集合和其他语言中的集合的一些不同之处。

首先,移动和借用无处不在。Rust使用移动来避免深拷贝。这就是为什么\texttt{Vec<T>::push(item)}方法以值获取参数,而不是以引用。值会被移动进vector。\hyperref[ch04]{第4章}中的图展示了实践中的表现:在Rust中把一个\texttt{String}添加到\texttt{Vec<String>}中很快,因为Rust不需要拷贝字符串的字符数据,字符串的所有权归属也总是很清楚。

其次,当集合改变大小或者被修改的同时还有指向它们的数据的指针时,Rust不会有无效性错误——即悬垂指针。无效性错误是C++中另一种未定义行为的来源,即使在内存安全的语言中也可能导致\texttt{ConcurrentModificationException}。Rust借用检查器会在编译器检查出它们。

最后,Rust没有\texttt{null},因此我们将在其他语言中需要\texttt{null}的地方看到\texttt{Option}。

除了这些不同之外,Rust的集合可能正是你需要的。如果你是经验丰富的程序员并且时间不多,你可以跳过这部分,但不要跳过“\nameref{entry}”。

\section{概述}

\autoref{t16-1}展示了Rust的8种标准集合。它们都是泛型类型。

\begin{table}[htbp]
    \centering
    \caption{标准集合总结}
    \label{t16-1}
    \begin{tabular}{p{0.2\textwidth}p{0.2\textwidth}lll}
        \hline
        \multirow{2}{*}{\textbf{集合}}  & \multirow{2}{*}{\textbf{说明}} & \multicolumn{3}{l}{\textbf{其他语言中的类似集合类型}} \\
        \cline{3-5}
         & & \textbf{C++} & \textbf{Java} & \textbf{Python} \\
        \hline
        
        \texttt{Vec<T>} & 可增长的数组  & \texttt{vector} & \texttt{ArrayList} & \texttt{list}  \\
        \rowcolor{tablecolor}
        \texttt{VecDeque<T>} & 双端队列(可增长环形缓冲区) & \texttt{deque} & \texttt{ArrayDeque} & \texttt{collections.deque} \\
        \texttt{LinkedList<T>} & 双向链表 & \texttt{list} & \texttt{LinkedList} & —— \\
        \rowcolor{tablecolor}
        \texttt{BinaryHeap<T> where T: Ord} & 大顶堆 & \texttt{priority\_queue} & \texttt{PriorityQueue} & \texttt{heapq} \\
        \texttt{HashMap<K, V> where K: Eq + Hash} & 键值哈希表 & \texttt{unordered\_map} & \texttt{HashMap} & \texttt{dict} \\
        \rowcolor{tablecolor}
        \texttt{BTreeMap<K, V> where K: Ord} & 有序键值表 & \texttt{map} & \texttt{TreeMap} & —— \\
        \texttt{HashSet<T> where T: Eq + Hash} & 基于哈希的无序集合 & \texttt{unordered\_set} & \texttt{HashSet} & \texttt{set} \\
        \rowcolor{tablecolor}
        \texttt{BTreeSet<T> where T: Ord} & 有序集合 & \texttt{set} & \texttt{TreeSet} & —— \\
    \end{tabular}
\end{table}

\texttt{Vec<T>}、\texttt{HashMap<K, V>}、\texttt{HashSet<T>}是最常用的集合类型。其他的集合都有适用的场景。这一章将轮流讨论每一个集合类型:

\codeentry{Vec<T>}
\hangparagraph{一个客增唱的、在堆上分配的、\texttt{T}类值的数组。本章中大约一半的篇幅专门介绍\texttt{Vec}和它的有用的方法。}

\codeentry{VecDeque<T>}
\hangparagraph{类似于\texttt{Vec<T>},但是用作先进先出队列会更好。它支持高效地在首部和尾部添加或移除元素,但这种能力的代价是其他操作会稍微慢一点。}

\codeentry{BinaryHeap<T>}
\hangparagraph{一个优先队列。\texttt{BinaryHeap}中的值按照一定结构组织,因此总是可以高效地找到和移除最大值。}

\codeentry{HashMap<K, V>}
\hangparagraph{一个键值对的表。通过键查找值很快速。表中的条目以任意顺序存储。}

\codeentry{BTreeMap<K, V>}
\hangparagraph{类似于\texttt{HashMap<K, V>},但按键的顺序保持条目有序。一个\texttt{BTreeMap<String, i32>}按照\texttt{String}的比较顺序存储条目。除非你需要条目保持有序,否则\texttt{HashMap}会更快。}

\codeentry{HashSet<T>}
\hangparagraph{类型\texttt{T}的值的集合。添加和删除元素都很快,查询一个值是否在集合中也很快。}

\codeentry{BTreeSet<T>}
\hangparagraph{类似于\texttt{HashSet<T>},但保持元素有序。同样,除非你想要数据保持有序,否则\texttt{HashSet}会更快。}

因为\texttt{LinkedList}很少使用(并且在大多数情况下都有更好的替代,无论是性能还是接口),因此我们不会在这里介绍它。

\section{\texttt{Vec<T>}}

我们假设你对\texttt{Vec}已经有了一定了解,因为我们在本书的很多地方都已经使用过它。简要的介绍见“\nameref{vector}”。这里我们只会描述它的方法以及深入它的内部工作原理。

最简单的创建vector的方式是使用\texttt{vec!}宏:
\begin{minted}{Rust}
    // 创建一个空的vector
    let mut numbers: Vec<i32> = vec![];

    // 用给定的内容创建一个vector
    let words = vec!["step", "on", "no", "pets"];
    let mut buffer = vec![0u8; 1024];   // 1024个0字节
\end{minted}

正如我们在“\hyperref[ch04]{第4章}”所述,vector有三个字段:长度、容量、和一个指向堆上分配的缓冲区的指针。\autoref{f16-1}展示了上面的vector在内存中的视图。空vector,\texttt{numbers},初始长度为0。在它添加第一个元素之前不会有堆内存被分配。

\begin{figure}[htbp]
    \centering
    \includegraphics[width=0.9\textwidth]{../img/f16-1.png}
    \caption{vector的内存布局:words的每个元素是一个由指针和长度组成的\&str值}
    \label{f16-1}
\end{figure}

类似于所有集合,\texttt{Vec}实现了\texttt{std::iter::FromIterator},因此你可以对任何迭代器调用\texttt{.collect()}方法来创建一个vector,正如“\nameref{BuildColl}”中所述:
\begin{minted}{Rust}
    // 将一个其他集合转换成vector
    let my_vec = my_set.into_iter().collect::<Vec<String>>();
\end{minted}

\subsection{访问元素}
通过索引访问数组、切片或vector的元素非常直观:
\begin{minted}{Rust}
    // 获取一个元素的引用
    let first_line = &lines[0];

    // 获取一个元素的拷贝 
    let fifth_number = numbers[4];          // 需要Copy
    let second_number = lines[1].clone();   // 需要Clone

    // 获取一个切片的引用
    let my_ref = &buffer[4..12];

    // 获取一个切片的拷贝
    let my_copy = buffer[4..12].to_vec();   // 需要Clone
\end{minted}

当索引越界时所有这些方式都会panic。

Rust对数字类型很挑剔,vector也不例外。vector的长度和索引都是\texttt{usize}类型。尝试使用\texttt{u32}、\texttt{u64}、\texttt{isize}作为vector的索引会导致错误。必要时你可以使用\texttt{n as usize}来转换,见“\nameref{cast}”。

有几种方法提供了便捷地访问vector或切片的特定元素的方法(注意所有的切片方法都能用于数组和vector):
\codeentry{slice.first()}
\hangparagraph{返回\texttt{slice}的第一个元素的引用。返回类型是\texttt{Option<\&T>},因此如果\texttt{slice}为空时返回值为\texttt{None},不为空时返回值为\texttt{Some(\&slice[0])}}:
\begin{minted}{Rust}
    if let Some(item) = v.first() {
        println!("We got one! {}", item);
    }
\end{minted}

\codeentry{slice.last()}
\hangparagraph{和上边相似,不过返回最有一个元素的引用。}

\codeentry{slice.get(index)}
\hangparagraph{返回\texttt{slice[index]}的引用,如果存在的话。如果\texttt{slice}的元素数量小于\texttt{index+1},那么返回\texttt{None}}:
\begin{minted}{Rust}
    let slice = [0, 1, 2, 3];
    assert_eq!(slice.get(2), Some(&2));
    assert_eq!(slice.get(4), None);
\end{minted}

\codeentry{slice.first\_mut(), slice.last\_mut(), slice.get\_mut(index)}
\hangparagraph{与上面的类似,不过借用\texttt{mut}引用:}
\begin{minted}{Rust}
    let mut slice = [0, 1, 2, 3];
    {
        let last = slice.last_mut().unwrap();   // 最后一个元素类型:&mut i32
        assert_eq!(*last, 3);
        *last = 100;
    }
\end{minted}

因为以值返回\texttt{T}意味着移动它,因此访问元素的方法通常返回元素的引用。

一个例外是\texttt{.to\_vec()}方法,它获取拷贝:

\codeentry{slice.to\_vec()}
\hangparagraph{克隆整个切片,返回一个新的vector:}
\begin{minted}{Rust}
    let v = [1, 2, 3, 4, 5, 6, 7, 8, 9];
    assert_eq!(v.to_vec(),
               vec![1, 2, 3, 4, 5, 6, 7, 8, 9]);
    assert_eq!(v[0..6].to_vec(),
               vec![1, 2, 3, 4, 5, 6]);
\end{minted}
\hangparagraph{只有当元素可以拷贝时这个方法才可用,即\texttt{where T: Clone}}

\subsection{迭代}
vector和切片可以以值或者以引用迭代,遵循“\nameref{IntoIter}”中介绍的模式:
\begin{itemize}
    \item 迭代\texttt{Vec<T>}会产生\texttt{T}类型的item。元素被逐个移出vector消耗掉。
    \item 迭代\texttt{\&[T; N], \&[T], \&Vec<T>}——即数组、切片或vector的引用——会产生\texttt{\&T}类型的item,每一个item指向一个元素,不会移动元素。
    \item 迭代\texttt{\&mut [T; N], \&mut [T], \&mut Vec<T>}产生\texttt{\&mut T}类型的item。
\end{itemize}

数组、切片和vector还有\texttt{.iter()}和\texttt{.iter\_mut()}方法(见“\nameref{IterMethod}”)创建产生元素的引用的迭代器。

我们将在“\nameref{split}”中介绍一些更有趣的迭代切片的方法。

\subsection{增长和缩减vector}
数组、切片或vector的\emph{长度(length)}是它包含的元素的数量:

\codeentry{slice.len()}
\hangparagraph{返回一个\texttt{slice}的长度,类型为\texttt{usize}。}

\codeentry{slice.is\_empty()}
\hangparagraph{当\texttt{slice}不包含元素时为真(即\texttt{slice.len() == 0})。}

本节剩余的方法都是关于增长和缩减vector。它们不能用于数组和切片,因为它们一旦被创建之后就不能改变大小。

vector的所有元素都存储在一个在堆上分配的连续内存块中。vector的\emph{容量(capacity)}是指当前的内存块中最多能存储的元素数量。\texttt{Vec}通常会替你管理容量,当需要增长时它会自动分配更大的缓冲区并把元素都移动过去。还有一些显式管理容量的方法:

\codeentry{Vec::with\_capacity(n)}
\hangparagraph{创建一个容量为\texttt{n}的新的空vector。}

\codeentry{vec.capacity()}
\hangparagraph{返回\texttt{vec}的容量,类型是\texttt{usize}。\texttt{vec.capacity() >= vec.len()}总是为真。}

\codeentry{vec.reserve(n)}
\hangparagraph{保证vector的剩余空间至少还能再存储\texttt{n}个或更多元素:即\texttt{vec.capacity()}至少是\texttt{vec.len() + n}。如果已经有足够的空间,它不做任何事。否则,它会分配一个更大的缓冲区并且把vector的内容移动过去。}

\codeentry{vec.reserve\_exact(n)}
\hangparagraph{类似于\texttt{vec.reserve(n)},但告诉\texttt{vec}不要为未来的增长分配额外的空间。调用它之后,\texttt{vec.capacity()}等于\texttt{vec.len() + n}。}

\codeentry{vec.shrink\_to\_fit()}
\hangparagraph{当\texttt{vec.capacity()}大于\texttt{vec.len()}时尝试释放额外的内存。}

\texttt{Vec<T>}有很多添加或移除元素的方法,同时改变vector的长度。所有这些方法都以\texttt{mut}引用获取\texttt{self}参数。

下面这两个方法在vector的末尾添加或移除一个元素:

\codeentry{vec.push(value)}
\hangparagraph{把\texttt{value}添加到\texttt{vec}的末尾。}

\codeentry{vec.pop()}
\hangparagraph{移除并返回最后一个元素。返回类型是\texttt{Option<T>}。当vector已经为空时返回\texttt{None},否则返回\texttt{Some(x)}。}

注意\texttt{.push()}以值而不是以引用获取参数。类似的,\texttt{.pop()}返回被弹出的值,而不是引用。本节中剩余的大部分方法也是这样。它们从vector移出或移进值。

这两个方法向vector中添加值或者从vector中移出值:
\codeentry{vec.insert(index, value)}
\hangparagraph{在\texttt{vec[index]}处插入给定的\texttt{value},把\texttt{vec[index..]}中的值都向后移动一个位置来腾出空间。如果\texttt{index > vec.len()}会panic。}

\codeentry{vec.remove(index)}
\hangparagraph{移除并返回\texttt{vec[index]},把\texttt{vec[index+1..]}中的值向左移动一个位置来消除缝隙。}

\texttt{.insert()}和\texttt{.remove()}都很慢,因为有很多元素需要移动。

有四个方法可以将vector的长度调整为指定值:

\codeentry{vec.resize(new\_len, value)}
\hangparagraph{将\texttt{vec}的长度设为\texttt{new\_len}。如果这会增大\texttt{vec}的长度,将会用\texttt{value}的拷贝填充新空间。元素的类型必须实现了\texttt{Clone} trait。}

\codeentry{vec.resize\_with(new\_len, closure)}
\hangparagraph{类似于\texttt{vec.resize},但调用闭包来构造每一个新元素。它可以用于元素没有实现\texttt{Clone}的vector。}

\codeentry{vec.truncate(new\_len)}
\hangparagraph{将\texttt{vec}的长度缩减到\texttt{new\_len},丢弃\texttt{vec[new\_len..]}范围内的所有元素。如果\texttt{vec.len()}小于等于\texttt{new\_len},那么什么也不做。}

\codeentry{vec.clear()}
\hangparagraph{删除\texttt{vec}的所有元素。等价于\texttt{vec.truncate(0)}。}

有四个方法可以一次添加或移除很多元素:

\codeentry{vec.extend(iterable)}
\hangparagraph{将\texttt{iterable}的所有item按顺序添加到\texttt{vec}的末尾。它类似于多值版本的\texttt{.push()}。\texttt{iterable}参数可以是任何实现了\texttt{IntoIterator<Item=T>}。}
\hangparagraph{这个方法如此有用以至于有一个专门的trait \texttt{Extend},所有的标准集合都实现了它。不幸的是,这导致\texttt{rustdoc}将\texttt{.extend()}和其他trait的方法放在生成的HTML底部的一堆方法中,因此当你需要它时很难找到它。你必须记住它!更多内容见“\nameref{extend}”。}

\codeentry{vec.split\_off(index)}
\hangparagraph{类似于\texttt{vec.truncate(index)},除了它返回一个\texttt{Vec<T>}包含\texttt{vec}尾部被移除的元素。它类似于\texttt{.pop()}的多值版本。}

\codeentry{vec.append(\&mut vec2)}
\hangparagraph{这会把\texttt{vec2}的所有元素移动进\texttt{vec},其中\texttt{vec2}是另一个\texttt{Vec<T>}类型的vector。调用之后,\texttt{vec2}变为空。}
\hangparagraph{这类似于\texttt{vec.extend(vec2)},除了调用之后\texttt{vec2}仍然存在,并且容量不变。}

\codeentry{vec.drain(range)}
\hangparagraph{这会从\texttt{vec}中移除范围\texttt{vec[range]},并返回一个迭代被移除元素的迭代器,其中\texttt{range}是一个范围值,例如\texttt{..}或\texttt{0..4}。}

还有一些选择性移除vector元素的古怪方法:
\codeentry{vec.retain(test)}
\hangparagraph{移除所有没有通过给定测试的方法。\texttt{test}参数是一个实现了\texttt{FnMut(\&T) -> bool}的函数或闭包。对于\texttt{vec}的每一个元素,它会调用\texttt{test(\&element)},如果返回\texttt{false},元素将会被移出vector然后丢弃。}
\hangparagraph{不考虑性能的话,这类似于:}
\begin{minted}{Rust}
    vec = vec.into_iter().filter(test).collect();
\end{minted}

\codeentry{vec.dedup()}
\hangparagraph{丢弃相邻的重复元素。它类似于Unix的\texttt{uniq} shell工具。它会扫描\texttt{vec}中寻找相邻的重复元素,然后丢弃掉多余的重复值,只留下一个:}
\begin{minted}{Rust}
    let mut byte_vec = b"Missssssissippi".to_vec();
    byte_vec.dedup();
    assert_eq!(&byte_vec, b"Misisipi");
\end{minted}
\hangparagraph{注意最后的输出中仍然有两个\texttt{'s'}字符。这个方法只移除\emph{相邻的(adjacent)}重复值。为了移除所有的重复值,你有三种选择:调用\texttt{.dedup()}之前先排序vector,将数据移动到一个“\nameref{set}”,或者(为了保持元素原本的顺序)使用这个\texttt{.retain()}技巧:}
\begin{minted}{Rust}
    let mut byte_vec = b"Missssssissippi".to_vec();

    let mut seen = HashSet::new();
    byte_vec.retain(|r| seen.insert(*r));

    assert_eq!(&byte_vec, b"Misp");
\end{minted}
\hangparagraph{这段代码的原理是当集合中已经包含要插入的item时\texttt{.insert()}会返回\texttt{false}。}

\codeentry{vec.dedup\_by(same)}
\hangparagraph{类似于\texttt{vec.dedup()},但它使用函数或者闭包\texttt{same(\&mut elem1, \&mut elem2)},而不是\texttt{==}运算符,来检查两个相邻元素是否被认为相等。}

\codeentry{vec.dedup\_by\_key(key)}
\hangparagraph{类似于\texttt{vec.dedup()},但当\texttt{key(\
&mut elem1) == key(\&mut elem2)}时它认为两个元素相等。}
\hangparagraph{例如,如果\texttt{errors}是一个\texttt{Vec<Box<dyn Error>>},你可以写:}
\begin{minted}{Rust}
    // 移除消息重复的错误。
    errors.dedup_by_key(|err| err.to_string());
\end{minted}

这一节介绍的所有方法中,只有\texttt{.resize()}可能会拷贝值。其他的通过移动值来工作。

\subsection{连接}






\subsection{分割}\label{split}

\section{\texttt{HashMap<K, V>}和\texttt{BTreeMap<K, V>}}

\subsection{条目}\label{entry}
    \chapter{字符串与文本}\label{ch17}
    \chapter{输入输出}\label{ch18}

\section{Reader和Writer}

\subsection{Reader}

\subsection{Buffered Reader}

\subsection{读取行}

\subsection{收集行}

\subsection{Writer}

\subsection{File}\label{file}
    \chapter{并发}\label{ch19}
    \chapter{异步编程}\label{ch20}

假设你正在编写一个聊天服务器。每个网络连接都有很多要解析的到来的包、要组装的发送的包、要管理的安全参数、要追踪的聊天组订阅等。同时为许多连接管理所有这些信息需要进行一些组织。

理想情况下,你可以简单地为每一个到来的连接启动一个单独的线程:
\begin{minted}{Rust}
    use std::{net, thread};

    let listener = net::TcpListener::bind(address)?;

    for socket_result in listener.incoming() {
        let socket = socket_result?;
        let groups = chat_group_table.clone();
        thread::spawn(|| {
            log_error(serve(socket, groups));
        });
    }
\end{minted}

这样对每一个连接都会创建一个新的线程运行\texttt{serve}函数,这个函数专门处理一个连接的需求。

这可以正常工作,直到一切都比计划的更加顺利很多,然后突然你就已经有了几万名用户。一个线程的栈增长到100KB或更多并不罕见,你可能不想就这样花费几GB的内存。要把任务分发到多个处理器上,线程是合适并且必须的,但它们的内存需求太大以至于我们通常需要一些补充的方式和线程一起使用,来减小资源占用。

你可以使用Rust的\emph{异步任务(asynchronous task)}来在单个线程或者线程池中交替执行很多独立的任务。异步任务类似于线程,但可以更快地创建、更高效地传递控制权、并且内存开销比线程少一个数量级。在单个程序中同时运行数十万个异步任务是完全可行的。当然,你的应用仍然可能被其他因素限制,例如网络带宽、数据库速度、计算、或者任务本身的内存需求,但使用异步任务的固有内存开销比使用线程的要小很多。

一般来讲,异步Rust代码看起来和普通的多线程代码非常相似,除了那些可能阻塞的操作,例如I/O或获取锁的处理有一点不同。特殊对待这些操作让Rust有更多的信息了解你的代码的行为,这为进一步优化提供了可能。上面代码的异步版本看起来像这样:
\begin{minted}{Rust}
    use async_std::{net, task};

    let listener = net::TcpListener::bind(address).await?;

    let mut new_connections = listener.incoming();
    while let Some(socket_result) = new_connections.next().await {
        let socket = socket_result?;
        let groups = chat_group_table.clone();
        task::spawn(async {
            log_error(serve(socket, groups).await);
        });
    }
\end{minted}

这里使用了\texttt{async\_std} crate的网络和任务模块,并在可能阻塞的调用后加上了\texttt{.await}。但整体的结构和基于线程的版本一样。

这一章的目标不止是帮你编写异步代码,还要向你展示它的工作细节,以便你可以预测它在你的应用中的表现,并了解它在哪些方面最有价值。

\begin{enumerate}
    \item 为了展示异步编程的机制,我们列出了涵盖所有核心概念的最小语言功能集:future、异步函数、\texttt{await}表达式、task、\texttt{block\_on}和\texttt{spawn\_local} executor。
    \item 然后我们会展示异步块和\texttt{spawn} executor。它们是完成真实工作的最基础的部分,但从概念上讲,它们只是我们刚才提到过的功能的变体。在这个过程中,我们会指出一些你可能会遇到的异步编程特有的问题并解释如何处理它们。
    \item 为了展示所有功能的协调工作,我们会展示一个聊天服务器和客户端的完整代码,上面的代码片段就是其中一部分。
    \item 为了演示原语future和executor如何工作,我们会展示\texttt{spawn\_blocking}和\texttt{block\_on}的简单实现。
    \item 最后,我们介绍了异步接口中经常出现的\texttt{Pin}类型,它被用来确保异步函数和块future被安全地使用。
\end{enumerate}

\section{从同步到异步}

考虑当你调用下面的(不是异步的)函数时会发生什么:
\begin{minted}{Rust}
    use std::io::prelude::*;
    use std::net;

    fn cheapo_request(host: &str, port: u16, path: &str)
                          -> std::io::Result<String>
    {
        let mut socket = net::TcpStream::connect((host, port))?;

        let request = format!("GET {} HTTP/1.1\r\nHost: {}\r\n\r\n", path, host);
        socket.write_all(request.as_bytes())?;
        socket.shutdown(net::Shutdown::Write)?;

        let mut response = String::new();
        socket.read_to_string(&mut response)?;

        Ok(response)
    }
\end{minted}

这会打开一个到web服务器的TCP连接,以过时的协议向它发送一个简单的HTTP请求,\footnote{如果你真的需要一个HTTP客户端,考虑使用一些非常优秀的crate例如\texttt{surf}或\texttt{reqwest},它们会正确并且异步地完成任务。这个客户端基本只是设法获得HTTPS重定向。}然后读取响应。\autoref{f20-1}展示了这个函数的执行过程随时间的变化。

\begin{figure}[htbp]
    \centering
    \includegraphics[width=0.8\textwidth]{../img/f20-1.png}
    \caption{一个同步HTTP请求的过程(深颜色的区域标识等待操作系统)}
    \label{f20-1}
\end{figure}

图中展示了从左到右随着时间的推移,函数的调用栈的变化。每一个函数调用都是一个方块,位于它的调用者上面。显然,\texttt{cheapo\_request}函数贯穿整个执行过程。它调用了Rust标准库里的函数例如\texttt{TcpStream::connect}和\texttt{TcpStream}的\texttt{write\_all}和\texttt{read\_to\_string}实现。这些对其他函数的调用依次进行,但最终程序会进行\emph{系统调用},请求操作系统完成真正的工作,例如打开TCP连接,或者读写一些数据。

深灰色的区域表示程序正在等待操作系统完成系统调用。我们并没有按比例绘制这些时间。因为加入我们按比例绘制,整个图都将是深灰色:在实践中,这个函数把几乎所有时间都用在等待操作系统上。上面代码的执行时间将是系统调用之间的窄条。

当函数等待系统调用返回时,它所在的线程会阻塞住:它不能做任何事,直到系统调用结束。一个线程的栈达到几百或几千字节并不罕见,因此如果这是一个更大的系统的一部分,并且有很多线程做类似的任务,锁住这些线程的资源但除了等待什么也不做的代价是非常昂贵的。

为了解决这个问题,一个线程需要能在等待系统调用完成的同时去执行其他的任务。但如何实现这一点并不明显。例如,我们用来从套接字读取响应的函数的签名是:
\begin{minted}{Rust}
    fn read_to_string(&mut self, buf: &mut String) -> std::io::Result<usize>;
\end{minted}

它的类型表明了:这个函数直到工作完成或者出错时才会返回。这个函数是\emph{同步的}:当操作完成时调用者才恢复。如果我们想在操作系统进行工作的同时用我们的线程去做别的任务,那么我们需要一个新的提供这个函数的\emph{异步}版本的I/O库。

\subsection{\texttt{Future}}
Rust支持异步操作的方法是引入一个trait:\texttt{std::future::Future}:
\begin{minted}{Rust}
    trait Future {
        type Output;
        // 现在,把`Pin<&mut Self>`看作`&mut Self`就好了。
        fn poll(self: Pin<&mut Self>, cx: &mut Context<'_>) -> Poll<Self::Output>;
    }

    enum Poll<T> {
        Ready(T),
        Pending,
    }
\end{minted}

一个\texttt{Future}代表一个可以测试是否完成的操作。一个future的\texttt{poll}方法从来不会等待操作完成:它总是立即返回。如果操作完成了,\texttt{poll}会返回\texttt{Poll::Ready(output)},其中\texttt{output}是最后的结果。否则,它会返回\texttt{Pending}。当且仅当future值得再次poll时,它会通过调用一个\emph{waker}来通知我们,这是一个由\texttt{Context}提供的回调函数。我们称之为异步编程的“piñata 模型” :你唯一能对future做的就是使用\texttt{poll}敲打它,直到有一个值掉出来。

所有现代的操作系统都包含一些系统调用的变体,我们可以用它们来实现这种poll接口。例如在Unix和Windows上,如果你把网络套接字设置为非阻塞模式,那么如果它们正在阻塞时进行read和write会返回错误,你必须稍后再试。

因此\texttt{read\_to\_string}的一个异步版本的签名大概是这样:
\begin{minted}{Rust}
    fn read_to_string(&mut self, buf: &mut String)
        -> impl Future<Output = Result<usize>;
\end{minted}

除了返回类型之外,这和我们之前展示的签名一样:异步的版本返回\emph{一个\texttt{Result<usize>}的future}。你需要poll这个future,直到从它得到一个\texttt{Ready(result)}。每次它被poll时,都会尽可能地继续读取。最后的\texttt{result}给你成功的值或者错误的值,就像普通的I/O操作一样。这是通常的模式:异步版本的任何函数和同步版本的函数获取相同的参数,但返回类型有一个\texttt{Future}包装。

调用这个版本的\texttt{read\_to\_string}并不会真的读取任何内容;它所有的任务就是构造并返回一个future,这个future会在被poll时进行真正的工作。这个future必须包含处理请求所需的所有信息。例如,这个\texttt{read\_to\_string}返回的future必须记住调用它的输入流,和它需要写入数据的\texttt{String}。事实上,因为这个future持有了\texttt{self}和\texttt{buf}的引用,因此这个\texttt{read\_to\_string}的真正的签名必须是:
\begin{minted}{Rust}
    fn read_to_string<'a>(&'a mut self, buf: &'a mut String)
        -> impl Future<Output = Result<usize>> + 'a;
\end{minted}

这个附加的生命周期指示了返回的future和它借用的\texttt{self}和\texttt{buf}的生命周期一样长。

\texttt{async-std} crate提供了\texttt{std}的所有I/O设施的异步版本,包括一个有\texttt{read\_to\_string}方法的异步\texttt{Read} trait。\texttt{async-std}密切地遵循了\texttt{std}的设计,尽可能地在自己的接口中重用\texttt{std}的类型,因此这两个世界中的错误、结果、网络地址、和其他大多数相关的数据都是兼容的。熟悉\texttt{std}有助于使用\texttt{async-std},反之亦然。

\texttt{Future} trait的一个规则是,一旦一个future返回了\texttt{Poll::Ready},它会假设它决不会再次被poll。一些future在自己被overpoll时简单地永远返回\texttt{Poll::Pending};其它的可能panic或者挂起。(它们绝不能违反内存或线程安全性,或者导致未定义行为)\texttt{Future} trait的\texttt{fuse}适配器将任何future转换成overpoll时永远返回\texttt{Poll::Pending}。但通常消费future的方法都遵循这个规则,因此\texttt{fuse}通常不是必须的。

如果poll听起来效率底下,请不必担心。Rust的异步架构是精心设计的,所以只要你的基本I/O函数例如\texttt{read\_to\_string}是正确实现的,那么你只会在值的时poll一个future。每一次\texttt{poll}被调用时,某个东西应该返回\texttt{Ready},或者至少向目标前进一步。我们将在“\nameref{WhenPoll}”中解释这是如何工作的。

但使用future看起来有一个挑战:当你poll时,如果你得到了\texttt{Poll::Pending}那你应该怎么做?你将不得不四处寻找这个线程暂时可以做的其他工作,并记住一段时间之后返回到这个future,然后再次poll。你的整个系统将因为持续追踪谁正在pending和当它们完成时应该做什么而变得杂乱无章。我们的\texttt{cheapo\_request}函数的简洁性会被破坏。

好消息是:它并不是这样的!

\subsection{\texttt{async}函数和\texttt{await}表达式}
这里有一个\emph{异步函数}版本的\texttt{cheapo\_request}:
\begin{minted}{Rust}
    use async_std::io::prelude::*;
    use async_std::net;

    async fn cheapo_request(host: &str, port: u16, path: &str)
                                -> std::io::Result<String>
    {
        let mut socket = net::TcpStream::connect((host, port)).await?;

        let request = format!("GET {} HTTP/1.1\r\nHost: {}\r\n\r\n", path, host);
        socket.write_all(request.as_bytes()).await?;
        socket.shutdown(net::Shutdown::Write)?;

        let mut response = String::new();
        socket.read_to_string(&mut response).await?;

        Ok(response)
    }
\end{minted}

这和之前的版本基本相同,除了:
\begin{enumerate}
    \item 函数以\texttt{async}代替\texttt{fn}开头。
    \item 它使用了\texttt{async\_std} crate里的异步版本的\texttt{TcpStream::connect, write\_all, read\_to\_string}。它们都返回结果的future(本节中的示例使用了\texttt{async\_std}的1.7版本)。
    \item 每一次调用返回future的函数之后,代码都会加上\texttt{.await}。尽管这看起来像是访问一个结构体的\texttt{await}字段,但它实际上是语言内置的一个特殊语法,它会等待一个future直到它准备好。\texttt{await}表达式会求出future的最终值。这个函数正是通过它获取\texttt{connect, write\_all, read\_to\_string}的结果。
\end{enumerate}

和普通的函数不同,当你调用异步函数时,它会在执行实际的主体代码之前立即返回。显然,调用的返回值还没有被计算出来;你得到的是它的最终值的\emph{future}。因此如果你执行这行代码:
\begin{minted}{Rust}
    let response = cheapo_request(host, port, path);
\end{minted}

那么\texttt{response}将是一个\texttt{std::io::Result<String>}的future,\texttt{cheapo\_request}的函数体还没有开始执行。你不需要调整异步函数的返回类型;Rust会自动把\texttt{async fn f(...) -> T}看做一个返回\texttt{T}的future而不是直接返回\texttt{T}的函数。

一个异步函数返回的future包含了函数体运行时所需的所有信息:函数的参数、局部变量所需的空间,等等。(就好像你把调用栈捕获为了一个普通的Rust值。)因此\texttt{response}必须包含传入的\texttt{host, port, path},因为\texttt{cheapo\_request}的函数体需要它们才能运行。

future的具体类型由编译器根据函数体和参数自动生成。这个类型并没有名称;你只知道它实现了\texttt{Future<Output=R>},其中\texttt{R}是异步函数的返回类型。从这一点来看,异步函数的future类似于闭包:闭包也有匿名类型、也是由编译器生成并且实现了\texttt{FnOnce}、\texttt{Fn}和\texttt{FnMut} trait。

当你第一次poll \texttt{cheapo\_request}返回的future时,将会从函数体的尾部开始运行到第一个由\texttt{TcpStream::connect}返回的future的\texttt{await}。这个\texttt{await}表达式会poll \texttt{connect} future,如果它还没准备好,那么它会向调用者返回\texttt{Poll::Pending}:直到\texttt{TcpStream::connect}的future返回\texttt{Poll::Ready}时,对\texttt{cheapo\_request}的future的poll才能继续通过第一个\texttt{await}。因此表达式\texttt{TcpStream::connect(...).await}的一个大概等价的写法是:
\begin{minted}{Rust}
    {
        // 注意:这是伪代码,不是有效的Rust代码
        let connect_future = TcpStream::connect(...);
        'retry_point:
        match connect_future.poll(cx) {
            Poll::Ready(value) => value,
            Poll::Pending => {
                // 设置`cheapo_request`的future的下一次`poll`
                // 从'retry_point处恢复执行。
                ...
                return Poll::Pending
            }
        }
    }
\end{minted}

一个\texttt{await}表达式会获取future的所有权然后poll它。如果它已经准备好,那么future的最终值就是\texttt{await}表达式的值,并且会继续往下执行。否则,它向调用者返回\texttt{Poll::Pending}。

但关键的是,下一次poll \texttt{cheapo\_request}的future时将不会再次从函数的首部开始:相反,它从poll \texttt{connect\_future}的地方开始\emph{恢复(resume)}执行。直到future准备好之后我们才会继续执行这个异步函数的其他部分。

随着\texttt{cheapo\_request}的future继续被poll,它会从函数体里的一个\texttt{await}开始执行到下一个,只有当它正在等待的子future ready时才会继续。因此,\texttt{cheapo\_request}的future将会被poll多少次取决于子future的行为和函数本身的控制流。\texttt{cheapo\_request}的future会追踪下一次\texttt{poll}时的恢复点和所有的局部状态——变量、参数、临时值——恢复需要这些。

在函数中间挂起并稍后恢复执行的能力是异步函数独有的。当普通函数返回时,它的栈帧就消失了。因为\texttt{await}表达式依赖于恢复执行的能力,所以你只能在异步函数里使用它们。

在撰写本书时,Rust还不允许trait有异步方法。只有自由函数和特定类型固有的方法才可以是异步的。取消这个限制需要对语言进行一些修改。与此同时,如果你需要定义包含异步函数的trait,可以考虑使用\texttt{async-trait} crate,它提供了一个基于宏的解决方案。

\subsection{在同步代码中调用异步函数:\texttt{block\_on}}
某种意义上讲,异步函数只是在推卸责任。在异步函数里很容易获取一个future的值:只需要\texttt{await}它。但异步函数\emph{本身}也返回一个future,因此现在调用者需要负责poll它。最后,总有一个地方必须实际等待一个值。

我们可以使用\texttt{async\_std}的\texttt{task::block\_on}函数在普通的同步函数(例如\texttt{main})中调用\texttt{cheapo\_request},它获取一个future并且poll它直到它产生一个值:
\begin{minted}{Rust}
    fn main() -> std::io::Result<()> {
        use async_std::task;

        let response = task::block_on(cheapo_request("example.com", 80, "/))?;
        println!("{}", response);
        Ok(())
    }
\end{minted}

因为\texttt{block\_on}是一个产生异步函数的最终值的同步函数,你可以将它看作是异步世界到同步世界的适配器。但它阻塞的特性也意味着你永远不应该在一个异步函数里使用\texttt{block\_on}:它会阻塞整个线程直到值准备好。作为代替,请使用\texttt{await}。

\autoref{f20-2}展示了\texttt{main}的一个可能的执行过程。

\begin{figure}[htbp]
    \centering
    \includegraphics[width=0.9\textwidth]{../img/f20-2.png}
    \caption{阻塞等待一个异步函数}
    \label{f20-2}
\end{figure}

上面的时间线,“简化视图”,展示了程序的异步调用的抽象视图:\texttt{cheapo\_request}首先调用了\texttt{TcpStream::connect}来获取一个套接字,然后对套接字调用了\texttt{write\_all}和\texttt{read\_to\_string}。然后它返回。这和本章前面的同步版本的\texttt{cheapo\_request}的时间线非常相似。

但这里每一个异步调用都是多阶段的过程:一个future被创建,然后被poll直到它准备好,可能还会创建并poll其他子future。下面的时间线,“实现”,展示了实现了这个异步行为的实际同步调用。这是一个介绍普通的异步执行过程中到底发生了什么的好机会:
\begin{enumerate}
    \item 首先,\texttt{main}调用\texttt{cheapo\_request},它返回最终结果的future \texttt{A}。然后\texttt{main}把这个future传给了\texttt{async\_std::block\_on},它会poll \texttt{A}。
    \item poll future \texttt{A}允许\texttt{cheapo\_request}的函数体开始执行。函数里调用了\texttt{TcpStream::connect}来获取一个套接字的future \texttt{B}并await它。更确切地说,因为\texttt{TcpStream::connect}可能会遇到错误,因此\texttt{B}是一个\texttt{Result<TcpStream, std::io::Error}的future。
    \item future \texttt{B}被\texttt{await} poll。因为网络连接还没有建立好,所以\texttt{B.poll}返回\texttt{Poll::Pending},但会设置好当套接字准备好后唤醒调用它的任务。
    \item 因为future \texttt{B}还没有准备好,\texttt{A.poll}也会向它的调用者\texttt{block\_on}返回\texttt{Poll::Pending}。
    \item 因为\texttt{block\_on}没有别的事情可做,它会陷入睡眠。这时整个线程会阻塞。
    \item 当\texttt{B}的连接准备好之后,它会唤醒poll它的任务。这促使\texttt{block\_on}开始行动,它会尝试再次poll future \texttt{A}。
    \item poll \texttt{A}促使\texttt{cheapo\_request}在它的第一个\texttt{await}处恢复执行,然后再次poll \texttt{B}。
    \item 这一次,\texttt{B}准备好了:套接字已经创建完毕,因此它返回\texttt{Poll::Ready(Ok(socket))}到\texttt{A.poll}。
    \item 到此\texttt{TcpStream::connect}的异步调用就完成了。\texttt{TcpStream::connect(...).await}表达式的值就是\texttt{Ok(socket)}。
    \item \texttt{cheapo\_request}的函数体会继续正常执行,使用\texttt{format!}宏构造请求字符串并传递给\texttt{socket.write\_all}。
    \item 因为\texttt{socket.write\_all}是一个异步函数,它会返回一个future \texttt{C},\texttt{cheapo\_request}会await \texttt{C}。
\end{enumerate}

剩余的流程和之前相似。在\autoref{f20-2}所示的执行流程中,\texttt{socket.read\_to\_string}在准备好之前被poll了四次,每一次都会从套接字读取\emph{一些}数据,但\texttt{read\_to\_string}被指定为一直读取到输入的末尾,这需要好几次的操作。

听起来编写一个一直调用\texttt{poll}的循环并不难。但让\texttt{async\_std::task::block\_on}真正有价值的是:它知道怎么睡眠到恰好future值得再次poll,而不是浪费处理器的时间和电量来进行几十亿次无用的\texttt{poll}调用。基本的I/O函数例如\texttt{connect}和\texttt{read\_to\_string}返回的future保留了传递给\texttt{poll}的\texttt{Context}参数提供的唤醒器,并在\texttt{block\_on}应该醒来并再次尝试poll时调用唤醒器。我们将在“\nameref{WhenPoll}”中通过实现一个简单版本的\texttt{block\_on}来展示这具体是怎么工作的。

和我们之前展示的原始的同步版本一样,这个异步版本的\texttt{cheapo\_request}方法也把几乎所有的时间花费在等待操作完成上。如果时间轴是按比例绘制的,那么图将几乎完全是深灰色的,只有当程序被唤醒时会有几个计算过程对应的很细的条。

这里讲了很多细节。幸运的是,你通常可以只考虑简化的上层时间线:一些函数调用是同步的,其他是异步的并需要一个\texttt{await},但它们都只是函数调用。Rust的异步支持的成功取决于帮助程序员在实践中只需要考虑简化的视图,不会被实现的来回跳转干扰。

\subsection{spawn异步任务}
\texttt{async\_std::task::block\_on}函数会阻塞知道一个future的值准备好。但在单个future上完全阻塞一个线程并不比同步调用更好:本章的目的是让线程在等待的同时\emph{做别的工作}。

为了实现这一点,你可以使用\texttt{async\_std::task::spawn\_local}。这个函数接受一个future并把它添加到一个池,当\texttt{block\_on}阻塞等待的future还没准备好时\texttt{block\_on}会poll这个池。因此如果你把一堆future传递给\texttt{spawn\_local}并且之后对最终结果的future调用\texttt{block\_on},\texttt{block\_on}会poll每一个被spawn的future(当它们可以进一步执行时),并发运行整个池,直到结果准备好。

在撰写本书时,只有当你启用\texttt{async-std} crate的\texttt{unstable}特性时\texttt{spawn\_local}才可用。你需要在\emph{Cargo.toml}中用这样的一行引入\texttt{async-std}:
\begin{minted}{toml}
    async-std = { version = "1", features = ["unstable"] }
\end{minted}

\texttt{spawn\_local}函数是标准中用于启动新线程的\texttt{std::thread::spawn}函数的异步版本的类似物:
\begin{enumerate}
    \item \texttt{std::thread::spawn(c)}接收闭包\texttt{c}然后启动一个线程运行它,返回一个\texttt{std::thread::JoinHandle},它的\texttt{join}方法会等待线程结束并返回\texttt{c}返回的内容。
    \item \texttt{async\_std::task::spawn\_local(f)}接收future \texttt{f}并把它添加到当前线程调用\texttt{block\_on}时会poll的池里。\texttt{spawn\_local}会返回它自己的\texttt{async\_std::task::JoinHandle}类型,它本身是一个future,你可以await它来获取\texttt{f}的最终值。
\end{enumerate}

例如,假设我们想让一个HTTP请求的集合并发执行。这是第一次尝试:
\begin{minted}{Rust}
    pub async fn many_requests(requests: Vec<(String, u16, String)>)
                               -> Vec<std::io::Result<String>>
    {
        use async_std::task;

        let mut handles = vec![];
        for (host, port, path) in requests {
            handles.push(task::spawn_local(cheapo_request(&host, port, &path)));
        }

        let mut results = vec![];
        for handle in handles {
            results.push(handle.await);
        }

        results
    }
\end{minted}

这个函数对\texttt{requests}的每个元素调用\texttt{cheapo\_request},将每一个调用返回的future传给\texttt{spawn\_local}。它把最后的\texttt{JoinHandle}收集到一个vector并且await每一个。以任意顺序await join handles都是没问题的:因为请求已经被spawn,它们的future将会被按需poll,即这个线程调用了\texttt{block\_on}并且无事可做时。所有的请求会并发运行。一旦它们完成,\texttt{many\_requests}会向调用者返回结果。

上面的代码几乎是正确的,但Rust的借用检查器担心\texttt{cheapo\_request}的future的生命周期:
\begin{minted}{text}
    error: `host` does not live long enough
        handles.push(task::spawn_local(cheapo_request(&host, port, &path)));
                                       ---------------^^^^^-------------
                                       |              |
                                       |              borrowed value does not
                                       |              live long enough
                         argument requires that `host` is borrowed for `'static`
    }
    - `host` dropped here while still borrowed
\end{minted}

\texttt{path}也有一个类似的错误。

自然地,如果我们向异步函数传递引用,那么它们返回的future就必须持有这些引用,因此出于安全性future不能比它们借用的值生存的更久。任何持有引用的其他值也有相同的限制。

问题在于\texttt{spawn\_local}不能确保你会在\texttt{host}和\texttt{path}被drop之前等待任务结束。事实上,\texttt{spawn\_local}只接受生命周期是\texttt{'static}的future,因为你可以简单地忽略它返回的\texttt{JoinHandle}并让任务在程序的剩余部分执行时继续运行。这并不是异步任务独有的问题:当你尝试用\texttt{std::thread::spawn}启动一个线程,并且它的闭包捕获了局部变量的引用时也会遇到类似的错误。

一种解决这个问题的方法是创建另一个版本的获取参数所有权的异步函数:
\begin{minted}{Rust}
    async fn cheapo_owning_request(host: String, port: u16, path: String)
                                   -> std::io::Result<String> {
        cheapo_request(&host, port, &path).await
    }
\end{minted}

这个函数接收\texttt{String}而不是\texttt{\&str}引用,因此它的future自身将拥有\texttt{host}和\texttt{path},并且生命周期是\texttt{'static}。借用检查器可以看到它立刻await了\texttt{cheapo\_request}的future,并且因此如果这个future被poll,它借用的\texttt{host}和\texttt{path}变量肯定还在。一切都没有问题。

使用\texttt{cheapo\_owning\_request},你可以像这样spwan所有的请求:
\begin{minted}{Rust}
    for (host, port, path) in requests {
        handles.push(task::spawn_local(cheapo_owning_request(host, port, path)));
    }
\end{minted}

你可以使用\texttt{block\_on}在同步的\texttt{main}函数中调用\texttt{many\_requests}:
\begin{minted}{Rust}
    let requests = vec![
        ("example.com".to_string(),      80, "/".to_string()),
        ("www.red-bean.com".to_string(), 80, "/".to_string()),
        ("en.wikipedia.org".to_string(), 80, "/".to_string()),
    ];

    let results = async_std::task::block_on(many_requests(requests));
    for result in results {
        match result {
            Ok(response) => println!("{}", response),
            Err(err) => eprintln!("error: {}", err),
        }
    }
\end{minted}

这段代码会在\texttt{block\_on}的调用中并发运行三个请求。每个请求会在当其他的请求阻塞时抓住几乎继续执行,它们全部在调用者线程中执行。\autoref{f20-3}中展示了三个\texttt{cheapo\_request}调用的可能的执行过程。

\begin{figure}[htbp]
    \centering
    \includegraphics[width=0.9\textwidth]{../img/f20-3.png}
    \caption{在单个线程中运行三个异步任务}
    \label{f20-3}
\end{figure}

(我们鼓励你自己尝试运行这段代码,使用\texttt{eprintln!}在\texttt{cheapo\_request}首部和每一个\texttt{await}表达式之后打印消息,这样你可以看到这些调用如何交错执行。)

对\texttt{many\_requests}的调用(为了简单没有展示)spawn了三个异步的任务,分别用\texttt{A}、\texttt{B}、\texttt{C}标记。\texttt{block\_on}开始时先poll \texttt{A},\texttt{A}会开始连接到\texttt{example.com}。这会立刻返回\texttt{Poll::Pending},\texttt{block\_on}会把注意移动到下一个spawn的任务,然后poll future \texttt{B},最后是\texttt{C},它们会开始连接各自的服务器。

当所有可以poll的future都返回了\texttt{Poll::Pending}之后,\texttt{block\_on}会进入睡眠,直到其中一个\texttt{TcpStream::connect} future指示它的任务值得再次poll。

在这次执行中,服务器\texttt{en.wikipedia.org}比其他的响应得更快,因此这个任务最先完成。当一个spawn的任务完成时,它会把值保存在\texttt{JoinHandle}并标记它已经准备好了,这样\texttt{many\_requests} await它时无需等待,可以继续执行。最后,其它的\texttt{cheapo\_request}要么成功要么返回错误,然后\texttt{many\_request}本身可以返回了。最后,\texttt{main}接收\texttt{block\_on}返回的结果的vector。

所有这些执行都发生在单个线程中,三个\texttt{cheapo\_request}的调用通过对future的poll实现交错执行。一个异步调用看起来像是一个运行到完成的单个函数调用,但实际上异步调用由一系列对future的\texttt{poll}方法的同步调用实现。每一个单独的\texttt{poll}调用都可以快速返回,让出线程从而让其他异步调用可以执行。

我们终于达成了我们在本章开头设置的目标:让一个线程在等待I/O完成的同时去执行其他的工作,这样线程的资源不会因等待而浪费。更妙的是,达成这个目标的代码看起来非常像普通的Rust代码:一些函数被标记为\texttt{async}、一些函数调用后面有\texttt{.await}、使用的函数来自\texttt{async\_std}而不是\texttt{std},但除此之外,它就是普通的Rust代码。

异步任务和线程有一个不同之处需要牢记:异步任务只有在\texttt{await}表达式处被await的future返回\texttt{Poll::Pending}时才会切换到另一个异步任务。这意味着如果你在\texttt{cheapo\_request}中放了一段长时间运行的计算代码,那么在它完成之前,任何传给\texttt{spawn\_local}的其他任务都没有机会运行。而使用线程时没有这个问题:操作系统可以在任何地方挂起任何线程并设置计时器来确保没有线程可以垄断处理器。

异步代码依赖于共享线程的future的协作。如果你需要让长时间计算和异步代码共存,本章后面的“\nameref{LongCompute}”中介绍了一些方法。

\subsection{\texttt{async}块}
除了异步函数之外,Rust还支持\emph{异步块(asynchronous block)}。与一个普通的快块返回最后一个表达式的值不同,一个异步块返回最后一个表达式的\emph{值的future}。你可以在异步块里使用\texttt{await}表达式。

异步块看起来就像普通的块表达式,在前边加上\texttt{async}关键字:
\begin{minted}{Rust}
    let serve_one = async {
        use async_std::net;

        // 监听连接并接受
        let listener = net::TcpListener::bind("localhost:8087").await?;
        let (mut socket, _add) = listener.accept().await?;

        // 通过`socket`与客户端交互
        ...
    };
\end{minted}

这里用一个future初始化了\texttt{serve\_one},当poll它时,它会监听并处理单个TCP连接。块的代码直到\texttt{serve\_one}被poll才会执行,就像异步函数只有在它的future被poll时才会执行一样。

如果你在异步块里使用了\texttt{?}操作符,它会从块里返回,而不是从所处的函数返回。例如,如果上面的\texttt{bind}调用返回一个错误,那么\texttt{?}操作符会返回它作为\texttt{serve\_one}的最终值。类似的,\texttt{return}表达式会从异步块里返回,而不是从外层的函数返回。

如果一个异步块引用了周围代码里的变量,它的future会捕获那些变量,就像闭包一样。并且和\texttt{move}闭包一样(见“\nameref{StealClosure}”),你可以以\texttt{async move}来创建获取变量所有权的块,而不是持有变量的引用。

异步块提供了一种精确的分离出想要异步运行的部分代码的方法。例如,在上一节中,\texttt{spawn\_local}需要\texttt{'static} future,因此我们定义了\texttt{cheapo\_owning\_request}包装函数来得到一个获取参数所有权的future。你可以简单地在一个异步块里调用\texttt{cheapo\_request}而不需要分离出包装函数来实现相同的效果:
\begin{minted}{Rust}
    pub async fn many_requests(requests: Vec<(String, u16, String)>)
                               -> Vec<std::io::Result<String>>
    {
        use async_std::task;

        let mut handles = vec![];
        for (host, port, path) in requests {
            handles.push(task::spawn_local(async move {
                cheapo_request(&host, port, &path).await
            }));
        }
        ...
    }
\end{minted}

因为这是一个\texttt{async move}块,所以它的future获取了\texttt{String}值\texttt{host}和\texttt{path}的所有权,就类似\texttt{move}闭包一样。然后它向\texttt{cheapo\_request}传递引用,借用检查器可以看到块的\texttt{await}表达式获取了\texttt{cheapo\_request}的future的所有权,因此\texttt{host}和\texttt{path}的引用不可能比它们借用的被捕获的变量生存的更久。异步块和\texttt{cheapo\_owning\_request}完成了同样的事,但所需的样板代码更少。

一个你可能遇到的问题是没有语法能指定异步块的返回类型,即异步函数参数后跟的\texttt{-> T}。当使用\texttt{?}操作符时这可能会导致问题:
\begin{minted}{Rust}
    let input = async_std::io::stdin();
    let future = async {
        let mut line = String::new();

        // 这会返回`std::io::Result<usize>`。
        input.read_line(&mut line).await?;

        println!("Read line: {}", line);

        Ok(())
    };
\end{minted}

这会因为如下错误失败:
\begin{minted}{text}
    error: type annotations needed
       |
    42 |     let future = async {
       |         ------ consider giving `future` a type
    ...
    46 |         input.read_line(&mut line).await?;
       |         ^^^^^^^^^^^^^^^^^^^^^^^^^^^^^^^^^ cannot infer type
\end{minted}

Rust不能分辨异步块的返回类型应该是什么。\texttt{read\_line}方法返回\texttt{Result<(), std::io::Error>},但因为\texttt{?}操作符使用了\texttt{From} trait来在需要时转换成指定的错误类型,所以这个异步块的返回类型是\texttt{Result<(), E>},其中\texttt{E}可能是任何实现了\texttt{From<std::io::Error>}的类型。

Rust未来的版本可能会添加指定\texttt{async}块的返回类型的语法。但现在,可以通过手动写出最后的\texttt{Ok}的类型来解决这个问题:
\begin{minted}{Rust}
    let future = async {
        ...
        Ok::<(), std::io::Error>(())
    };
\end{minted}

因为\texttt{Result}是一个需要成功和错误类型作为参数的泛型类型,我们可以像这里一样使用\texttt{Ok}或\texttt{Err}指定那些类型参数。

\subsection{从异步块中构建异步函数}
异步块给了我们另一种实现和异步函数相同效果的方法,并且更加灵活一点。例如,我们可以将我们的\texttt{cheapo\_request}写成一个普通的、同步的返回异步块的future的函数:
\begin{minted}{Rust}
    use std::io;
    use std::future::Future;

    fn cheapo_request<'a>(host: &'a str, port: u16, path: &'a str)
        -> impl Future<Output = io::Result<String>> + 'a
    {
        async move {
            ... function body ...
        }
    }
\end{minted}

当你调用这个版本的函数时,它会立刻返回异步块的值的future。这个future会捕获函数的参数并且和异步函数返回的future的行为一样。因为我们没有使用\texttt{async fn}语法,我们需要在返回值中写出\texttt{impl Future},但对调用者来说,这两个定义是同一个函数签名的两种可替换的实现。

如果你想让函数被调用时立刻进行一些计算然后再构造返回的future,那么第二种方法更有用。例如,另一种协调\texttt{cheapo\_request}和\texttt{spawn\_local}的方法是让它变成一个返回\texttt{'static} future的同步函数,并让这个future捕获参数的拷贝的所有权:
\begin{minted}{Rust}
    fn cheapo_request(host: &str, port: u16, path: &str)    
        -> impl Future<Output = io::Result<String> + 'static
    {
        let host = host.to_string();
        let path = path.to_string();

        async move {
            ... use &*host, port, and path ...
        }
    }
\end{minted}

这个版本让异步块捕获\texttt{host}和\texttt{path}为\texttt{String}值,而不是\texttt{\&str}引用。因为future拥有自己运行所需的所有数据,所以它是有效的\texttt{'static}生命周期。(我们在上面的签名中写出了\texttt{+ 'static},但\texttt{-> impl}返回的类型默认是\texttt{'static}的,因此省略它不会有影响。)

因为这个版本的\texttt{cheapo\_request}返回的future是\texttt{'static}的,我们可以直接把它们传递给\texttt{spawn\_local}:
\begin{minted}{Rust}
    let join_handle = async_std::task::spawn_local(
        cheapo_request("areweasyncyet.rs", 80, "/")
    );

    ... other work ...

    let response = join_handle.await?;
\end{minted}

\subsection{在一个线程池中spawn异步任务}
我们至今为止展示过的例子几乎把所有时间花费在等待I/O上,但一些负载是更多处理器工作和阻塞的组合。当你有太多的计算以至于单个处理器不能进行快速处理,你可以使用\texttt{async\_std::task::spawn}来把一个future spawn到一个工作线程池里,这些线程会poll可以进一步执行的future。

\texttt{async\_std::task::spawn}的使用方法类似\texttt{async\_std::task::spawn\_local}:
\begin{minted}{Rust}
    use async_std::task;

    let mut handles = vec![];
    for (host, port, path) in requests {
        handles.push(task::spawn(async move {
            cheapo_request(&host, port, &path).await
        }));
    }
    ...
\end{minted}

类似于\texttt{spawn\_local},\texttt{spawn}也返回一个\texttt{JoinHandle}值,你可以await它来获取future的最终值。但和\texttt{spawn\_local}不同的是,这个future不会等到你调用\texttt{block\_on}才会被poll,只要线程池中有一个空闲的线程,它就会尝试poll这个future。

在实践中,\texttt{spawn}比\texttt{spawn\_local}使用得更加广泛,因为人们更希望他们的负载不管计算和阻塞怎么混合,都能在机器上均衡地执行。

当使用\texttt{spawn}时一个需要记住的点是线程池会尝试保持忙碌,因此只要有一个线程空闲你的future就会被poll。一个异步调用可能在一个线程中开始执行,在一个\texttt{await}表达式处阻塞,最后在另一个不同的线程中恢复执行。因此将一个异步函数调用看作单个函数调用是一个合理的简化(事实上,异步函数和\texttt{await}表达式的目的就是鼓励你以这种方式思考)。和代码的执行情况有关,异步调用可能实际上会在很多不同线程中移动。

如果你正在使用thread-local存储,你可能会惊讶地发现你在\texttt{await}表达式之前放置的一些数据在恢复之后被替换成了某些完全不同的东西,这是因为你的任务现在正在被池中的另一个线程poll。如果这导致了问题,你应该使用\emph{task-local storage};细节见\texttt{async-std} crate中\texttt{task\_local!}宏的文档。

\subsection{但你的Future实现了\texttt{Send}吗?}
有一个\texttt{spawn}要求但\texttt{spawn\_local}不要求的限制。因为future被送到另一个线程运行,因此future必须实现了\texttt{Send}标记trait。我们在“\nameref{threadsafe}”中介绍过\texttt{Send}。只有当future包含的所有值都是\texttt{Send}时future才是\texttt{Send}:所有的函数参数、局部变量、甚至匿名的临时值都必须能安全地移动到另一个线程。

和之前一样,这个要求也不是异步任务独有的:如果你尝试使用\texttt{std::thread::spawn}启动一个捕获了非\texttt{Send}值的闭包也会遇到一个类似的错误。不同之处在于,传给\texttt{std::thread::spawn}的闭包会留在新创建的线程中运行,而spawn到线程池里的future可能会在await时从一个线程移动到另一个线程。

这个限制很容易意外触发。例如,下面的代码看起来足够合法:
\begin{minted}{Rust}
    use async_std::task;
    use std::rc::Rc;

    async fn reluctant() -> String {
        let string = Rc::new("ref-counted String".to_string());

        some_asynchronous_thing().await;

        format!("Your splendid string: {}", string)
    }

    task::spawn(reluctant());
\end{minted}

一个异步函数的future必须持有足够的信息来让它可以从一个\texttt{await}表达式继续执行。在这个例子中,\texttt{reluctant}的future必须在\texttt{await}之后使用\texttt{string},因此这个future将会,或至少有时会,包含一个\texttt{Rc<String>}值。因为\texttt{Rc}指针不能安全地在线程之间共享,所以这个future本身不能是\texttt{Send}。并且因为\texttt{spawn}只接受\texttt{Send}的future,所以Rust会报错:
\begin{minted}{text}
    error: future cannot be sent between threads safely
        |
    17  |     task::spawn(reluctant());
        |     ^^^^^^^^^^^ future returned by `reluctant` is not `Send`
        |

        |
    127 | T: Future + Send + 'static,
        |             ---- required by this bound in `async_std::task::spawn`
        |
        = help: within `impl Future`, the trait `Send` is not implemented
                for `Rc<String>`
    note: future is not `Send` as this value is used across an await
    10  |         let string = Rc::new("ref-counted string".to_string());
        |             ------ has type `Rc<String>` which is not `Send`
    11  |
    12  |         some_asynchronous_thing().await;
        |         ^^^^^^^^^^^^^^^^^^^^^^^^^^^^^^^
                      await occurs here, with `string` maybe used later
    ...
    15  |    }
        |    - `string` is later dropped here
\end{minted}

这一段错误信息很长,但包含很多有用的细节:
\begin{enumerate}
    \item 它解释了为什么future需要是\texttt{Send}:\texttt{task::spawn}的要求。
    \item 它解释了什么样的值不是\texttt{Send},局部变量\texttt{string},它的类型是\texttt{Rc<String>}。
    \item 它解释了为什么\texttt{string}会影响future:它的作用域跨过了\texttt{await}。
\end{enumerate}

有两种解决这个问题的方法。一个是限制非\texttt{Send}的值的作用域,让它不包含任何\texttt{await}表达式,因此就不需要保存在函数的future里:
\begin{minted}{Rust}
    async fn reluctant() -> String {
        let return_value = {
            let string = Rc::new("ref-counted string".to_string());
            format!("Your splendid string: {}", string)
            // `Rc<String>`在这里离开作用域...
        };

        // ...因此当我们在这里挂起时不需要保存它。
        some_asynchronous_thing().await;

        return_value
    }
\end{minted}

另一种解决方案是简单地用\texttt{std::sync::Arc}替换\texttt{Rc}。\texttt{Arc}使用原子更新来管理它的引用计数,这意味着它会稍微慢一点,不过\texttt{Arc}指针是\texttt{Send}。

尽管最终你会学会识别和避免非\texttt{Send}类型,但一开始它们可能令人惊讶。(至少,你的作者通常会很惊讶。)例如,较旧的Rust代码有时会像这样使用泛型结果类型:
\begin{minted}{Rust}
    // 不推荐!
    type GenericError = Box<dyn std::error::Error>;
    type GenericResult<T> = Result<T, GenericError>;
\end{minted}

这个\texttt{GenericError}类型使用了一个trait对象来存储任何实现了\texttt{std::error::Error}的类型。但并没有给它施加更严格的限制:如果有一个非\texttt{Send}类型实现了\texttt{Error},它们将能转换成一个\texttt{GenericError}类型。因为这种可能性,\texttt{GenericError}将不是\texttt{Send},下面的代码将不能工作:
\begin{minted}{Rust}
    fn some_fallible_thing() -> GenericResult<i32> {
        ...
    }

    // 这个函数的future不是`Send`...
    async fn unfortunate() {
        // ...因为这个调用返回的值...
        match some_fallible_thing() {
            Err(error) => {
                report_error(error);
            }
            Ok(output) => {
                // ...到这个await处仍然存在...
                use_output(output).await;
            }
        }
    }

    // ...因此这个`spawn`会导致错误。
    async_std::task::spawn(unfortunate());
\end{minted}

和前面的例子一样,编译器的错误消息解释了发生了什么,指出了那个\texttt{Result}是罪魁祸首。因为Rust考虑到\texttt{some\_fallible\_thing}的结果在整个\texttt{match}表达式中生效,包括\texttt{await}表达式,它决定了\texttt{unfortunate}的future不是\texttt{Send}。这个错误是因为Rust过度谨慎:尽管\texttt{GenericError}不能安全地发送到另一个线程,但\texttt{await}只会在结果是\texttt{Ok}的时候发生,因此当我们await \texttt{use\_output}的future时错误的值永远不会存在。

一个理想的解决方法是使用更加严格的泛型错误类型,例如我们在“\nameref{MultiErr}”中建议的这个:
\begin{minted}{Rust}
    type GenericError = Box<dyn std::error::Error + Send + Sync + 'static>;
    type GenericResult<T> = Result<T, GenericError>;
\end{minted}

这个trait对象显式地要求底层的错误类型要实现了\texttt{Send},这样就一切顺利了。

如果你的future不是\texttt{Send}并且不能方便地将它变成\texttt{Send},那么你可以使用\texttt{spawn\_local}来在当前线程运行它。当然,你需要保证这个线程在某个地方调用\texttt{block\_on},以给它运行的机会,并且你将不能从多处理器中收益。

\subsection{长时间计算:\texttt{yield\_now}和\texttt{spawn\_blocking}}\label{LongCompute}

对于一个和其它任务共享它的线程的future,它的\texttt{poll}方法应该总是尽可能快速地返回。但如果你在进行长时间的计算,它可能需要很长时间才会到达下一个\texttt{await},让其它的异步任务等待比你预想得更长的时间。

一种避免这种情况的方法是偶尔就\texttt{await}一次。\texttt{async\_std::task::yield\_now}函数返回一个为此设计的简单future:
\begin{minted}{Rust}
    while computation_not_done() {
        // ... 进行中等规模的计算 ...
        async_std::task::yield_now().await;
    }
\end{minted}

\texttt{yield\_now}的future第一次被poll时,它会返回\texttt{Poll::Pending},但它会很快声明它值得再次poll。效果就是你的异步调用可以放弃线程,其他的任务可以得到运行的机会,但很快又会轮到你的调用。\texttt{yield\_now}的future第二次被poll时,它会返回\texttt{Poll::Ready(())},因此你的异步函数可以恢复执行。

然而这个方法并不总是可行。如果你正在使用一个外部的crate来做长时间计算或者调用外部的C或C++代码,那么并不方便修改代码来变得更加异步友好。或者可能很难确保计算的每一条路径都会经过\texttt{await}。

对于这种情况,你可以使用\texttt{async\_std::task::spawn\_blocking}。这个函数接受一个闭包,在它自己的线程中运行它,并返回一个返回值的future。异步代码可以await这个future,把它的线程让给其他的任务,直到计算完成。通过把困难的任务放在单独的线程,可以让操作系统负责让它很好地共享处理器。

例如,假设我们需要检查用户输入的密码和我们在认证数据库中存储的哈希过的版本是否一致。为了安全性,验证密码需要是计算密集的,这样即使攻击者获取了数据库的拷贝,他们也不能简单地尝试几万亿个可能的密码来看看是否匹配。\texttt{argonautica} crate提供了一个专为存储密码设计的哈希函数:一个正确生成的\texttt{argonautica}哈希值需要几分之一秒来验证。我们可以像这样在我们的异步应用中使用\texttt{argonautica}(版本\texttt{0.2}):
\begin{minted}{Rust}
    async fn verify_password(password: &str, hash: &str, key: &str)
                            -> Result<bool, argonautica::Error>
    {
        // 获取参数的拷贝,以让闭包变为'static
        let password = password.to_string();
        let hash = hash.to_string();
        let key = key.to_string();

        async_std::task::spawn_blocking(move || {
            argonautica::Verifier::default()
                .with_hash(hash)
                .with_password(password)
                .with_secret_key(key)
                .verify()
        }).await
    }
\end{minted}

如果\texttt{password}匹配\texttt{hash}它会返回\texttt{Ok(true)},其中的\texttt{key}是数据库里的一个键。在传给\texttt{spawn\_blocking}的闭包里进行验证,可以把昂贵的计算放到它自己的线程里,确保它不会影响对其它用户的请求返回响应。

\subsection{比较异步设计}
Rust的异步编程的方案在很多方面都和其他语言采用的方案很像。例如,JavaScript、C\#和Rust都有带有\texttt{await}表达式的异步函数。所有这些语言都有值来表示还未完成的计算:Rust称之为“future”,JavaScript称之为“promise”,C\#称之为“task”,但它们都代表一个可能要等待的值。

然而Rust中poll的使用并不寻常。在JavaScript和C\#中,一个异步函数被调用后会立刻执行,有一个内置在系统库中的全局的事件循环负责当它们等待的值可用时恢复挂起的异步函数调用。然而在Rust中,异步函数调用什么都不做,直到把它的future传递给\texttt{block\_on}、\texttt{spawn}或者\texttt{spawn\_local},这些函数会poll它并驱动工作完成。这些函数,称为\emph{executor},扮演了其它语言中的全局事件循环的角色。

因为Rust允许你——程序员来选择一个executor来poll你的future,所以Rust不需要内置在系统中的全局事件循环。\texttt{async-std} crate提供了我们在本章中用过的executor函数,但我们在本章稍后会使用的\texttt{tokio} crate,定义了它自己的类似的executor函数集。并且作为本章的终结,我们会实现自己的executor。你可以在同一个程序中使用这三种executor。

\subsection{一个真实的异步HTTP客户端}
如果我们不展示一个使用合适的异步HTTP客户端crate的例子将是我们的疏忽,因为它是如此简单,并且有好几个好的crate可以选择,包括\texttt{reqwest}和\texttt{surf}。

这里有一个使用\texttt{surf}来并发运行一系列请求的重写的\texttt{many\_requests},甚至比基于\texttt{cheapo\_request}的版本还要简单,你需要在\emph{Cargo.toml}中加上这些依赖:
\begin{minted}{toml}
    [dependencies]
    async-std = "1.7"
    surf = "1.0"
\end{minted}

然后,我们可以像这样定义\texttt{many\_requests}:
\begin{minted}{Rust}
    pub async fn many_requests(urls: &[String])
                               -> Vec<Result<String, surf::Exception>>
    {
        let client = surf::Client::new();

        let mut handles = vec![];
        for url in urls {
            let request = client.get(&url).recv_string();
            handles.push(async_std::task::spawn(request));
        }

        let mut results = vec![];
        for handle in handles {
            results.push(handle.await);
        }

        results
    }

    fn main() {
        let requests = &["http://example.com".to_string(),
                         "https://www.red-bean.com".to_string(),
                         "https://en.wikipedia.org/wiki/Main_Page".to_string()];

        let results = async_std::task::block_on(many_requests(requests));
        for result in results {
            match result {
                Ok(response) => println!("*** {}\n", response),
                Err(err) => eprintln!("error: {}\n", err),
            }
        }
    }
\end{minted}

使用单个\texttt{surf::Client}来进行所有请求让我们可以在其中某些请求指向同一个服务器时重用HTTP连接。并且不需要异步块:因为\texttt{recv\_string}是一个返回\texttt{Send + 'static} future的异步方法,我们可以直接把它的future传给\texttt{spawn}。

\section{一个异步的客户端和服务器}
是时候整理一下我们至今为止讨论过的关键思路并将它们组合成一个可以工作的程序了。很大程度上来说,异步应用类似于普通的多线程应用,但有新机会写出紧凑且富有表现力的代码。

这一节的示例是一个聊天服务器和客户端。\href{https://github.com/ProgrammingRust/async-chat}{完整的代码}见这里。真实的聊天系统很复杂,从安全和重连到隐私和现代化都是需要考虑的因素,但我们将只实现一组简单的功能子集,这样能更加关注几个我们感兴趣的点。

特别地,我们想很好的处理\emph{背压(backpressure)}。意思是如果一个客户端的网络连接很慢或者完全丢失了连接,必须不影响其他客户端交换信息的能力。并且因为一个慢速的客户端不应该让服务器花费无限制的内存来保存它不断增长的累积消息,我们的服务器应该丢弃一些不能跟上速度的客户端的消息,但要通知它们它们的消息流是不完整的。(一个真实的服务器应该把消息记录到磁盘上并让客户端去获取它们错过的消息,不过我们省略了这个功能。)

我们以命令\texttt{cargo new --lib async-chat}开始项目,首先把以下内容添加到\emph{async-chat/Cargo.toml}:
\begin{minted}{toml}
    [package]
    name = "async-chat"
    version = "0.1.0"
    authors = ["You <you@example.com>"]
    edition = "2018"

    [dependencies]
    async-std = { version = "1.7", features = ["unstable"] }
    tokio = { version = "1.0", features = ["sync"] }
    serde = { version = "1.0", features = ["derive", "rc"] }
    serde_json = "1.0"
\end{minted}

我们依赖四个crate:
\begin{enumerate}
    \item \texttt{async-std} crate是我们在本章中一直在用的异步I/O原语和工具的集合。
    \item \texttt{tokio} crate是另一个类似\texttt{async-std}的异步原语的集合,它是最古老和成熟的之一。它被广泛使用并保持设计和实现的高标准,但相比\texttt{async-std}还需要一些别的crate才能使用。
    
    \texttt{tokio}是要给很大的crate,但我们只需要它的一个组件,因此\emph{Cargo.toml}中的\texttt{features = ["sync"]}字段将\texttt{tokio}消减到只有我们需要的部分,让它更加轻量一些。

    当异步库的生态系统不够主流时,人们会避免同时在一个程序中使用\texttt{tokio}和\texttt{async-std},但这两个项目一直在合作来确保可以正确工作,只要遵守它们的文档中的每一条规则。
    \item \texttt{serde}和\texttt{serde\_json} crate我们之前已经在\hyperref[ch18]{第18章}中见过。它们给了我们便利且高效地生成和解析JSON的工具,我们的聊天协议将使用JSON来在网络中表示数据。我们想使用\texttt{serde}中的一些可选特性,因此在我们指定依赖时选择了那些特性。
\end{enumerate}

我们的聊天应用的整体架构,包括客户端和服务器,看起来像这样:
\begin{minted}{text}
    async-chat
    |—— Cargo.toml
    |—— src
        |—— lib.rs
        |—— utils.rs
        |—— bin
            |—— client.rs
            |—— server
                |—— main.rs
                |—— connection.rs
                |—— group.rs
                |—— group_table.rs
\end{minted}

这个包的布局使用了我们在“\nameref{src/bin}”中介绍过的一个Cargo的特性:除了主要的库crate \emph{src/lib.rs}和它的子模块\emph{src/utils.rs}之外,它还包含两个可执行文件:
\begin{enumerate}
    \item \emph{src/bin/client.rs}是聊天客户端的单文件可执行程序。
    \item \emph{src/bin/server}是聊天服务器的可执行程序,它被分成四个文件:\emph{main.rs}保存\texttt{main}函数,还有三个子模块\texttt{connection.rs}、\texttt{group.rs}、\texttt{group\_table.rs}。
\end{enumerate}

我们将在本章中展示每个源文件的内容,等它们都就位之后,如果在目录树中输入\texttt{cargo build},就会编译库crate并且构建两个可执行程序。Cargo会自动把库crate当作一个依赖,这使得它变为一个放置客户端和服务器共享的定义的好地方。类似的,\texttt{cargo check}会检查整个源码树。为了运行其中某一个可执行程序,你可以使用像这样的命令:
\begin{minted}{text}
    $ cargo run --release --bin server -- localhost:8088
    $ cargo run --release --bin client -- localhost:8088
\end{minted}

\texttt{--bin}选项指示了要运行哪一个可执行程序,并且任何跟在\texttt{--}选项后面的参数都会被传给可执行程序本身。我们的客户端和服务器要想知道服务器的地址和TCP端口。

\subsection{\texttt{Error}和\texttt{Result}类型}
库crate的\texttt{utils}模块定义了整个应用中用到的结果和错误类型。\emph{src/utils.rs}:
\begin{minted}{Rust}
    use std::error::Error;

    pub type = ChatError = Box<dyn Error + Send + Sync + 'static>;
    pub type = ChatResult<T> = Result<T, ChatError>;
\end{minted}

这是我们在“\nameref{MultiErr}”中建议过的通用的错误类型。\texttt{async\_std}、\texttt{serde\_json}和\texttt{tokio} crate都定义了它们自己的错误类型,但\texttt{?}运算符可以自动把它们全部转换成一个\texttt{ChatError},使用标准库的\texttt{From} trait的实现可以把任何合适的错误类型转换成\texttt{Box<dyn Error + Send + Sync + 'static>}。\texttt{Send}和\texttt{Sync}约束确保了如果一个被spawn到其他线程的任务失败了,它可以安全地把错误汇报给主线程。

在一个真实的应用中,请考虑使用\texttt{anyhow} crate,它提供了类似于这里的\texttt{Error}和\texttt{Result}类型。\texttt{anyhow} crate易于使用并且提供了一些我们的\texttt{ChatError}和\texttt{ChatResult}没有的很棒的特性。

\subsection{协议}
库crate把整个聊天协议封装在两个类型里,在\emph{lib.rs}中定义:
\begin{minted}{Rust}
    use serde::{Deserialize, Serialize};
    use std::sync::Arc;

    pub mod utils;

    #[derive(Debug, Deserialize, Serialize, PartialEq)]
    pub enum FromClient {
        Join { group_name: Arc<String> },
        Post {
            group_name: Arc<String>,
            message: Arc<String>,
        },
    }

    #[derive(Debug, Deserialize, Serialize, PartialEq)]
    pub enum FromServer {
        Message {
            group_name: Arc<String>,
            message: Arc<String>,
        },
        Error(String),
    }

    #[test]
    fn test_fromclient_json() {
        use std::sync::Arc;

        let from_client = FromClient::Post {
            group_name: Arc::new("Dogs".to_string()),
            message: Arc::new("Samoyeds rock!".to_string()),
        };

        let json = serde_json::to_string(&from_client).unwrap();
        assert_eq!(json,
                   r#"{"Post":{"group_name":"Dogs","message":"Samoyeds rock!"}}"#);
        
        assert_eq!(serde_json::from_str::<FromClient>(&json).unwrap(),
                   from_client);
    }
\end{minted}

\texttt{FromClient}枚举表示一个客户端可能发送给服务器的包:它可以要求加入一个房间并向它加入的房间发送消息。\texttt{FromServer}表示服务器可能返回给客户端的包:被发送到组的消息和错误消息。使用引用计数指针\texttt{Arc<String>}来代替普通的\texttt{String}帮助服务器在管理组和分发消息时避免拷贝字符串。

\texttt{\#[derive]}属性告诉\texttt{serde} crate为\texttt{FromClient}和\texttt{FromServer}生成它的\texttt{Serialize}和\texttt{Deserialize} trait的实现。这让我们可以调用\texttt{serde\_json::to\_string}来把它们转换成JSON值、通过网络发送它们、并且最终调用\texttt{serde\_json::from\_str}来把它们转换回Rust形式。

\texttt{test\_fromclient\_json}单元测试展示了这该如何使用。有了\texttt{serde}生成的\texttt{Serialize}实现,我们可以调用\texttt{serde\_json::to\_string}来把给定的\texttt{FromClient}值转换成这个JSON:
\begin{minted}{json}
    {"Post":{"group_name":"Dogs","message":"Samoyeds rock!"}}
\end{minted}

然后生成的\texttt{Deserialize}实现会把它转换成一个等价的\texttt{FromClient}值。注意\texttt{FromClient}中的\texttt{Arc}指针对序列化的形式没有影响:引用计数的字符串直接作为JSON的对象成员值出现。

\subsection{获取用户输入:异步流}
我们的聊天客户端的第一个功能是读取用户的命令并向服务器发送相应的包。管理一个合适的用户接口超出了本章的范围,因此我们只准备完成能工作的最简单的实现:直接从标准输入读取。下面的代码在\emph{src/bin/client.rs}:
\begin{minted}{Rust}
    use async_std::prelude::*;
    use async_chat::utils::{self, ChatResult};
    use async_std::io;
    use async_std::net;

    async fn send_commands(mut to_server: net::TcpStream) -> ChatResult<()> {
        println!("Commands:\n\
                  join GROUP\n\
                  post GROUP MESSAGE...\n\
                  Type Control-D (on Unix) or Control-Z (on Windows) \
                  to close the connection.");

        let mut command_lines = io::BufReader::new(io::stdin()).lines();
        while let Some(command_result) = command_lines.next().await {
            let command = command_result?;
            // `parse_command`的定义见Github仓库
            let request = match parse_command(&command) {
                Some(request) => request,
                None => continue,
            };

            utils::send_as_json(&mut to_server, &request).await?;
            to_server.flush().await?;
        }
    }
\end{minted}

这段代码中调用了\texttt{async\_std::io::stdin}来获取一个客户端的标准输入的异步handle,用\texttt{async\_std::io::BufReader}包装它来进行缓冲,然后调用\texttt{lines}来逐行处理用户的输入。它尝试把输入的每一行命令行解析为\texttt{FromClient}值,并且如果成功就把值发送给服务器。如果用户输入了未知的命令,\texttt{parse\_command}会打印出错误消息并返回\texttt{None},因此\texttt{send\_commands}可以继续循环。如果用户输入了end-of-file标志,那么\texttt{lines}流会返回\texttt{None},因此\texttt{send\_commands}会返回。这和以普通的同步程序的方式编写的代码非常相似,除了它使用了\texttt{async\_std}版本的库特性。

异步的\texttt{BufReader}的\texttt{lines}方法很有趣。它并不像标准库一样返回一个迭代器:标准库中\texttt{Iterator::next}方法是一个普通的同步函数,因此调用\texttt{commands.next()}将会阻塞线程直到读取到下一行。作为代替,它返回一个\texttt{Result<String>}值的\emph{流}。流是异步中和迭代器类似的概念:它按需以一种异步友好的风格产生一个值的序列。这里是\texttt{Stream} trait的定义,来自于\texttt{async\_std::stream}模块:
\begin{minted}{Rust}
    trait Stream {
        type Item;

        // 现在,把`Pin<&mut Self>`看作`&mut Self`就好。
        fn poll_next(self: Pin<&mut Self>, cx: &mut Context<'_>)
            -> Poll<Option<Self::Item>>;
    }
\end{minted}

你可以将它看作\texttt{Iterator}和\texttt{Future} trait的结合。类似于迭代器,\texttt{Stream}有关联的\texttt{Item}类型并使用\texttt{Option}来指示序列何时结束。但类似于future,流必须被poll才能得到下一个item(或者知道stream已经结束),你必须调用\texttt{poll\_next}直到它返回\texttt{Poll::Ready}。一个流的\texttt{poll\_next}实现应该总是快速返回,不能阻塞。如果一个流返回\texttt{Poll::Pending},它必须在值得再次poll时通过\texttt{Context}提醒调用者。

\texttt{poll\_next}方法直接使用起来很别扭,但你通常不需要这么做。类似迭代器,流也有很多工具方法例如\texttt{filter}和\texttt{map}。其中一个是\texttt{next}方法,它返回流的下一个\texttt{Option<Self::Item>}的future。你可以调用\texttt{next}并await future返回而不是显式地poll流。

将这些组合起来,\texttt{send\_commands}通过使用\texttt{next}和\texttt{while let}迭代一个流产生的值并消耗输入的行:
\begin{minted}{Rust}
    while let Some(item) = stream.next().await {
        ... use item ...
    }
\end{minted}

(未来的Rust版本可能会引入一种\texttt{for}循环语法的异步变体来消耗流,就像普通的\texttt{for}循环消耗\texttt{Iterator}值一样。

在流结束后poll它——即在它返回\texttt{Poll::Ready(None)}来指示流结束之后——就类似于在一个迭代器返回\texttt{None}之后调用\texttt{next}或在一个future返回\texttt{Poll::Ready}之后poll它一样:\texttt{Stream} trait没有指定流的行为,因此有些流可能行为不当。类似于future和迭代器,流有一个\texttt{fuse}方法来确保这样的调用结果是可预测的,更多细节见文档。

当处理流时,要记得use \texttt{async\_std}的 prelude:
\begin{minted}{Rust}
    use async_std::prelude::*;
\end{minted}

这是因为\texttt{Stream} trait的工具方法,例如\texttt{next, map, filter}等等,并不是真的定义在\texttt{Stream}自身里。实际上,它们是另一个trait \texttt{StreamExt}的默认方法,这个trait自动为所有\texttt{Stream}实现:
\begin{minted}{Rust}
    pub trait StreamExt: Stream {
        // ... 以默认方法的方式定义工具方法 ...
    }

    impl<T: Stream> StreamExt for T { }
\end{minted}

这是我们在“\nameref{OrphanRule}”中介绍过的\emph{扩展trait(extension trait)}的一个例子。\texttt{async\_std::prelude}模块把\texttt{StreamExt}的方法引入作用域,因此要记得use这个prelude来确保这些方法在你的代码中可见。

\subsection{发送包}
为了通过网络套接字传输包,我们的客户端和服务器使用了我们的库crate的\texttt{utils}模块中的\texttt{send\_as\_json}函数:
\begin{minted}{Rust}
    use async_std::prelude::*;
    use serde::Serialize;
    use std::marker::Unpin;

    pub async fn send_as_json<S, P>(outbound: &mut S, packet: &P) -> ChatResult<()>
    where
        S: async_std::io::Write + Unpin,
        P: Serialize,
    {
        let mut json = serde_json::to_string(&packet)?;
        json.push('\n');
        outbound.write_all(json.as_bytes()).await?;
        Ok(())
    }
\end{minted}

这个函数构建\texttt{packet}的JSON \texttt{String}表示,在末尾加上了一个换行符,然后全部写入\texttt{outbound}。

通过它的where子句,你可以看到\texttt{send\_as\_json}非常灵活。要发送的包的类型\texttt{P},可以是任何实现了\texttt{serde::Serialize}的类型。输出流\texttt{S}可以是任何实现了\texttt{async_std::io::Write}的类型,这个trait是\texttt{std::io::Write} trait的异步版本。这足够我们在一个异步的\texttt{TcpStream}上发送\texttt{FromClient}和\texttt{FromServer}值。保持\texttt{send\_as\_json}的定义是泛型的可以确保它不依赖流或包的类型的细节:不过\texttt{send\_as\_json}只能使用那些trait的方法。

为了使用\texttt{write\_all}方法,\texttt{S}中的\texttt{Unpin}约束是必须的。我们将在本章稍后介绍pin和unpin,但现在只需要给需要的类型参数添加上\texttt{Unpin}约束即可,如果你哪里忘了,Rust编译器会指出来的。

\texttt{send\_as\_json}把包序列化到临时的\texttt{String}中,然后写入到\texttt{outbound}中,而不是直接序列化到\texttt{outbound}流中。\texttt{serde\_json} crate确实提供了一些函数把值直接序列化到输出流,但那些函数只支持同步流。写入到异步流需要同时修改\texttt{serde\_json}和\texttt{serde} crate的格式无关的核心,因为这些trait是为同步方法设计的。

和流一样,\texttt{async\_std}的I/O trait的很多方法实际上也是定义在扩展trait中,因此无论何时使用它们都要记得\texttt{use async\_std::prelude::*}。

\subsection{接收包:更多异步流}


\section{原语future和executor:何时一个future值得再次poll}\label{WhenPoll}

    \chapter{宏}\label{ch21}

\emph{A cento (from the Latin for “patchwork”) is a poem made up entirely of lines quoted from another poet.}
\begin{flushright}
    ——Matt Madden
\end{flushright}

Rust支持\emph{宏(macros)}:一种普通函数无法做到的扩展语言的方式。例如,我们已经看到过\texttt{assert\_eq!}宏,它可以方便地用来测试:
\begin{minted}{Rust}
    assert_eq!(gcd(6, 10), 2);
\end{minted}

这也可以写成一个泛型函数,但\texttt{assert\_eq!}可以做到几件函数做不到的事。一是当断言失败时,\texttt{assert\_eq!}会生成包含断言所在的文件名和行号的错误消息。函数没有办法获得这些信息。但宏可以,因为它们工作的方式完全不同。

宏是一种缩写。在编译期间,在类型检查之前、更在生成任何机器码之前,每一个宏调用都会被\emph{展开(expand)}——即,被替换为一些Rust代码。上面的宏调用会展开成类似这样的代码:
\begin{minted}{Rust}
    match (&gcd(6, 10), &2) {
        (left_val, right_val) => {
            if !(*left_val == *right_val) {
                panic!("assertion failed: `(left == right)`, \
                        (left: `{:?}`, right: `{:?}`)", left_val, right_val));
            }
        }
    }
\end{minted}

\texttt{panic!}也是一个宏,它自己会展开成更多Rust代码(这里没有展示)。那些代码里用到了两个别的宏,\texttt{file!()}和\texttt{line!()}。一旦crate中的每一个宏调用都被完全展开,Rust会进入编译的下一个阶段。

在运行时,一个断言失败看起来像这样(并且可能指示\texttt{gcd()}函数中的一个bug,因为\texttt{2}是正确的答案):
\begin{minted}{text}
    thread 'main' panicked at 'assertion failed: `(left == right)`, (left: `17`,
    right: `2`)', gcd.rs:7
\end{minted}

如果你是从C++来的,你可能经历过一些宏的糟糕体验。Rust的宏采用了一种不同的方式,类似于Scheme的\texttt{syntax-rules}。相比于C++的宏,Rust的宏和语言的其他部分集成得更好,并且因此更不容易出错。宏调用总是用感叹号标记,这样当你阅读代码时它们会很显眼,并且不会在你想调用函数时偶然错误地调用成了宏。Rust的宏从来不会插入不匹配的花括号或圆括号。并且Rust的宏带有模式匹配,这使得编写既可维护又易于使用的宏变得更容易。

在本章中,我们将通过几个简单的示例展示如何编写宏。但和Rust中的很多部分一样,宏值得深入理解,所以我们将介绍一个更复杂的宏的设计,它允许我们直接在我们的程序中嵌入JSON字面量。但除了本书中介绍的部分之外,宏还有更多的内容。因此我们将以一些进一步学习的点结束,既有我们将在这里向你展示的高级技巧,也有功能更加强大的设施称为\emph{过程宏(procedural macros)}。

\section{宏基础}
\autoref{f21-1}展示了\texttt{assert\_eq!}宏的部分源码。

\begin{figure}[htbp]
    \centering
    \includegraphics[width=0.9\textwidth]{../img/f21-1.png}
    \caption{\texttt{assert\_eq!}宏}
    \label{f21-1}
\end{figure}

\texttt{macro\_rules!}是Rust中定义宏的主要方法。注意,宏定义里\texttt{assert\_eq}后边没有\texttt{!}:只有在调用宏时才需要\texttt{!},定义时不需要。

并不是所有的宏都是以这种方式定义的:少数的宏,例如\texttt{file!}、\texttt{line!}和\texttt{macro\_rules!}自身,是编译器内建的。我们将在本章的末尾讨论另一种方法,称为过程宏。但我们的主要精力还是集中在\texttt{macro\_rules!},这是(目前为止)最容易的编写自己的宏的方法。

一个用\texttt{macro\_rules!}定义的宏完全靠模式匹配工作。宏的主体只是一系列规则:
\begin{minted}{text}
    ( pattern1 ) => ( template1 );
    ( pattern2 ) => ( template2 );
    ...
\end{minted}

\autoref{f21-1}中的\texttt{assert\_eq!}版本只有一个模式和一个模板。

顺便,你可以使用方括号或者花括号来代替模式或模板两侧的圆括号,对Rust来说它们并没有任何区别。另外,当你调用一个宏时,这些都是等价的:
\begin{minted}{Rust}
    assert_eq!(gcd(6, 10), 2);
    assert_eq![gcd(6, 10), 2];
    assert_eq!{gcd(6, 10), 2}
\end{minted}

唯一的不同是花括号后边的分号是可选的。为了方便,我们在调用\texttt{assert\_eq!}时使用圆括号,调用\texttt{vec!}时使用方括号,\texttt{macro\_rules!}时使用花括号。

现在我们展示了一个宏展开的简单示例和生成这个宏的定义,我们可以深入了解它工作的细节:
\begin{itemize}
    \item 我们将详细地解释Rust是怎么发现和展开你的程序中的宏定义的。
    \item 我们将指出在根据宏模板生成代码时的一些细节之处。
    \item 最后,我们将展示模式如何处理重复的结构。
\end{itemize}

\subsection{宏展开基础}
Rust会在编译的前期展开宏。编译器会从头到尾读取源码,在这个过程中定义和展开宏。宏只有在定义之后才能被调用,因为Rust会立刻展开每一个宏调用,而不会去看程序的剩余部分。(相反,函数和其他的\hyperref[static]{item}的定义不需要按照任何顺序。调用一个后面才定义的函数也是OK的。

当Rust展开一个\texttt{assert\_eq!}宏调用时,行为和执行一个\texttt{match}表达式非常像。Rust首先根据模式匹配参数,如\autoref{f21-2}所示:
\begin{figure}[htbp]
    \centering
    \includegraphics[width=0.8\textwidth]{../img/f21-2.png}
    \caption{展开一个宏,第1部分:用模式匹配参数}
    \label{f21-2}
\end{figure}

宏的模式是Rust的mini语言。它们本质上是匹配代码的正则表达式。但普通的正则表达式是操作字符,而模式操作\emph{token(词元)}——数字、变量名、标点符号等Rust中的构建块。这意味着你可以在宏模式中自由地使用注释和空格来提升它们的可读性。注释和空格不是token,因此它们不会影响到匹配。

正则表达式和宏模式的另一个重要不同之处是圆括号、方括号、花括号在Rust中总是成对出现。这一点会在宏展开之前就检查,不仅仅是在宏模式中,而且是在整个语言中。

在这个例子中,我们的模式包含了\emph{fragment(片段)} \texttt{\$left:expr},这告诉Rust去匹配一个表达式(在这个例子中,就是\texttt{gcd(6, 10)}并把它复制到名称\texttt{\$left}。然后Rust用\texttt{gcd}调用后边的逗号匹配模式中的逗号。类似于正则表达式,模式只有少数特殊字符会触发有趣的匹配行为;其它的所有字符,例如逗号,都必须逐字匹配相同的字符,否则就会匹配失败。

这个模式中的两个代码片段都是\texttt{expr}类型:它们代表表达式。我们将在“\nameref{FragType}”中看到其他类型的代码片段。

因为这个模式匹配到了所有的参数,Rust会展开相应的\emph{template(模板)}(\autoref{f21-3}):
\begin{figure}[htbp]
    \centering
    \includegraphics[width=0.9\textwidth]{../img/f21-3.png}
    \caption{展开一个宏,第2部分:填充模板}
    \label{f21-3}
\end{figure}

Rust会用匹配阶段发现的代码片段来替换\texttt{\$left}和\texttt{\$right}。

一个常见的错误是在输出模板中包含片段的类型:写\texttt{\$left:expr}而不是\texttt{\$left}。Rust不会立刻检测出这种错误。它会把\texttt{\$left}看错一个整体,然后把\texttt{:expr}看作和模板中其他部分一样的东西:要包含在宏的输出中的词元。因此只有当你\emph{调用}这个宏时这个错误才会出现;然后它会生成错误的不能编译的输出。如果你在使用一个新的宏时得到了类似\texttt{cannot find type `expr` in this scope}和\texttt{help: maybe you meant to use a path separator here}这样的错误消息,可以检查下它是不是有这个错误。(“\nameref{DebugMacro}”提供了更多类似这种情况的建议。)

宏模板和web编程中常用的很多种模板语言中的任意一种都没有太大的区别。唯一的不同——也是很重要的一点——就是它的输出是Rust代码。

\subsection{意外的结果}
把代码片段插入模板和普通的处理值的代码有一些区别。这些区别一开始可能并不明显。我们之前看到的\texttt{assert\_eq!}宏包含了一些有些奇怪的代码,这是宏编程特有的。让我们重点看看其中两个有趣的部分。

首先,为什么这个宏创建了两个变量\texttt{left\_val}和\texttt{right\_val}?为什么我们不能把模板简化成这样?
\begin{minted}{Rust}
    if !($left == $right) {
        panic!("assertion failed: `(left == right)` \
                (left: `{:?}`, right: `{:?}`)", $left, $right)
    }
\end{minted}
为了回答这个问题,尝试手动展开宏调用\texttt{assert\_eq!(letters.pop(), Some('z'))}。输出将会是什么?Rust会把匹配到的表达式插入模板中的多个位置。看起来在构建错误消息时重新求值表达式是一个坏主意,原因不仅仅是因为它需要消耗两倍的时间:因为\texttt{letters.pop()}从vector中移除一个值,所以在我们第二次调用它时会产生一个和第一次不同的值!这就是为什么真正的宏只计算一次\texttt{\$left}和\texttt{\$right},然后存储它们的值。

来到第二个问题:为什么这个宏借用了\texttt{\$left}和\texttt{\$right}的值的引用?为什么不直接把它们的\emph{值}存进变量中,像这样?
\begin{minted}{Rust}
    macro_rules! bad_assert_eq {
        ($left:expr, $right:expr) => ({
            match ($left, $right) {
                (left_val, right_val) => {
                    if !(left_val == right_val) {
                        panic!("assertion failed" /* ... */);
                    }
                }
            }
        });
    }
\end{minted}

对于我们展示过的特殊情况,即宏的参数是整数的情况下,这可以正常工作。但如果调用者传递了一个\texttt{String}变量作为\texttt{\$left}或\texttt{\$right},这段代码将会移动变量的值!
\begin{minted}{Rust}
    fn main() {
        let s = "a rose".to_string();
        bad_assert_eq!(s, "a rose");
        println!("confirmed: {} is a rose", s);    // error: use of moved value "s"
    }
\end{minted}

因为我们不想让断言移动值,所以宏里使用了引用。

(你可能想知道为什么这个宏使用了\texttt{match}而不是\texttt{let}来定义变量。我们也想知道。事实证明并没有特殊的原因这么做。\texttt{let}是等价的。)

简而言之,宏可能做出令人惊讶的行为。如果一个你写的宏附近发生了奇怪的事,那很可能是这个宏有问题。

C++里的这个经典bug你\emph{将不会}看到:
\begin{minted}{Rust}
    // buggy C++ macro to add 1 to a number
    #define ADD_ONE(n) n + 1
\end{minted}

原因大部分的C++程序员应该很熟悉了,并且不值得详细解释,类似\texttt{ADD\_ONE(1) * 10}或者\texttt{ADD\_ONE(1 << 4)}这样的代码可能会产生令人惊讶的行为。为了修复它,你需要在宏定义中加上更多括号。这在Rust中是不必要的,因为Rust宏和语言集成的更好。Rust知道它什么时候是在处理表达式,因此在把一个表达式粘贴到另一个地方时它会自动添加括号。

\subsection{重复}
标准的\texttt{vec!}宏有两种形式:
\begin{minted}{Rust}
    // 重复一个值N次
    let buffer = vec![0_u8; 1000];

    // 一个逗号分隔的值的列表
    let numbers = vec!["udon", "ramen", "soba"];
\end{minted}

它可以像这样实现:
\begin{minted}{Rust}
    macro_rules! vec {
        ($elem:expr ; $n:expr) => {
            ::std::vec::from_elem($elem, $n)
        };
        ( $( $x:expr ),* ) => {
            <[_]>::into_vec(Box::new([ $( $x ),* ]))
        };
        ( $( $x:expr ),+ ,) => {
            vec![ $( $x ),* ]
        };
    }
\end{minted}

这里有三个规则。我们将解释多个规则是如何工作的,然后依次看看每一个规则。

当Rust展开一个例如\texttt{vec![1, 2, 3]}的宏调用时,它首先尝试匹配参数\texttt{1, 2, 3}和第一条规则的模式,即\texttt{\$elem:expr ; \$n:expr}。这会匹配失败:\texttt{1}是一个表达式,但这个模式要求它后边有一个分号,然而并没有。因此Rust会移动到第二个规则,等等。如果没有规则可以匹配,就会报错。

第一个规则处理类似\texttt{vec![0u8; 1000]}这样的调用。恰好有一个标准(但不在文档里的)函数\texttt{std::vec::from\_elem},正好可以完成我们需要的功能,因此这个规则很直观。

第二个规则处理\texttt{vec!["udon", "ramen", "soba"]}。模式\texttt{\$( \$x:expr ),*}使用了一个我们之前没有提过的的特性:重复。它匹配0个或多个逗号分隔的表达式。更一般地来说,语法\texttt{\$( PATTERN ),*}用来匹配任意逗号分割的列表,其中列表中的每一项匹配\texttt{PATTERN}。

这里的\texttt{*}和正则表达式中的\texttt{*}有相同的含义(“0次或多次”),尽管正则表达式没有特殊的\texttt{,*}重复器。你也可以用\texttt{+}来要求至少一次匹配,或者用\texttt{?}要求0次或1次匹配。\autoref{t21-1}列出了全套的重复模式。

\begin{table}[htbp]
    \centering
    \caption{重复模式}
    \label{t21-1}
    \begin{tabular}{ll}
        \hline
        \textbf{模式} & \textbf{含义} \\
        \hline
        \texttt{\$( ... )*}     & 匹配0次或多次,无分隔符   \\
        \rowcolor{tablecolor}
        \texttt{\$( ... ),*}    & 匹配0次或多次,逗号分隔   \\
        \texttt{\$( ... );*}    & 匹配0次或多次,分号分隔   \\
        \rowcolor{tablecolor}
        \texttt{\$( ... )+}     & 匹配1次或多次,无分隔符   \\
        \texttt{\$( ... ),+}    & 匹配1次或多次,逗号分隔   \\
        \rowcolor{tablecolor}
        \texttt{\$( ... );+}    & 匹配1次或多次,分号分隔   \\
        \texttt{\$( ... )?}     & 匹配0次或1次,无分隔符    \\
        \rowcolor{tablecolor}
        \texttt{\$( ... ),?}    & 匹配0次或1次,逗号分隔    \\
        \texttt{\$( ... );?}    & 匹配0次或1次,分号分隔    \\
    \end{tabular}
\end{table}

代码片段\texttt{\$x}不是单个表达式,而是一个表达式的列表。这个规则的模板也使用了重复语法:
\begin{minted}{Rust}
    <[_]>::into_vec(Box::new([ $( $x ),* ]))
\end{minted}

再一次,恰好有标准方法可以满足我们的需要。这段代码创建了一个装箱的数组,然后使用\texttt{[T]::into\_vec}方法把这个装箱的数组转换成一个vector。

开头的\texttt{<[\_]>},是一种不常见的写法,它表示类型“某些东西的切片”,由Rust来推断元素类型。名字是普通标识符的类型可以直接在表达式中使用,但类似\texttt{fn()}、\texttt{\&str},或者\texttt{[\_]}这样的类型必须用尖括号包裹。

重复模式出现在模板的末尾,即\texttt{\$(\$x),*}。这个\texttt{\$(...),*}和我们在模式中看到的是相同的语法。它迭代\texttt{\$x}匹配到的表达式列表并把它们全部插入模板,用逗号分隔。

在这种情况下看,重复的输出看起来和输入一样。但并不总是这样。我们可以编写类似这样的规则:
\begin{minted}{Rust}
    ( $( $x:expr ),* ) => {
        {
            let mut v = Vec::new();
            $( v.push($x); )*
            v
        }
    };
\end{minted}

这里,模板中\texttt{\$( v.push(\$x); )*}这一部分会对\texttt{\$x}中的每个表达式插入一个\texttt{v.push()}调用。一个宏分支可以展开成一个表达式序列,但这里我们只需要单个表达式,所以我们把vector的处理包装在一个块中。

和Rust中其他部分不同,使用\texttt{\$( ... ),*}并不能自动支持可选的尾部逗号。然而,有一种标准的技巧是通过添加一个额外的规则来支持尾部逗号。也就是我们的\texttt{vec!}宏的第三条规则所做的:
\begin{minted}{Rust}
    ( $( $x:expr ),+ ,) => {    // 如果存在尾部的逗号,
        vec![ $( $x ),* ]       // 重试没有它的情况
    }
\end{minted}

我们使用\texttt{\$( ... ),+ ,}来匹配一个有额外逗号的列表。然后,我们在模板中递归调用了\texttt{vec!},但排除了那个逗号。这一次第二条规则将会匹配。

\section{内建的宏}
Rust编译器提供了几个宏,如果你要定义自己的宏,它们可能会发挥作用。这些宏都不能使用\texttt{macro\_rules!}来实现。它们被硬编码进\texttt{rustc}:

\codeentry{file!(), line!(), column!()}

\hangparagraph{\texttt{file!()}展开成一个字符串字面量:当前的文件名。\texttt{line!()}和\texttt{column!()}展开成\texttt{u32}字面量,表示当前的行号和列号(从1开始计数)。}

\hangparagraph{如果一个宏调用了另一个宏,那个宏又调用了另一个宏,这三个宏在不同的文件中,并且最后一个宏调用了\texttt{file!(), line!()}或者\texttt{column!()},它会展开成\emph{第一个}宏调用的位置。}

\codeentry{stringify!(...tokens...)}

\hangparagraph{展开成一个包含给定token的字符串字面量。\texttt{assert!}宏就是使用了它来生成一条包含了断言代码的错误信息。}

\hangparagraph{参数中的宏调用\emph{不会}被展开:\texttt{stringify!(line!())}会展开为\texttt{"line!()"}。}

\hangparagraph{Rust根据token构建字符串,因此生成的字符串里没有换行符或者注释。}

\codeentry{concat!(str0, str1, ...)}

\hangparagraph{连接参数开展为单个字符串字面量。}

Rust还定义了下面这些宏用来查询构建环境:

\codeentry{cfg!(...)}

\hangparagraph{展开为一个bool常量,如果当前的构建环境满足括号里的条件则为\texttt{true}。例如,如果在编译时启用了调试断言那么\texttt{cfg!(debug\_assertions)}为真。}

\hangparagraph{这个宏和\nameref{attribute}中介绍过的\texttt{\#[cfg(...)]}属性的语法完全相同,但它不是条件编译,而是得到一个true或者false。}

\codeentry{env!("VAR\_NAME")}

\hangparagraph{展开为一个字符串:在编译时该环境变量的值。如果这个变量不存在,会产生编译错误。}

\hangparagraph{除了Cargo在编译crate时设置的几个有趣的环境变量之外,这个宏毫无价值。例如,为了得到crate当前的版本,你可以写:}
\begin{minted}{Rust}
        let version = env!("CARGO_PKG_VERSION");
\end{minted}

\hangparagraph{这些环境变量的完整列表见\href{https://doc.rust-lang.org/cargo/reference/environment-variables.html\#environment-variables-cargo-sets-for-crates}{Cargo文档}。}

\codeentry{option\_env!("VAR\_NAME")}

\hangparagraph{它和\texttt{env!}宏一样,除了它返回一个\texttt{Option<\&'static str>},当环境变量没有设置时返回\texttt{None}。}

还有更多内建的宏可以让你引入另一个文件中的代码或者数据:
\codeentry{include!("file.rs")}

\hangparagraph{展开为指定文件的内容,必须是有效的Rust代码——表达式或者\hyperref[declaration]{item}的序列。}

\codeentry{include\_str!("file.txt")}

\hangparagraph{展开成一个包含指定文件内容的\texttt{\&'static str}。你可以像这样使用它:}

\begin{minted}{Rust}
        const COMPOSITOR_SHADER: &str =
            include_str!("../resources/compositor.glsl");
\end{minted}

\hangparagraph{如果文件不存在或者不是有效的UTF-8,会产生编译错误。}

\codeentry{include\_bytes!("file.dat")}

\hangparagraph{和上一个基本相同,除了它把文件当作二进制数据而不是UTF-8文本,结果是一个\texttt{\&'static [u8]}。}

和所有的宏一样,这些宏也是在编译时进行处理。如果文件不存在或者不能被读取,就会编译失败。它们不可能在运行时出错。在任何情况下,如果文件名是一个相对路径,它会被解析为相对于当前文件所在的目录的路径。

Rust还提供了几个方便的宏:
\codeentry{todo!(), unimplemented!()}

\hangparagraph{这些等价于\texttt{panic!()},但用于表示不同的意图。\texttt{unimplemented!()}出现在\texttt{if}分支、\texttt{march}分支,以及其它还未处理的case中。它总是会panic。\texttt{todo!()}大致相同,但传达的意图是代码还没写完;一些IDE使用它来进行标记。}

\codeentry{matches!(value, pattern)}

\hangparagraph{比较一个值和一个模式,当它们匹配时返回\texttt{true},否则返回\texttt{false}。它等价于写:}

\begin{minted}{Rust}
        match value {
            pattern => true,
            _ => false
        }
\end{minted}

\hangparagraph{如果你在寻找基本的编写宏的练习,这是一个很好的例子——尤其是你可以在标准库文档中看到它的实际实现非常简单。}

\section{调试宏}\label{DebugMacro}
调试宏可能很有挑战性。最大的问题是在宏展开的过程中缺少可视性。Rust总是展开所有宏,找到一些错误,然后打印出一条错误信息,但这个错误信息并没有显示出完整的展开后的代码!

这里有三个工具可以帮助你调试宏。(这些特性都是unstable的,但因为它们被设计为用在开发的过程中,而不是最后的代码中,因此在实践中这不是一个很大的问题。)

第一个也是最简单的一个,你可以让\texttt{rustc}显示你的代码在展开所有宏之后是什么样的。使用\texttt{cargo build --verbose}来看看Cargo是怎么调用\texttt{rustc}的。拷贝\texttt{rustc}的命令行并加上\texttt{-Z unstable-options --pretty expanded}选项。完全展开后的代码会输出到终端。不幸的是,只有当你的代码没有语法错误时这种方式才能生效。

第二,Rust提供了一个\texttt{log\_syntax!()}宏简单地在编译期把它的参数打印到终端。你可以使用它来进行\texttt{println!}风格的调试。这个宏需要\texttt{\#![feature(log\_syntax)]}特性标记。

第三,你可以让Rust编译器把所有宏调用记录到终端。在代码中插入\texttt{trace\_macros!(true)},之后每当Rust展开一个宏时,它都会打印出宏的名字和参数。例如,考虑这个程序:
\begin{minted}{Rust}
    #![feature(trace_macros)]

    fn main() {
        trace_macros!(true);
        let numbers = vec![1, 2, 3];
        trace_macros!(false);
        println!("total: {}", numbers.iter().sum::<u64>());
    }
\end{minted}

它会产生如下输出:
\begin{minted}{text}
    $ rustup override set nightly
    ...
    $ rustc trace_example.rs
    note: trace_macro
     --> trace_example.rs:5:19
      |
    5 |     let numbers = vec![1, 2, 3];
      |                   ^^^^^^^^^^^^^
      |
      = note: expanding `vec! { 1 , 2 , 3 }`
      = note: to `< [ _ ] > :: into_vec ( box [ 1 , 2 , 3 ] )`
\end{minted}
编译器会显示每一个宏调用的代码,包括展开之前和展开之后的代码。\texttt{trace\_macros!(false);}这一行关闭了追踪,因此\texttt{println!()}的调用不会被追踪。

\section{构建\texttt{json!}宏}



\subsection{片段类型}\label{FragType}

    \chapter{unsafe代码}\label{ch22}

\emph{Let no one think of me that I am humble or weak or passive; \\
Let them understand I am of a different kind: dangerous to my enemies, loyal to my friends. \\
To such a life glory belongs.}

系统级编程的乐趣在于,在每一个安全的语言和精心设计的抽象之下,都是不安全的机器语言和比特位。你也可以在Rust中写出这样的代码。

到目前为止本书中介绍的语言部分,例如类型、生命周期、约束检查等都可以自动保证你的程序完全没有内存错误和数据竞争。但这种自动的技术有它的局限性;Rust并不能识别出来很多有价值的技术是安全的。

\emph{unsafe代码}让你可以告诉Rust,“我要使用一些你不能保证安全的特性”。把一个块或者函数标记为unsafe之后,你就可以调用标准库中的\texttt{unsafe}函数、解引用unsafe指针、调用其他语言例如C和C++编写的函数等。Rust的其他安全性检查依然生效:类型检查、生命周期检查、约束检查等仍然和之前一样。unsafe代码只是允许了一小部分额外的特性。

正是因为有了允许超出safe Rust界限的能力,Rust才能实现它自身的大部分基础特性,和C/C++一样,Rust也被用来实现它自己的标准库。unsafe代码可以让\texttt{Vec}类型更高效地管理它的缓冲区;让\texttt{std::io}模块和操作系统交互;让\texttt{std::thread}和\texttt{std::sync}模块提供并发原语。

本章介绍了unsafe特性的一些基础:

\begin{enumerate}
    \item Rust的\texttt{unsafe}块区分开了普通的safe Rust代码和使用unsafe特性的代码。
    \item 你可以把函数标记为\texttt{unsafe},提醒调用者他们必须遵守的一些额外约束来避免未定义行为。
    \item 原始指针和它们的方法运行对内存进行不受限的访问,并允许你构建Rust的类型系统可能禁止的数据结构。Rust的引用虽然安全,但却是受限制的,原始指针正如每个C或C++程序员所知,是一个强大而锋利的工具。
    \item 理解未定义行为的定义将会帮助你理解为什么它们比得到错误结果还要糟糕的多。
    \item unsafe trait,类似于\texttt{unsafe}函数,隐含了每个实现(而不是每个调用者)都要遵守的规则。
\end{enumerate}

\section{unsafe从何而来?}
在本书的开头处,我们曾经展示过一个C程序,它以一种非常令人惊讶的方式崩溃。这是因为它违背了C标准的一个规则。你可以在Rust中实现相同的效果:

\begin{minted}{bash}
    $cat crash.rs
    fn main() {
        let mut a: usize = 0;
        let ptr = &mut a as *mut usize;
        unsafe {
            *ptr.offset(3) = 0x7ffff72f484c;
        }
    }
    $ cargo build
       Compiling unsafe-samples v0.1.0
        Finished debug [unoptimized + debuginfo] target(s) in 0.44s
    $ ../../target/debug/crash
    crash: Error: .netrc file is readable by others.
    crash: Remove password or make file unreadable by others.
    Segmentation fault (core dumped)
    $
\end{minted}

这个程序借用了局部变量\texttt{a}的一个可变引用,然后把它转换成了\texttt{*mut usize}类型的原始指针,然后使用了它的\texttt{offset}方法来产生一个指向三个字之后的位置的指针。这里恰巧存储了\texttt{main}的返回地址。这个程序用一个常量覆盖了返回地址,因此从\texttt{main}返回之后程序的行为就很奇怪。这个崩溃之所以可行,是因为程序错误使用了unsafe的特性——在这个例子中,就是解引用原始指针的能力。

一个unsafe特性通常隐含了一份\emph{合约(contract)}:即一组Rust不能自动强制,但你必须遵守才能避免\emph{未定义行为}的规则。

一份合约超出了普通的类型检查和生命周期检查,它们隐含着一些unsafe特性特定的规则。通常来说,Rust本身完全不知道这些合约,它们只在unsafe特性的文档里得到解释。例如,原始指针类型的合约是禁止解引用一个指向的位置超出原来指向物的末尾的原始指针。这个例子中的表达式\texttt{*ptr.offset(3) = ...}打破了这个合约。但是,正如上面所示,Rust没有任何警告,成功编译了这个程序:它的安全检查并没有检测出来这个违规行为。当你使用unsafe特性时,你作为程序员,需要负责检查你的代码遵守了它们的合约。

很多特性如果想正确使用,都要遵守一定的规则。但这些规则并不是我们这里说的合约,除非它们可能会导致未定义行为。未定义行为是一种Rust假设你的代码中绝对不会出现的行为。例如,Rust假设你不会用别的值覆盖函数的返回地址。通过了Rust通常的安全检查并且遵守了使用到的unsafe特性的合约的代码不可能会出现这样的行为。因为这个程序违反了原始指针的合约,因此它的行为变得未定义,并最终崩溃。

如果你的代码出现了未定义行为,说明你打破了你负责的一部分,Rust拒绝预测结果。从系统库的深处抛出来一个错误并崩溃是一种可能的结果;把你的计算机的控制权交给攻击者是另一种可能的结果。不同的Rust版本可能也会有不同的行为。然而,有时未定义行为不一定会产生可见的结果。例如,如果这里的\texttt{main}函数永远不会返回(可能调用了\texttt{std::process::exit})来提前终止程序),那么错误的返回地址也无关紧要。

你只能在\texttt{unsafe}块或者\texttt{unsafe}函数里使用unsafe特性;我们将在接下来的小节介绍它们。它们让unsafe特性不容易被忽略:通过强迫你写一个\texttt{unsafe}块或者函数,Rust能确保你知道你的代码可能要遵守一些额外的规则。

\section{unsafe块}
\texttt{unsafe}块看起来就像一个以\texttt{unsafe}关键字开头的普通块,区别在于你可以在unsafe块里使用unsafe特性:
\begin{minted}{Rust}
    unsafe {
        String::from_utf8_unchecked(ascii)
    }
\end{minted}

如果没有块前面的\texttt{unsafe}关键字,Rust将会禁止使用\texttt{from\_utf8\_unchecked},因为它是一个\texttt{unsafe}的函数。在\texttt{unsafe}块中,你可以随意使用它。

和普通的Rust块一样,\texttt{unsafe}块的值也是最后一条表达式的值,或者是\texttt{()}。这个例子中\texttt{String::from\_utf8\_unchecked}的调用提供了块的值。

一个\texttt{unsafe}块为你解锁了5个额外的功能:
\begin{enumerate}
    \item 你可以调用\texttt{unsafe}函数。每一个\texttt{unsafe}函数都有它自己的合约,这取决于它的功能。
    \item 你可以解引用原始指针。safe代码可以传递、比较、通过引用转换创建原始指针,但只有unsafe代码才可以使用它们来访问内存。我们将在\nameref{rawp}中详细介绍原始指针并解释如何安全地使用它们。
    \item 你可以访问\texttt{union}的字段,尽管编译器不能确定它们含有相应类型的有效值。
    \item 你可以访问可变的\texttt{static}变量。正如\nameref{globalvar}中解释的一样,Rust不能确保什么时候有线程正在使用可变的\texttt{static}变量,因此它们的合约要求你要确保所有的访问都是正确同步的。
    \item 你可以访问通过Rust的外部函数接口声明的函数和变量。即使它们是不可变的,也会被认为是\texttt{unsafe}的,因为它们是使用其他语言编写的,这些语言可能不遵守Rust的安全规则。
\end{enumerate}

把unsafe特性约束在\texttt{unsafe}块里并不会真的阻止你做任何想做的事。你完全只需要在你的代码里加上一个\texttt{unsafe}块,然后就可以继续了。这个规则的作用主要是为了把人类的注意力吸引到那些Rust不能保证安全性的代码上:
\begin{enumerate}
    \item 你不会意外地使用到unsafe特性,然后发现你要为甚至不知道它的存在的合约负责。
    \item 一个\texttt{unsafe}块可以吸引reviewer更多的注意力。一些项目甚至有一些自动化流程来确保这一点,例如标记出会影响\texttt{unsafe}块的代码来吸引更多注意力。
    \item 当你正在考虑编写一个\texttt{unsafe}块时,你可以花费一点时间来问问自己你的任务是否真的需要这些特性。如果是为了性能,你是否有测量数据表明这真的是一个性能瓶颈?可能有一种在safe Rust中也可以实现相同效果的方法。
\end{enumerate}

\section{示例:一个高效的ASCII字符类型}
这里有一个示例,\texttt{Ascii}是一个字符串类型,它确保它的内容总是有效的ASCII字符。这个类型使用一个unsafe特性来提供到\texttt{String}的0开销转换:
\begin{minted}{Rust}
    mod my_ascii {
        /// 一个ASCII编码的字符串
        #[derive(Debug, Eq, PartialEq)]
        pub struct Ascii(
            // 它必须只存有有效的ASCII文本:从`0`到`0x7f`的字节序列
            Vec<u8>
        );

        impl Ascii {
            /// 从`bytes`中的Ascii文本创建一个`Ascii`。
            /// 如果`bytes`中含有任何非ASCII字符就返回一个`NotAsciiError`错误。
            pub fn from_bytes(bytes: Vec<u8>) -> Result<Ascii, NotAsciiError> {
                if bytes.iter().any(|&byte| !byte.is_ascii()) {
                    return Err(NotAsciiError(bytes));
                }
                Ok(Ascii(bytes))
            }
        }

        // 当转换失败时,我们会给出不能转换的vector。
        // 它应该实现`std::error::Error`,这里为了简洁就省略了。
        #[derive(Debug, Eq, PartialEq)]
        pub struct NotAsciiError(pub Vec<u8>);

        // 安全、高效的转换,使用unsafe代码实现。
        impl From<Ascii> for String {
            fn from(ascii: Ascii) -> String {
                // 如果这个模块没有bug的话,这里就是安全的,
                // 因为有效的ASCII文本也是有效的UTF-8文本。
                unsafe { String::from_utf8_unchecked(ascii.0) }
            }
        }
        ...
    }
\end{minted}

这个模块的关键是\texttt{Ascii}类型的定义。这个类型本身被标记为\texttt{pub},来让它在\texttt{my\_ascii}模块外可见。但它的\texttt{Vec<u8>}元素\emph{不是}public的,因此只有\texttt{my\_ascii}模块里的方法可以创建一个\texttt{Ascii}值或者访问它的元素。这完全控制了模块里哪些代码是公开的哪些是不公开的。只要public的构造器和方法能确保新创建的\texttt{Ascii}值是有效的,并始终保持有效,那么程序的其他部分就不可能违反规则。并且public的构造器\texttt{Ascii::from\_bytes}确实小心地检查了给定的vector来确保能从它构建出一个有效的\texttt{Ascii}。出于简洁性的考虑,我们并没有展示出每一个方法,但你可以想象还有一些处理文本的方法,这些方法同样确保\texttt{Ascii}的值总是有效的ASCII文本,就像\texttt{String}的方法确保它的内容总是有效的UTF-8.

这样的安排让我们可以非常高效地为\texttt{String}实现\texttt{From<Ascii>}。unsafe函数\texttt{String::from\_utf8\_unchecked}获取一个字节vector并根据它的内容构建一个\texttt{String},并且不检查它的内容是否是有效的UTF-8文本;这个函数的合约就是调用者要负责这一点。幸运的是,\texttt{Ascii}类型强迫的规则正是满足\texttt{from\_utf8\_unchecked}的合约所需的条件。正如我们在\nameref{utf8}中解释的一样,任何有效的ASCII文本都是有效的UTF-8文本,因此\texttt{Ascii}内部的\texttt{Vec<u8>}可以立刻作为一个\texttt{String}的缓冲区使用。

有了这些定义之后,你可以写这样的代码:
\begin{minted}{Rust}
    use my_ascii::Ascii;

    let bytes: Vec<u8> = b"ASCII and ye shall receive".to_vec();

    // 这里的调用没有任何内存分配或者文本拷贝,只进行一次扫描。
    let ascii: Ascii = Ascii::from_bytes(bytes)
        .unwrap();  // 我们已经知道了bytes是没问题的。

    // 这里的调用是0开销的:没有内存分配、拷贝、扫描。
    let string = String::from(ascii);

    assert_eq!(string, "ASCII and ye shall receive");
\end{minted}

使用\texttt{Ascii}不需要unsafe块。我们已经使用unsafe操作实现了一个safe的接口,并且安排好只依赖模块自己的代码而不是用户的行为来满足它的合约。

\texttt{Ascii}只是一个\texttt{Vec<u8>}的包装,并在模块里隐藏了一些强迫它的内容需要满足的规则。一个这样的类型被称为\emph{newtype},它是Rust中非常普遍的一种模式。Rust自己的\texttt{String}类型就使用完全相同的方式定义的,区别只有它的内容被限制为UTF-8,而不是ASCII。事实上,这是标准库里\texttt{String}的定义:
\begin{minted}{Rust}
    pub struct String {
        vec: Vec<u8>,
    }
\end{minted}

在机器语言的层面上,是完全没有Rust的类型信息的,一个newtype和它的元素有完全相同的内存表示,因此构建一个newtype完全不需要任何额外的机器指令。在\texttt{Ascii::from\_bytes}中,表达式\texttt{Ascii(bytes)}只是表明\texttt{Vec<u8>}现在的内存表示持有的是一个\texttt{Ascii}值。类似的,\texttt{String::from\_utf8\_unchecked}在内联的情况下可能不包含任何机器指令:它只表明\texttt{Vec<u8>}现在被认为是一个\texttt{String}。

\section{unsafe函数}
\texttt{unsafe}函数的定义就像一个以\texttt{unsafe}开头的普通函数。\texttt{unsafe}函数的函数体自动被认为是一个\texttt{unsafe}块。

你只能在\texttt{unsafe}块里调用\texttt{unsafe}函数。这意味着将一个函数标记为\texttt{unsafe}可以警告调用者这个函数有一个额外的合约,必须满足这个合约才能避免未定义行为。

例如,这里有一个新的\texttt{Ascii}类的构造器,这个构造器从一个字节vector构建一个\texttt{Ascii},并且不检查内容是否是有效的ASCII:
\begin{minted}{Rust}
    // 这段代码必须放在`my_ascii`模块中。
    impl Ascii {
        /// 从`bytes`构建一个`Ascii`值,不检查`bytes`是否是有效的ASCII文本。
        ///
        /// 这个函数直接返回一个`Ascii`,而不是像`from_bytes`一样返回一个
        /// `Result<Ascii, NotAsciiError>`。
        ///
        /// # 安全性
        ///
        /// 调用者必须确保`bytes`只包含ASCII字符:每个字节都不大于0x7f。
        /// 否则,最后的结果是未定义的。
        pub unsafe fn from_bytes_unchecked(bytes: Vec<u8>) -> Ascii {
            Ascii(bytes)
        }
    }
\end{minted}

如果调用\texttt{Ascii::from\_bytes\_unchecked}的代码总是知道vector中只包含有效的ASCII字符,那么\texttt{Ascii::from\_bytes}里的检查就只是在浪费时间,并且调用者还必须处理永远不会出现的\texttt{Err}结果。\texttt{Ascii::from\_bytes}可以简化这种情况下的调用和错误处理。

但之前我们曾经强调过\texttt{Ascii}的public构造器和方法保证\texttt{Ascii}的值是有效的的重要性。\texttt{from\_bytes\_unchecked}是不是没有遵守这个规则?

不完全是:\texttt{from\_bytes\_unchecked}把它的责任通过它的合约交给了调用者。这个合约的存在正是它应该被标记为\texttt{unsafe}的原因:虽然这个函数本身没有进行unsafe的操作,但它的调用者必须遵守一些Rust不能强制的规则才能避免未定义行为。

你真的能通过打破\texttt{Ascii::from\_bytes\_unchecked}的合约来导致未定义行为吗?是的。你可以像下面这样构造一个无效的\texttt{String}:
\begin{minted}{Rust}
    // 想象这个vector是一些我们认为会产生ASCII文本的操作的结果,
    // 但这个操作出错了。
    let bytes = vec![0xf7, 0xbf, 0xbf, 0xbf];
    let ascii = unsafe {
        // 当`bytes`含有非ASCII值时这个unsafe的合约就被打破了
        Ascii::from_bytes_unchecked(bytes)
    };

    let bogus: String = ascii.into();

    // `bogus` 现在包含无效的UTF-8。
    // 解析它的第一个字符会产生一个无效的Unicode码点的`char`,
    // 这是未定义行为,因此Rust不知道这个断言的行为会是什么样的。
    assert_eq!(bogus.chars().next().unwrap() as u32, 0x1fffff);
\end{minted}

在特定版本的Rust和特定的平台上,这个断言会输出下面的错误信息并失败:
\begin{minted}{text}
    thread 'main' panicked at 'assertion failed: `(left == right)`
      left: `2097151`
     right: `2097151`, src/main.rs:42:5
\end{minted}

这两个数字在我们看来似乎是相等的,但这不是Rust的问题;这是之前的\texttt{unsafe}块的问题。当我们说未定义行为会导致无法预料的结果时,这就是其中一种情况。

这个例子展示了两个有关bug和unsafe代码的关键事实:
\begin{enumerate}
    \item \emph{\texttt{unsafe}块之前发生的bug可能会打破合约}。一个\texttt{unsafe}块是否会导致未定义行为可能不仅仅取决于这个块本身,还取决于提供它要操作的值的代码。你的\texttt{unsafe}代码依赖的任何东西都是和安全相关的。只有当模块的其他部分正确的维护了\texttt{Ascii}相关的内容时,基于\texttt{String::from\_utf\_unchecked}的\texttt{Ascii}到\texttt{String}的转换才是安全的。
    \item \emph{打破合约的结果可能在你离开\texttt{unsafe}块之后才会出现}。不遵守unsafe特性而导致的未定义行为通常不会在\texttt{unsafe}块本身里出现。如上面所示,构造一个bogus \texttt{String}可能不会有问题,直到程序的程序执行中才出现问题。
\end{enumerate}

本质上讲,Rsut的类型检查、借用检查和其他的静态检查都是在分析你的程序并尝试证明它不可能会出现未定义行为。当Rust成功编译你的程序时,这意味着它成功地证明了这一点。一个\texttt{unsafe}块是这个证明中的例外:等于你在告诉Rust“它没有问题,相信我”。你的声明是否正确可能依赖程序的任何会影响到\texttt{unsafe}块的部分,并且出错时产生的结果也可能出现在任何被\texttt{unsafe}块影响的地方。\texttt{unsafe}关键字也是在提醒你,你无法完全享受到它的安全检查的好处。

如果可以选择的话,你应该尽量选择使用安全的没有合约的接口。它们更容易使用,因为用户可以依赖Rust的安全检查来保证他们的代码不可能出现未定义行为。即使你的实现使用了unsafe特性,最好使用Rust的类型、生命周期和模块系统来满足它们的合约,同事只使用你自己就可以保证的,而不是把责任传递给调用者。

不幸的是,在实际编程中遇到懒得解释它们的合约的unsafe函数并不罕见。他们期望你能依靠自己的经验和知识自己推导出这些规则。

\section{unsafe块还是unsafe函数?}
你可能会想知道是使用\texttt{unsafe}块还是直接把整个函数标记为unsafe。我们推荐的方法是首先判断函数:
\begin{enumerate}
    \item 如果这个函数可能被误用,可以成功编译但可能导致未定义行为,那么你应该将它标记为unsafe。正确使用这个函数的规则就是它的合约,也正是合约的存在让它变得unsafe。
    \item 否则,这个函数是safe的:没有调用能让它产生未定义行为。它不应该被标记为\texttt{unsafe}。
\end{enumerate}

这个函数在函数体里是否使用unsafe特性并不这个重要,关键是合约的存在。之前我们展示过一个没有使用unsafe特性的unsafe函数,也展示过一个使用了unsafe特性的safe函数。

不要只因为你在函数体里使用了unsafe特性就把safe的函数标记为\texttt{unsafe}。这只会让函数更难用,并且迷惑调用者,让他以为这里有一个合约。正确的做法是使用一个\texttt{unsafe}块,即使这个块就是整个函数体。

\section{未定义行为}
在引言中,我们说过术语\emph{未定义行为}意思是“Rust假设你的代码绝对不会出现的行为”。这是一个奇怪的说法,尤其是我们通过其他语言积累的经验告诉我们这些行为\emph{确实}会偶然出现。为什么这个概念有助于规定unsafe代码的义务?

编译器是从一种编程语言到另一种语言的转换器。Rust编译器接收一个Rust程序并把它翻译成等价的机器语言程序。但两个差别大的语言,我们说它们等价到底是什么意思?

幸运的是,相比于语言学家,对程序员来说这个问题简单的多。如果两个程序执行时总是有相同的可见的行为,那么我们说这两个程序是等价的:它们进行相同的系统调用、以等价的方式和外部函数交互等等。这有点像程序的图灵测试:如果你不能分辨出你是在和原始的程序交互还是和翻译后的程序交互,那么它们就是等价的。

现在考虑下面的代码:
\begin{minted}{Rust}
    let i = 10;
    very_trustworthy(&i);
    println!("{}", i * 100);
\end{minted}

即使我们完全不知道\texttt{very\_trustworthy}的定义,我们可以看到它只接收一个\texttt{i}的共享引用,因此这个调用不可能改变\texttt{i}的值。因此传递给\texttt{println!}的值将总是\texttt{1000},Rust可以把这段代码翻译成机器语言,就好像我们写的是:
\begin{minted}{Rust}
    very_trustworthy(&10);
    println!("{}", 1000);
\end{minted}

这个转换后的版本和原本的有相同的可见的行为,而且它可能还要更快一点。但只有在我们认同它真的和原始的版本相同的时候考虑它的性能才有意义。如果\texttt{very\_trustworthy}被定义成这样呢?
\begin{minted}{Rust}
    fn very_trustworthy(shared: &i32) {
        unsafe {
            // 把这个共享引用转换成一个可变的指针。
            // 这是未定义行为。
            let mutable = shared as *const i32 as *mut i32;
            *mutable = 20;
        }
    }
\end{minted}

这段代码打破了共享引用的规则:它把\texttt{i}的值改成了\texttt{20},但\texttt{i}是以共享的方式借用的。因此,现在对这个函数的调用者进行转换会产生非常明显的效果:如果Rust转换了这段代码,程序会打印出\texttt{1000};如果它保留了这段代码并使用\texttt{i}的新值,它会打印出\texttt{2000}。在\texttt{very\_trustworthy}中打破共享引用的规则意味着共享引用的行为并不会如调用者所预期。

这类问题出现在几乎每种Rust会尝试进行的转换中。包括把一个函数内联到调用者中、当调用结束后控制流返回到调用处,等等。但是我们以一个打破了这种假设的的例子来开始这一章。

对Rust(或其他任何语言)来说基本不可能判断对程序的转换是否能保持它的含义,除非它可以信任语言的基础特性的行为和预期一样。它们是否会进行这种转换不仅依赖于眼下的代码,还可能依赖潜在的很远之外的代码。为了对你的代码做一点改动,Rust必须假设程序的其他部分的行为都是正常的。

然后这里是Rust对行为正确程序的规则:
\begin{enumerate}
    \item 程序绝对不能读取未初始化的内存。
    \item 程序绝对不能创建无效的基础值:
    \begin{enumerate}
        \item 引用、box或函数指针为\texttt{null}
        \item 既不是\texttt{0}也不是\texttt{1}的\texttt{bool}值
        \item 判断值无效的\texttt{enum}
        \item 无效的\texttt{char}值,非Unicode码点
        \item 内容不是有效的UTF-8的\texttt{str}值
        \item 虚表或者切片长度无效的胖指针
        \item \texttt{!}类型的任何值
    \end{enumerate}
    \item \autoref{ch05}中介绍的引用的规则必须要遵守。不能有引用的生命周期比引用的对象更长;共享的访问是只读访问,可变的访问是独占的访问。
    \item 程序绝对不能解引用空的、错误对齐的、或悬垂的指针。
    \item 程序绝对不能用一个指针去访问超出这个指针关联的对象的内存范围之外的位置。我们将在\nameref{DerefRawP}中详细解释这个规则。
    \item 程序必须没有数据竞争。数据竞争发生在两个线程在未同步的情况下访问相同的内存位置,并且其中至少有一个访问是写入访问。
    \item 程序绝对不能在一个其他语言通过外部函数接口所进行的调用中进行栈展开,正如在\nameref{unwind}中解释的一样。
    \item 程序必须遵守标准库函数的合约。
\end{enumerate}

由于我们还没有Rust的\texttt{unsafe}语义的完整模型,这个列表可能会随着时间的推移而改变,但这些内容很可能仍然是禁止的。

任何违反这些规则的行为都可能构成未定义行为,还会阻止Rust优化你的程序并把它们转换成机器语言。如果你打破了最后一个规则把无效的UTF-8传递给\texttt{String::from\_utf8\_unchecked},那么之后可能2097151不等于2097151。

不使用unsafe特性的Rust代码只要能编译就能被保证遵守上述所有规则(假设编译器没有bug,我们正在逐渐靠近这个目标,但曲线和渐近线永远不会相交)。只有当你使用unsafe特性时,这些规则才会变成你自己的责任。

在C和C++中,即使你的程序没有报错成功通过了编译也意义不大;正如我们在这本书的引言中解释的,即使是用那些保持高标准代码的广受欢迎的库编写的最好的C和C++程序在实践中也会出现未定义行为。

\section{unsafe trait}\label{UnsafeTrait}
\emph{\texttt{unsafe} trait}是一种特殊的trait,它们有一个Rust无法检查或者强制实现必须遵守的规则,实现必须遵守这些规则才能避免未定义行为。为了实现一个unsafe trait,你必须将实现标记为unsafe的。理解trait的合约并确保你的实现满足合约是你的责任。

一个用unsafe trait来约束类型变量的函数通常自身也会使用unsafe特性,并且它们只依赖这些unsafe trait的合约来满足自己的合约。一个错误的trait实现可能会导致这样的函数出现未定义行为。

\texttt{std::marker::Send}和\texttt{std::marker::Sync}是unsafe trait的典型例子。这些trait并没有定义任何方法,因此可以很容易地为任何类型实现它们。但它们确实有合约:\texttt{Send}要求实现者可以安全地移动到另一个线程中,\texttt{Sync}要求实现者必须能安全地通过共享引用在线程间共享。为一个不恰当的类型实现\texttt{Send}将会使\texttt{std::sync::Mutex}不能再保证没有数据竞争。

这里有个简单的例子,Rust标准库曾经包含了一个叫\texttt{core::nonzero::Zeroable}的unsafe trait,它用来表示那些可以通过把所有字节置为0来安全地初始化的类型。举个例子,把一个\texttt{usize}置0是可以的,但把一个\texttt{\&T}置0会产生空引用,如果解引用就会崩溃。对于那些实现了\texttt{Zeroable}的类型,有一些可行的优化:你可以使用\texttt{std::ptr::write\_bytes}(\texttt{memset}在Rust中的等价函数)或者一个分配置0内存页的系统调用来快速地初始化它们的数组。(\texttt{Zeroable}是unstable的,并且在Rust 1.26中被移到只在\texttt{num} crate中内部使用,但它是一个好的、简单的、真实的例子。)

\texttt{Zeroable}是一个类型标记trait,没有方法或者关联类型:
\begin{minted}{Rust}
    pub unsafe trait Zeroable {}
\end{minted}

为恰当的类型实现这个trait非常的直观:
\begin{minted}{Rust}
    unsafe impl Zeroable for u8 {}
    unsafe impl Zeroable for i32 {}
    unsafe impl Zeroable for usize {}
    // 其他的整数类型同理
\end{minted}

有了这些定义,我们可以编写一个函数,它可以快速地分配一个给定长度的\texttt{Zeroable}类型的vector:
\begin{minted}{Rust}
    use core::nonzero::Zeroable;

    fn zeroed_vector<T>(len: usize) -> Vec<T>
        where T: Zeroable
    {
        let mut vec = Vec::with_capacity(len);
        unsafe {
            std::ptr::write_bytes(vec.as_mut_ptr(), 0, len);
            vec.set_len(len);
        }
        vec
    }
\end{minted}

这个函数首先用给定的容量创建一个空的\texttt{Vec},然后调用\texttt{write\_bytes}用0来填充未初始化的缓冲区。(\texttt{write\_bytes}函数把\texttt{len}看做\texttt{T}类型元素的数量,而不是字节的数量,因此这个调用确实填充了整个缓冲区。)vector的\texttt{set\_len}方法只修改它的长度,不对缓冲区进行任何操作;这是unsafe的,因为你必须保证新的缓冲区空间内都包含正确初始化的\texttt{T}类型的值。但这正是\texttt{T: Zeroable}约束的:一个0字节的块代表一个有效的\texttt{T}值。我们对\texttt{set\_len}的使用是安全的。

这里,我们来使用它:
\begin{minted}{Rust}
    let v: Vec<usize> = zeroed_vector(100_000);
    assert!(v.iter().all(|&u| u == 0));
\end{minted}

显然\texttt{Zeroable}必须是一个unsafe的trait,因为一个不遵守合约的实现可能导致未定义行为:
\begin{minted}{Rust}
    struct HoldsRef<'a>(&'a mut i32);

    unsafe impl<'a> Zeroable for HoldsRef<'a> { }

    let mut v: Vec<HoldsRef> = zeroed_vector(1);
    *v[0].0 = 1;    // 崩溃:解引用空指针
\end{minted}

Rust不知道\texttt{Zeroable}表示什么,所以它不能分辨出哪些类型的实现是不恰当的。和其他的unsafe特性一样,理解并遵守unsafe trait的合约是你的责任。

注意unsafe代码绝对不能依赖正确实现的普通的safe trait。例如,假设有一个\texttt{std::hash::Hasher} trait的实现简单地返回一个随机的哈希值,这个值和被哈希的值没有一点关系。这个trait要求同样的值两次被哈希时必须产生相同的哈希值,但这个现实并不满足这个要求;它很显然是错误的。但因为\texttt{Hasher}并不是unsafe的trait,unsafe代码在使用这个哈稀器的时候不应该出现未定义行为。\texttt{std::collections::HashMap}类型是被精心编写的,它遵守所有用到的unsafe特性的合约,不管哈稀器的行为是什么样的。具体的来说,即使哈希表不能正确地工作:查找可能失败,表项可能随机出现或者消,整个表也不会出现未定义行为。

\section{原始指针}\label{rawp}
Rust中\emph{原始指针}指的是没有约束的指针。你可以使用原始指针来组织Rust的普通指针类型无法做到的数据结构,例如双向链表或者任意的图对象。但因为原始指针太过灵活,Rust无法分辨出你是否正在安全地使用它们,因此你只能在\texttt{unsafe}块中解引用它们。

原始指针基本等价于C或C++中的指针,因此在和这些语言编写的代码交互时原始指针非常有用。

有两种原始指针:
\begin{enumerate}
    \item \texttt{*mut T}是可以修改引用对象的指针。
    \item \texttt{*const T}是只能读取引用对象的指针。
\end{enumerate}
(没有\texttt{*T}类型,你必须指定\texttt{const}或者\texttt{mut}。)

你可以通过转换引用来创建原始指针,并使用\texttt{*}操作符来解引用它:
\begin{minted}{Rust}
    let mut x = 10;
    let ptr_x = &mut x as *mut i32;

    let y = Box::new(20);
    let ptr_y = &*y as *const i32;

    unsafe {
        *ptr_x += *ptr_y;
    }
    assert_eq!(x, 30);
\end{minted}

和box指针以及引用不同,原始指针可以为null,类似C中的\texttt{NULL}和C++中的\texttt{nullptr}:
\begin{minted}{Rust}
    fn option_to_raw<T>(opt: Option<&T>) -> *const T {
        match opt {
            None => std::ptr::null(),
            Some(r) => r as *const T
        }
    }

    assert!(!option_to_raw(Some(&("pea", "pod"))).is_null());
    assert_eq!(option_to_raw::<i32>(None), std::ptr::null());
\end{minted}

这个例子中没有\texttt{unsafe}块:创建、传递、比较原始指针都是safe的。只有解引用原始指针才是unsafe的。

unsized类型的原始指针是胖指针,就像相应的引用或\texttt{Box}指针一样。一个\texttt{*const [u8]}的指针除了地址之外还包括长度,一个trait对象的原始指针例如\texttt{*mut dyn std::io::Write}指针还附带一个虚表。

尽管Rust在很多场景可以隐式解引用safe的指针类型,但原始指针的解引用必须是显式的:
\begin{enumerate}
    \item \texttt{.}运算符不会隐式解引用原始指针,你必须用\texttt{(*raw).field}或者\texttt{(*raw).method(...)}。
    \item 原始指针并没有实现\texttt{Deref},因此强制解引用并不适用于它们。
    \item \texttt{==}和\texttt{<}之类的运算符以地址比较原始指针:只有两个原始指针指向同一个内存位置它们才是相等的。类似,哈希一个原始指针会对它指向的地址进行哈希,而不是对它指向的对象的值进行哈希。
    \item 格式化trait例如\texttt{std::fmt::Dispaly}会自动解引用,但无法处理原始指针。例外的是\texttt{std::fmt::Debug}和\texttt{std::fmt::Pointer},它们会以16进制地址的形式显示原始指针,不会解引用它们。
\end{enumerate}

和C/C++中的\texttt{+}运算符不同,Rust的\texttt{+}运算符不能用于原始指针,但你可以使用原始指针的\texttt{offset}、\texttt{wrapping\_offset}或者更方便的\texttt{add}、\texttt{sub}、\texttt{wrapping\_add}、\texttt{wrapping\_sub}方法对它们进行算数操作。\texttt{offset\_from}方法可以给出两个指针之间的距离,不过我们必须确保起点和终点在相同的内存区域(例如在同一个\texttt{Vec})里:
\begin{minted}{Rust}
    let trucks = vec!["grabage truck", "dump truck", "moonstruck"];
    let first: *const &str = &trucks[0];
    let last: *const &str = &trucks[2];
    assert_eq!(unsafe { last.offset_from(first) }, 2);
    assert_eq!(unsafe { first.offset_from(last) }, -2);
\end{minted}

\texttt{first}和\texttt{last}不需要隐式转换,只要指明类型就够了。Rust隐式地把引用强制转换为原始指针(当然反过来不行)。

\texttt{as}运算符允许几乎把任何引用转换成原始指针或者转换两个原始指针类型。然而,你必须把一个复杂的转换拆分成一系列简单的转换。例如:
\begin{minted}{Rust}
    &vec![42_u8] as *const String;  // 错误:无效转换
    &vec![42_u8] as *const Vec<u8> as *const String;    // 允许
\end{minted}

注意\texttt{as}不能把原始指针转换为引用。这样的转换是unsafe的,而\texttt{as}应该保证是safe的操作。要想做到这一点,你必须解引用原始指针(在一个\texttt{unsafe}块中)然后借用得到的值的引用。

这么做的时候一定要小心:这种方式产生的引用将会有无限的生命周期:它的生存时间没有任何限制,因为原始指针并没有给Rust提供推断这一点的信息。在后面的\nameref{SafeInter}一节中,我们将展示几个例子来演示如何正确地约束生命周期。

很多类型都有\texttt{as\_ptr}和\texttt{as\_mut\_ptr}方法可以返回它们的内容的原始指针。例如,数组的切片和字符串会返回它们的第一个元素的指针,一些迭代器会返回它们要产生的下一个元素的指针。拥有所有权的指针类型例如\texttt{Box}、\texttt{Rc}和\texttt{Arc}有\texttt{into\_raw}和\texttt{from\_raw}函数可以转换成或转换自原始指针。其中一些方法的合约有一些令人惊讶的要求,因此在使用之前要仔细阅读它们的文档。

你也可以把整数转换成原始指针,尽管你唯一可以信任的整数是从之前的指针得到整数。\nameref{RefWithFlag}以这种方式使用了原始指针。

和引用不同,原始指针既没有实现\texttt{Send}也没有实现\texttt{Sync}。因此,任何包含原始指针的类型默认都没有实现这两个trait。在线程间发送或者共享原始指针本质上并没有什么不安全的,毕竟,不管它们去了哪,在解引用它们的时候仍然需要一个\texttt{unsafe}块。但考虑到原始指针通常扮演的角色,语言的设计者认为默认是这样会更有帮助。我们已经在\nameref{UnsafeTrait}中讨论过如何自己实现\texttt{Send}和\texttt{Sync}了。

\subsection{安全地解引用原始指针}\label{DerefRawP}
这里有一些安全使用原始指针的基本规则:
\begin{enumerate}
    \item 解引用空指针或悬垂指针是未定义行为,指向未初始化内存或超出作用域的值的指针也是如此。
    \item 解引用没有按照类型正确对齐的指针是未定义行为。
    \item 你可以从解引用原始指针获得的值借用引用,不过只有当这么做满足\nameref{ch05}中介绍的引用安全性规则时才可以:引用不能超出被引用对象的生命周期、共享的访问是只读的访问、可变的访问是独占的访问。(这个规则很容易在无意中被违反,因为原始指针通常被用来创建非标准共享或所有权的数据结构。)
    \item 只有当一个原始指针指向的对象是正确的该类型的值时你才能使用它指向的对象。例如,你必须确保解引用一个\texttt{*const char}返回一个正确的Unicode码点。
    \item 在使用原始指针的\texttt{offset}和\texttt{wrapping\_offset}方法时你必须确保最后指向的位置还在一开始指向的那个对象的值或者内存块里。\\ 如果你先把一个指针转换成整数,然后进行任何的算术运算,再把它转换回指针,那么结果必须是\texttt{offset}的规则允许你产生的指针。
    \item 如果你对原始指针指向的对象赋值,你必须保证不违反其中任何一个类型的不变量。例如,如果你有一个指向一个\texttt{String}的字节的\texttt{*mut u8}指针,你对这个\texttt{u8}赋的值必须保证\texttt{String}持有的仍是有效的UTF-8。
\end{enumerate}

除了借用规则之外,这些都是在C和C++中使用指针时必须要遵守的基本规则。

不能违背类型的不变量的原因应该很清楚。很多Rust的标准类型在实现里都使用了unsafe代码,但仍然提供了safe的接口。它们假设Rust的安全检查、模块系统和可见性规则都没有被违反。使用原始指针来绕开这些保护措施可能会导致未定义行为。

完整又精确的原始指针的合约很难简单地说清楚,也可能会随着语言的改进发生改变。但这里列出的原则应该能保证代码是安全的。

\subsection{示例:\texttt{RefWithFlag}}\label{RefWithFlag}

    \chapter{外部函数}\label{ch23}
\end{document}
