\chapter{枚举与模式}\label{ch10}

\emph{Surprising how much computer stuff makes sense viewed as tragic deprivation of sum types (cf. deprivation of lambdas).}

\begin{flushright}
    ——Graydon Hoare
\end{flushright}

这一章的第一个话题将是一个古老的、强有力的、可以帮你在短期内完成很多工作的(有代价地)、并且在许多语言中以不同的名字广为人知的特性。但它并不是魔鬼。而是一种用户自定义的数据类型,它是ML和Haskell程序员们熟知的和类型、也是互斥的联合、还是代数数据类型。在Rust中,它们被称为\emph{枚举(enumerations)},或者简写为\emph{enum}。和魔鬼不同的是,它们非常安全、索取的代价也很小。

C++和C\#都有枚举,你可以使用它们来定义自己的类型,这种类型的取值范围是一些命名常量的集合。例如,你可能定义过一个叫\texttt{Color}的类型,取值范围为\texttt{Red}、\texttt{Orange}、\texttt{Yellow}等等。这种枚举在Rust中也能工作,但Rust进一步扩展了枚举。一个Rust枚举可以包含数据,包括多种不同类型的数据。例如,Rust的\texttt{Result<String, io::Error}类型是一个枚举;这样一个值要么是一个包含\texttt{String}的\texttt{Ok}值要么是一个包含\texttt{io::Error}的\texttt{Err}值。这就超出了C++和C\#中枚举的能力。它更像C中的\texttt{union}——但和联合不同的是,Rust的枚举是类型安全的。

枚举适用于一个值有多种可能的情况。使用它们的“代价”是你必须使用模式匹配来安全地访问数据,这也是我们这一章中的第二个话题。

如果你使用过Python的解包或者JavaScript中的解构,那你可能觉得模式也很熟悉,但Rust同样扩展了模式。Rust的模式有点像匹配数据的正则表达式。它们被用来测试一个值是否具有特定的期望的形态。它们可以一次从结构体或这元组中提取出多个字段存入局部变量。并且和正则表达式类似,它们很简洁,通常只用单行代码就能完成任务。

这一章将以枚举的基础开始,展示数据怎么被关联到枚举选项以及枚举是怎么存储在内存中的。然后我们会展示Rust的模式和\texttt{match}表达式如何简洁地指定基于枚举、结构体、数组、切片的逻辑。模式也可以包含引用、move和\texttt{if}条件,这让它们的功能更加强大。

\section{枚举}\label{enum}

简单的C风格枚举非常直观:
\begin{minted}{Rust}
    enum Ordering {
        Less,
        Equal,
        Greater,
    }
\end{minted}

这里声明了一个有三个可能的值的\texttt{Ordering}类型,这些值被称为\emph{variant}或者\emph{constructor}:\texttt{Ordering::Less}、\texttt{Ordering::Equal}、\texttt{Ordering::Greater}。这个枚举是标准库的一部分,因此Rust代码可以导入它:
\begin{minted}{Rust}
    use std::cmp::Ordering;

    fn compare(n: i32, m: i32) -> Ordering {
        if n < m {
            Ordering::Less
        } else if n > m {
            Ordering::Greater
        } else {
            Ordering::Equal
        }
    }
\end{minted}

或者它的所有constructor:
\begin{minted}{Rust}
    use std::cmp::Ordering::{self, *};  // `*`意思是导入所有的子item

    fn compare(n: i32, m: i32) -> Ordering {
        if n < m {
            Less
        } else if n > m {
            Greater
        } else {
            Equal
        }
    }
\end{minted}

导入constructor之后,我们可以写\texttt{Less}来代替\texttt{Ordering::Less}等,但因为这样不够明显,因此一般认为\emph{不要}导入它们式更好的风格,除非它能是你的代码的可读性更强。

为了导入一个在当前模块中声明的枚举的constructor,可以使用\texttt{self}:
\begin{minted}{Rust}
    enum Pet {
        Orca,
        Giraffe,
        ...
    }

    use self::Pet::*;
\end{minted}

在内存中,C风格的枚举值被存储为整数。有时告诉Rust使用哪些整数会很有用:
\begin{minted}{Rust}
    enum HttpStatus {
        Ok = 200,
        NotModified = 304,
        NotFound = 404,
        ...
    }
\end{minted}

否则,Rust会从0开始自动分配值。

默认情况下,Rust用能容纳所有值的最小的内建整数类型来存储C风格枚举。大多数情况下都是一个单独的字节:
\begin{minted}{Rust}
    use std::mem::size_of;
    assert_eq!(size_of::<Ordering>(), 1);
    assert_eq!(size_of::<HttpStatus>(), 2); // 404不能存储在u8中
\end{minted}

你可以通过添加\texttt{\#[repr]}属性来覆盖Rust选择的内存表示方式。更多的细节见“\nameref{repr}”。

将C风格的枚举转换为整数是允许的:
\begin{minted}{Rust}
    assert_eq!(HttpStatus::Ok as i32, 200);
\end{minted}

然而,反过来把整数转换为枚举是不允许的。和C和C++不同,Rust保证枚举的值只能是\texttt{enum}生命中列出的值之一。未经检查的从整数类型到枚举类型的转换会打破这种保证,所以它是不允许的。你可以写出你自己的带检查的版本:
\begin{minted}{Rust}
    fn http_status_from_u32(n: u32) -> Option<HttpStatus> {
        match n {
            200 => Some(HttpStatus::Ok),
            304 => Some(HttpStatus::NotModified),
            404 => Some(HttpStatus::NotFound),
            ...
            _ => None,
        }
    }
\end{minted}

或者使用\href{https://crates.io/crates/enum_primitive}{\texttt{enum\_primitive}} crate。它包含一个宏可以为你自动生成这种类型的转换代码。

和结构体一样,编译器也可以为你自动生成类似\texttt{==}运算符这样的特性,但你需要显式地要求这样:
\begin{minted}{Rust}
    #[derive(Copy, Clone, Debug, PartialEq, Eq)]
    enum TimeUnit {
        Seconds, Minutes, Hours, Days, Months, Years,
    }
\end{minted}

枚举也和结构体一样可以拥有方法:
\begin{minted}{Rust}
    impl TimeUnit {
        /// 返回该时间单位的复数名词。
        fn plural(self) -> &'static str {
            match self {
                TimeUnit::Seconds => "seconds",
                TimeUnit::Minutes => "minutes",
                TimeUnit::Hours => "hours",
                TimeUnit::Days => "days",
                TimeUnit::Months => "months",
                TimeUnit::Years => "years",
            }
        }

        /// 返回该时间单位的单数名词。
        fn singular(self) -> &'static str {
            self.plural().trim_end_matches('s')
        }
    }
\end{minted}

C风格的枚举就这么多内容了。Rust中最有趣的一类枚举是那些带有数据的枚举。我们将展示这些枚举如何存储在内存中、如何通过添加类型参数将它们变为泛型的,以及如何通过枚举构建复杂的数据结构。

\subsection{带有数据的枚举}

一些程序总是需要显示完整的日期和时间,并且精确到毫秒。但对于大多数程序,显示大概的时间范围会更加友好,例如“两个月以前”。我们可以用之前定义的枚举编写一个新的枚举来实现这一点:
\begin{minted}{Rust}
    /// 一个故意舍入的时间戳,因此我们的程序会显示“6个月以前”
    /// 而不是“February 9, 2016, at 9:49 AM”。
    #[derive(Copy, Clone, Debug, PartialEq)]
    enum RoughTime {
        InThePast(TimeUnit, u32),
        JustNow,
        InTheFuture(TimeUnit, u32),
    }
\end{minted}

这个枚举中的两个variant,即\texttt{InThePast}和\texttt{InTheFuture}都有参数。这些被称为\emph{tuple variant}。就像类元组结构体一样,它们的constructor是创建新的\texttt{RoughTime}值的函数:
\begin{minted}{Rust}
    let four_score_and_seven_years_ago =
        RoughTime::InThePast(TimeUnit::Years, 4 * 20 + 7);

    let three_hours_from_now =
        RoughTime::InTheFuture(TimeUnit::Hours, 3);
\end{minted}

枚举也可以有\emph{struct variant},它们和普通的结构体一样拥有命名字段:
\begin{minted}{Rust}
    enum Shape {
        Sphere { center: Point3d, radius: f32 },
        Cuboid { corner1: Point3d, corner2: Point3d },
    }

    let unit_sphere = Shape::Sphere {
        center: ORIGIN,
        radius: 1.0,
    };
\end{minted}

总的来说,Rust有三种枚举variant,分别对应我们在上一章中展示的三种结构体。没有数据的variant对应类单元结构体。元组variant对应类元组结构体。结构体variant对应有花括号和命名字段的结构体。一个枚举可以同时有这三种variant:
\begin{minted}{Rust}
    enum RelationshipStatus {
        Single,
        InARelationship,
        ItsComplicated(Option<String>),
        ItsExtremelyComplicated {
            car: DifferentialEquation, 
            cdr: EarlyModernistPoem,
        },
    }
\end{minted}

所有种类的constructor都和枚举自身有相同的可见性。

\subsection{内存中的枚举}

在内存中,带有数据的枚举被存储为一个很小的整数\emph{标签(tag)},加上一块足够存储所有variant中最大的那个的内存。标签字段是Rust内部要使用的,它表示是哪一个constructor创建了这个值,进而得知这个值有哪些字段。

在Rust 1.50中,\texttt{RoughTime}存储为8个字节,如\hyperref[f10-1]{图10-1}所示。

\begin{figure}[htbp]
    \centering
    \includegraphics[width=0.9\textwidth]{../img/f10-1.png}
    \caption{内存中的\texttt{RoughTime}值}
    \label{f10-1}
\end{figure}

对于枚举的布局Rust不做任何保证。然而,为了给将来的优化留下余地,在一些情况下它可能会用比图中所示更加高效的方式包装一个枚举。例如,一些泛型结构体可以不用标签存储,我们稍后会讲到它。

\subsection{使用枚举实现富数据结构}

枚举在实现树形结构时也很有用。例如,假设一个Rust程序要处理任意的Json数据。在内存中,任何Json文档都可以被表示为一个这种Rust类型的值:
\begin{minted}{Rust}
    use std::collections::HashMap;

    enum Json {
        Null,
        Boolean(bool),
        Number(f64),
        String(String),
        Array(Vec<Json>),
        Object(Box<HashMap<String, Json>>),
    }
\end{minted}

与Rust代码相比,用英文来解释这个数据结构也不会再有太大的改进了。JSON标准定义了可以出现在JSON文档中的数据类型:\texttt{null}、布尔值、数字、字符串、JSON值的数组、以及带有字符串键和JSON值的对象。这个\texttt{Json}枚举简单地列出了这些类型。

这并不是一个假想的例子。你可以在\texttt{serde\_json} crate中找到一个非常相似的枚举,它是一个用于Rust结构体序列化的库,也是crates.io上下载次数最多的crate之一。

用于表示\texttt{Object}的\texttt{HashMap}外层的\texttt{Box}只是为了让\texttt{Json}值更加紧凑。在内存中,\texttt{Json}类型的值将占据4个机器字。\texttt{String}和\texttt{Vec}都是3个字,Rust会再添加一个字节的标签,再加上对齐所以总共是4个字。\texttt{Null}和\texttt{Boolean}值没有足够的数据利用全部的空间,但所有的\texttt{Json}值大小必须相同,因此这时多余的空间就被浪费了。\hyperref[f10-2]{图10-2}展示了一些示例的\texttt{Json}值在内存中的实际视图。

\begin{figure}[htbp]
    \centering
    \includegraphics[width=0.8\textwidth]{../img/f10-2.png}
    \caption{内存中的\texttt{Json}值}
    \label{f10-2}
\end{figure}

一个\texttt{HashMap}会更大如果我们一定要在每一个\texttt{Json}值中给它留出空间,它们将会变得更大,也就是8个字。但\texttt{Box<HashMap>}是单个字:它只是一个指向堆上分配的数据的指针。我们甚至可以通过装箱更多的字段来让\texttt{Json}变得更加紧凑。

这里优秀的地方在于,我们如此简单的就完成了这一切。如果是在C++中,可能要写一个这样的一个类才行:
\begin{minted}{Rust}
    class JSON {
    private:
        enum Tag {
            Null, Boolean, Number, String, Array, Object
        };
        union Data {
            bool boolean;
            double number;
            shared_ptr<string> str;
            shared_ptr<vector<JSON>> array;
            shared_ptr<unordered_map<string, JSON>> object;

            Data() {}
            ~Data() {}
            ...
        };
        
        Tag tag;
        Data data;
    
    public:
        bool is_null() const { return tag == Null; }
        bool is_boolean const { return tag == Boolean; }
        bool get_boolean() const {
            assert(is_boolean());
            return data.boolean;
        }
        void set_boolean(bool value) {
            this->~JSON();  // 清除string/array/object值
            tag = Boolean;
            data.boolean = value;
        }
        ...
    };
\end{minted}

30行代码,我们才刚刚开始。这个类还需要构造函数、析构函数、一个赋值运算符。另一种方案是通过继承,首先创建一个基类\texttt{JSON}和它的子类\texttt{JSONBoolean}、\texttt{JSONString}等等。无论哪种方式,等到完成之后,我们的C++ JSON库都要有一堆代码了。其他程序员需要花费不少精力来阅读和使用它。而Rust的整个枚举只需要8行代码。

\subsection{泛型枚举}
枚举可以是泛型的。标准库的两个例子几乎是整个语言中使用最广泛的数据类型:
\begin{minted}{Rust}
    enum Option<T> {
        None,
        Some(T),
    }

    enum Result<T, E> {
        Ok(T),
        Err(E),
    }
\end{minted}

到现在这些类型你应该已经很熟悉了,泛型枚举的语法和泛型结构体完全相同。

一个不明显的细节是当类型\texttt{T}是引用、\texttt{Box}或其他智能指针类型时Rust可以省略\texttt{Option<T>}的标签字段。因为这些指针类型中的任何一个都不允许为0,所以Rust可以用单个机器字来表示\texttt{Option<Box<i32>>}:用0表示\texttt{None},用非0表示\texttt{Some}指针。这使得这样的\texttt{Option}类型与C和C++中可以为空的指针值非常相似。不同之处在于Rust的类型系统要求你必须先检查\texttt{Option}的值是\texttt{Some},然后才能使用它内含的值。这有效的避免了空指针解引用。

泛型数据结构体可以用很少的几行代码构建:
\begin{minted}{Rust}
    // 一个`T`类型的有序集合
    enum BinaryTree<T> {
        Empty,
        NonEmpty(Box<TreeNode<T>>),
    }

    // 二叉树的一部分
    struct TreeNode<T> {
        element: T,
        left: BinaryTree<T>,
        right: BinaryTree<T>,
    }
\end{minted}

这几行代码定义了一个可以存储任意数量的\texttt{T}类型值的\texttt{BinaryTree}类型。

这两个定义包含了大量信息,所以我们将花费一些时间把代码翻译为中文。每一个\texttt{BinaryTree}值是\texttt{Empty}或者\texttt{NonEmpty}。如果它是\texttt{Empty},那么它不包含任何数据。如果是\texttt{NonEmpty},那么它会包含一个\texttt{Box},这个指针指向一个在堆上分配的\texttt{TreeNode}值。

每一个\texttt{TreeNode}值包含一个实际的元素,和两个\texttt{BinaryTree}值。这意味着一棵树可以包含子树,因此一个\texttt{NonEmpty}树可以包含任意数量的后台节点。

一个\texttt{BinaryTree<\&str>}类型的值的视图如\hyperref[f10-3]{图10-3}所示。因为对于\texttt{Option<Box<T>>},Rust会省略标签字段,所以一个\texttt{BinaryTree}值只占一个机器字。

\begin{figure}[htbp]
    \centering
    \includegraphics[width=0.9\textwidth]{../img/f10-3.png}
    \caption{一个包含6个字符串的\texttt{BinaryTree}}
    \label{f10-3}
\end{figure}

构建这棵树中的节点非常直观:
\begin{minted}{Rust}
    use self::BinaryTree::*;
    let jupiter_tree = NonEmpty(Box::new(TreeNode {
        element: "Jupiter",
        left: Empty,
        right: Empty,
    }));
\end{minted}

更大的树可以通过较小的树构建:
\begin{minted}{Rust}
    let mars_tree = NonEmpty(BOx::new(TreeNode {
        element: "Mars",
        left: jupiter_tree,
        right: mercury_tree,
    }));
\end{minted}

自然地,这个赋值会把\texttt{jupiter\_node}和\texttt{mercury\_node}的所有权移动到新的父节点里。

树的其他部分遵循相同的模式。根节点和其它节点不同:
\begin{minted}{Rust}
    let tree = NonEmpty(Box::new(TreeNode {
        element: "Saturn",
        left: mars_tree,
        right: uranus_tree,
    }));
\end{minted}

在这一章的后续部分中,我们将介绍怎么在\texttt{BinaryTree}类型上实现一个\texttt{add}方法,这样我们就可以这样写:
\begin{minted}{Rust}
    let mut tree = BinaryTree::Empty;
    for planet in planets {
        tree.add(planet);
    }
\end{minted}

无论你之前用什么语言,在Rust中创建像\texttt{BinaryTree}这样的数据结构都需要一些练习。一开始把\texttt{Box}放在哪可能并不明显。一种寻找设计的方法是画一幅像\hyperref[f10-3]{图10-3}这样的内存布局图。然后根据图设计代码:每一个矩形都是一个结构体或者元组,每一个箭头都是一个\texttt{Box}或者其他智能指针。搞清楚每个字段的类型有点困难,但解决难题的回报是控制程序的内存使用。 

现在就到了我们在本章开始时提到的“代价”。枚举的标签字段要占用很小的内存,最糟的情况下要占用8个字节,但这种情况通常非常少见。枚举真正的缺点(如果它能被称为缺点的话)是Rust不能忽略安全性、不管当前的值是什么直接尝试访问字段:
\begin{minted}{Rust}
    let r = shape.radius;   // 错误:`Shape`类型没有字段`radius`
\end{minted}

访问枚举中的值的唯一方式是:使用枚举,这是一种安全的方式。

\section{模式}

回顾一下我们在本章中定义过的\texttt{RoughTime}:
\begin{minted}{Rust}
    enum RoughTime {
        InThePast(TimeUnit, u32),
        JustNow,
        InTheFuture(TimeUnit, u32),
    }
\end{minted}

假设你有一个\texttt{RoughTime}值并且你想在网页中显示它。你需要访问值里的\texttt{TimeUnit}和\texttt{u32}字段。Rust不允许你直接通过\texttt{rough\_time.0}和\texttt{rough\_time.1}访问它们,因为毕竟此时值也可能是\texttt{RoughTime::JustNow},而它没有字段。那么,你怎么获取数据呢?

你需要一个\texttt{match}表达式:
\begin{minted}[linenos,numbersep=-1em]{Rust}
    fn rough_time_to_english(rt: RoughTime) -> String {
        match rt {
            RoughTime::InThePast(units, count) =>
                format!("{} {} ago", count, units.plural()),
            RoughTime::JustNow =>
                format!("just now"),
            RoughTime::InTheFuture(units, count) =>
                format!("{} {} from now", count, units.plural()),
        }
    }
\end{minted}
\texttt{match}会进行模式匹配。在这个例子中,\emph{模式}是第3、5、7行中出现在\texttt{=>}符号左边的部分。匹配\texttt{RoughTime}值的模式看起来就像是一个创建\texttt{RoughTime}值的表达式。这并不是巧合。表达式\emph{产生}值,模式\emph{消耗}值。它们使用相同的语法。

让我们逐步看看运行这个\emph{match}表达式时发生了什么。假设\texttt{rt}的值是\texttt{RoughTime::InTheFuture(TimeUnit::Months, 1)}。Rust首先尝试将这个值和第3行的模式匹配。正如\hyperref[f10-4]{图10-4}所示,它并不能匹配。

\begin{figure}[htbp]
    \centering
    \includegraphics[width=0.8\textwidth]{../img/f10-4.png}
    \caption{一个\texttt{RoughTime}值和不匹配的模式}
    \label{f10-4}
\end{figure}

Rust中用于匹配一个枚举、结构体或者元组的模式的工作原理就好像简单地从左到右扫描,检查模式中的每个部分来看看是不是和值匹配。如果不是,Rust会移动到下一个模式。

第3和第5行的模式都匹配失败。但第7行的模式成功了(\hyperref[f10-5]{图10-5})。

\begin{figure}[htbp]
    \centering
    \includegraphics[width=0.8\textwidth]{../img/f10-5.png}
    \caption{一个成功的匹配}
    \label{f10-5}
\end{figure}

当一个模式包含像\texttt{units}和\texttt{count}这样的简单标识符时,在匹配之后的代码中它们会变为局部变量。当前值里的任何内容都会被拷贝或移动到新的变量中。Rust把\texttt{TimeUnit::Months}存储在\texttt{units}中,把\texttt{1}存储在\texttt{count}中,然后运行第8行的代码,最后返回字符串\texttt{"1 months from now"}。

这个输出有一点语法上的错误,可以通过给\texttt{match}添加另一个分支来修正:
\begin{minted}{Rust}
    RoughTime::InTheFuture(unit, 1) =>
        format!("a {} from now", unit.singular()),
\end{minted}

只有当\texttt{count}字段恰好是1时这个分支才能匹配。注意这一段新代码必须添加到第7行之前。如果我们把它添加在最后,那么执行流永远不会到达它,因为第7行匹配所有的\texttt{InTheFuture}值。如果你犯了这种错误,Rust编译器会给出一个“不可达的模式”警告。

即使有了新的代码,\texttt{RoughTime::InTheFuture(TimeUnit::Hours, 1)}仍然有一个问题:结果\texttt{"a hour from now"}在英语中并不是完全正确。这可以通过给\texttt{match}添加另一个分支来修复。

正如这个例子所示,模式匹配和枚举协同工作,甚至可以测试它们包含的值,这使得\texttt{match}表达式成为C的\texttt{switch}语句的一个更强大、更灵活的替代。

到目前为止,我们只见到了匹配枚举值的模式。其实它还有更多用途。Rust的模式有它们自己的语言,\hyperref[t10-1]{表10-1}中进行了总结。我们将用本章中剩下的大部分内容来展示表中的特性。

\begin{table}[htbp]
    \centering
    \caption{模式}
    \label{t10-1}
    \begin{tabular}{p{0.15\textwidth}p{0.35\textwidth}p{0.4\textwidth}}
        \hline
        \textbf{模式类型} & \textbf{示例} & \textbf{注释} \\
        \hline
        字面量  & \makecell[l]{\texttt{100} \\ \texttt{"name"}} & 匹配一个精确值,也可以使用一个\texttt{const}的值的名称    \\
        \rowcolor{tablecolor}
        范围    & \makecell[l]{\texttt{0 ..= 100} \\ \texttt{'a' ..= 'k'}}  & 匹配范围内的任何值,包括终点值 \\
        通配符  & \texttt{\_}   & 匹配任何值并忽略  \\
        \rowcolor{tablecolor}
        变量    & \makecell[l]{\texttt{name} \\ \texttt{mut count}} & 类似\texttt{\_}但是把值移动或拷贝进新的局部变量   \\
        \texttt{ref}变量    & \makecell[l]{\texttt{ref field} \\ \texttt{ref mut field}}    & 借用匹配的值的引用,而不是移动或拷贝它    \\
        \rowcolor{tablecolor}
        带子模式的绑定  & \makecell[l]{\texttt{val @ 0 ..= 99} \\ \texttt{ref circle @ Shape::Circle \{ .. \}}} & 匹配@右侧的模式,使用左侧作为变量名   \\
        枚举模式    & \makecell[l]{\texttt{Some(value)} \\ \texttt{None} \\ \texttt{Pet::Orca}} & \\
        \rowcolor{tablecolor}
        元组模式    & \makecell[l]{\texttt{(key, value)} \\ \texttt{(r, g, b)}} & \\
        数组模式    & \makecell[l]{\texttt{[a, b, c, d, e, f, g]} \\ \texttt{[heading, carom, correction]}} & \\
        \rowcolor{tablecolor}
        切片模式    & \makecell[l]{\texttt{[first, second]} \\ \texttt{[first, \_, third]} \\ \texttt{[first, .., nth]} \\ \texttt{[]}}  & \\
        结构体模式  & \makecell[l]{\texttt{Color(r, g, b)} \\ \texttt{Point \{ x, y \}} \\ \texttt{Card \{ suit: Clubs, rank: n \}} \\ \texttt{Account \{ id, name, .. \}}} & \\
        \rowcolor{tablecolor}
        引用    & \makecell[l]{\texttt{\&value} \\ \texttt{\&(k, v)}}   & 只匹配引用值 \\
        多重模式    & \texttt{'a' | 'A'}    & 只能用作可反驳的模式(\texttt{match, if let, while let}) \\
        \rowcolor{tablecolor}
        守卫表达式  & \texttt{x if x * x <= r2} & 只能在\texttt{match}中使用(在\texttt{let}等表达式中无效) \\
    \end{tabular}
\end{table}

\subsection{模式中的字面量、变量和通配符}
到目前为止,我们已经展示了\texttt{match}表达式和枚举一起使用,其实其它类型也可以用模式来匹配。当你需要类似C的\texttt{switch}语句的功能时,可以使用处理整数值的\texttt{match}表达式。整数字面量例如\texttt{0}和\texttt{1}可以用作模式:
\begin{minted}{Rust}
    match meadow.count_rabbits() {
        0 => {} // 什么都不输出
        1 => println!("A rabbit is nosing around in the clover."),
        n => println!("There are {} rabbits hopping about in the meadow", n),
    }
\end{minted}

当草地上没有兔子时模式\texttt{0}会匹配,当只有一只时\texttt{1}会匹配。如果有两只或者更多兔子,就会到达第三个模式\texttt{n}。这个模式只有一个变量名。它可以匹配任何值,被匹配的值会被移动或拷贝进新的局部变量。因此在这个例子中,\texttt{meadow.count\_rabbits()}的值被存储在一个新的局部变量\texttt{n}中,然后我们打印出它。

其他的字面量也可以用作模式,包括布尔值、字符、甚至字符串:
\begin{minted}{Rust}
    let calendar = match settings.get_string("calendar") {
        "gregorian" =>  Calendar::Gregorian,
        "chinese" => Calendar::Chinese,
        "ethiopian" => Calendar::Ethiopian,
        other => return parse_error("calendar", other),
    };
\end{minted}

在这个例子中,\texttt{other}和上个例子中的\texttt{n}一样用作匹配任何值的模式。这些模式和\texttt{switch}语句中的\texttt{default}标签一样,用来匹配其他所有模式都匹配不了的值。

如果你需要一个匹配所有值的模式,但又不关心匹配到的值,你可以使用单个下划线\texttt{\_}作为模式,也就是\emph{通配模式}:
\begin{minted}{Rust}
    let caption = match photo.tagged_pet() {
        Pet::Tyrannosaur => "RRRAAAAAHHHHHH",
        Pet::Samoyed => "*dog thoughts*",
        _ => "I'm cute, love me",   // 通用标题,用于任何宠物
    };
\end{minted}

通配模式匹配任何值,但并不存储它。因为Rust要求每一个\texttt{match}表达式要能处理所有可能的值,因此最后通常需要一个通配符。即使你非常确信其他的情况不会发生,你也必须至少添加一个fallback分支,这个分支里可以直接panic:
\begin{minted}{Rust}
    // 有很多形状,但我们只支持“选择”文本或者一个矩形区域。
    // 你不能选择一个椭圆或者梯形。
    match document.selection() {
        Shape::TextSpan(start, end) => paint_text_selection(start, end),
        Shape::Rectangle(rect) => paint_rect_selection(rect),
        _ => panic!("unexpected selection type"),
    }
\end{minted}

\subsection{元组和结构体模式}
元组模式匹配元组。当你想在单个\texttt{match}中获得数据的多个部分时它们会很有用:
\begin{minted}{Rust}
    fn describe_point(x: i32, y: i32) -> &'static str {
        use std::cmp::Ordering::*;
        match (x.cmp(&0), y.cmp(&0)) {
            (Equal, Equal) => "at the origin",
            (_, Equal) => "on the x axis",
            (Equal, _) => "on the y axis",
            (Greater, Greater) => "in the first quadrant",
            (Less, Greater) => "in the second quadrant",
            _ => "somewhere else",
        }
    }
\end{minted}

结构体模式要使用花括号,就和结构体表达式一样。它们可以包含每个字段的子模式:
\begin{minted}{Rust}
    match balloon.location {
        Point { x: 0, y: height } =>
            println!("straight up {} meters", height),
        Point { x: x, y: y } =>
            println!("at ({}m, {}m)", x, y),
    }
\end{minted}

在这个例子中,如果第一个分支匹配了,那么\texttt{balloon.location.y}会被存储到新的局部变量\texttt{height}。

假设\texttt{balloon.location}是\texttt{Point \{ x: 30, y: 40 \}}。和之前一样,Rust会按照\hyperref[f10-6]{图10-6}的顺序检查每一个模式的每一个部分。

\begin{figure}[htbp]
    \centering
    \includegraphics[width=0.8\textwidth]{../img/f10-6.png}
    \caption{结构体模式匹配}
    \label{f10-6}
\end{figure}

第二个分支可以匹配,因此输出将是\texttt{at (30m, 40m)}。

当匹配结构体时类似\texttt{Point \{ x: x, y: y \}}的模式非常常见,多余的名字也只会扰乱视觉,因此Rust为此支持一种缩写形式\texttt{Point \{x, y\}}。含义和之前相同,这个模式也会把点的\texttt{x}字段存储在新的局部变量\texttt{x}、把\texttt{y}字段存储在新的局部变量\texttt{y}。

即使有了缩写形式,如果我们要匹配一个很大的结构体但又只关心少数字段时还是会很麻烦:
\begin{minted}{Rust}
    match get_account(id) {
        ...
        Some(Account {
                name, language, // 我们关心的两个字段
                id: _, status: _, address: _, birthday: _, eye_color: _,
                pet: _, security_question: _, hashed_innermost_secret: _,
                is_adamantium_preferred_customer: _, }) =>
            language.show_custom_greeting(name),
    }
\end{minted}

为了避免这种情况,可以使用\texttt{..}告诉Rust你不关心其他的字段:
\begin{minted}{Rust}
    Some(Account { name, language, .. }) =>
        language.show_custom_greeting(name),
\end{minted}

\subsection{数字和切片模式}
数组模式匹配数组。它们被通常被用来过滤出某些特殊值,当数组的不同位置的含义不同时它们也会变得很有用。

例如,当把色相、饱和度、亮度(HSL)颜色值转换为红绿蓝(RGB)颜色值时,亮度为0的颜色就是黑、而亮度为满的颜色就是白。我们可以使用\texttt{match}表达式来简单地处理这些情况:
\begin{minted}{Rust}
    fn hsl_to_rgb(hsl: [u8; 3]) -> [u8; 3] {
        match hsl {
            [_, _, 0] => [0, 0, 0],
            [_, _, 255] => [255, 255, 255],
            ...
        }
    }
\end{minted}

切片模式与此类似,单核数组不同的是,切片的长度可以变化。因此切片模式并不只匹配值,还要匹配长度。切片模式中的\texttt{..}匹配任意数量的元素:
\begin{minted}{Rust}
    fn greet_people(names: &[&str]) {
        match names {
            [] => { println!("Hello, nobody.") },
            [a] => { println!("Hello, {}.", a) },
            [a, b] => { println!("Hello, {} and {}.", a, b) },
            [a, .., b] => { println!("Hello, everyone from {} to {}.", a, b) }
        }
    }
\end{minted}


