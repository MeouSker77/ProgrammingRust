\chapter{异步编程}\label{ch20}

假设你正在编写一个聊天服务器。每个网络连接都有很多要解析的到来的包、要组装的发送的包、要管理的安全参数、要追踪的聊天组订阅等。同时为许多连接管理所有这些信息需要进行一些组织。

理想情况下,你可以简单地为每一个到来的连接启动一个单独的线程:
\begin{minted}{Rust}
    use std::{net, thread};

    let listener = net::TcpListener::bind(address)?;

    for socket_result in listener.incoming() {
        let socket = socket_result?;
        let groups = chat_group_table.clone();
        thread::spawn(|| {
            log_error(serve(socket, groups));
        });
    }
\end{minted}

这样对每一个连接都会创建一个新的线程运行\texttt{serve}函数,这个函数专门处理一个连接的需求。

这可以正常工作,直到一切都比计划的更加顺利很多,然后突然你就已经有了几万名用户。一个线程的栈增长到100KB或更多并不罕见,你可能不想就这样花费几GB的内存。要把任务分发到多个处理器上,线程是合适并且必须的,但它们的内存需求太大以至于我们通常需要一些补充的方式和线程一起使用,来减小资源占用。

你可以使用Rust的\emph{异步任务(asynchronous task)}来在单个线程或者线程池中交替执行很多独立的任务。异步任务类似于线程,但可以更快地创建、更高效地传递控制权、并且内存开销比线程少一个数量级。在单个程序中同时运行数十万个异步任务是完全可行的。当然,你的应用仍然可能被其他因素限制,例如网络带宽、数据库速度、计算、或者任务本身的内存需求,但使用异步任务的固有内存开销比使用线程的要小很多。

一般来讲,异步Rust代码看起来和普通的多线程代码非常相似,除了那些可能阻塞的操作,例如I/O或获取锁的处理有一点不同。特殊对待这些操作让Rust有更多的信息了解你的代码的行为,这为进一步优化提供了可能。上面代码的异步版本看起来像这样:
\begin{minted}{Rust}
    use async_std::{net, task};

    let listener = net::TcpListener::bind(address).await?;

    let mut new_connections = listener.incoming();
    while let Some(socket_result) = new_connections.next().await {
        let socket = socket_result?;
        let groups = chat_group_table.clone();
        task::spawn(async {
            log_error(serve(socket, groups).await);
        });
    }
\end{minted}

这里使用了\texttt{async\_std} crate的网络和任务模块,并在可能阻塞的调用后加上了\texttt{.await}。但整体的结构和基于线程的版本一样。

这一章的目标不止是帮你编写异步代码,还要向你展示它的工作细节,以便你可以预测它在你的应用中的表现,并了解它在哪些方面最有价值。

\begin{enumerate}
    \item 为了展示异步编程的机制,我们列出了涵盖所有核心概念的最小语言功能集:future、异步函数、\texttt{await}表达式、task、\texttt{block\_on}和\texttt{spawn\_local} executor。
    \item 然后我们会展示异步块和\texttt{spawn} executor。它们是完成真实工作的最基础的部分,但从概念上讲,它们只是我们刚才提到过的功能的变体。在这个过程中,我们会指出一些你可能会遇到的异步编程特有的问题并解释如何处理它们。
    \item 为了展示所有功能的协调工作,我们会展示一个聊天服务器和客户端的完整代码,上面的代码片段就是其中一部分。
    \item 为了演示原语future和executor如何工作,我们会展示\texttt{spawn\_blocking}和\texttt{block\_on}的简单实现。
    \item 最后,我们介绍了异步接口中经常出现的\texttt{Pin}类型,它被用来确保异步函数和块future被安全地使用。
\end{enumerate}

\section{从同步到异步}

考虑当你调用下面的(不是异步的)函数时会发生什么:
\begin{minted}{Rust}
    use std::io::prelude::*;
    use std::net;

    fn cheapo_request(host: &str, port: u16, path: &str)
                          -> std::io::Result<String>
    {
        let mut socket = net::TcpStream::connect((host, port))?;

        let request = format!("GET {} HTTP/1.1\r\nHost: {}\r\n\r\n", path, host);
        socket.write_all(request.as_bytes())?;
        socket.shutdown(net::Shutdown::Write)?;

        let mut response = String::new();
        socket.read_to_string(&mut response)?;

        Ok(response)
    }
\end{minted}

这会打开一个到web服务器的TCP连接,以过时的协议向它发送一个简单的HTTP请求,\footnote{如果你真的需要一个HTTP客户端,考虑使用一些非常优秀的crate例如\texttt{surf}或\texttt{reqwest},它们会正确并且异步地完成任务。这个客户端基本只是设法获得HTTPS重定向。}然后读取响应。\autoref{f20-1}展示了这个函数的执行过程随时间的变化。

\begin{figure}[htbp]
    \centering
    \includegraphics[width=0.8\textwidth]{../img/f20-1.png}
    \caption{一个同步HTTP请求的过程(深颜色的区域标识等待操作系统)}
    \label{f20-1}
\end{figure}

图中展示了从左到右随着时间的推移,函数的调用栈的变化。每一个函数调用都是一个方块,位于它的调用者上面。显然,\texttt{cheapo\_request}函数贯穿整个执行过程。它调用了Rust标准库里的函数例如\texttt{TcpStream::connect}和\texttt{TcpStream}的\texttt{write\_all}和\texttt{read\_to\_string}实现。这些对其他函数的调用依次进行,但最终程序会进行\emph{系统调用},请求操作系统完成真正的工作,例如打开TCP连接,或者读写一些数据。

深灰色的区域表示程序正在等待操作系统完成系统调用。我们并没有按比例绘制这些时间。因为加入我们按比例绘制,整个图都将是深灰色:在实践中,这个函数把几乎所有时间都用在等待操作系统上。上面代码的执行时间将是系统调用之间的窄条。

当函数等待系统调用返回时,它所在的线程会阻塞住:它不能做任何事,直到系统调用结束。一个线程的栈达到几百或几千字节并不罕见,因此如果这是一个更大的系统的一部分,并且有很多线程做类似的任务,锁住这些线程的资源但除了等待什么也不做的代价是非常昂贵的。

为了解决这个问题,一个线程需要能在等待系统调用完成的同时去执行其他的任务。但如何实现这一点并不明显。例如,我们用来从套接字读取响应的函数的签名是:
\begin{minted}{Rust}
    fn read_to_string(&mut self, buf: &mut String) -> std::io::Result<usize>;
\end{minted}

它的类型表明了:这个函数直到工作完成或者出错时才会返回。这个函数是\emph{同步的}:当操作完成时调用者才恢复。如果我们想在操作系统进行工作的同时用我们的线程去做别的任务,那么我们需要一个新的提供这个函数的\emph{异步}版本的I/O库。

\subsection{\texttt{Future}}
Rust支持异步操作的方法是引入一个trait:\texttt{std::future::Future}:
\begin{minted}{Rust}
    trait Future {
        type Output;
        // 现在,把`Pin<&mut Self>`看作`&mut Self`就好了。
        fn poll(self: Pin<&mut Self>, cx: &mut Context<'_>) -> Poll<Self::Output>;
    }

    enum Poll<T> {
        Ready(T),
        Pending,
    }
\end{minted}

一个\texttt{Future}代表一个可以测试是否完成的操作。一个future的\texttt{poll}方法从来不会等待操作完成:它总是立即返回。如果操作完成了,\texttt{poll}会返回\texttt{Poll::Ready(output)},其中\texttt{output}是最后的结果。否则,它会返回\texttt{Pending}。当且仅当future值得再次poll时,它会通过调用一个\emph{waker}来通知我们,这是一个由\texttt{Context}提供的回调函数。我们称之为异步编程的“piñata 模型” :你唯一能对future做的就是使用\texttt{poll}敲打它,直到有一个值掉出来。

所有现代的操作系统都包含一些系统调用的变体,我们可以用它们来实现这种poll接口。例如在Unix和Windows上,如果你把网络套接字设置为非阻塞模式,那么如果它们正在阻塞时进行read和write会返回错误,你必须稍后再试。

因此\texttt{read\_to\_string}的一个异步版本的签名大概是这样:
\begin{minted}{Rust}
    fn read_to_string(&mut self, buf: &mut String)
        -> impl Future<Output = Result<usize>;
\end{minted}

除了返回类型之外,这和我们之前展示的签名一样:异步的版本返回\emph{一个\texttt{Result<usize>}的future}。你需要poll这个future,直到从它得到一个\texttt{Ready(result)}。每次它被poll时,都会尽可能地继续读取。最后的\texttt{result}给你成功的值或者错误的值,就像普通的I/O操作一样。这是通常的模式:异步版本的任何函数和同步版本的函数获取相同的参数,但返回类型有一个\texttt{Future}包装。

调用这个版本的\texttt{read\_to\_string}并不会真的读取任何内容;它所有的任务就是构造并返回一个future,这个future会在被poll时进行真正的工作。这个future必须包含处理请求所需的所有信息。例如,这个\texttt{read\_to\_string}返回的future必须记住调用它的输入流,和它需要写入数据的\texttt{String}。事实上,因为这个future持有了\texttt{self}和\texttt{buf}的引用,因此这个\texttt{read\_to\_string}的真正的签名必须是:
\begin{minted}{Rust}
    fn read_to_string<'a>(&'a mut self, buf: &'a mut String)
        -> impl Future<Output = Result<usize>> + 'a;
\end{minted}

这个附加的生命周期指示了返回的future和它借用的\texttt{self}和\texttt{buf}的生命周期一样长。

\texttt{async-std} crate提供了\texttt{std}的所有I/O设施的异步版本,包括一个有\texttt{read\_to\_string}方法的异步\texttt{Read} trait。\texttt{async-std}密切地遵循了\texttt{std}的设计,尽可能地在自己的接口中重用\texttt{std}的类型,因此这两个世界中的错误、结果、网络地址、和其他大多数相关的数据都是兼容的。熟悉\texttt{std}有助于使用\texttt{async-std},反之亦然。

\texttt{Future} trait的一个规则是,一旦一个future返回了\texttt{Poll::Ready},它会假设它决不会再次被poll。一些future在自己被overpoll时简单地永远返回\texttt{Poll::Pending};其它的可能panic或者挂起。(它们绝不能违反内存或线程安全性,或者导致未定义行为)\texttt{Future} trait的\texttt{fuse}适配器将任何future转换成overpoll时永远返回\texttt{Poll::Pending}。但通常消费future的方法都遵循这个规则,因此\texttt{fuse}通常不是必须的。

如果poll听起来效率底下,请不必担心。Rust的异步架构是精心设计的,所以只要你的基本I/O函数例如\texttt{read\_to\_string}是正确实现的,那么你只会在值的时poll一个future。每一次\texttt{poll}被调用时,某个东西应该返回\texttt{Ready},或者至少向目标前进一步。我们将在“\nameref{WhenPoll}”中解释这是如何工作的。

但使用future看起来有一个挑战:当你poll时,如果你得到了\texttt{Poll::Pending}那你应该怎么做?你将不得不四处寻找这个线程暂时可以做的其他工作,并记住一段时间之后返回到这个future,然后再次poll。你的整个系统将因为持续追踪谁正在pending和当它们完成时应该做什么而变得杂乱无章。我们的\texttt{cheapo\_request}函数的简洁性会被破坏。

好消息是:它并不是这样的!

\subsection{\texttt{async}函数和\texttt{await}表达式}
这里有一个\emph{异步函数}版本的\texttt{cheapo\_request}:
\begin{minted}{Rust}
    use async_std::io::prelude::*;
    use async_std::net;

    async fn cheapo_request(host: &str, port: u16, path: &str)
                                -> std::io::Result<String>
    {
        let mut socket = net::TcpStream::connect((host, port)).await?;

        let request = format!("GET {} HTTP/1.1\r\nHost: {}\r\n\r\n", path, host);
        socket.write_all(request.as_bytes()).await?;
        socket.shutdown(net::Shutdown::Write)?;

        let mut response = String::new();
        socket.read_to_string(&mut response).await?;

        Ok(response)
    }
\end{minted}

这和之前的版本基本相同,除了:
\begin{enumerate}
    \item 函数以\texttt{async}代替\texttt{fn}开头。
    \item 它使用了\texttt{async\_std} crate里的异步版本的\texttt{TcpStream::connect, write\_all, read\_to\_string}。它们都返回结果的future(本节中的示例使用了\texttt{async\_std}的1.7版本)。
    \item 每一次调用返回future的函数之后,代码都会加上\texttt{.await}。尽管这看起来像是访问一个结构体的\texttt{await}字段,但它实际上是语言内置的一个特殊语法,它会等待一个future直到它准备好。\texttt{await}表达式会求出future的最终值。这个函数正是通过它获取\texttt{connect, write\_all, read\_to\_string}的结果。
\end{enumerate}

和普通的函数不同,当你调用异步函数时,它会在执行实际的主体代码之前立即返回。显然,调用的返回值还没有被计算出来;你得到的是它的最终值的\emph{future}。因此如果你执行这行代码:
\begin{minted}{Rust}
    let response = cheapo_request(host, port, path);
\end{minted}

那么\texttt{response}将是一个\texttt{std::io::Result<String>}的future,\texttt{cheapo\_request}的函数体还没有开始执行。你不需要调整异步函数的返回类型;Rust会自动把\texttt{async fn f(...) -> T}看做一个返回\texttt{T}的future而不是直接返回\texttt{T}的函数。

一个异步函数返回的future包含了函数体运行时所需的所有信息:函数的参数、局部变量所需的空间,等等。(就好像你把调用栈捕获为了一个普通的Rust值。)因此\texttt{response}必须包含传入的\texttt{host, port, path},因为\texttt{cheapo\_request}的函数体需要它们才能运行。

future的具体类型由编译器根据函数体和参数自动生成。这个类型并没有名称;你只知道它实现了\texttt{Future<Output=R>},其中\texttt{R}是异步函数的返回类型。从这一点来看,异步函数的future类似于闭包:闭包也有匿名类型、也是由编译器生成并且实现了\texttt{FnOnce}、\texttt{Fn}和\texttt{FnMut} trait。

当你第一次poll \texttt{cheapo\_request}返回的future时,将会从函数体的尾部开始运行到第一个由\texttt{TcpStream::connect}返回的future的\texttt{await}。这个\texttt{await}表达式会poll \texttt{connect} future,如果它还没准备好,那么它会向调用者返回\texttt{Poll::Pending}:直到\texttt{TcpStream::connect}的future返回\texttt{Poll::Ready}时,对\texttt{cheapo\_request}的future的poll才能继续通过第一个\texttt{await}。因此表达式\texttt{TcpStream::connect(...).await}的一个大概等价的写法是:
\begin{minted}{Rust}
    {
        // 注意:这是伪代码,不是有效的Rust代码
        let connect_future = TcpStream::connect(...);
        'retry_point:
        match connect_future.poll(cx) {
            Poll::Ready(value) => value,
            Poll::Pending => {
                // 设置`cheapo_request`的future的下一次`poll`
                // 从'retry_point处恢复执行。
                ...
                return Poll::Pending
            }
        }
    }
\end{minted}

一个\texttt{await}表达式会获取future的所有权然后poll它。如果它已经准备好,那么future的最终值就是\texttt{await}表达式的值,并且会继续往下执行。否则,它向调用者返回\texttt{Poll::Pending}。

但关键的是,下一次poll \texttt{cheapo\_request}的future时将不会再次从函数的首部开始:相反,它从poll \texttt{connect\_future}的地方开始\emph{恢复(resume)}执行。直到future准备好之后我们才会继续执行这个异步函数的其他部分。

随着\texttt{cheapo\_request}的future继续被poll,它会从函数体里的一个\texttt{await}开始执行到下一个,只有当它正在等待的子future ready时才会继续。因此,\texttt{cheapo\_request}的future将会被poll多少次取决于子future的行为和函数本身的控制流。\texttt{cheapo\_request}的future会追踪下一次\texttt{poll}时的恢复点和所有的局部状态——变量、参数、临时值——恢复需要这些。

在函数中间挂起并稍后恢复执行的能力是异步函数独有的。当普通函数返回时,它的栈帧就消失了。因为\texttt{await}表达式依赖于恢复执行的能力,所以你只能在异步函数里使用它们。

在撰写本书时,Rust还不允许trait有异步方法。只有自由函数和特定类型固有的方法才可以是异步的。取消这个限制需要对语言进行一些修改。与此同时,如果你需要定义包含异步函数的trait,可以考虑使用\texttt{async-trait} crate,它提供了一个基于宏的解决方案。

\subsection{在同步代码中调用异步函数:\texttt{block\_on}}
某种意义上讲,异步函数只是在推卸责任。在异步函数里很容易获取一个future的值:只需要\texttt{await}它。但异步函数\emph{本身}也返回一个future,因此现在调用者需要负责poll它。最后,总有一个地方必须实际等待一个值。

我们可以使用\texttt{async\_std}的\texttt{task::block\_on}函数在普通的同步函数(例如\texttt{main})中调用\texttt{cheapo\_request},它获取一个future并且poll它直到它产生一个值:
\begin{minted}{Rust}
    fn main() -> std::io::Result<()> {
        use async_std::task;

        let response = task::block_on(cheapo_request("example.com", 80, "/))?;
        println!("{}", response);
        Ok(())
    }
\end{minted}

因为\texttt{block\_on}是一个产生异步函数的最终值的同步函数,你可以将它看作是异步世界到同步世界的适配器。但它阻塞的特性也意味着你永远不应该在一个异步函数里使用\texttt{block\_on}:它会阻塞整个线程直到值准备好。作为代替,请使用\texttt{await}。

\autoref{f20-2}展示了\texttt{main}的一个可能的执行过程。

\begin{figure}[htbp]
    \centering
    \includegraphics[width=0.9\textwidth]{../img/f20-2.png}
    \caption{阻塞等待一个异步函数}
    \label{f20-2}
\end{figure}

上面的时间线,“简化视图”,展示了程序的异步调用的抽象视图:\texttt{cheapo\_request}首先调用了\texttt{TcpStream::connect}来获取一个套接字,然后对套接字调用了\texttt{write\_all}和\texttt{read\_to\_string}。然后它返回。这和本章前面的同步版本的\texttt{cheapo\_request}的时间线非常相似。

但这里每一个异步调用都是多阶段的过程:一个future被创建,然后被poll直到它准备好,可能还会创建并poll其他子future。下面的时间线,“实现”,展示了实现了这个异步行为的实际同步调用。这是一个介绍普通的异步执行过程中到底发生了什么的好机会:
\begin{enumerate}
    \item 首先,\texttt{main}调用\texttt{cheapo\_request},它返回最终结果的future \texttt{A}。然后\texttt{main}把这个future传给了\texttt{async\_std::block\_on},它会poll \texttt{A}。
    \item poll future \texttt{A}允许\texttt{cheapo\_request}的函数体开始执行。函数里调用了\texttt{TcpStream::connect}来获取一个套接字的future \texttt{B}并await它。更确切地说,因为\texttt{TcpStream::connect}可能会遇到错误,因此\texttt{B}是一个\texttt{Result<TcpStream, std::io::Error}的future。
    \item future \texttt{B}被\texttt{await} poll。因为网络连接还没有建立好,所以\texttt{B.poll}返回\texttt{Poll::Pending},但会设置好当套接字准备好后唤醒调用它的任务。
    \item 因为future \texttt{B}还没有准备好,\texttt{A.poll}也会向它的调用者\texttt{block\_on}返回\texttt{Poll::Pending}。
    \item 因为\texttt{block\_on}没有别的事情可做,它会陷入睡眠。这时整个线程会阻塞。
    \item 当\texttt{B}的连接准备好之后,它会唤醒poll它的任务。这促使\texttt{block\_on}开始行动,它会尝试再次poll future \texttt{A}。
    \item poll \texttt{A}促使\texttt{cheapo\_request}在它的第一个\texttt{await}处恢复执行,然后再次poll \texttt{B}。
    \item 这一次,\texttt{B}准备好了:套接字已经创建完毕,因此它返回\texttt{Poll::Ready(Ok(socket))}到\texttt{A.poll}。
    \item 到此\texttt{TcpStream::connect}的异步调用就完成了。\texttt{TcpStream::connect(...).await}表达式的值就是\texttt{Ok(socket)}。
    \item \texttt{cheapo\_request}的函数体会继续正常执行,使用\texttt{format!}宏构造请求字符串并传递给\texttt{socket.write\_all}。
    \item 因为\texttt{socket.write\_all}是一个异步函数,它会返回一个future \texttt{C},\texttt{cheapo\_request}会await \texttt{C}。
\end{enumerate}

剩余的流程和之前相似。在\autoref{f20-2}所示的执行流程中,\texttt{socket.read\_to\_string}在准备好之前被poll了四次,每一次都会从套接字读取\emph{一些}数据,但\texttt{read\_to\_string}被指定为一直读取到输入的末尾,这需要好几次的操作。

听起来编写一个一直调用\texttt{poll}的循环并不难。但让\texttt{async\_std::task::block\_on}真正有价值的是:它知道怎么睡眠到恰好future值得再次poll,而不是浪费处理器的时间和电量来进行几十亿次无用的\texttt{poll}调用。基本的I/O函数例如\texttt{connect}和\texttt{read\_to\_string}返回的future保留了传递给\texttt{poll}的\texttt{Context}参数提供的唤醒器,并在\texttt{block\_on}应该醒来并再次尝试poll时调用唤醒器。我们将在“\nameref{WhenPoll}”中通过实现一个简单版本的\texttt{block\_on}来展示这具体是怎么工作的。

和我们之前展示的原始的同步版本一样,这个异步版本的\texttt{cheapo\_request}方法也把几乎所有的时间花费在等待操作完成上。如果时间轴是按比例绘制的,那么图将几乎完全是深灰色的,只有当程序被唤醒时会有几个计算过程对应的很细的条。

这里讲了很多细节。幸运的是,你通常可以只考虑简化的上层时间线:一些函数调用是同步的,其他是异步的并需要一个\texttt{await},但它们都只是函数调用。Rust的异步支持的成功取决于帮助程序员在实践中只需要考虑简化的视图,不会被实现的来回跳转干扰。

\subsection{spawn异步任务}
\texttt{async\_std::task::block\_on}函数会阻塞知道一个future的值准备好。但在单个future上完全阻塞一个线程并不比同步调用更好:本章的目的是让线程在等待的同时\emph{做别的工作}。

为了实现这一点,你可以使用\texttt{async\_std::task::spawn\_local}。这个函数接受一个future并把它添加到一个池,当\texttt{block\_on}阻塞等待的future还没准备好时\texttt{block\_on}会poll这个池。因此如果你把一堆future传递给\texttt{spawn\_local}并且之后对最终结果的future调用\texttt{block\_on},\texttt{block\_on}会poll每一个被spawn的future(当它们可以进一步执行时),并发运行整个池,直到结果准备好。

在撰写本书时,只有当你启用\texttt{async-std} crate的\texttt{unstable}特性时\texttt{spawn\_local}才可用。你需要在\emph{Cargo.toml}中用这样的一行引入\texttt{async-std}:
\begin{minted}{toml}
    async-std = { version = "1", features = ["unstable"] }
\end{minted}

\texttt{spawn\_local}函数是标准中用于启动新线程的\texttt{std::thread::spawn}函数的异步版本的类似物:
\begin{enumerate}
    \item \texttt{std::thread::spawn(c)}接收闭包\texttt{c}然后启动一个线程运行它,返回一个\texttt{std::thread::JoinHandle},它的\texttt{join}方法会等待线程结束并返回\texttt{c}返回的内容。
    \item \texttt{async\_std::task::spawn\_local(f)}接收future \texttt{f}并把它添加到当前线程调用\texttt{block\_on}时会poll的池里。\texttt{spawn\_local}会返回它自己的\texttt{async\_std::task::JoinHandle}类型,它本身是一个future,你可以await它来获取\texttt{f}的最终值。
\end{enumerate}

例如,假设我们想让一个HTTP请求的集合并发执行。这是第一次尝试:
\begin{minted}{Rust}
    pub async fn many_requests(requests: Vec<(String, u16, String)>)
                               -> Vec<std::io::Result<String>>
    {
        use async_std::task;

        let mut handles = vec![];
        for (host, port, path) in requests {
            handles.push(task::spawn_local(cheapo_request(&host, port, &path)));
        }

        let mut results = vec![];
        for handle in handles {
            results.push(handle.await);
        }

        results
    }
\end{minted}

这个函数对\texttt{requests}的每个元素调用\texttt{cheapo\_request},将每一个调用返回的future传给\texttt{spawn\_local}。它把最后的\texttt{JoinHandle}收集到一个vector并且await每一个。以任意顺序await join handles都是没问题的:因为请求已经被spawn,它们的future将会被按需poll,即这个线程调用了\texttt{block\_on}并且无事可做时。所有的请求会并发运行。一旦它们完成,\texttt{many\_requests}会向调用者返回结果。

上面的代码几乎是正确的,但Rust的借用检查器担心\texttt{cheapo\_request}的future的生命周期:
\begin{minted}{text}
    error: `host` does not live long enough
        handles.push(task::spawn_local(cheapo_request(&host, port, &path)));
                                       ---------------^^^^^-------------
                                       |              |
                                       |              borrowed value does not
                                       |              live long enough
                         argument requires that `host` is borrowed for `'static`
    }
    - `host` dropped here while still borrowed
\end{minted}

\texttt{path}也有一个类似的错误。

自然地,如果我们向异步函数传递引用,那么它们返回的future就必须持有这些引用,因此出于安全性future不能比它们借用的值生存的更久。任何持有引用的其他值也有相同的限制。

问题在于\texttt{spawn\_local}不能确保你会在\texttt{host}和\texttt{path}被drop之前等待任务结束。事实上,\texttt{spawn\_local}只接受生命周期是\texttt{'static}的future,因为你可以简单地忽略它返回的\texttt{JoinHandle}并让任务在程序的剩余部分执行时继续运行。这并不是异步任务独有的问题:当你尝试用\texttt{std::thread::spawn}启动一个线程,并且它的闭包捕获了局部变量的引用时也会遇到类似的错误。

一种解决这个问题的方法是创建另一个版本的获取参数所有权的异步函数:
\begin{minted}{Rust}
    async fn cheapo_owning_request(host: String, port: u16, path: String)
                                   -> std::io::Result<String> {
        cheapo_request(&host, port, &path).await
    }
\end{minted}

这个函数接收\texttt{String}而不是\texttt{\&str}引用,因此它的future自身将拥有\texttt{host}和\texttt{path},并且生命周期是\texttt{'static}。借用检查器可以看到它立刻await了\texttt{cheapo\_request}的future,并且因此如果这个future被poll,它借用的\texttt{host}和\texttt{path}变量肯定还在。一切都没有问题。

使用\texttt{cheapo\_owning\_request},你可以像这样spwan所有的请求:
\begin{minted}{Rust}
    for (host, port, path) in requests {
        handles.push(task::spawn_local(cheapo_owning_request(host, port, path)));
    }
\end{minted}

你可以使用\texttt{block\_on}在同步的\texttt{main}函数中调用\texttt{many\_requests}:
\begin{minted}{Rust}
    let requests = vec![
        ("example.com".to_string(),      80, "/".to_string()),
        ("www.red-bean.com".to_string(), 80, "/".to_string()),
        ("en.wikipedia.org".to_string(), 80, "/".to_string()),
    ];

    let results = async_std::task::block_on(many_requests(requests));
    for result in results {
        match result {
            Ok(response) => println!("{}", response),
            Err(err) => eprintln!("error: {}", err),
        }
    }
\end{minted}

这段代码会在\texttt{block\_on}的调用中并发运行三个请求。每个请求会在当其他的请求阻塞时抓住几乎继续执行,它们全部在调用者线程中执行。\autoref{f20-3}中展示了三个\texttt{cheapo\_request}调用的可能的执行过程。

\begin{figure}[htbp]
    \centering
    \includegraphics[width=0.9\textwidth]{../img/f20-3.png}
    \caption{在单个线程中运行三个异步任务}
    \label{f20-3}
\end{figure}

(我们鼓励你自己尝试运行这段代码,使用\texttt{eprintln!}在\texttt{cheapo\_request}首部和每一个\texttt{await}表达式之后打印消息,这样你可以看到这些调用如何交错执行。)

对\texttt{many\_requests}的调用(为了简单没有展示)spawn了三个异步的任务,分别用\texttt{A}、\texttt{B}、\texttt{C}标记。\texttt{block\_on}开始时先poll \texttt{A},\texttt{A}会开始连接到\texttt{example.com}。这会立刻返回\texttt{Poll::Pending},\texttt{block\_on}会把注意移动到下一个spawn的任务,然后poll future \texttt{B},最后是\texttt{C},它们会开始连接各自的服务器。

当所有可以poll的future都返回了\texttt{Poll::Pending}之后,\texttt{block\_on}会进入睡眠,直到其中一个\texttt{TcpStream::connect} future指示它的任务值得再次poll。

在这次执行中,服务器\texttt{en.wikipedia.org}比其他的响应得更快,因此这个任务最先完成。当一个spawn的任务完成时,它会把值保存在\texttt{JoinHandle}并标记它已经准备好了,这样\texttt{many\_requests} await它时无需等待,可以继续执行。最后,其它的\texttt{cheapo\_request}要么成功要么返回错误,然后\texttt{many\_request}本身可以返回了。最后,\texttt{main}接收\texttt{block\_on}返回的结果的vector。

所有这些执行都发生在单个线程中,三个\texttt{cheapo\_request}的调用通过对future的poll实现交错执行。一个异步调用看起来像是一个运行到完成的单个函数调用,但实际上异步调用由一系列对future的\texttt{poll}方法的同步调用实现。每一个单独的\texttt{poll}调用都可以快速返回,让出线程从而让其他异步调用可以执行。

我们终于达成了我们在本章开头设置的目标:让一个线程在等待I/O完成的同时去执行其他的工作,这样线程的资源不会因等待而浪费。更妙的是,达成这个目标的代码看起来非常像普通的Rust代码:一些函数被标记为\texttt{async}、一些函数调用后面有\texttt{.await}、使用的函数来自\texttt{async\_std}而不是\texttt{std},但除此之外,它就是普通的Rust代码。

异步任务和线程有一个不同之处需要牢记:异步任务只有在\texttt{await}表达式处被await的future返回\texttt{Poll::Pending}时才会切换到另一个异步任务。这意味着如果你在\texttt{cheapo\_request}中放了一段长时间运行的计算代码,那么在它完成之前,任何传给\texttt{spawn\_local}的其他任务都没有机会运行。而使用线程时没有这个问题:操作系统可以在任何地方挂起任何线程并设置计时器来确保没有线程可以垄断处理器。

异步代码依赖于共享线程的future的协作。如果你需要让长时间计算和异步代码共存,本章后面的“\nameref{LongCompute}”中介绍了一些方法。

\subsection{\texttt{async}块}
除了异步函数之外,Rust还支持\emph{异步块(asynchronous block)}。与一个普通的快块返回最后一个表达式的值不同,一个异步块返回最后一个表达式的\emph{值的future}。你可以在异步块里使用\texttt{await}表达式。

异步块看起来就像普通的块表达式,在前边加上\texttt{async}关键字:
\begin{minted}{Rust}
    let serve_one = async {
        use async_std::net;

        // 监听连接并接受
        let listener = net::TcpListener::bind("localhost:8087").await?;
        let (mut socket, _add) = listener.accept().await?;

        // 通过`socket`与客户端交互
        ...
    };
\end{minted}

这里用一个future初始化了\texttt{serve\_one},当poll它时,它会监听并处理单个TCP连接。块的代码直到\texttt{serve\_one}被poll才会执行,就像异步函数只有在它的future被poll时才会执行一样。

如果你在异步块里使用了\texttt{?}操作符,它会从块里返回,而不是从所处的函数返回。例如,如果上面的\texttt{bind}调用返回一个错误,那么\texttt{?}操作符会返回它作为\texttt{serve\_one}的最终值。类似的,\texttt{return}表达式会从异步块里返回,而不是从外层的函数返回。

如果一个异步块引用了周围代码里的变量,它的future会捕获那些变量,就像闭包一样。并且和\texttt{move}闭包一样(见“\nameref{StealClosure}”),你可以以\texttt{async move}来创建获取变量所有权的块,而不是持有变量的引用。

异步块提供了一种精确的分离出想要异步运行的部分代码的方法。例如,在上一节中,\texttt{spawn\_local}需要\texttt{'static} future,因此我们定义了\texttt{cheapo\_owning\_request}包装函数来得到一个获取参数所有权的future。你可以简单地在一个异步块里调用\texttt{cheapo\_request}而不需要分离出包装函数来实现相同的效果:
\begin{minted}{Rust}
    pub async fn many_requests(requests: Vec<(String, u16, String)>)
                               -> Vec<std::io::Result<String>>
    {
        use async_std::task;

        let mut handles = vec![];
        for (host, port, path) in requests {
            handles.push(task::spawn_local(async move {
                cheapo_request(&host, port, &path).await
            }));
        }
        ...
    }
\end{minted}

因为这是一个\texttt{async move}块,所以它的future获取了\texttt{String}值\texttt{host}和\texttt{path}的所有权,就类似\texttt{move}闭包一样。然后它向\texttt{cheapo\_request}传递引用,借用检查器可以看到块的\texttt{await}表达式获取了\texttt{cheapo\_request}的future的所有权,因此\texttt{host}和\texttt{path}的引用不可能比它们借用的被捕获的变量生存的更久。异步块和\texttt{cheapo\_owning\_request}完成了同样的事,但所需的样板代码更少。

一个你可能遇到的问题是没有语法能指定异步块的返回类型,即异步函数参数后跟的\texttt{-> T}。当使用\texttt{?}操作符时这可能会导致问题:
\begin{minted}{Rust}
    let input = async_std::io::stdin();
    let future = async {
        let mut line = String::new();

        // 这会返回`std::io::Result<usize>`。
        input.read_line(&mut line).await?;

        println!("Read line: {}", line);

        Ok(())
    };
\end{minted}

这会因为如下错误失败:
\begin{minted}{text}
    error: type annotations needed
       |
    42 |     let future = async {
       |         ------ consider giving `future` a type
    ...
    46 |         input.read_line(&mut line).await?;
       |         ^^^^^^^^^^^^^^^^^^^^^^^^^^^^^^^^^ cannot infer type
\end{minted}

Rust不能分辨异步块的返回类型应该是什么。\texttt{read\_line}方法返回\texttt{Result<(), std::io::Error>},但因为\texttt{?}操作符使用了\texttt{From} trait来在需要时转换成指定的错误类型,所以这个异步块的返回类型是\texttt{Result<(), E>},其中\texttt{E}可能是任何实现了\texttt{From<std::io::Error>}的类型。

Rust未来的版本可能会添加指定\texttt{async}块的返回类型的语法。但现在,可以通过手动写出最后的\texttt{Ok}的类型来解决这个问题:
\begin{minted}{Rust}
    let future = async {
        ...
        Ok::<(), std::io::Error>(())
    };
\end{minted}

因为\texttt{Result}是一个需要成功和错误类型作为参数的泛型类型,我们可以像这里一样使用\texttt{Ok}或\texttt{Err}指定那些类型参数。

\subsection{从异步块中构建异步函数}
异步块给了我们另一种实现和异步函数相同效果的方法,并且更加灵活一点。例如,我们可以将我们的\texttt{cheapo\_request}写成一个普通的、同步的返回异步块的future的函数:
\begin{minted}{Rust}
    use std::io;
    use std::future::Future;

    fn cheapo_request<'a>(host: &'a str, port: u16, path: &'a str)
        -> impl Future<Output = io::Result<String>> + 'a
    {
        async move {
            ... function body ...
        }
    }
\end{minted}

当你调用这个版本的函数时,它会立刻返回异步块的值的future。这个future会捕获函数的参数并且和异步函数返回的future的行为一样。因为我们没有使用\texttt{async fn}语法,我们需要在返回值中写出\texttt{impl Future},但对调用者来说,这两个定义是同一个函数签名的两种可替换的实现。

如果你想让函数被调用时立刻进行一些计算然后再构造返回的future,那么第二种方法更有用。例如,另一种协调\texttt{cheapo\_request}和\texttt{spawn\_local}的方法是让它变成一个返回\texttt{'static} future的同步函数,并让这个future捕获参数的拷贝的所有权:
\begin{minted}{Rust}
    fn cheapo_request(host: &str, port: u16, path: &str)    
        -> impl Future<Output = io::Result<String> + 'static
    {
        let host = host.to_string();
        let path = path.to_string();

        async move {
            ... use &*host, port, and path ...
        }
    }
\end{minted}

这个版本让异步块捕获\texttt{host}和\texttt{path}为\texttt{String}值,而不是\texttt{\&str}引用。因为future拥有自己运行所需的所有数据,所以它是有效的\texttt{'static}生命周期。(我们在上面的签名中写出了\texttt{+ 'static},但\texttt{-> impl}返回的类型默认是\texttt{'static}的,因此省略它不会有影响。)

因为这个版本的\texttt{cheapo\_request}返回的future是\texttt{'static}的,我们可以直接把它们传递给\texttt{spawn\_local}:
\begin{minted}{Rust}
    let join_handle = async_std::task::spawn_local(
        cheapo_request("areweasyncyet.rs", 80, "/")
    );

    ... other work ...

    let response = join_handle.await?;
\end{minted}

\subsection{在一个线程池中spawn异步任务}


\subsection{长时间计算:\texttt{yield\_now}和\texttt{spawn\_blocking}}\label{LongCompute}

\section{原语future和executor:何时一个future值得再次poll}\label{WhenPoll}
