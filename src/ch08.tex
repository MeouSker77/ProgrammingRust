\chapter{crate与模块}\label{ch08}

\emph{This is one note in a Rust theme: systems programmers can have nice things.}

\begin{flushright}
    ——Robert O'Callahan, “\href{https://robert.ocallahan.org/2016/08/random-thoughts-on-rust-cratesio-and.html}{Random Thoughts on Rust: crates.io and IDEs}”
\end{flushright}

假设你在编写一个仿真蕨类植物从细胞开始生长的程序。你的程序就像蕨类一样,一开始非常简单,可能所有代码都在单个文件里——就像一个孢子。随着它逐渐成长,它开始逐渐建立起内部的结构,不同的片段负责不同的功能。它将分裂为多个文件,可能覆盖整个目录树。随着时间的推移,它可能会成为整个软件生态系统的重要组成部分 。对于任何成长到不仅仅是几个数据结构和几百行代码的程序,都必须要对代码进行组织。

这一章将会介绍Rust中用于组织程序的特性:crate和模块。我们还会介绍Rust crate的结构和分发相关的话题,包括如何编写文档和测试Rust代码,如何禁用不需要的编译器警告,如何使用Cargo来管理项目依赖和版本,如何在Rust的公开crate仓库:crates.io上发布开源的库,crate的版本如何演变,等等。我们将使用蕨类仿真程序作为我们的例子。

\section{Crate}

Rust程序由\emph{crates}组成。每一个crate都是一个完整的、一体的单元:一个库或可执行文件的所有代码、加上相关的测试、示例、工具、配置、以及一些其他东西。为了编写你自己的蕨类模拟器,你可能需要使用和3D图形、生物信息学、并行计算等相关的第三方库。这些库就像箱子一样(见\hyperref[f8-1]{图8-1})。

\begin{figure}[htbp]
    \centering
    \includegraphics[width=0.9\textwidth]{../img/f8-1.png}
    \caption{一个crate和它的依赖}
    \label{f8-1}
\end{figure}

查看crate是什么以及它们是如何工作的最简单方法就是使用带有\texttt{--verbose}参数的\texttt{cargo build}来构建一个有一些依赖的程序。我们用“\hyperref[mandelbrot]{一个并发的曼德勃罗集}”作为示例。结果如下所示:
\begin{minted}{text}
    $ cd mandelbrot
    $ cargo clean   # delete previously compiled code
    $ cargo build --verbose
        Updating registry `https://github.com/rust-lang/crates.io-index`
     Downloading autocfg v1.0.0
     Downloading semver-parser v0.7.0
     Downloading gif v0.9.0
     Downloading png v0.7.0
    
    ... (downloading and compiling many more crates)

        Compiling jpeg-decoder v0.1.18
          Running `rustc
             --crate-name jpeg_decoder
             --crate-type lib
             ...
             --extern byteorder=.../libbyteorder-29efdd0b59c6f920.rmeta
             ...
        Compiling image v0.13.0
          Running `rustc
             --crate--name image
             --crate-type lib
             ...
             --extern byteorder=.../libbyteorder-29efdd0b59c6f920.rmeta
             --extern gif=.../libgif-a7006d35f1b58972.rmeta
             --extern jpeg_decoder=.../libjped_decoder-5c10558d0d57d300.rmeta
        Compiling mandelbrot v0.1.0 (/tmp/rustbook-test-files/mandelbrot)
          Running `rustc
             --edition=2018
             --crate-name mandelbrot
             --crate-type bin
             ...
             --extern crossbeam=.../libcrossbeam-f87b4b3d3284acc2.rlib
             --extern image=.../libimage-b5737c12bd641c43.rlib
             --extern num=.../libnum-1974e9a1dc582ba7.rlib -C link-arg=-fuse-ld=lld`
         Finished dev [unoptimized + debuginfo] target(s) in 16.94s
\end{minted}

我们重新格式化了\texttt{rustc}的命令行来改善可读性,并且删掉了很多和我们的讨论无关的编译器选项,用省略号(\ldots)代替了它们。

你可能还记得,当我们完成曼德勃罗集程序时,它的\texttt{main.rs}包含几个引入其它crate的\texttt{use}声明:
\begin{minted}{Rust}
    use num::Complex;
    // ...
    use image::ColorType;
    use image::png::PNGEncoder;
\end{minted}

哦我们还在\texttt{Cargo.toml}中指定了每个crate的版本:
\begin{minted}{toml}
    [dependencies]
    num = "0.4"
    image = "0.13"
    crossbeam = "0.8"
\end{minted}

这里的\emph{依赖}指这个程序使用的其它crate,也就是我们依赖的代码。我们可以在\href{https://crates.io}{crates.io}中找到这些crate,那是Rust社区用于存放开源的crate的网站。例如,我们可以访问crates.io并搜索图片库来找到\texttt{image}库。crates.io上的每个crate的页面上会显示它的\texttt{README.md}文件和到文档和源代码的链接,还有一行配置例如\texttt{image = "0.13"},你可以复制这一行并添加到你的\texttt{Crago.toml}中。这里显示的版本号直接用了我们在编写这个程序时这三个包的最新版本。

Cargo的输出说明了这些信息是如何被使用的。当我们运行\texttt{cargo build}时,Cargo会首先从crates.io下载这些crate的指定版本的源码。然后,它读取那些crate的\texttt{Cargo.toml}文件,下载\emph{它们}的依赖,然后递归操作。例如,\texttt{image} crate的0.13.0版本的源代码中包含一个\texttt{Cargo.toml}文件,内容如下:
\begin{minted}{toml}
    [dependencies]
    byteorder = "1.0.0"
    num-iter = "0.1.32"
    num-rational = "0.1.32"
    num-traits = "0.1.32"
    enum_primitive = "0.1.0"
\end{minted}

看到这些内容,Cargo知道在它可以使用\texttt{image}之前,它必须先拉取这些crate。我们称它们为\texttt{mandelbrot}的\emph{间接(transitive)}依赖。所有这些依赖的集合告诉了Cargo需要知道的有关如何构建和构建顺序的一切信息,它被称为crate的\emph{依赖图}。Cargo自动处理依赖图和间接依赖的能力是程序员们付出时间和努力的一大胜利。

当获得了源代码之后,Cargo会编译所有的crate。它会运行Rust的编译器\texttt{rustc},一次编译依赖图中的一个crate。当编译这些库时,Cargo会使用\texttt{--crate-type lib}选项。这告诉\texttt{rustc}不要寻找\texttt{main()}函数,而是产生一个包含编译过代码的\texttt{.rlib}文件,这个文件可以被用于创建可执行文件和其他\texttt{.rlib}文件。

当编译程序时,Cargo会使用\texttt{--crate-type bin},编译的结果将是一个目标平台的二进制可执行文件:例如在Windows上就是\texttt{mandelbrot.exe}。

对于每一个\texttt{rustc}命令,Cargo都会传递\texttt{--extern}选项,给出crate用到的每一个库的名称。这样,当\texttt{rustc}看到一行类似于\texttt{use image::png::PNGEncoder}的代码时,它可以分辨出\texttt{image}是另一个crate的名字,而且Cargo传递的选项让它知道该从哪里寻找编译好的crate。Rust的编译器需要访问这些\texttt{.rlib}文件,因为它们包含编译好的库中的代码。Rust将会将代码静态链接到最终的可执行文件中。\texttt{.rlib}还包含类型信息,因此Rust可以通过检查确保我们在代码中使用的库的特性确实存在而且被正确使用。它还包含一份crate的public内联函数、泛型、宏、特性的拷贝,这些东西只有当Rust看到我们如何使用它们时才可以将它们编译为机器代码。

\texttt{cargo build}支持各种选项,其中的大部分都超出了本书的范围,不过我们在这里会提到其中一个:\texttt{cargo build --release}会生成优化后的构建。Release构建运行得更快,但需要更长的时间来编译,而且它们不检查整数溢出、跳过\texttt{debug\_assert!()}断言,并且它们在panic生成的堆栈追踪通常不太可靠。

\subsection{版本}

Rust有极强的兼容性保证。任何在Rust 1.0中能编译的代码必须在Rust 1.50或者1.900(如果发布了的话)中也能编译。


但有时社区会遇到一些令人信服的扩展语言的建议,这可能会导致旧代码不能再编译。例如,经过了多次讨论之后,Rust确定了一种支持异步编程的语法,将标识符\texttt{async}和\texttt{await}重新用作关键字(见\hyperref[ch20]{第20章})。但这项语言的改变可能会导致使用\texttt{async}或者\texttt{await}作为变量名的代码不能再编译。

为了在不破坏这些现有代码的前提下演变,Rust使用了\emph{版本}。Rust的2015版本和Rust 1.0兼容。2018版本将\texttt{async}和\texttt{await}改为关键字、精简了模块系统、还引入了一些和2015版本不兼容的其它语言更改。每个crate在\texttt{Cargo.toml}文件中的\texttt{package}节中用一行类似如下的说明指定Rust的版本:
\begin{minted}{Rust}
    edition = "2018"
\end{minted}

如果缺少这个关键字,将会假设使用2015版本,因此旧的crate完全不需要做任何更改。但如果你想使用异步函数或者新的模块系统,你需要确保\texttt{Cargo.toml}中有\texttt{edition = "2018"}(或者可能更新的版本)。

Rust保证编译器将总是接受语言的所有版本,并且程序可以自由混合使用不同版本编写的crate。即使一个2015版本的crate依赖一个2018版本的crate也没有问题。换句话说,一个crate的版本只影响它的代码是如何被构建的,版本的区别只体现在代码编译的时候。这意味着没有必要更新旧的版本来适配现代Rust的生态。类似的,也没有必要将crate保持在旧版本来避免影响到它的用户。你只需要在想使用新的语言特性时更改自己代码中的版本。

版本并不是每年都会更新,只有当Rust项目觉得有必要出新版本的时候才会更新。例如,没有2020版本。把\texttt{edition}设置为\texttt{"2020"}将会导致错误。\href{https://doc.rust-lang.org/stable/edition-guide}{Rust版本指南}介绍了每一个版本中的变化,并提供了版本系统的背景知识。

使用最新版本几乎总是一个好主意,尤其是新编写代码时。\texttt{cargo new}会默认创建最新版本的项目。这本书中将始终使用2018版本。

如果你有一个用更旧版本的Rust编写的crate,\texttt{cargo fix}命令也许可以帮你自动把代码更新到更新的版本。Rust版本指南详细解释了\texttt{cargo fix}命令。

\subsection{构建配置}
